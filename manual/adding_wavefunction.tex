\section{Adding a wavefunction}\label{sec:adding_wavefunction}

The total wavefunction is stored in \texttt{TrialWaveFunction} as a product of all
the components.  Each component derives from \texttt{WaveFunctionComponent}.
The code contains an example of a wavefunction component for a Helium atom using a simple form and
is described in Section~\ref{sec:helium_wavefunction_example}.


\subsection{Mathematical preliminaries}

The wavefunction evaluation functions compute the log of the wavefunction,
the gradient and the Laplacian of the log of the wavefunction.
Expanded, the gradient and Laplacian are
\begin{eqnarray}
G &=& \grad \log(\psi) = \frac{\grad \psi}{\psi} \\
L &=& \lap \log(\psi) = \frac{\lap \psi}{\psi} - \frac{\grad \psi}{\psi} \cdot \frac{\grad \psi}{\psi} \\
                &=& \frac{\lap \psi}{\psi} - G \cdot G
\end{eqnarray}

However, the local energy formula needs $\frac{\lap \psi}{\psi}$.  The conversion from the Laplacian of the log of the wavefunction to the local energy value is performed
in \texttt{QMCHamiltonians/BareKineticEnergy.h} (i.e. $L + G \cdot G$.)


\subsection{Wavefunction evaluation}

The process for creating a new wavefunction component class is to derive
from WaveFunctionComponent and implement a number pure virtual functions.
To start most of them can be empty.

The following four functions evaluate the wavefunction values and spatial derivatives:

\begin{description}
\item{\texttt{evaluateLog}} Computes the log of the wavefunction and the gradient
and Laplacian (of the log of the wavefunction) for all particles.
The input is the\texttt{ParticleSet}(\texttt{P}) (of the electrons).
The return value is the log of wavefunction, and the gradient is in \texttt{G} and Laplacian in \texttt{L}.

\item{\texttt{ratio}} Computes the wavefunction ratio (not the log) for a single particle move ($\psi_{new}/\psi_{old}$).
The inputs are the \texttt{ParticleSet}(\texttt{P}) and the particle index (\texttt{iat}).

\item{\texttt{evalGrad}} Computes the gradient for a given particle.
The inputs are the \texttt{ParticleSet}(\texttt{P}) and the particle index (\texttt{iat}).

\item{\texttt{ratioGrad}} Computes the wavefunction ratio and the gradient at the new position for a single particle move.
The inputs are the \texttt{ParticleSet}(\texttt{P}) and the particle index (\texttt{iat}).
The output gradient is in \texttt{grad\_iat};
\end{description}

The \texttt{updateBuffer} function needs to be implemented, but to start it can simply
call \texttt{evaluateLog}.

The \texttt{put} function should be implemented to read parameter specifics from the input XML file.

\subsection{Function use}

For debugging it can be helpful to know the under what conditions the various
routines are called.

The VMC and DMC loops initialize the walkers by calling \texttt{evaluateLog}.
For all-electron moves, each timestep advance calls \texttt{evaluateLog}.
If the \texttt{use\_drift} parameter is no, then only the wavefunction value is used for sampling.
The gradient and Laplacian are used for computing the local energy.

For particle-by-particle moves, each timestep advance
\begin{enumerate}
\item calls \texttt{evalGrad}
\item computes a trial move
\item calls \texttt{ratioGrad} for the wavefunction ratio and the gradient at the trial position.
(If the \texttt{use\_drift} parameter is no, the \texttt{ratio} function is called instead.)
\end{enumerate}


The following example shows part of an input block for VMC with all-electron moves and drift.

\begin{verbatim}
<qmc method="vmc" target="e" move="alle">
  <parameter name="use_drift">yes</parameter>
</qmc>
\end{verbatim}



\subsection{Particle distances}

The \texttt{ParticleSet} parameter in these functions refers to the electrons.
The distance tables that store the inter-particle distances are stored as an array.

To get the electron-ion distances, add the ion \texttt{ParticleSet} using \texttt{addTable}
and save the returned index. Use that index to get the ion-electron distance table.
\begin{verbatim}
const int ei_id = elecs.addTable(ions, DT_SOA); // in the constructor only
const auto& ei_table = elecs.getDistTable(ei_id); // when consuming a distance table
\end{verbatim}
Getting the electron-electron distances is very similar, just add the electron \texttt{ParticleSet} using \texttt{addTable}.

Only the lower triangle for the electron-electron table should be used.
It is the only part of the distance table valid throughout the run.
During particle-by-particle move, there are extra restrictions.
When a move of electron iel is proposed, only the lower triangle parts [0,iel)[0,iel) [iel, Nelec)[iel, Nelec) and the row [iel][0:Nelec) are valid.
In fact, the current implementation of distance based two and three body Jastrow factors in QMCPACK only needs the row [iel][0:Nelec).

In \texttt{ratioGrad}, the new distances are stored in the \texttt{Temp\_r} and \texttt{Temp\_dr}
members of the distance tables.

\subsection{Setup}

A builder processes XML input, creates the wavefunction, and adds it to \texttt{targetPsi}.
Builders derive from \texttt{WaveFunctionComponentBuilder}.

The new builder hooks into the XML processing in \texttt{WaveFunctionFactory.cpp} in the \texttt{build} function.


\subsection{Caching values}
The \texttt{acceptMove} and \texttt{restore} methods are called on accepted and rejected moves for
the component to update cached values.


\subsection{Threading}
The \texttt{makeClone} function needs to be implemented to work correctly with OpenMP threading.
There will be one copy of the component created for each thread.
If there is no extra storage, calling the copy constructor will be sufficient.
If there are cached values, the clone call may need to create space.


\subsection{Parameter optimization}

The \texttt{checkInVariables}, \texttt{checkOutVariables}, and \texttt{resetParameters} functions manage the variational parameters.
Optimizable variables also need to be registered when the XML is processed.

Variational parameter derivatives are computed in the \texttt{evaluateDerivatives} function.
The first output value is an array with parameter derivatives of log of the wavefunction.
The second output values is an array with parameter derivatives of
the Laplacian divided by the wavefunction (and not the Laplacian of the log of the wavefunction)
The kinetic energy term contains a $-1/2m$ factor.
The $1/m$ factor is applied in \texttt{TrialWaveFunction.cpp}, but the $-1/2$ is not and must be accounted for in this function.


%See \texttt{QMCWaveFunctions/Jastrow/DiffTwoBodyJastrowOrbital.h} for an
%application.
%Also note that it calls into the radial functions for derivatives, but
%it expects the derivative of the logs.

%The conversion is
%\begin{equation}
%\frac{\partial }{\partial B} = \frac{\partial}{\partial B} \lap \log(\psi) +
%2 \frac{\grad \psi}{\psi} \frac{\partial}{\partial B} \frac{\grad \psi}{\psi}
%\end{equation}

\subsection{Helium Wavefunction Example}\label{sec:helium_wavefunction_example}
The code contains an example of a wavefunction component for a Helium atom using STO orbitals and a Pade Jastrow.

The wavefunction is
\begin{equation}
\psi = \frac{1}{\sqrt{\pi}} \exp(-Z r_1) \exp(-Z r_2) \exp(A / (1 + B r_{12}))
\end{equation}
where $Z = 2$ is the nuclear charge, $A=1/2$ is the electron-electron cusp, and $B$ is a variational parameter.
The electron-ion distances are $r_1$ and $r_2$, and $r_{12}$ is the electron-electron distance.
The wavefunction is the same as the one expressed with built-in components in \texttt{examples/molecules/He/he\_simple\_opt.xml}.

The code is in \texttt{src/QMCWaveFunctions/ExampleHeComponent.cpp}.
The builder is in \texttt{src/QMCWaveFunctions/ExampleHeBuilder.cpp}.
The input file is in \texttt{examples/molecules/He/he\_example\_wf.xml}.
A unit test compares results from the wavefunction evaluation functions for
consistency in \texttt{src/QMCWaveFunctions/tests/test\_example\_he.cpp}.

The recommended approach for creating a new wavefunction component is to copy the example and the unit test.
Implement the evaluation functions and ensure the unit test passes.
