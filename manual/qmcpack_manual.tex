\documentclass[11pt,letterpaper]{report}
\usepackage{qmcpack_manual}
\usepackage{bibtopic}
\bibliographystyle{ieeetr}
\usepackage{amsmath}
\usepackage{amssymb}
\usepackage{delarray}
\usepackage{algorithmic}
\usepackage{algorithm}
\usepackage{makeidx}
\usepackage{fancyhdr}
\usepackage{xcolor}
\usepackage[colorlinks=true,linkcolor=blue,urlcolor=blue]{hyperref} %for urls
\usepackage{tabularx}
\usepackage{placeins}
\usepackage{caption}
\usepackage{graphicx}

% making listing behave properly
%   with setting below, listings now render correctly
%   copy/paste from pdf is still messed up (is this even possible to fix?)
%     -indentation whitespace is not preserved (needed for Python)
%     -copy/paste can result in mangled text
%     -mangling depends on pdf viewer (it is different for acroread and evince)
%     -verbatim suffers from this also

\usepackage{upquote}  % render ' properly
\usepackage{qmcpack_listings}

% set margins for whole document, lots of wasted space at top and bottom originally
\usepackage[left=1.0in,right=1.0in,top=1.0in,bottom=1.0in]{geometry}




\newcommand{\HRule}{\rule{\linewidth}{0.5mm}}
%% \newcommand{\courier}[1]{{\fontfamily{pcr}\selectfont #1}}

% for markup, as needed
\newcommand{\red}[1]{{\color{red} #1}}
\newcommand{\blue}[1]{{\color{blue} #1}}

% hide or show text relevant to developers
\newcommand{\dev}[1]{#1}
%\newcommand{\dev}[1]{}

% efficiently comment out/hide blocks of text for any purpose
\newcommand{\hide}[1]{}


% control display of instructions in the labs
%   normally one only wants to show the 'workstation' way of running the labs
\newif\ifws
\wstrue
%   for the pdf used during the labs, one wants to show the host supercomputer way
%\wsfalse
%  command for switching inline text (do not wrap verbatim environments with this!)
\ifws
\newcommand{\labsw}[2]{#1}
\else
\newcommand{\labsw}[2]{#2}
\fi


\oddsidemargin 0cm
\evensidemargin 0cm
\textwidth 6.5in


% proper rendering of qmcpack
\newcommand{\qmcpack}{{QMCPACK} } % apparently the trailing whitespace is significant

% mathematics convenience commands
\newcommand{\abs}[1]{\lvert #1 \rvert}
\newcommand{\norm}[1]{\lVert #1 \rVert}
\newcommand{\pnorm}[2]{\lVert #1 \rVert_{#2}}
\newcommand{\mean}[1]{\langle #1 \rangle}
\newcommand{\ket}[1]{\lvert #1 \rangle}
\newcommand{\bra}[1]{\langle #1 \rvert}
\newcommand{\expval}[3]{\bra{#1}#2\ket{#3}}
\newcommand{\expvalh}[3]{\bra{#1}\hat{#2}\ket{#3}}
\newcommand{\overlap}[2]{\langle #1 \lvert #2 \rangle}
\newcommand{\operator}[3]{\ket{#1} #2 \bra{#3}}
\newcommand{\idop}{\hat{\mathbb{1}}}
\newcommand{\bs}{\boldsymbol}
\newcommand{\tr}{\text{tr}} % trace
\newcommand{\grad}{\nabla}
\newcommand{\lap}{\nabla^2}  % laplacian

% urls are were too large
% hyperref gives us an easy way to control that
\renewcommand{\UrlFont}{\ttfamily\small}

% latex itself doesn't give us an easy way to deal with \texttt's font size
% so we need to define this command
\usepackage{letltxmacro}
% https://tex.stackexchange.com/q/88001/5764
\LetLtxMacro\oldttfamily\ttfamily
\DeclareRobustCommand{\ttfamily}{\oldttfamily\csname ttsize\endcsname}
\newcommand{\setttsize}[1]{\def\ttsize{#1}}%

% while \texttt use should be sparing (see contributing.tex) its necessary for the
% QMCPACK input XML spec tables, but those are also nearly too big for the page
\setttsize{\footnotesize}

% We have a huge number of overfull boxes, this adds another pass to
% typesetting a paragraph properly
\setlength{\emergencystretch}{3em}


\begin{document}

\input{title.tex}
\newpage
\tableofcontents
\newpage

\begin{btUnit}

\chapter{Introduction}
\label{chap:introduction}

QMCPACK is an open-source, high-performance electronic structure code
that implements numerous Quantum Monte Carlo (QMC) algorithms. Its main
applications are electronic structure calculations of molecular,
periodic 2D, and periodic 3D solid-state systems. Variational Monte
Carlo (VMC), diffusion Monte Carlo (DMC), and a number of other
advanced QMC algorithms are implemented. By directly solving the
Schrodinger equation, QMC methods offer greater accuracy than methods
such as density functional theory but at a trade-off of much greater
computational expense. Distinct from many other correlated many-body
methods, QMC methods are readily applicable to both bulk
(periodic) and isolated molecular systems.

QMCPACK is written in C++ and is designed with the modularity afforded by
object-oriented programming. It makes extensive use of template
metaprogramming to achieve high computational efficiency. Because of the
modular architecture, the addition of new wavefunctions, algorithms,
and observables is relatively straightforward. For parallelization,
QMCPACK uses a fully hybrid (OpenMP,CUDA)/MPI approach to optimize
memory usage and to take advantage of the growing number of cores per
SMP node or graphical processing units (GPUs) and accelerators. High
parallel and computational efficiencies are achievable on the largest
supercomputers. Finally, QMCPACK uses standard file formats for
input and output in XML and HDF5 to facilitate data exchange.

This manual currently serves as an introduction to the essential features
of QMCPACK and as a guide to installing and running it. Over time this
manual will be expanded to include a fuller introduction to QMC
methods in general and to include more of the specialized features in
QMCPACK.
%Test of the bibliography\cite{CeperleyAlderPRL1980}.

\section{Quickstart and a first QMCPACK calculation}
In case you are keen to get started, this section describes how to quickly
build and run QMCPACK on a standard UNIX or Linux-like system. The
autoconfiguring build system usually works without much fuss on these
systems.  If C++, MPI, BLAS/LAPACK, FFTW, HDF5, and CMake are already
installed, QMCPACK can be built and run within five minutes. For
supercomputers, cross-compilation systems, and other computer clusters,
the build system might require hints on the locations of libraries and
which versions to use, typical of any code; see Chapter
\ref{chap:obtaininginstalling}. Section \ref{sec:installexamples}
includes complete examples of installations for common workstations and supercomputers that you can reuse.

To build QMCPACK:

\begin{enumerate}
\item Download the latest QMCPACK distribution from
  \url{http://www.qmcpack.org}.
\item Untar the archive (e.g., \ishell{tar xvf
    qmcpack\_v1.3.tar.gz}).
\item Check the instructions in the README file.
\item Run CMake in a suitable build directory to configure QMCPACK for
  your system: \ishell{cd qmcpack/build; cmake ..}
\item If CMake is unable to find all needed libraries, see Chapter
  \ref{chap:obtaininginstalling} for instructions and specific build
  instructions for common systems.
\item Build QMCPACK: \ishell{make} or \ishell{make -j 16}; use the latter
  for a faster parallel build on a system using, for example, 16 processes.
\item The QMCPACK executable is \ishell{bin/qmcpack}.
\end{enumerate}

QMCPACK is distributed with examples illustrating different
capabilities. Most of the examples are designed to run quickly with
modest resources. We'll run a short diffusion Monte Carlo calculation
of a water molecule:

\begin{enumerate}
\item Go to the appropriate example directory: \ishell{cd
    ../examples/molecules/H2O}.
\item (Optional) Put the QMCPACK binary on your path:\\ \ishell{export PATH=\$PATH:location-of-qmcpack/build/bin}.
\item Run QMCPACK: \ishell{../../../build/bin/qmcpack simple-H2O.xml} or
  \ishell{qmcpack simple-H2O.xml} if you followed the step above.
\item The run will output to the screen and generate a number of files:
\begin{verbatim}
$ls H2O*
H2O.HF.wfs.xml      H2O.s001.scalar.dat H2O.s002.cont.xml
H2O.s002.qmc.xml    H2O.s002.stat.h5    H2O.s001.qmc.xml
H2O.s001.stat.h5    H2O.s002.dmc.dat    H2O.s002.scalar.dat
\end{verbatim}
\item Partially summarized results are in the standard text files with the
  suffixes scalar.dat and dmc.dat. They are viewable with any standard editor.
\end{enumerate}

If you have Python and matplotlib installed, you can use the
\ishell{qmca} analysis utility to produce statistics and plots of the
data. See Chapter \ref{chap:analyzing} for information on analyzing
QMCPACK data.
\begin{verbatim}
export PATH=$PATH:location-of-qmcpack/nexus/bin 
export PYTHONPATH=$PYTHONPATH:location-of-qmcpack/nexus/library
qmca H2O.s002.scalar.dat         # For statistical analysis of the DMC data
qmca -t -q e H2O.s002.scalar.dat # Graphical plot of DMC energy
\end{verbatim}

The last command will produce a graph as per
Fig. \ref{fig:quick_qmca_dmc_trace}. This shows the average energy of
the DMC walkers at each timestep. In a real simulation we would have
to check equilibration, convergence with walker population, time step, etc.

Congratulations, you have completed a DMC calculation with QMCPACK!

\begin{figure}
  \centering
  \includegraphics[width=10cm]{./figures/quick_qmca_dmc_trace.png}
  \caption{Trace of walker energies produced by the qmca tool for a simple
    water molecule example.}
  \label{fig:quick_qmca_dmc_tracej}
\end{figure}

\section{Authors and History}
\label{sec:history}
QMCPACK was initially written by Jeongnim Kim while in the group of
Professor David Ceperley at the University of Illinois at
Urbana-Champaign, with later contributations being made at Oak Ridge National Laboratory (ORNL). Over the years, many others have contributed, particularly
students and researchers in the groups of Professor David Ceperley
and Professor Richard M. Martin, as well as staff at Lawrence Livermore
National Laboratory, Sandia National Laboratories, Argonne National
Laboratory, and ORNL.

Additional developers, contributors, and advisors include
Anouar Benali,
Mark A. Berrill,  
David M. Ceperley, 
Simone Chiesa,
Raymond C. III Clay,
Bryan Clark,
Kris T. Delaney,
Kenneth P. Esler,
Paul R. C. Kent,
Jaron T. Krogel,
Ying Wai Li,
Ye Luo,
Jeremy McMinis,
Miguel A. Morales,
William D. Parker,
Nichols A. Romero,
Luke Shulenburger,
Norman M. Tubman,
and Jordan E. Vincent.

If you should be added to this list, please let us know.

Development of QMCPACK has been supported financially by
several grants, including the following:

\begin{itemize}
\item ``Network for ab initio many-body methods: development, education
  and training'' supported through the Predictive
  Theory and Modeling for Materials and Chemical Science program by
  the U.S. Department of Energy Office of Science, Basic Energy
  Sciences
\item ``QMC Endstation,'' supported by Accelerating Delivery of Petascale
  Computing Environment at the DOE Leadership Computing Facility at
  ORNL
\item PetaApps, supported by the US National Science
  Foundation
\item Materials Computation Center (MCC), supported by the
  US National Science Foundation
\end{itemize}


\section{Support and Contacting the Developers}
\label{sec:support}

Questions about installing, applying, or extending QMCPACK can be
posted on the QMCPACK Google group at
\url{https://groups.google.com/forum/#!forum/qmcpack}. You may also
email any of the developers, but we recommend checking the group
first. Particular attention is given to any problem reports.

\section{Performance}
\label{sec:performance}

QMCPACK implements modern Monte Carlo (MC) algorithms, is highly parallel,
and is written using very efficient code for high per-CPU or on-node performance. In particular, the code is highly vectorizable,
giving high performance on modern central processing units (CPUs) and GPUs. We believe QMCPACK
delivers performance either comparable to or better than other QMC
codes when similar calculations are run, particularly for the most
common QMC methods and for large systems. If you find a calculation where this is not the
case, or you simply find performance slower than expected, please post on the Google
group or contact one of the developers. These reports are valuable. If your calculation is
sufficiently mainstream we will optimize QMCPACK to improve
the performance.

\section{Open source license}
\label{sec:license}

QMCPACK is distributed under the University of Illinois at
Urbana-Champaign/National Center for Supercomputing Applications (UIUC/NCSA) Open
Source License. 

\begin{verbatim}
		  University of Illinois/NCSA Open Source License

Copyright (c) 2003, University of Illinois Board of Trustees.
All rights reserved.

Developed by:   
  Jeongnim Kim
  Condensed Matter Physics,
  National Center for Supercomputing Applications, University of Illinois
  Materials computation Center, University of Illinois
  http://www.mcc.uiuc.edu/qmc/

Permission is hereby granted, free of charge, to any person obtaining a
copy of this software and associated documentation files (the
``Software''), to deal with the Software without restriction, including
without limitation the rights to use, copy, modify, merge, publish,
distribute, sublicense, and/or sell copies of the Software, and to
permit persons to whom the Software is furnished to do so, subject to
the following conditions:

        * Redistributions of source code must retain the above copyright 
          notice, this list of conditions and the following disclaimers.
        * Redistributions in binary form must reproduce the above copyright 
          notice, this list of conditions and the following disclaimers in 
          the documentation and/or other materials provided with the 
          distribution.
        * Neither the names of the NCSA, the MCC, the University of Illinois, 
          nor the names of its contributors may be used to endorse or promote 
          products derived from this Software without specific prior written 
          permission.

THE SOFTWARE IS PROVIDED "AS IS", WITHOUT WARRANTY OF ANY KIND, EXPRESS
OR IMPLIED, INCLUDING BUT NOT LIMITED TO THE WARRANTIES OF MERCHANTABILITY, 
FITNESS FOR A PARTICULAR PURPOSE AND NONINFRINGEMENT. IN NO EVENT SHALL 
THE CONTRIBUTORS OR COPYRIGHT HOLDERS BE LIABLE FOR ANY CLAIM, DAMAGES OR 
OTHER LIABILITY, WHETHER IN AN ACTION OF CONTRACT, TORT OR OTHERWISE, 
ARISING FROM, OUT OF OR IN CONNECTION WITH THE SOFTWARE OR THE USE OR 
OTHER DEALINGS WITH THE SOFTWARE.
\end{verbatim}

Copyright is generally believed to remain with the authors of the
individual sections of code. See the various notations in the source code as
well as the code history.

\section{Contributing to QMCPACK}
\label{sec:contributing}

QMCPACK is fully open source, and we welcome contributions. If you are
planning a development, early discussions are encouraged. Please
post on the QMCPACK Google group or contact the developers. We can tell you whether anyone else is working on a similar feature or whether
any related work has been done in the past.  Credit for your
contribution can be obtained, for example, through citation of a paper or by
becoming one of the authors on the next version of the standard
QMCPACK reference citation.

A guide to developing for QMCPACK, including instructions on how to
work with GitHub and make pull requests (contributions) to the main
source are listed on the QMCPACK GitHub wiki:
\url{https://github.com/QMCPACK/qmcpack/wiki}.

Contributions are made under the same license as QMCPACK, the
UIUC/NCSA open source license. If this is problematic, please discuss
with a developer.

Please note the following guidelines for contributions:
\begin{itemize}
\item Additions should be fully synchronized with the latest release
  version and ideally the latest develop branch on github. Merging of code
  developed on older versions is error prone.
\item Code should be cleanly formatted, commented, portable, and accessible to
  other programmers. That is, if you need to use any clever tricks, add a comment
  to note this, why the trick is needed, how it works, etc. Although we like
  high performance, ease of maintenance and accessibility are also
  considerations.
\item Comment your code. You are not only writing it for the compiler
  for also for other humans! (We know this is a repeat of the previous
  point, but it is important enough to repeat.)
\item Write a brief description of the method, algorithms, and inputs and outputs
  suitable for inclusion in this manual.
\item Develop some short tests that exercise the
  functionality that can be used for validation and for examples. We
  can help with this and their integration into the test system.
\end{itemize}

\section{QMCPACK Roadmap}
\label{sec:roadmap}

A general outline of the QMCPACK roadmap is given in Sections 1.7.1 and 1.7.2 . Suggestions
for improvements are welcome, particularly those that would facilitate new
scientific applications. For example, if an interface to a particular
quantum chemical or density functional code would help, this would be
given strong consideration.

\subsection{Code}

We will continue to improve the accessibility and usability of
QMCPACK through combinations of more convenient input parameters, improved
workflow, integration with more quantum chemical and density
functional codes, and a wider range of examples.

In terms of methodological development, we expect to significantly
increase the range of QMC algorithms in QMCPACK in the near future.

Computationally, we are porting QMCPACK to the next generation of
supercomputer systems. The internal changes required to run efficiently on these
systems are expected to benefit \emph{all} platforms due
to improved vectorization, cache utilization, and memory performance.

\subsection{Documentation}

This manual describes the core features of QMCPACK that are
required for routine research calculations, i.e., the VMC and DMC
methods, how to obtain and optimize trial wavefunctions, and simple
observables. Over time this manual will be expanded to include a
broader introduction to QMC methods and to describe more features of
the code.

Because of its history as a research code, QMCPACK contains a variety of
additional QMC methods, trial wavefunction forms, potentials, etc.,
that, although not critical, might be very useful for specialized
calculations or particular material or chemical systems. These
``secret features'' (every code has these) are not actually secret but
simply lack descriptions, example inputs, and tests. You are
encouraged to browse and read the source code to find them. New
descriptions will be added over time but can also be prioritized and
added on request (e.g., if a specialized Jastrow factor would help or
a historical Jastrow form is needed for benchmarking).



\chapter{Features of QMCPACK}
\label{chap:features}
\section{Production-level features}
The following list contains the main production-level features of QMCPACK. If
you do not see a specific feature that you are interested in,
see the remainder of this manual and ask whether that specific feature is
available or can be quickly  brought to the full production level.

\begin{itemize}
\item Variational Monte Carlo (VMC)
\item Diffusion Monte Carlo (DMC)
\item Reptation Monte Carlo
\item Single and multideterminant Slater Jastrow wavefunctions
\item Wavefunction updates using optimized multideterminant algorithm of Clark et al.
\item Backflow wavefunctions
\item One, two, and three-body Jastrow factors
\item Excited state calculations via flexible occupancy assignment of Slater determinants
\item All electron and nonlocal pseudopotential calculations
\item Casula T-moves for variational evaluation of nonlocal
  pseudopotentials (non-size-consistent and size-consistent variants)
\item Wavefunction optimization using the ``linear method'' of Umrigar
  and coworkers, with arbitrary mix of variance and energy in the
  objective function
\item Blocked, low memory adaptive shift optimizer of Zhao and Neuscamman 
\item Gaussian, Slater, plane-wave, and real-space spline basis sets for orbitals
\item Interface and conversion utilities for plane-wave wavefunctions from Quantum Espresso (Plane-Wave Self-Consistent Field package [PWSCF])
\item Interface and conversion utilities for Gaussian-basis wavefunctions from GAMESS
\item Easy extension and interfacing to other electronic structure codes via standardized XML and HDF5 inputs
\item MPI parallelism
\item Fully threaded using OpenMP
\item GPU (NVIDIA CUDA) implementation (limited functionality)
\item HDF5 input/output for large data
\item Nexus: advanced workflow tool to automate all aspects of QMC calculation from initial DFT calculations through to final analysis
\item Analysis tools for minimal environments (Perl only) through to
  Python-based environments with graphs produced via matplotlib (included with Nexus)
\end{itemize}

\section{SoA optimizations and improved algorithms}
The Structure-of-Arrays (SoA) optimizations \cite{IPCC_SC17} are a set
of improved data layouts facilitating vectorization on modern CPUs
with wide SIMD units.  \textbf{For many calculations and
  architectures, the SoA implementation more than doubles the speed of
  the code.}  This so-called SoA implementation replaces the older, less efficient
Array-of-Structures (AoS) code and can be enabled or disabled at compile time. The memory footprint is
also reduced in the SoA implementation by better
algorithms, enabling more systems to be run.

The SoA build was made the default for QMCPACK v3.7.0. As described in Section \ref{sec:cmakeoptions}, the SoA
implementation can be disabled by configuring with \ishell{-DENABLE_SOA=0}.

The SoA code path currently does \textit{not} support:
\begin{itemize}
\item Backflow wavefunctions
\item Many observables
\end{itemize}
  
The code should abort with a message referring to AoS vs SoA features
if any unsupported feature is invoked. In this case the AoS build
should be used by configuring with \ishell{-DENABLE_SOA=0}. In
addition, please inform the developers via GitHub or Google Groups so
that porting these features can be prioritized.

Core features are heavily tested in both SoA and AoS versions. If
using untested and noncore features in the SoA code,
please compare the AoS and SoA results carefully.

\section{Supported GPU features}

The GPU implementation supports multiple GPUs per node, with MPI tasks assigned
in a round-robin order to the GPUs. Currently, for large runs, 1 MPI task should
be used per GPU per node. For smaller calculations, use of multiple
MPI tasks per GPU might yield improved performance. Supported GPU features:

\begin{itemize}
  \item VMC, wavefunction optimization, DMC.
  \item Periodic and open boundary conditions. Mixed boundary conditions are not yet supported.
  \item Wavefunctions:
    \begin{enumerate}
        \item Single Slater determinants with 3D B-spline orbitals. Twist-averaged boundary conditions and complex wavefunctions are fully supported. Gaussian type orbitals are not yet supported.
        \item Hybrid mixed basis representation in which orbitals are represented as 1D splines times spherical harmonics in spherical regions (muffin tins) around atoms and 3D B-splines in the interstitial region.
        \item One-body and two-body Jastrows represented as 1D
          B-splines. Three-body Jastrow functions are
          not yet supported.
    \end{enumerate}
  \item Semilocal (nonlocal and local) pseudopotentials, Coulomb interaction (electron-electron, electron-ion), and model periodic Coulomb (MPC) interaction.
\end{itemize}

\section{Beta test features}

This section describes developmental features in QMCPACK that might be
ready for production but that require additional testing, features, or
documentation to be ready for general use. We describe them here
because they offer significant benefits and are well tested in
specific cases.

\subsection{Auxiliary-Field Quantum Monte Carlo}

The orbital-space Auxiliary-Field Quantum Monte Carlo (AFMQC) method is now available in QMCPACK. The main input for the code is the matrix elements of the Hamiltonian in a given single particle basis set, which must be produced from mean-field calculations such as Hartree-Fock or density functional theory. The code and many features are in development. Check the latest version of QMCPACK for an up-to-date description of available features. A partial list of the current capabilities of the code follows. For a detailed description of the available features, see chapter \ref{chap:afqmc}.
 
\begin{itemize}
    \item Phaseless AFQMC algorithm of Zhang et al. (S. Zhang and H. Krakauer. 2003. ``Quantum Monte Carlo Method using Phase-Free Random Walks with Slater Determinants." \textit{PRL} 90: 136401).
    \item ``Hybrid" and ``local energy" propagation schemes.
    \item Hamiltonian matrix elements from (1) Molpro's FCIDUMP format (which can be produced by Molpro, PySCF, and VASP) and (2) internal HDF5 format produced by PySCF (see AFQMC section below).
    \item AFQMC calculations with RHF (closed-shell doubly occupied), ROHF (open-shell doubly occupied), and UHF (spin polarized broken symmetry) symmetry. 
    \item Single and multideterminant trial wavefunctions. Multideterminant expansions with either orthogonal or nonorthogonal determinants. 
    \item Fast update scheme for orthogonal multideterminant expansions.
    \item Distributed propagation algorithms for large systems. Enables calculations where data structures do not fit on a single node.
    \item Complex implementation for PBC calculations with complex integrals.
    \item Sparse representation of large matrices for reduced memory usage.
    \item Mixed and back-propagated estimators.   
    \item Specialized implementation for solids with k-point symmetry (e.g. primitive unit cells with kpoints).
    \item Efficient GPU implementation (currently limited to solids with k-point symmetry).
\end{itemize}

\subsection{Sharing of spline data across multiple GPUs}

Sharing of GPU spline data enables distribution of the data
across multiple GPUs on a given computational node. For example, on a
two-GPU-per-node system, each GPU would have half of the
orbitals. This allows use of larger overall spline tables than would fit in
the memory of individual GPUs and potentially up to
the total GPU memory on a node. To obtain high performance, large
electron counts or a high-performing CPU-GPU interconnect is required.

To use this feature, the following needs to be done:

\begin{itemize}
    \item The CUDA Multi-Process Service (MPS) needs to be used
      (e.g., on OLCF Summit/SummitDev use ``-alloc\_flags gpumps" for
      bsub). If MPI is not detected, sharing will be disabled.
    \item CUDA\_VISIBLE\_DEVICES needs to be properly set to control each
      rank's visible CUDA devices (e.g., on OLCF Summit/SummitDev create a resource set containing all GPUs with the
      respective number of ranks with ``jsrun --task-per-rs Ngpus -g
      Ngpus").
    \item In the determinant set definition of the <wavefunction>
      section, the ``gpusharing" parameter needs to be set
      (i.e., <determinantset gpusharing=``yes">). See Section \ref{sec:spo_spline}.
\end{itemize}

\chapter{Obtaining, installing, and validating QMCPACK}
\label{chap:obtaininginstalling}

This chapter describes how to obtain, build, and validate QMCPACK. This process is designed to be as simple as
possible and should be no harder than building a modern plane-wave density
functional theory code such as Quantum ESPRESSO, QBox, or
VASP. Parallel builds enable a complete
compilation in under 2 minutes on a fast multicore system. If you
are unfamiliar with building codes we suggest working with your system
administrator to install QMCPACK.

\section{Installation steps}
To install QMCPACK, follow the steps below. Full details of
each step are given in the referenced sections.
\begin{enumerate}
\item Download the source code (Section \ref{sec:obrelease} or \ref{sec:obdevelopment}).
\item Verify that you have the required compilers, libraries, and tools
  installed (Section \ref{sec:prerequisites}).
\item Run the cmake configure step and build with make (Sections
  \ref{sec:cmake} and \ref{sec:cmakequick}). Examples for common
  systems are given in Section \ref{sec:installexamples}.
\item Run the tests to verify QMCPACK (Section \ref{sec:testing}).
\item Build the ppconvert utility in QMCPACK (Section \ref{sec:buildppconvert}).
\item Download and patch Quantum ESPRESSO. This patch adds the
  pw2qmcpack utility (Section \ref{sec:buildqe}).
\end{enumerate}

Hints for high performance are in Section \ref{sec:buildperformance}. Troubleshooting suggestions are in Section \ref{sec:troubleshoot}.

Note that there are two different QMCPACK executables that can be
produced: the general one, which is the default, and the ``complex''
version, which supports periodic calculations at arbitrary twist angles and
k-points. This second version is enabled via a cmake configuration
parameter (see Section \ref{sec:cmakeoptions}). The general version
 supports only wavefunctions that can be made real. If you run a
calculation that needs the complex version, QMCPACK will stop and inform you.

\section{Obtaining the latest release version}
\label{sec:obrelease}
Major releases of QMCPACK are distributed from
\url{http://www.qmcpack.org}. Because these versions undergo the most testing, we encourage using them for all production calculations unless there are specific reasons not to do so.

Releases are usually compressed tar files that indicate the version
number, date, and often the source code revision control number
corresponding to the release. To obtain the latest release: 

\begin{itemize}
\item Download the latest QMCPACK distribution from \url{http://www.qmcpack.org}.
\item Untar the archive (e.g., \ishell{tar xvf qmcpack_v1.3.tar.gz}).
\end{itemize}

Releases can also be obtained from the `master' branch of the QMCPACK
git repository, similar to obtaining the development version (Section \ref{sec:obdevelopment}).

\section{Obtaining the latest development version}
\label{sec:obdevelopment}
The most recent development version of QMCPACK can be obtained anonymously via
\begin{shade}
git clone https://github.com/QMCPACK/qmcpack.git
\end{shade}
Once checked out,
updates can be made via the standard \ishell{git pull}.

The `develop' branch of the git repository contains the day-to-day development source
with the latest updates, bug fixes, etc. This version might be useful
for updates to the build system to support new machines, for support
of the latest versions of Quantum ESPRESSO, or for updates to the
documentation.  Note that the development version might not be fully
consistent with the online documentation.  We attempt to keep
the development version fully working. However, please be sure to run tests and
compare with previous release versions before using for any serious
calculations. We try to keep bugs out, but occasionally they crawl
in! Reports of any breakages are appreciated.

\section{Prerequisites}
\label{sec:prerequisites}
The following items are required to build QMCPACK. For workstations, these are available via the standard
package manager. On shared supercomputers this software is usually
installed by default and is often
accessed via a modules environment---check your system
documentation.

\textbf{Use of the latest versions of all compilers and libraries is
strongly encouraged} but not absolutely essential. Generally, newer versions are faster; see
Section \ref{sec:buildperformance} for performance suggestions.

\begin{itemize}
\item C/C++ compilers such as GNU, Clang, Intel, and IBM XL. C++ compilers
  are required to support the C++ 14 standard. Use of recent (``current
  year version'') compilers is strongly encouraged.
\item An MPI library such as OpenMPI (\url{http://open-mpi.org}) or a vendor-optimized MPI.
\item BLAS/LAPACK, numerical, and linear algebra libraries. Use
  platform-optimized libraries where available, such as Intel MKL.
  ATLAS or other optimized open source libraries can also be used
  (\url{http://math-atlas.sourceforge.net}).
\item CMake, build utility (\url{http://www.cmake.org}).
\item Libxml2, XML parser (\url{http://xmlsoft.org}).
\item HDF5, portable I/O library (\url{http://www.hdfgroup.org/HDF5/}). Good performance at large scale requires parallel version $>=$ 1.10.
\item BOOST, peer-reviewed portable C++ source libraries  (\url{http://www.boost.org}).   Minimum version is 1.61.0.
\item FFTW, FFT library (\url{http://www.fftw.org/}).
\end{itemize}

To build the GPU accelerated version of QMCPACK, an installation of
NVIDIA CUDA development tools is required. Ensure that this is
compatible with the C and C++ compiler versions you plan to
use. Supported versions are included in the NVIDIA release notes.

Many of the utilities provided with QMCPACK use Python (v2). The numpy
and matplotlib libraries are required for full functionality.

Note that the standalone einspline library used by previous versions of QMCPACK
is no longer required. A more optimized version is included
inside. The standalone version should \emph{not} be on any standard
search paths because conflicts between the old and new include files
can result.

\section{C++ 14 standard library}
The C++ standard consists of language features---which are implemented in the compiler---and library features---which are implemented in the standard library.
GCC includes its own standard library and headers, but many compilers do not and instead reuse those from an existing GCC install.
Depending on setup and installation, some of these compilers might not default to using a GCC with C++ 14 headers (e.g., GCC 4.8 is common as a base system compiler, but its standard library only supports C++ 11).

The symptom of having header files that do not support the C++ 14 standard is usually
compile errors involving standard include header files.
Look for the GCC library version, which should be present in the path to the include file in the error message, and ensure that it is 5.0 or greater.
To avoid these errors occurring at compile time, QMCPACK tests for a C++ 14 standard
library during configuration and will halt with an error if one is not found.

At sites that use modules, running \ishell{module load gcc} is often sufficient to
load a newer GCC and resolve the issue.

\subsection{Intel compiler}
The Intel compiler version must be 18 or newer.
The version 17 compiler cannot compile some of the C++ 14 constructs in the code.

If a newer GCC is needed, the \ishell{-cxxlib} option can be used to point to a different GCC installation.
(Alternately, the \ishell{-gcc-name} or \ishell{-gxx-name} options can be used.)
Be sure to pass this flag to the C compiler in addition to the C++ compiler.
This is necessary because CMake extracts some library paths from the C compiler, and those paths usually also contain to the C++ library.
The symptom of this problem is C++ 14 standard library functions not found at link time.

\section{Building with CMake}
\label{sec:cmake}
The build system for QMCPACK is based on CMake.  It will autoconfigure
based on the detected compilers and libraries. The most recent
version of CMake has the best detection for the greatest variety of
systems.  The minimum required version of CMake is 3.6, which is the
oldest version to support correct application of C++ 14 flags for the Intel compiler.
Most computer installations have a sufficiently recent CMake, though it might not be
the default.

If no appropriate version CMake is available, building it from source is straightforward.
Download a version from \url{https://cmake.org/download/} and unpack the files.
Run \ishell{./boostrap} from the CMake directory, and then run \ishell{make}
when that finishes.  The resulting CMake executable will be in the \ishell{bin/} directory.
The executable can be run directly from that location.

Previously, QMCPACK made extensive use of toolchains, but the build system
has since been updated to eliminate the use of toolchain files for
most cases.  The build system is verified to work with GNU, Intel, and IBM XLC
compilers.  Specific compile options can be specified either through
specific environment or CMake variables.  When the libraries are
installed in standard locations (e.g., /usr, /usr/local), there is no
need to set environment or CMake variables for the packages.

\subsection{Quick build instructions (try first)}
\label{sec:cmakequick}

If you are feeling lucky and are on a standard UNIX-like system such
as a Linux workstation, the following might quickly give a
working QMCPACK:

The safest quick build option is to specify the C and C++ compilers
through their MPI wrappers. Here we use Intel MPI and Intel
compilers. Move to the build directory, run CMake, and make

\begin{shade}
cd build
cmake -DCMAKE_C_COMPILER=mpiicc -DCMAKE_CXX_COMPILER=mpiicpc ..
make -j 8
\end{shade}
You can increase the ``8'' to the number of cores on your system for
faster builds. Substitute mpicc and mpicxx or other wrapped compiler names to suit
  your system. For example, with OpenMPI use

\begin{shade}
cd build
cmake -DCMAKE_C_COMPILER=mpicc -DCMAKE_CXX_COMPILER=mpicxx ..
make -j 8
\end{shade}

If you are feeling particularly lucky, you can skip the compiler specification:

\begin{shade}
cd build
cmake ..
make -j 8
\end{shade}

The complexities of modern computer hardware and software systems are
such that you should check that the autoconfiguration system has made
good choices and picked optimized libraries and compiler settings
before doing significant production. That is, check the following details. We
give examples for a number of common systems in Section \ref{sec:installexamples}.

\subsection{Environment variables}
\label{sec:envvar}
A number of environment variables affect the build.  In particular
they can control the default paths for libraries, the default
compilers, etc.  The list of environment variables is given below:
%
\begin{shade}
CXX              C++ compiler
CC               C Compiler
MKL_ROOT         Path for MKL
LIBXML2_HOME     Path for libxml2
HDF5_ROOT        Path for HDF5
BOOST_ROOT       Path for Boost
FFTW_HOME        Path for FFTW
\end{shade}

\subsection{Configuration options}
\label{sec:cmakeoptions}
In addition to reading the environment variables, CMake provides a
number of optional variables that can be set to control the build and
configure steps.  When passed to CMake, these variables will take
precedent over the environment and default variables.  To set them,
add -D FLAG=VALUE to the configure line between the CMake command and
the path to the source directory.

\begin{itemize}
\item  Key QMCPACK build options
%
\begin{shade}
QMC_CUDA              Enable CUDA and GPU acceleration (1:yes, 0:no)
QMC_COMPLEX           Build the complex (general twist/k-point) version (1:yes, 0:no)
QMC_MIXED_PRECISION   Build the mixed precision (mixing double/float) version
                      (1:yes (GPU default), 0:no (CPU default)).
                      The CPU support is experimental.
                      Use float and double for base and full precision.
                      The GPU support is quite mature.
                      Use always double for host side base and full precision
                      and use float and double for CUDA base and full precision.
ENABLE_SOA            Enable data layout and algorithm optimizations using  
                      Structure-of-Array (SoA) datatypes (1:yes (default), 0:no).  
ENABLE_TIMERS         Enable fine-grained timers (1:yes, 0:no (default)).
                      Timers are off by default to avoid potential slowdown in small
                      systems. For large systems (100+ electrons) there is no risk.
\end{shade}

\item General build options
%
\begin{shade}
CMAKE_BUILD_TYPE     A variable which controls the type of build
                     (defaults to Release). Possible values are:
                     None (Do not set debug/optmize flags, use
                     CMAKE_C_FLAGS or CMAKE_CXX_FLAGS)
                     Debug (create a debug build)
                     Release (create a release/optimized build)
                     RelWithDebInfo (create a release/optimized build with debug info)
                     MinSizeRel (create an executable optimized for size)
CMAKE_C_COMPILER     Set the C compiler
CMAKE_CXX_COMPILER   Set the C++ compiler
CMAKE_C_FLAGS        Set the C flags.  Note: to prevent default
                     debug/release flags from being used, set the CMAKE_BUILD_TYPE=None
                     Also supported: CMAKE_C_FLAGS_DEBUG,
                     CMAKE_C_FLAGS_RELEASE, and CMAKE_C_FLAGS_RELWITHDEBINFO
CMAKE_CXX_FLAGS      Set the C++ flags.  Note: to prevent default
                     debug/release flags from being used, set the CMAKE_BUILD_TYPE=None
                     Also supported: CMAKE_CXX_FLAGS_DEBUG,
                     CMAKE_CXX_FLAGS_RELEASE, and CMAKE_CXX_FLAGS_RELWITHDEBINFO
CMAKE_INSTALL_PREFIX Set the install location (if using the optional install step)
INSTALL_NEXUS        Install Nexus alongside QMCPACK (if using the optional install step)
\end{shade}

\item Additional QMCPACK build options

\begin{shade}
QMC_INCLUDE            Add extra include paths
QMC_EXTRA_LIBS         Add extra link libraries
QMC_BUILD_STATIC       Add -static flags to build
QMC_DATA               Specify data directory for QMCPACK performance and integration tests
QE_BIN                 Location of Quantum Espresso binaries including pw2qmcpack.x
QMC_SYMLINK_TEST_FILES Set to zero to require test files to be copied. Avoids space
                       saving default use of symbolic links for test files. Useful
                       if the build is on a separate filesystem from the source, as
                       required on some HPC systems.
\end{shade}

\item Intel MKL related
%
\begin{shade}
ENABLE_MKL          Enable Intel MKL libraries (1:yes (default for intel compiler),
                                                0:no (default otherwise)).
MKL_ROOT            Path to MKL libraries (only necessary for non intel compilers
                    or intel without standard environment variables.)
                    One of the above environment variables can be used.
\end{shade}

\item libxml2 related
%
\begin{shade}
Libxml2_INCLUDE_DIRS   Specify include directories for libxml2
Libxml2_LIBRARY_DIRS   Specify library directories for libxml2
\end{shade}

\item HDF5 related
%
\begin{shade}
ENABLE_PHDF5    1(default)/0, enables/disable parallel collective IO.
\end{shade}

\item FFTW related
%
\begin{shade}
FFTW_INCLUDE_DIRS   Specify include directories for FFTW
FFTW_LIBRARY_DIRS   Specify library directories for FFTW
\end{shade}

\item CTest related
%
\begin{shade}
MPIEXEC                Specify the mpi wrapper, e.g. srun, aprun, mpirun, etc.
MPIEXEC_NUMPROC_FLAG   Specify the number of mpi processes flag,
                       e.g. "-n", "-np", etc.
\end{shade}

\item LLVM/Clang Developer Options\\
\begin{shade}
LLVM_SANITIZE_ADDRESS     link with the %*\href{https://clang.llvm.org/docs/AddressSanitizer.html}{Clang address sanitizer library}*
LLVM_SANITIZE_MEMORY      link with the %*\href{https://clang.llvm.org/docs/MemorySanitizer.html}{Clang memory sanitizer library}*
\end{shade}

See Section \ref{tool:LLVM-Sanitizer-Libraries} for more information.
\end{itemize}

\subsection{Installation from CMake}
Installation is optional. The QMCPACK executable can be run from the \ishell{bin} directory in the build location.
If the install step is desired, run the \ishell{make install} command to install the QMCPACK executable, the converter,
and some additional executables.
Also installed is the \ishell{qmcpack.settings} file that records options used to compile QMCPACK.
Specify the \ishell{CMAKE_INSTALL_PREFIX} CMake variable during configuration to set the install location.


\subsection{Role of QMC\_DATA}
QMCPACK includes a variety of optional performance and integration
tests that use research quality wavefunctions to obtain meaningful
performance and to more thoroughly test the code. The necessarily
large input files are stored in the location pointed to by QMC\_DATA (e.g., scratch or long-lived project space on a supercomputer). These
files are not included in the source code distribution to minimize
size. The tests are activated if CMake detects the files when
configured. See tests/performance/NiO/README,
tests/solids/NiO\_afqmc/README, and tests/performance/C-graphite/README
for details of the current tests and input files and to download them.

Currently the files must be downloaded via\\
\url{https://anl.box.com/s/pveyyzrc2wuvg5tmxjzzwxeo561vh3r0} and \\
\url{https://anl.box.com/s/j5d6jeazpgx5441s04ajtv77iogoyjsh}.

Current complete set of files:
%
\begin{shade}
QMC_DATA/C-graphite/lda.pwscf.h5
QMC_DATA/NiO/NiO-fcc-supertwist111-supershift000-S1.h5
QMC_DATA/NiO/NiO-fcc-supertwist111-supershift000-S2.h5
QMC_DATA/NiO/NiO-fcc-supertwist111-supershift000-S4.h5
QMC_DATA/NiO/NiO-fcc-supertwist111-supershift000-S8.h5
QMC_DATA/NiO/NiO-fcc-supertwist111-supershift000-S16.h5
QMC_DATA/NiO/NiO-fcc-supertwist111-supershift000-S32.h5
QMC_DATA/NiO/NiO-fcc-supertwist111-supershift000-S64.h5
QMC_DATA/NiO/NiO-fcc-supertwist111-supershift000-S128.h5
QMC_DATA/NiO/NiO-fcc-supertwist111-supershift000-S256.h5
QMC_DATA/NiO/NiO_afm_fcidump.h5
QMC_DATA/NiO/NiO_afm_wfn.dat
QMC_DATA/NiO/NiO_nm_choldump.h5
\end{shade}


\subsection{Configure and build using CMake and make}
To configure and build QMCPACK, move to build directory, run CMake, and make

\begin{shade}
cd build
cmake ..
make -j 8
\end{shade}

As you will have gathered, CMake encourages ``out of source'' builds,
where all the files for a specific build configuration reside in their
own directory separate from the source files. This allows multiple
builds to be created from the same source files, which is very useful
when the file system is shared between different systems. You can also
build versions with different settings (e.g., QMC\_COMPLEX) and
different compiler settings. The build directory does not have to be
called build---use something descriptive such as build\_machinename or
build\_complex. The ``..'' in the CMake line refers to the directory
containing CMakeLists.txt. Update the ``..'' for other build
directory locations.

\subsection{Example configure and build}
\begin{itemize}
\item Set the environments (the examples below assume bash, Intel compilers, and MKL library)

\begin{shade}
export CXX=icpc
export CC=icc
export LIBXML2_HOME=/usr/local
export MKL_ROOT=/usr/local/intel/mkl/10.0.3.020
export HDF5_ROOT=/usr/local
export BOOST_ROOT=/usr/local/boost
export FFTW_HOME=/usr/local/fftw
\end{shade}

\item Move to build directory, run CMake, and make

\begin{shade}
cd build
cmake -D CMAKE_BUILD_TYPE=Release ..
make -j 8
\end{shade}
\end{itemize}

\subsection{Build scripts}
We recommended creating a helper script that contains the
configure line for CMake.  This is particularly useful when avoiding
environment variables, packages are installed in custom locations,
or the configure line is long or complex.  In this case it is also
recommended to add ``rm -rf CMake*" before the configure line to remove
existing CMake configure files to ensure a fresh configure each time
the script is called. Deleting all the files in the build
directory is also acceptable. If you do so we recommend adding some sanity
checks in case the script is run from the wrong directory (e.g.,
checking for the existence of some QMCPACK files).

Some build script examples for different systems are given in the
config directory. For example, on Cray systems these scripts might
load the appropriate modules to set the appropriate programming
environment, specific library versions, etc.

An example script build.sh is given below. It is much more complex
than usually needed for comprehensiveness:


\begin{shade}
export CXX=mpic++
export CC=mpicc
export ACML_HOME=/opt/acml-5.3.1/gfortran64
export HDF5_ROOT=/opt/hdf5
export BOOST_ROOT=/opt/boost

rm -rf CMake*

cmake                                                \
  -D CMAKE_BUILD_TYPE=Debug                         \
  -D Libxml2_INCLUDE_DIRS=/usr/include/libxml2      \
  -D Libxml2_LIBRARY_DIRS=/usr/lib/x86_64-linux-gnu \
  -D FFTW_INCLUDE_DIRS=/usr/include                 \
  -D FFTW_LIBRARY_DIRS=/usr/lib/x86_64-linux-gnu    \
  -D QMC_EXTRA_LIBS="-ldl ${ACML_HOME}/lib/libacml.a -lgfortran" \
  -D QMC_DATA=/projects/QMCPACK/qmc-data            \
  ..
\end{shade}

\subsection{Using vendor-optimized numerical libraries (e.g., Intel MKL})

Although QMC does not make extensive use of linear algebra, use of
vendor-optimized libraries is strongly recommended for highest
performance. BLAS routines are used in the Slater determinant update, the VMC wavefunction optimizer,
and to apply orbital coefficients in local basis calculations. Vectorized
math functions are also beneficial (e.g., for the phase factor
computation in solid-state calculations). CMake is generally successful
in finding these libraries, but specific combinations can require
additional hints, as described in the following:

\subsubsection{Using Intel MKL with non-Intel compilers}

To use Intel MKL with, e.g. an MPICH wrapped gcc:

\begin{shade}
cmake \
 -DCMAKE_C_COMPILER=mpicc -DCMAKE_CXX_COMPILER=mpicxx \
 -DENABLE_MKL=1 -DMKL_ROOT=$MKLROOT/lib \
 ..
\end{shade}

MKLROOT is the directory containing the MKL binary, examples, and lib
directories (etc.) and is often /opt/intel/mkl.

\subsubsection{Serial or multithreaded library}
\label{sec:threadedlibrary}
Vendors might provide both serial and multithreaded versions of their libraries.
Using the right version is critical to QMCPACK performance.
QMCPACK makes calls from both inside and outside threaded regions.
When being called from outside an OpenMP parallel region, the multithreaded version is preferred for the possibility of using all the available cores.
When being called from every thread inside an OpenMP parallel region, the serial version is preferred for not oversubscribing the cores.
Fortunately, nowadays the multithreaded versions of many vendor libraries (MKL, ESSL) are OpenMP aware.
They use only one thread when being called inside an OpenMP parallel region.
This behavior meets exactly both QMCPACK needs and thus is preferred.
If the multithreaded version does not provide this feature of dynamically adjusting the number of threads,
the serial version is preferred. In addition, thread safety is required no matter which version is used.

\subsection{Cross compiling}
Cross compiling is often difficult but is required on supercomputers
with distinct host and compute processor generations or architectures.
QMCPACK tried to do its best with CMake to facilitate cross compiling.

\begin{itemize}
  \item On a machine using a Cray programming environment, we rely on
      compiler wrappers provided by Cray to correctly set architecture-specific
      flags. The CMake configure log should indicate that a
      Cray machine was detected.
  \item If not on a Cray machine, by default we assume building for
    the host architecture (e.g., -xHost is added for the Intel compiler
    and -march=native is added for GNU/Clang compilers).
  \item If -x/-ax or -march is specified by the user in CMAKE\_C\_FLAGS and CMAKE\_CXX\_FLAGS,
    we respect the user's intention and do not add any architecture-specific flags.
\end{itemize}

The general strategy for cross compiling should therefore be to
manually set CMAKE\_C\_FLAGS and CMAKE\_CXX\_FLAGS for the target
architecture. Using \ishell{make VERBOSE=1} is a useful way to check the
final compilation options.  If on a Cray machine, selection of the
appropriate programming environment should be sufficient.

\section{Installation instructions for common workstations and
  supercomputers}
\label{sec:installexamples}

This section describes how to build QMCPACK on various common systems
including multiple Linux distributions, Apple OS X, and various
supercomputers. The examples should serve as good starting points for
building QMCPACK on similar machines. For example, the software
environment on modern Crays is very consistent. Note that updates to
operating systems and system software might require small modifications
to these recipes. See Section \ref{sec:buildperformance} for key
points to check to obtain highest performance and
Section \ref{sec:troubleshoot} for troubleshooting hints.

\subsection{Installing on Ubuntu Linux or other apt-get--based distributions}
\label{sec:buildubuntu}

The following is designed to obtain a working QMCPACK build on, for example, a
student laptop, starting from a basic Linux installation with none of
the developer tools installed. Fortunately, all the required packages
are available in the default repositories making for a quick
installation. Note that for convenience we use a generic BLAS. For
production, a platform-optimized BLAS should be used.


\begin{shade}
apt-get cmake g++ openmpi-bin libopenmpi-dev libboost-dev
apt-get libatlas-base-dev liblapack-dev libhdf5-dev libxml2-dev fftw3-dev
export CXX=mpiCC
cd build
cmake ..
make -j 8
ls -l bin/qmcpack
\end{shade}

For qmca and other tools to function, we install some Python libraries:

\begin{shade}
sudo apt-get install python-numpy python-matplotlib
\end{shade}

\subsection{Installing on CentOS Linux or other yum-based distributions}

The following is designed to obtain a working QMCPACK build on, for example, a
student laptop, starting from a basic Linux installation with none of
the developer tools installed. CentOS 7 (Red Hat compatible) is using
gcc 4.8.2. The installation is complicated only by the need to install
another repository to obtain HDF5 packages that are not available by
default. Note that for convenience we use a generic BLAS. For
production, a platform-optimized BLAS should be used.


\begin{shade}
sudo yum install make cmake gcc gcc-c++ openmpi openmpi-devel fftw fftw-devel \
                  boost boost-devel libxml2 libxml2-devel
sudo yum install blas-devel lapack-devel atlas-devel
module load mpi
\end{shade}

To setup repoforge as a source for the HDF5 package, go to
\url{http://repoforge.org/use}. Install the appropriate up-to-date
release package for your operating system. By default, CentOS Firefox will offer
to run the installer. The CentOS 6.5 settings were still usable for HDF5 on
CentOS 7 in 2016, but use CentOS 7 versions when they become
available.


\begin{shade}
sudo yum install hdf5 hdf5-devel
\end{shade}

To build QMCPACK:

\begin{shade}
module load mpi/openmpi-x86_64
which mpirun
# Sanity check; should print something like   /usr/lib64/openmpi/bin/mpirun
export CXX=mpiCC
cd build
cmake ..
make -j 8
ls -l bin/qmcpack
\end{shade}

\subsection{Installing on Mac OS X using Macports}
These instructions assume a fresh installation of macports
and use the gcc 6.1 compiler. Older versions are fine, but it is vital to ensure that 
matching compilers and libraries are used for all
packages and to force use of what is installed in /opt/local.  Performance should be very reasonable.
Note that we use the Apple-provided Accelerate framework for
optimized BLAS.

Follow the Macports install instructions at \url{https://www.macports.org/}.

\begin{itemize}
\item Install Xcode and the Xcode Command Line Tools.
\item Agree to Xcode license in Terminal: sudo xcodebuild -license.
\item Install MacPorts for your version of OS X.
\end{itemize}


Install the required tools:


\begin{shade}
sudo port install gcc6
sudo port select gcc mp-gcc6
sudo port install openmpi-devel-gcc6
sudo port select --set mpi openmpi-devel-gcc61-fortran

sudo port install fftw-3 +gcc6
sudo port install libxml2
sudo port install cmake
sudo post install boost +gcc6
sudo port install hdf5 +gcc6

sudo port select --set python python27
sudo port install py27-numpy +gcc6
sudo port install py27-matplotlib  #For graphical plots with qmca
\end{shade}

QMCPACK build:

\begin{shade}
cd build
cmake -DCMAKE_C_COMPILER=mpicc -DCMAKE_CXX_COMPILER=mpiCXX ..
make -j 6 # Adjust for available core count
ls -l bin/qmcpack
\end{shade}

Cmake should pickup the versions of HDF5 and libxml (etc.) installed in
/opt/local by macports. If you have other copies of these libraries
installed and wish to force use of a specific version, use the
environment variables detailed in Section \ref{sec:envvar}.

This recipe was verified on July 1, 2016, on a Mac running OS X 10.11.5
``El Capitain.''

\subsection{Installing on Mac OS X using Homebrew (brew)}
Homebrew is a package manager for OS X that provides a convenient
route to install all the QMCPACK dependencies. The
following recipe will install the latest available versions of each
package. This was successfully tested under OS X 10.12 ``Sierra'' in December 2017. Note that it is necessary to build the MPI software from
source to use the brew-provided gcc instead of Apple CLANG.

\begin{enumerate}
\item Install Homebrew from \url{http://brew.sh/}:
%
\begin{shade}
/usr/bin/ruby -e "$(curl -fsSL
    https://raw.githubusercontent.com/Homebrew/install/master/install)"
\end{shade}
 
\item Install the prerequisites:
%
\begin{shade}
brew install gcc # installs gcc 7.2.0 on 2017-12-19
export HOMEBREW_CXX=g++-7
export HOMEBREW_CC=gcc-7
brew install mpich2 --build-from-source
# Build from source required to use homebrew compiled compilers as
# opposed to Apple CLANG. Check "mpicc -v" indicates Homebrew gcc
brew install cmake
brew install fftw
brew install boost
brew install homebrew/science/hdf5
#Note: Libxml2 is not required via brew since OS X already includes it.
\end{shade}
\item Configure and build QMCPACK:
%
\begin{shade}
cmake -DCMAKE_C_COMPILER=/usr/local/bin/mpicc \
      -DCMAKE_CXX_COMPILER=/usr/local/bin/mpicxx ..
make -j 12
\end{shade}
\item Run the short tests. When MPICH is used for the first time, OS
  X will request approval of the network connection for each executable.
%
\begin{shade}
ctest -R short -LE unstable
\end{shade}
\end{enumerate}

\subsection{Installing on ANL ALCF Mira/Cetus IBM Blue Gene/Q}
\label{sec:buildbgq}
Mira/Cetus is a Blue Gene/Q supercomputer at Argonne National Laboratory's Argonne Leadership Computing Facility (ANL ALCF).
Mira has 49,152 compute nodes, and each node has a 16-core PowerPC A2 processor with 16 GB DDR3 memory.
Because the login nodes and the compute nodes have different processors with distinct instruction sets,
cross compiling is required on this platform. See details about using Blue Gene/Q at \url{http://www.alcf.anl.gov/user-guides/compiling-linking}.
On Mira, compilers are loaded via softenv, and users need to add +mpiwrapper-bgclang-mpi3 and +cmake-3.8.1 in \$HOME/.soft.
To build QMCPACK, a toolchain file is provided for setting up CMake.
\textbf{BGClang is required for C++ 14 support. IBM XL C/C++ compiler should not be used.}

%
\begin{shade}
cd build
cmake -DCMAKE_TOOLCHAIN_FILE=../config/BGQ_Clang_ToolChain.cmake ..
make -j 16
ls -l bin/qmcpack
\end{shade}

\subsection{Installing on ALCF Theta, Cray XC40}
Theta is a 9.65 petaflops system manufactured by Cray with 3,624 compute nodes.
Each node features a second-generation Intel Xeon Phi 7230 processor and 192 GB DDR4 RAM.

%
\begin{shade}
export CRAYPE_LINK_TYPE=dynamic
# Do not use cmake 3.9.1, it causes trouble with parallel HDF5.
module load cmake/3.11.4
module unload cray-libsci
module load cray-hdf5-parallel
module load gcc   # Make C++ 14 standard library available to the Intel compiler
export BOOST_ROOT=/soft/libraries/boost/1.64.0/intel
cmake ..
make -j 24
ls -l bin/qmcpack
\end{shade}

\subsection{Installing on ORNL OLCF Titan Cray XK7 (NVIDIA GPU
  accelerated)}
\label{sec:titanbuildgpu}
Titan is a GPU-accelerated supercomputer at Oak Ridge National
Laboratory's  Oak Ridge Leadership Computing Facility  (ORNL OLCF). Each
compute node has a 16 core AMD 2.2GHz Opteron 6274 (Interlagos) and an
NVIDIA Kepler accelerator. The standard Cray software environment is
available, with libraries accessed via modules. The only extra
settings required to build the GPU version are the cudatoolkit module
and specifying -DQMC\_CUDA=1 on the CMake configure line.

Note that on Crays, the compiler wrappers ``CC'' and ``cc'' are
used. The build system checks for these and does not (should not) use
the compilers directly.

%
\begin{shade}
export CRAYPE_LINK_TYPE=dynamic
module swap PrgEnv-pgi PrgEnv-gnu # Use gnu compilers
module load cudatoolkit           # CUDA for GPU build
module load cray-hdf5-parallel
module load cmake3
module load fftw
export FFTW_HOME=$FFTW_DIR/..
module load boost
mkdir build_titan_gpu
cd build_titan_gpu
export CC=cc
export CXX=CC
cmake -DQMC_CUDA=1 ..             # Must enable CUDA capabilities
make -j 8
ls -l bin/qmcpack
\end{shade}

\subsection{Installing on ORNL OLCF Titan Cray XK7 (CPU version)}
As noted in Section \ref{sec:titanbuildgpu} for the GPU, building on
Crays requires only loading the appropriate library modules.

%
\begin{shade}
export CRAYPE_LINK_TYPE=dynamic
module swap PrgEnv-pgi PrgEnv-gnu # Use gnu compilers
module unload cudatoolkit         # No CUDA for CPU build
module load cray-hdf5-parallel
module load cmake3
module load fftw
export FFTW_HOME=$FFTW_DIR/..
module load boost
mkdir build_titan_cpu
cd build_titan_cpu
export CC=cc
export CXX=CC
cmake ..
make -j 8
ls -l bin/qmcpack
\end{shade}

\subsection{Installing on ORNL OLCF Eos Cray XC30}
Eos is a Cray XC30 with 16 core Intel Xeon E5-2670 processors connected
by the Aries interconnect. The build process is identical to Titan
except that we use the default Intel programming environment. This is
usually preferred to GNU.

\begin{shade}
export CRAYPE_LINK_TYPE=dynamic
module unload cray-libsci
module load cray-hdf5
module load cmake3/3.6.1   # Or newer
module load fftw
export FFTW_HOME=$FFTW_DIR/..
module load boost
module swap gcc gcc/6.3.0  # Make C++ 14 standard library available to the Intel compiler
mkdir build_eos
cd build_eos
cmake ..
make -j 8
ls -l bin/qmcpack
\end{shade}

\subsection{Installing on ORNL OLCF Summit}
Summit is an IBM system at the ORNL OLCF built with IBM Power System AC922
nodes. They have two IBM Power 9 processors and six NVIDIA Volta V100
accelerators.

\subsubsection{Building QMCPACK}
Note that these build instructions are preliminary as the
software environment is subject to change. As of December 2018, the
IBM XL compiler does not support C++14, so we currently use the
gnu compiler. 

\begin{shade}
module load gcc
module load essl
module load netlib-lapack #because ESSL does not provided needed LAPACK functionality
module load hdf5
module load fftw
export FFTW_HOME=$OLCF_FFTW_ROOT
module load cmake
module load boost
module load cuda
module load python/2.7.15-anaconda2-5.3.0
mkdir build_summit
cd build_summit
cmake -DCMAKE_C_COMPILER="mpicc" \
      -DCMAKE_CXX_COMPILER="mpicxx" \
      -DBUILD_LMYENGINE_INTERFACE=0 \
      -DQMC_CUDA=1 \
      -DCUDA_ARCH="sm_70" \
      ..
make -j 8
ls -l bin/qmcpack
\end{shade}

\subsubsection{Building Quantum Espresso}
The v6.3 release of Quantum Espresso (QE) does not officially support the
Power 9 architecture, and several steps of the configuration process
require updating to build successfully. v6.4 is expected to officially support
the new architecture. The following can be used to build a
CPU version of QE on Summit, placing the script in the
external\_codes/quantum\_espresso directory. Note that performance is
not yet optimized although vendor libraries are
used. Alternatively, the wavefunction files can be generated on
another system and the converted HDF5 files copied over.


\begin{shade}
#!/usr/bin/bash
module load gcc/6.4.0 #6.4.0 was default on 2019-01-03
module load essl
module load netlib-lapack #because ESSL does not provided needed LAPACK functionality
module load hdf5
module load fftw
export FFTW_HOME=$OLCF_FFTW_ROOT
module -t list
./download_and_patch_qe6.3.sh
cd qe-6.3
cd archive
mv fox.tgz fox.tgz_orig
wget https://gitlab.com/QEF/q-e/raw/develop/archive/fox.tgz # Latest FoX on develop is patched for PPC64 architectures
cd ..
./configure --with-hdf5=$OLCF_HDF5_ROOT
cp -p make.inc make.inc_orig
sed -e 's/libhdf5.a /libhdf5.a -L/g' make.inc_orig >make.inc # Incorrect awk in install/configure drops library dir for zlib
sed -i 's/-D__FFTW3/-D__LINUX_ESSL/g' make.inc
sed -i "s|-lblas|-L${OLCF_ESSL_ROOT}/lib64 -lessl|g" make.inc
sed -i 's/-lfftw3/ /g' make.inc # Use the ESSL FFTW
sed -i 's/^LAPACK_LIBS .*=.*/LAPACK_LIBS = -lessl -llapack -lessl/g' make.inc
echo --- Starting build `date`
make all
echo --- Finished build `date`
\end{shade}

\subsection{Installing on NERSC Edison Cray XC30}

Edison is a Cray XC30 with dual 12-core Intel ``Ivy Bridge" nodes
installed at the National Energy Research Scientific Computing Center (NERSC). The build settings are identical to Eos.


\begin{shade}
export CRAYPE_LINK_TYPE=dynamic
module unload cray-libsci
module load boost
module load cmake/3.11.4
module load libxml2
module load cray-hdf5-parallel
module load gcc   # Make C++ 14 standard library available to the Intel compiler
cmake ..
make -j 8
ls -l bin/qmcpack
\end{shade}
When the preceding was tested on October 30, 2018, the following module and
software versions were present:

\begin{shade}
build> module list
Currently Loaded Modulefiles:
  1) modules/3.2.10.6                                 14) alps/6.6.43-6.0.7.0_26.4__ga796da3.ari
  2) nsg/1.2.0                                        15) rca/2.2.18-6.0.7.0_33.3__g2aa4f39.ari
  3) intel/18.0.1.163                                 16) atp/2.1.1
  4) craype-network-aries                             17) PrgEnv-intel/6.0.4
  5) craype/2.5.14                                    18) craype-ivybridge
  6) udreg/2.3.2-6.0.7.0_33.18__g5196236.ari          19) cray-mpich/7.7.0
  7) ugni/6.0.14.0-6.0.7.0_23.1__gea11d3d.ari         20) altd/2.0
  8) pmi/5.0.13                                       21) darshan/3.1.4
  9) dmapp/7.1.1-6.0.7.0_34.3__g5a674e0.ari           22) boost/1.63
 10) gni-headers/5.0.12.0-6.0.7.0_24.1__g3b1768f.ari  23) cmake/3.11.4
 11) xpmem/2.2.15-6.0.7.1_5.8__g7549d06.ari           24) libxml2/2.9.4
 12) job/2.2.3-6.0.7.0_44.1__g6c4e934.ari             25) cray-hdf5-parallel/1.10.1.1
 13) dvs/2.7_2.2.112-6.0.7.1_6.4__ge96a422            26) gcc/7.3.0

\end{shade}

\subsection{Installing on NERSC Cori, Haswell Partition, Cray XC40}
Cori is a Cray XC40 with 16-core Intel "Haswell" nodes
installed at NERSC.


\begin{shade}
export CRAYPE_LINK_TYPE=dynamic
module unload cray-libsci
module load boost
module load cray-hdf5-parallel
module load cmake/3.11.4
module load gcc   # Make C++ 14 standard library available to the Intel compiler
mkdir build_cori_hsw
cd build_cori_hsw
cmake ..
make -j 16
ls -l bin/qmcpack
\end{shade}

When the preceding was tested on October 30, 2018, the following module and
software versions were present:


\begin{shade}
build_cori_hsw> module list
Currently Loaded Modulefiles:
  1) modules/3.2.10.6                                 14) alps/6.6.43-6.0.7.0_26.4__ga796da3.ari
  2) nsg/1.2.0                                        15) rca/2.2.18-6.0.7.0_33.3__g2aa4f39.ari
  3) intel/18.0.1.163                                 16) atp/2.1.1
  4) craype-network-aries                             17) PrgEnv-intel/6.0.4
  5) craype/2.5.14                                    18) craype-haswell
  6) udreg/2.3.2-6.0.7.0_33.18__g5196236.ari          19) cray-mpich/7.7.0
  7) ugni/6.0.14.0-6.0.7.0_23.1__gea11d3d.ari         20) altd/2.0
  8) pmi/5.0.13                                       21) darshan/3.1.4
  9) dmapp/7.1.1-6.0.7.0_34.3__g5a674e0.ari           22) boost/1.63
 10) gni-headers/5.0.12.0-6.0.7.0_24.1__g3b1768f.ari  23) cray-hdf5-parallel/1.10.1.1
 11) xpmem/2.2.15-6.0.7.1_5.8__g7549d06.ari           24) cmake/3.11.4
 12) job/2.2.3-6.0.7.0_44.1__g6c4e934.ari             25) gcc/7.3.0
 13) dvs/2.7_2.2.113-6.0.7.1_7.1__g1bbc03e

\end{shade}

\subsection{Installing on NERSC Cori, Xeon Phi KNL partition, Cray XC40}
The second phase of NERSC's Cori uses Intel
Xeon Phi Knight's Landing (KNL) nodes. The following build recipe ensures that the code
generation is appropriate for the KNL nodes:


\begin{shade}
export CRAYPE_LINK_TYPE=dynamic
module swap craype-haswell craype-mic-knl
module unload cray-libsci
module load boost
module load cray-hdf5-parallel
module load cmake/3.11.4
module load gcc   # Make C++ 14 standard library available to the Intel compiler
mkdir build_cori_knl
cd build_cori_knl
cmake ..
make -j 16
ls -l bin/qmcpack
\end{shade}

When the preceding was tested on October 30, 2018, the following module and
software versions were present:


\begin{shade}
build_cori_knl> module list
Currently Loaded Modulefiles:
  1) modules/3.2.10.6                                 14) alps/6.6.43-6.0.7.0_26.4__ga796da3.ari
  2) nsg/1.2.0                                        15) rca/2.2.18-6.0.7.0_33.3__g2aa4f39.ari
  3) intel/18.0.1.163                                 16) atp/2.1.1
  4) craype-network-aries                             17) PrgEnv-intel/6.0.4
  5) craype/2.5.14                                    18) craype-mic-knl
  6) udreg/2.3.2-6.0.7.0_33.18__g5196236.ari          19) cray-mpich/7.7.0
  7) ugni/6.0.14.0-6.0.7.0_23.1__gea11d3d.ari         20) altd/2.0
  8) pmi/5.0.13                                       21) darshan/3.1.4
  9) dmapp/7.1.1-6.0.7.0_34.3__g5a674e0.ari           22) boost/1.63
 10) gni-headers/5.0.12.0-6.0.7.0_24.1__g3b1768f.ari  23) cray-hdf5-parallel/1.10.1.1
 11) xpmem/2.2.15-6.0.7.1_5.8__g7549d06.ari           24) cmake/3.11.4
 12) job/2.2.3-6.0.7.0_44.1__g6c4e934.ari             25) gcc/7.3.0
 13) dvs/2.7_2.2.113-6.0.7.1_7.1__g1bbc03e

\end{shade}

\subsection{Installing on systems with ARMv8-based processors}
The following build recipe was verified using the `Arm Compiler for HPC' on the ANL JLSE Comanche system with Cavium ThunderX2 processors on November 6, 2018.
\begin{shade}
# load armclang compiler
module load Generic-AArch64/RHEL/7/arm-hpc-compiler/18.4
# load Arm performance libraries
module load ThunderX2CN99/RHEL/7/arm-hpc-compiler-18.4/armpl/18.4.0
# define path to pre-installed packages
export HDF5_ROOT=</path/to/hdf5/install/>
export BOOST_ROOT=</path/to/boost/install> # header-only, no need to build
\end{shade}
Then using the following command:
\begin{shade}
mkdir build_armclang
cd build_armclang
cmake -DCMAKE_C_COMPILER=armclang -DCMAKE_CXX_COMPILER=armclang++ -DQMC_MPI=0 \
      -DLAPACK_LIBRARIES="-L$ARMPL_DIR/lib -larmpl_mp" \
      -DFFTW_INCLUDE_DIR="$ARMPL_DIR/include" \
      -DFFTW_LIBRARIES="$ARMPL_DIR/lib/libarmpl_mp.a" \
      ..
make -j 56
\end{shade}
Note that armclang is recognized as an `unknown' compiler by CMake v3.13* and below. In this case, we need to force it as clang to apply necessary flags. To do so, pass the following additionals option to CMake:
\begin{shade}
      -DCMAKE_C_COMPILER_ID=Clang -DCMAKE_CXX_COMPILER_ID=Clang \
      -DCMAKE_CXX_COMPILER_VERSION=5.0 -DCMAKE_CXX_STANDARD_COMPUTED_DEFAULT=98 \
\end{shade}      

\subsection{Installing on Windows}
Install the Windows Subsystem for Linux and Bash on Windows.
Open a bash shell and follow the install directions for Ubuntu in Section \ref{sec:buildubuntu}.

\section{Installing via Spack}
Spack is a package manager for scientific software.
One of the primary goals of Spack is to reduce the barrier for the users to install scientific
software. Spack is intended to work on everything from laptop
computers to high-end supercomputers. More information about Spack can
be found at \url{https://spack.readthedocs.io/en/latest}. The major
advantage of installation with Spack is that all dependencies are
automatically built, potentially including all the compilers and libraries, and
different versions of QMCPACK can easily coexist with each other.
The QMCPACK Spack package also knows how to automatically build
and patch QE. In principle, QMCPACK can be installed with
a single Spack command.

\subsection{Known Limitations}
The QMCPACK Spack package inherits the limitations of the underlying
Spack infrastructure and its dependencies. The main limitation is that installation typically fails when building a
dependency such as HDF5, MPICH, etc. For
\ishell{spack install qmcpack} to succeed, it is very important to
leverge preinstalled packages on your computer or supercomputer. The
other frequently encountered challenge is that the compiler configuration
is nonintuitive.  This is especially the case with the Intel
compiler.

Here are some additional limitations of the QMCPACK Spack package that
will be resolved in future releases:
\begin{itemize}
\item Because of a conflict in the QE 6.3 package, the QE variant cannot build
  in serial. In other words, \verb|qmcpack install qmcpack~mpi| will issue
  a conflict.  This will not be a problem with the next release of QE.

\item Robust CUDA support in Spack is a work-in-progress.  It will
  catch only some compiler-CUDA conflicts.
\item CUDA support is not leveraging the Spack CUDA infrastructure;
  thus, compiler conflicts will not be caught until a build is attempted.
\item AFQMC is not part of the Spack package at this time.
\item The Spack package will install Nexus as part of the
  installation, but actual use of Nexus from within the Spack
  environment is untested.
\end{itemize}

\subsection{Setting up the Spack Environment}
Begin by cloning Spack from GitHub and configuring your shell as described at 
\url{https://spack.readthedocs.io/en/latest/getting_started.html}.

The goal of the next several steps is to set up the Spack environment
for building. First, we highly recommend limiting the number of build jobs to
a reasonable value for your machine. This can be
accomplished by modifying your \ishell{\~/.spack/config.yaml} file as follows:

\begin{lstlisting}[style=SHELL]
config:
  build_jobs: 16
\end{lstlisting}

Make sure any existing compilers are properly detected. For many
architectures, compilers are properly detected with no additional
effort.

\begin{shade}
your-laptop> spack compilers
==> Available compilers
-- gcc sierra-x86_64 --------------------------------------------
gcc@7.2.0  gcc@6.4.0  gcc@5.5.0  gcc@4.9.4  gcc@4.8.5  gcc@4.7.4  gcc@4.6.4
\end{shade}

However, if your compiler is not automatically detected, it is straightforward
to add one:

\begin{shade}
your-laptop> spack compiler add <path-to-compiler>
\end{shade}

The Intel compiler that requires checkout a license is particular
tricky. First go ahead and add the compiler by modifying
\ishell{\~/.spack/linux/compilers.yaml}. Here is an example of a typical configuration:
\begin{shade}
- compiler:
    environment:
      set:
        INTEL_LICENSE_FILE: server@national-lab.doe.gov
    extra_rpaths:  ['/soft/com/packages/intel/18/u3/compilers_and_libraries_2018.3.222/linux/compiler/lib/intel64',
    '/soft/apps/packages/gcc/gcc-6.2.0/lib64']
    flags: 
      cflags: -gcc-name=/soft/apps/packages/gcc/gcc-6.2.0/bin/gcc
      fflags: -gcc-name=/soft/apps/packages/gcc/gcc-6.2.0/bin/gcc
      cxxflags: -gxx-name=/soft/apps/packages/gcc/gcc-6.2.0/bin/g++
    modules: []
    operating_system: ubuntu14.04
    paths:
      cc: /soft/com/packages/intel/18/u3/compilers_and_libraries_2018.3.222/linux/bin/intel64/icc
      cxx: /soft/com/packages/intel/18/u3/compilers_and_libraries_2018.3.222/linux/bin/intel64/icpc
      f77: /soft/com/packages/intel/18/u3/compilers_and_libraries_2018.3.222/linux/bin/intel64/ifort
      fc: /soft/com/packages/intel/18/u3/compilers_and_libraries_2018.3.222/linux/bin/intel64/ifort
    spec: intel@18.0.3
    target: x86_64
\end{shade}

This last step is the most troublesome. Pre-installed packages are not
automatically detected. If vendor optimized libraries are already
installed, you will need to manually add them to your
\ishell{\~/.spack/packages.yaml}. For example, this works on Mac OS X
for the Intel MKL package.

\begin{shade}
your-laptop> cat \~/.spack/packages.yaml
packages:
    intel-mkl:
        paths:
            intel-mkl@2018.0.128: /opt/intel/compilers_and_libraries_2018.0.104/mac/mkl
        buildable: False
\end{shade}

Some trial-and-error might be involved to get the directory correct. If
you do not include enough of the tree path, Spack will not be able to
register the package in its database. More information about system
packages can be found at

\url{http://spack.readthedocs.io/en/latest/getting_started.html#system-packages}

\subsection{Building QMCPACK}
The QMCPACK Spack package has a number of variants to support different compile time
options and different versions of the application. A full list can be displayed by typing:

\begin{lstlisting}[style=SHELL]
your laptop> spack info qmcpack
Description:
    QMCPACK, is a modern high-performance open-source Quantum Monte Carlo
    (QMC) simulation code.

Homepage: http://www.qmcpack.org/

Tags:
    None

Preferred version:
    3.6.0      [git] https://github.com/QMCPACK/qmcpack.git at tag v3.6.0

Safe versions:
    develop  [git] https://github.com/QMCPACK/qmcpack.git
    3.6.0      [git] https://github.com/QMCPACK/qmcpack.git at tag v3.6.0
    3.5.0      [git] https://github.com/QMCPACK/qmcpack.git at tag v3.5.0
    3.4.0      [git] https://github.com/QMCPACK/qmcpack.git at tag v3.4.0
    3.3.0      [git] https://github.com/QMCPACK/qmcpack.git at tag v3.3.0
    3.2.0      [git] https://github.com/QMCPACK/qmcpack.git at tag v3.2.0
    3.1.1      [git] https://github.com/QMCPACK/qmcpack.git at tag v3.1.1
    3.1.0      [git] https://github.com/QMCPACK/qmcpack.git at tag v3.1.0

Variants:
    Name [Default]                 Allowed values          Description


    build_type [RelWithDebInfo]    Debug, Release,         CMake build type
                                   RelWithDebInfo,
                                   MinSizeRel
    complex [off]                  True, False             Build the complex (general
                                                           twist/k-point) version
    cuda [off]                     True, False             Enable CUDA and GPU
                                                           acceleration
    da [off]                       True, False             Install with support for basic
                                                           data analysis tools
    debug [off]                    True, False             Build debug version
    gui [off]                      True, False             Install with Matplotlib (long
                                                           installation time)
    mixed [off]                    True, False             Build the mixed precision
                                                           (mixture of single and double
                                                           precision) version for gpu and
                                                           cpu
    mpi [on]                       True, False             Build with MPI support
    phdf5 [on]                     True, False             Build with parallel collective
                                                           I/O
    qe [on]                        True, False             Install with patched Quantum
                                                           Espresso 6.3.0
    soa [off]                      True, False             Build with Structure-of-Array
                                                           instead of Array-of-Structure
                                                           code. Only for CPU codeand
                                                           only in mixed precision
    timers [off]                   True, False             Build with support for timers

Installation Phases:
    cmake    build    install

Build Dependencies:
    blas  boost  cmake  cuda  espresso  fftw-api  hdf5  lapack  libxml2  mpi

Link Dependencies:
    blas  boost  cuda  espresso  fftw  hdf5  lapack  libxml2  mpi

Run Dependencies:
    py-matplotlib  py-numpy

Virtual Packages:
    None
\end{lstlisting}

For example, to install the complex version of QMCPACK in mixed-precision use:

\begin{shade}
your-laptop> spack install qmcpack+mixed+complex%gcc@7.2.0 ^intel-mkl
\end{shade}
where

\begin{shade}
%gcc@7.2.0
\end{shade}
specifies the compiler version to be used and

\begin{shade}
^intel-mkl
\end{shade}
specifies that the Intel MKL should be used as the BLAS and LAPACK provider.

It is also possible to run the QMCPACK regression tests as part of the
installation process, for example:

\begin{shade}
your-laptop> spack install --test=root qmcpack+mixed+complex%gcc@7.2.0 ^intel-mkl
\end{shade}
will run the unit and short tests. The current behavior of the QMCPACK
Spack package is to complete the install as long as all the unit tests
pass. If the short tests fail, a warning is issued at the command prompt.

For CUDA, you will need to specify and extra \ishell{cuda_arch}
parameter otherwise, it will default to \ishell{cuda_arch=61}.
\begin{shade}
your-laptop> spack install qmcpack+cuda%intel@18.0.3 cuda_arch=61 ^intel-mkl
\end{shade}

Due to limitations in the Spack CUDA package, if your compiler and
CUDA combination give you a conflict you will need to specify a
specific verison of CUDA that is compatible with your compiler on the
command line. For example,
\begin{shade}
your-laptop> spack install qmcpack+cuda%intel@18.0.3 cuda_arch=61 ^cuda@10.0.130 ^intel-mkl
\end{shade}

\subsection{Loading QMCPACK into your environment}
Spack does not set up an environment to automatically find its packages. A few additional steps are needed. First, install the
modules package by executing:

\begin{shade}
your-laptop> spack install environment-modules
\end{shade}

Then modify your \ishell{\~/.cshrc} file with something like

\begin{shade}
source ${HOME}/spack/opt/spack/darwin-sierra-x86_64/\\
          clang-7.3.0-apple/environment-modules-3.2.10-kenvhysdws4zw26tvbrjl37fdyn7inhb/\
          Modules/init/csh
\end{shade}

You should then be able to load the qmcpack binary into your path by
invoking

\begin{shade}
your-laptop> spack load qmcpack+mixed+complex%gcc@7.2.0
\end{shade}
or more generally

\begin{shade}
your-laptop> spack load qmcpack+<spec>
\end{shade}

\subsection{Installing QMCPACK with Spack on Linux}
Spack works robustly on the standard flavors of Linux (Ubuntu, CentOS,
Ubuntu, etc.) using GCC or Intel compilers.

\subsection{Installing QMCPACK with Spack on Mac OS X}
Spack works on Mac OS X but requires installation of a few packages
using Homebrew. You will need to install at minimum the GCC compilers,
CMake, and pkg-config. The Intel compiler for Mac on OS X is not well
supported by Spack packages and will most likely lead to a compile
time failure in one of QMCPACK's dependencies.

\subsection{Installing QMCPACK with Spack on IBM Blue Gene}
This is untested at this time. In principle, it should work as long as each
package found in

\begin{shade}
Blue Gene prompt> spack spec qmcpack
\end{shade}
can be compiled for the compute nodes.

\subsection{Installing QMCPACK with Spack on Cray Supercomputers}
There are a number of issues in the Cray module environment. Spack
contributors are working to fix these problems. We will update this
section once we have a recipe that works reliably.

\subsection{Reporting Bugs}
Bugs with the QMCPACK Spack package should be filed at the main GitHub
Spack repo \url{https://github.com/spack/spack/issues}

In the GitHub issue, include \ishell{@naromero77} to get the attention
of our developer.

\section{Testing and validation of QMCPACK}
\label{sec:testing}
We \textbf{strongly encourage} running the included tests each time
QMCPACK is built. A range of unit and integration tests ensure that
the code behaves as expected and that results are consistent with
known-good mean-field, quantum chemical, and historical QMC results.

The tests include the following:
\begin{itemize}
\item Unit tests: to check fundamental behavior. These should always pass.
\item Stochastic integration tests: to check computed results from
  the Monte Carlo methods. These might fail statistically, but rarely
  because of the use of three sigma level statistics. These tests are
  further split into ``short'' tests, which have just sufficient
  length to have valid statistics, and ``long'' tests, to check
  behavior to higher statistical accuracy.
\item Converter tests: to check conversion of trial wavefunctions
  from codes such as QE and GAMESS to QMCPACK's
  formats. These should always pass.
\item Workflow tests: in the case of QE, we test the
  entire cycle of DFT calculation, trial wavefunction conversion, and
  a subsequent VMC run.  
\item Performance: to help performance monitoring. Only the timing of
  these runs is relevant.  
\end{itemize}


The test types are differentiated by prefixes in their names, for example, \ishell{short-LiH_dimer_ae_vmc_hf_noj_16-1} indicates a short VMC test
for the LiH dime. 

QMCPACK also includes tests for developmental features and features
that are unsupported on certain platforms. To indicate these, tests
that are unstable are labeled with the CTest label
``unstable.'' For example, they are unreliable, unsupported, or known to fail
from partial implementation or bugs.

When installing QMCPACK you should run at least the unit tests:

\begin{shade}
 ctest -R unit
 \end{shade}
 
These tests take only a few seconds to run. All should pass. A
failure here could indicate a major problem with the installation.

A wider range of deterministic integration
tests are being developed. The goal is to test much more of QMCPACK than the unit tests
do and to do so in a manner that is reproducible
across platforms. All of these should eventually pass 100\% reliably
and quickly. At present, some fail on some platforms and for certain
build types.

 \begin{shade}
 ctest -R deterministic -LE unstable
 \end{shade}

If time allows, the ``short'' stochastic tests should also be run.
The short tests take a few minutes each on a 16-core machine---about 1 hour total depending on the platform. You can run these tests using the following command in the
build directory:


\begin{shade}
ctest -R short -LE unstable  # Run the tests with "short" in their name.
                             # Exclude any known unstable tests.
\end{shade}
The output should be similar to the following:

\begin{shade}
Test project build_gcc
      Start  1: short-LiH_dimer_ae-vmc_hf_noj-16-1
 1/44 Test  #1: short-LiH_dimer_ae-vmc_hf_noj-16-1 ..............  Passed   11.20 sec
      Start  2: short-LiH_dimer_ae-vmc_hf_noj-16-1-kinetic
 2/44 Test  #2: short-LiH_dimer_ae-vmc_hf_noj-16-1-kinetic ......  Passed    0.13 sec
..
42/44 Test #42: short-monoO_1x1x1_pp-vmc_sdj-1-16 ...............  Passed   10.02 sec
      Start 43: short-monoO_1x1x1_pp-vmc_sdj-1-16-totenergy
43/44 Test #43: short-monoO_1x1x1_pp-vmc_sdj-1-16-totenergy .....  Passed    0.08 sec
      Start 44: short-monoO_1x1x1_pp-vmc_sdj-1-16-samples
44/44 Test #44: short-monoO_1x1x1_pp-vmc_sdj-1-16-samples .......  Passed    0.08 sec

100% tests passed, 0 tests failed out of 44

Total Test time (real) = 167.14 sec
\end{shade}

Note that the number of tests run varies between the
standard, complex, and GPU compilations. These tests should pass with three sigma reliability. That is, they should nearly always pass, and when rerunning a failed test it should usually pass. Overly frequent failures suggest a problem that should be addressed before any scientific production.

The  full set of tests consist of significantly longer versions of the short
tests, as well as tests of the conversion utilities. The runs require
several hours each for improved statistics and a much more
stringent test of the code. To run all the tests, simply run CTest in the build
directory:

\begin{shade}
ctest -LE unstable           # Run all the stable tests. This will take several hours.
\end{shade}

You can also run verbose tests, which direct the QMCPACK
output to the standard output:

\begin{shade}
ctest -V -R short   # Verbose short tests
\end{shade}

The test system includes specific tests for the complex version of the code.

The input data files for the tests are located in the \ishell{tests} directory.
The system-level test directories are grouped into \ishell{heg}, \ishell{molecules}, and \ishell{solids}, with particular physical systems under each (for example \ishell{molecules/H4_ae}
\footnote{The suffix ``ae'' is short for ``all-electron,'' and ``pp'' is short for ``pseudopotential.''})
Under each physical system directory there might be tests for multiple QMC methods or parameter variations.
The numerical comparisons and test definitions are in the \ishell{CMakeLists.txt} file in each physical system directory.

If \textit{all} the QMC tests fail it is likely
that the appropriate mpiexec (or mpirun, aprun, srun, jsrun) is not being
called or found. If the QMC runs appear to work but all the other
tests fail, it is possible that Python is not working on your system. We suggest checking some of the test console output in \ishell{build/Testing/Temporary/LastTest.log}
or the output files under \ishell{build/tests/}.
%The runs occur in \ishell{build/tests/test_dir/test_name}.


Note that because most of the tests are very small, consisting of only a few
electrons, the performance is not representative of larger
calculations. For example, although the calculations might fit in cache,
there will be essentially no vectorization because of the small electron
counts. \textbf{These tests should therefore not be used for any benchmarking or
performance analysis}. Example runs that can be used for testing performance are described in
Section \ref{sec:perftests}.

\subsection{Deterministic and unit tests}

QMCPACK has a set of deterministic tests, predominantly unit tests.
All of these tests can be run with the following command (in the build directory):

\begin{shade}
ctest -R deterministic -LE unstable
\end{shade}

These tests should always pass. Failure could indicate a major problem
with the compiler, compiler settings, or a linked library that would
give incorrect results.

The output should look similar to the following:

\begin{shade}
Test project qmcpack/build
      Start  1: unit_test_numerics
 1/11 Test  #1: unit_test_numerics ...............   Passed    0.06 sec
      Start  2: unit_test_utilities
 2/11 Test  #2: unit_test_utilities ..............   Passed    0.02 sec
      Start  3: unit_test_einspline
 ...
10/11 Test #10: unit_test_hamiltonian ............   Passed    1.88 sec
      Start 11: unit_test_drivers
11/11 Test #11: unit_test_drivers ................   Passed    0.01 sec

100% tests passed, 0 tests failed out of 11

Label Time Summary:
unit    =   2.20 sec

Total Test time (real) =   2.31 sec
\end{shade}

Individual unit test executables can be found in \ishell{build/tests/bin}.
The source for the unit tests is located in the \ishell{tests} directory under each directory in \ishell{src} (e.g. \ishell{src/QMCWavefunctions/tests}).

See Chapter \ref{chap:unit_testing} for more details about unit tests.

\subsection{Integration tests with Quantum Espresso}
\label{sec:integtestqe}
As described in Section \ref{sec:buildqe}, it is possible to test entire
workflows of trial wavefunction generation, conversion, and eventual
QMC calculation. A patched QE must be installed so that the
pw2qmcpack converter is available.

By adding \ishell{-D QE_BIN=your_QE_binary_path} in the CMake command line when building your QMCPACK,
tests named with the ``qe-'' prefix will be included in the test set of your build.
You can test the whole pw $\to$ pw2qmcpack $\to$ qmcpack workflow by

\begin{shade}
ctest -R qe
\end{shade}
This provides a very solid test of the entire QMC
toolchain for plane wave--generated wavefunctions.

\subsection{Performance tests}
\label{sec:perftests}
Performance tests representative of real research runs are included in the
tests/performance directory. They can be used for benchmarking, comparing machine
performance, or assessing optimizations. This is in
contrast to the majority of the conventional integration tests in which the particle
counts are too small to be representative. Care is still needed to
remove initialization, I/O, and compute a representative performance
measure.

The CTest integration is sufficient to run the benchmarks and measure
relative performance from version to version of QMCPACK and to assess
proposed code changes. Performance tests are prefixed with
``performance.'' To obtain the highest performance on a particular
platform, you must run the benchmarks in a standalone manner and tune
thread counts, placement, walker count (etc.). This is essential to
fairly compare different machines. Check with the
developers if you are unsure of what is a fair change.

For the largest problem sizes, the initialization of spline orbitals might
take a large portion of overall runtime. When QMCPACK is run at scale,
the initialization is fast because it is fully
parallelized. However, the performance runs are most usually run on a single node.
Consider running QMCPACK once with \ishell{--save_wfs=1} to save the
converted spline coefficients to the disk and load them for later runs in the same folder.

The delayed update algorithm in Section~\ref{sec:singledeterminant}
significantly changes the performance characteristics of QMCPACK.  A
parameter scan of the maximal number of delays specific to every
architecture and problem size is required to achieve the best
performance.

\subsubsection{NiO performance tests}

Follow the instructions in tests/performance/NiO/README to
enable and run the NiO tests.

The NiO tests are for bulk supercells of varying size. The QMC runs consist of short blocks of (1) VMC
without drift (2) VMC with drift term included, and (3) DMC with
constant population. The tests use spline wavefunctions that must be
downloaded as described in the README file because of their large size. You
will need to set ``\ishell{-DQMC_DATA=YOUR_DATA_FOLDER -DENABLE_TIMERS=1}''
when running CMake as
described in the README file.

Two sets of wavefunction are tested: spline orbitals with one- and
two-body Jastrow functions and a more complex form with an additional
three-body Jastrow function. The Jastrows are the same for each run
and are not reoptimized, as might be done for research purposes.  Runs
in the hundreds of electrons up to low thousands of electrons are representative of
research runs performed in 2017. The largest runs target
future machines and require very large memory.

\begin{table}[h]
\begin{center}
  \caption{System sizes and names for NiO performance tests. GPU performance
    tests are named similarly but have different walker counts.}
\begin{tabular}{|c|c|c|c|c|}
\hline
\bfseries Performance test name&  \bfseries Historical name &\bfseries Atoms& \bfseries Electrons&  \bfseries Electrons/spin \\
\hline
performance-NiO-cpu-a32-e384  & S8 & 32 & 384 & 192 \\
performance-NiO-cpu-a64-e768  & S16 & 64 & 768 & 384 \\
performance-NiO-cpu-a128-e1536 & S32 & 128 & 1536 & 768 \\
performance-NiO-cpu-a256-e3072 & S64 & 256 & 3072 & 1536 \\
performance-NiO-cpu-a512-e6144 & S128 & 512 & 6144 & 3072 \\
performance-NiO-cpu-a1024-e12288& S256 & 1024 & 12288 & 6144 \\
\hline
\end{tabular}

  \label{tab:niotests}
\end{center}
\end{table}

\subsection{Troubleshooting tests}
CTest reports briefly pass or fail of tests in printout and also collects all the standard outputs to help investigating how tests fail.
If the CTest execution is completed, look at \ishell{Testing/Temporary/LastTest.log}.
If you manually stop the testing (ctrl+c), look at \ishell{Testing/Temporary/LastTest.log.tmp}.
You can locate the failing tests by searching for the key word ``Fail.''

\subsection{Slow testing with OpenMPI}
OpenMPI has a default binding policy that makes all the threads run on a single core during testing when there are two or fewer MPI ranks.
This significantly increases testing time. If you are authorized to change the default setting, you can just add ``hwloc\_base\_binding\_policy=none'' in /etc/openmpi/openmpi-mca-params.conf.

\section{Automated testing of QMCPACK}

The QMCPACK developers run automatic tests of QMCPACK on several
different computer systems,  many on a continuous basis. See the reports at
\url{https://cdash.qmcpack.org/CDash/index.php?project=QMCPACK}.
We currently test
the following combinations nightly (workstations) and weekly (supercomputers):

\begin{itemize}
\item On a Linux Intel Xeon workstation, combinations of:
  \begin{itemize}
  \item GCC 8.2.0, Intel2019, Clang6, Clang7, PGI2018
  \item No MPI, Intel MPI, and OpenMPI
  \item CPU and GPU builds using CUDA 10.0
  \end{itemize}
\item On a Linux Intel Knight's Landing workstation:
  \begin{itemize}
  \item Intel 2017 with Intel MPI and MKL
  \item GCC 6.3.1 with Intel MPI and MKL
  \end{itemize}
\item On Eos, a Cray XC30 Intel machine:
  \begin{itemize}
\item The default Intel programming environment and compiler with Cray MPI and Intel MKL
  \end{itemize}
\item On Titan, a Cray XK7 CPU+GPU machine:
  \begin{itemize}
  \item The GCC programming environment and compiler with Cray MPI and CUDA
  \item The GCC programming environment and compiler with Cray MPI
  \end{itemize}
\item On Cetus, an IBM Blue Gene Q machine:
\begin{itemize}
\item Blue Gene Clang
\end{itemize}
\end{itemize}

\begin{figure}
  \centering
  \includegraphics[width=10cm]{./figures/QMCPACK_CDash_CTest_Results_20160129.png}
  \caption{Example test results for QMCPACK showing data for a
    workstation (Intel, GCC, both CPU and GPU builds) and for two ORNL
    supercomputers. In this example, four errors were found. This
    dashboard is accessible at \url{https://cdash.qmcpack.org}.}
\end{figure}

\section{Building ppconvert, a pseudopotential format converter}
\label{sec:buildppconvert}
QMCPACK includes a utility---ppconvert---to convert between different
pseudopotential formats. Examples include effective core potential
formats (in Gaussians), the UPF format used by QE, and
the XML format used by QMCPACK itself. The utility also enables the
atomic orbitals to be recomputed via a numerical density functional
calculation if they need to be reconstructed for use in an
electronic structure calculation.

The utility is a stand-alone C++ executable that is not built by default but that is accessible via adding
\ishell{-DBUILD_PPCONVERT=1} to CMake and then typing \ishell{make ppconvert}.
A user guide is provided in Section~\ref{sec:ppconvert}.

\section{Installing and patching Quantum ESPRESSO}
\label{sec:buildqe}
For trial wavefunctions obtained in a plane-wave basis, we mainly
support QE. Note that ABINIT and QBox were supported historically
and could be reactivated.

QE stores wavefunctions in a nonstandard internal
``save'' format. To convert these to a conventional HDF5 format file
we have developed a converter---pw2qmcpack---which is an add-on to the
QE distribution.

To simplify the process of patching QE we have developed
a script that will automatically download and patch the source
code. The patches are specific to each version. For example, to download and
patch QE v6.3:

\begin{shade}
cd external_codes/quantum_espresso
./download_and_patch_qe6.3.sh
\end{shade}
After running the patch, you must configure QE with
the HDF5 capability enabled in either way:
\begin{itemize}
\item If your system already has HDF5 installed with Fortran, use the -{}-with-hdf5 configuration option.

\begin{shade}
cd qe-6.3
./configure --with-hdf5=/opt/local   # Specify HDF5 base directory
\end{shade}
   {\bf Check} the end of the configure output if HDF5 libraries are found properly.
   If not, either install a complete library or use the other scheme. If using a parallel HDF5 library, be sure to use
   the same MPI with QE as used to build the parallel HDF5 library.

   Currently, HDF5 support in QE itself is preliminary. To enable use of pw2qmcpack
   but use the old non-HDF5 I/O within QE, replace \ishell{-D__HDF5} with \ishell{-D__HDF5_C} in make.inc.
\item If your system has HDF5 with C only, manually edit make.inc by adding \ishell{-D__HDF5_C} and \ishell{-DH5_USE_16_API}
   in \ishell{DFLAGS} and provide include and library path in \ishell{IFLAGS} and \ishell{HDF5_LIB}.
\end{itemize}

The complete process is described in external\_codes/quantum\_espresso/README.

The tests involving pw.x and pw2qmcpack.x have been integrated into the test suite of QMCPACK.
By adding \ishell{-D QE_BIN=your_QE_binary_path} in the CMake command line when building your QMCPACK,
tests named with the ``qe-'' prefix will be included in the test set of your build.
You can test the whole pw $\to$ pw2qmcpack $\to$ qmcpack workflow by

\begin{shade}
ctest -R qe
\end{shade}
See Section \ref{sec:integtestqe} and the testing section for more details.

\section{How to build the fastest executable version of QMCPACK}
\label{sec:buildperformance}
To build the fastest version of QMCPACK we recommend the following:
\begin{itemize}
\item Use the latest C++ compilers available for your
  system. Substantial gains have been made optimizing C++ in recent
  years.
\item Use a vendor-optimized BLAS library such as Intel MKL and AMD ACML. Although
  QMC does not make extensive use of linear algebra, it is used in the
  VMC wavefunction optimizer to apply the orbital coefficients in local basis
  calculations and in the Slater determinant update.
\item Use a vector math library such as Intel VML.  For periodic
  calculations, the calculation of the structure factor and Ewald
  potential benefit from vectorized evaluation of sin and
  cos. Currently we only autodetect Intel VML, as provided with MKL,
  but support for MASSV and AMD LibM is included via \#defines. See,
 for example, src/Numerics/e2iphi.h. For
  large supercells, this optimization can gain 10\% in performance.
\end{itemize}

Note that greater speedups of QMC calculations can usually be obtained by
carefully choosing the required statistics for each
investigation. That is, do not compute smaller error bars than necessary.

\section{Troubleshooting the installation}
\label{sec:troubleshoot}
Some tips to help troubleshoot installations of QMCPACK:
\begin{itemize}
\item First, build QMCPACK on a workstation you control or on any
  system with a simple and up-to-date set of development
  tools. You can compare the results of CMake and QMCPACK on this
  system with any more difficult systems you encounter.
\item Use up-to-date development software, particularly a recent
  CMake.
\item Verify that the compilers and libraries you expect are
  being configured. It is common to have multiple versions
  installed. The configure system will stop at the first version it
  finds, which might not be the most recent. If this occurs, directly specify the appropriate
  directories and files (Section
  \ref{sec:cmakeoptions}). For example, 
  \begin{shade}
  cmake -DCMAKE_C_COMPILER=/full/path/to/mpicc -DCMAKE_CXX_COMPILER=/full/path/to/mpicxx ..
  \end{shade}
\item To monitor the compiler and linker settings, use a verbose build, \ishell{make
  VERBOSE=1}. If an individual source file fails to compile you
  can experiment by hand using the output of the verbose build to
  reconstruct the full compilation line.
\end{itemize}

If you still have problems please post to the QMCPACK Google group with full
details, or contact a developer.

\chapter{Running QMCPACK}
\label{chap:running}

QMCPACK requires at least one xml input file, and is invoked via:

{\ishell{qmcpack [command line options] <XML input file(s)>}}

\section{Command line options}
\label{sec:commandline}
QMCPACK offers several command line options that affect how calculations
are performed. If the flag is absent, then the corresponding
option is disabled.

% unfortunately description is not compatible with much in its []
\begin{description}
\item[\texttt{-{}-dryrun}]{ Validate the input file without performing the simulation.
  This is a good way to ensure that QMCPACK will do what you think it will. }
\item[\texttt{-{}-enable-timers=none|coarse|medium|fine}]{ Control the timer granularity
  when the build option \ishell{ENABLE_TIMERS} is enabled. }
\item[\texttt{-{}-help}]{ Print version information as well as a list of optional
  command-line arguments. }
\item[\texttt{-{}-noprint}]{ Do not print extra information on Jastrow or pseudopotential.
  If this flag is not present, QMCPACK will create several \ishell{.dat} files
  that contain information about pseudopotentials (one file per PP) and Jastrow
  factors (one per Jastrow factor). These file might be useful for visual inspection
  of the Jastrow, for example. }
\item[\texttt{-{}-verbosity=low|high|debug}]{ Control the output verbosity. The default low verbosity is concise and, for example, does not include all electron or atomic positions for large systems to reduce output size. Use ``high'' to see this information and more details of initialization, allocations, QMC method settings, etc. }
\item[\texttt{-{}-version}]{ Print version information and optional arguments.
  Same as \ishell{--help}. }
\end{description}


\section{Input files}
\label{sec:inputs}
The input is one or more XML file(s), documented in Chapter~\ref{chap:input_overview}.

\section{Output files}
QMCPACK generates multiple files, documented in Chapter~\ref{chap:output_overview}.

\section{Running in parallel with MPI}
\label{sec:parallelrunning}

QMCPACK is fully parallelized with MPI. When performing an ensemble job, all
the MPI ranks are first equally divided into groups that perform individual
QMC calculations. Within one calculation, all the walkers are fully distributed
across all the MPI ranks in the group. Since MPI requires distributed memory,
there must be at least one MPI per node. To maximize the efficiency, more facts
should be taken into account. When using MPI+threads on compute nodes with more
than one NUMA domain (e.g., AMD Interlagos CPU on Titan or a node with multiple
CPU sockets), it is recommended to place as many MPI ranks as the number of
NUMA domains if the memory is sufficient (e.g., one MPI task per socket). On clusters with more than one
GPU per node (NVIDIA Tesla K80), it is necessary to use the same number of MPI
ranks as the number of GPUs per node to let each MPI rank take one GPU.

\section{Using OpenMP threads}
\label{sec:openmprunning}
Modern processors integrate multiple identical cores even with
hardware threads on a single die to increase the total performance and
maintain a reasonable power draw. QMCPACK takes advantage of this
compute capability by using threads and the OpenMP programming model
as well as threaded linear algebra libraries. By default, QMCPACK is
always built with OpenMP enabled. When launching calculations, users
should instruct QMCPACK to create the right number of threads per MPI
rank by specifying environment variable OMP\_NUM\_THREADS. Assuming
one MPI rank per socket, the number of threads should typically be the
number of cores on that socket. Even in the GPU-accelerated version,
using threads significantly reduces the time spent on the calculations
performed by the CPU.

\subsection{Nested OpenMP threads}
Nested threading is an advanced feature requiring experienced users to finely tune runtime parameters to reach the best performance.  

For small-to-medium problem sizes, using one thread per walker or for multiple walkers is most efficient. This is the default in QMCPACK and achieves the shortest time to solution.

For large problems of at least 1,000 electrons, use of nested OpenMP threading can be enabled to reduce the time to solution further, although at some loss of efficiency. In this scheme multiple threads are used in the computations of each walker. This capability is implemented for some of the key computational kernels: the 3D spline orbital evaluation, certain portions of the distance tables, and implicitly the BLAS calls in the determinant update. Use of the batched nonlocal pseudopotential evaluation is also recommended.

Nested threading is enabled by setting \ishell{OMP\_NUM\_THREADS=AA,BB}, \ishell{OMP\_MAX\_ACTIVE\_LEVELS=2} and \ishell{OMP\_NESTED=TRUE} where the additional \ishell{BB} is the number of second-level threads.  Choosing the thread affinity is critical to the performance.  QMCPACK provides a tool qmc-check-affinity (source file src/QMCTools/check-affinity.cpp for details), which might help users investigate the affinity. Knowledge of how the operating system logical CPU cores (/prco/cpuinfo) are bound to the hardware is also needed.

For example, on Blue Gene/Q with a Clang compiler, the best way to fully use the 16 cores each with 4 hardware threads is
\begin{shade}
OMP_NESTED=TRUE
OMP_NUM_THREADS=16,4
MAX_ACTIVE_LEVELS=2
OMP_PLACES=threads
OMP_PROC_BIND=spread,close
\end{shade}

On Intel Xeon Phi KNL with an Intel compiler, to use 64 cores without using hardware threads:
\begin{shade}
OMP_NESTED=TRUE
OMP_WAIT_POLICY=ACTIVE
OMP_NUM_THREADS=16,4
MAX_ACTIVE_LEVELS=2
OMP_PLACES=cores
OMP_PROC_BIND=spread,close
KMP_HOT_TEAMS_MODE=1
KMP_HOT_TEAMS_MAX_LEVEL=2
\end{shade}

Most multithreaded BLAS/LAPACK libraries do not spawn threads by default
when being called from an OpenMP parallel region. See the explanation in Section~\ref{sec:threadedlibrary}.
This results in the use of only a single thread in each second-level thread team for BLAS/LAPACK operations.
Some vendor libraries like MKL support using multiple threads when being called from an OpenMP parallel region.
One way to enable this feature is using environment variables to override the default behavior.
However, this forces all the calls to the library to use the same number of threads.
As a result, small function calls are penalized with heavy overhead and heavy function calls are slow for not being able to use more threads.
Instead, QMCPACK uses the library APIs to turn on nested threading only at selected performance critical calls.
In the case of using a serial library, QMCPACK implements nested threading to distribute the workload wherever necessary.
Users do not need to control the threading behavior of the library.

\subsection{Performance considerations}
\label{sec:cpu:performance}
As walkers are the basic units of workload in QMC algorithms, they are loosely coupled and distributed across all the threads. For this reason, the best strategy to run QMCPACK efficiently is to feed enough walkers to the available threads.

In a VMC calculation, the code automatically raises the actual number of walkers per MPI rank to the number of available threads
if the user-specified number of walkers is smaller, see ``walkers/mpi=XXX'' in the VMC output.

In DMC, for typical small to mid-sized calculations choose the total number of walkers to be a significant multiple of the total number of
threads (MPI tasks * threads per task). This will ensure a good load balance. e.g. For a calculation on a few nodes with a total
512 threads, using 5120 walkers may keep the load imbalance around 10\%. For the very largest calculations, the target number of
walkers should be chosen to be slightly smaller than a multiple of the total number of available threads across all the MPI ranks.
This will reduce occurrences worse-case load imbalance e.g. where one thread has two walkers while all the others have one.

To achieve better performance, a mixed-precision version (experimental) has been developed in the CPU code. The mixed-precision
CPU code uses a mixed of single precision (SP) and double precision (DP) operations, while the default code use DP exclusively.
This mixed precision version is more aggressive than the GPU CUDA version in using single precision (SP) operations. The Current implementation uses SP on most
calculations, except for matrix inversions and reductions where double precision is required to retain high accuracy. All the
constant spline data in wavefunction, pseudopotentials, and Coulomb potentials are initialized in double precision and later
stored in single precision. The mixed-precision code is as accurate as the double-precision code up to a certain system size, and
may have double the throughput.
Cross checking and verification of accuracy is always required but is particularly important above approximately 1,500 electrons. 

\subsection{Memory considerations}
When using threads, some memory objects are shared by all the threads. Usually these memory objects are read only when the walkers are evolving, for instance the ionic distance table and wavefunction coefficients.
If a wavefunction is represented by B-splines, the whole table is shared by all the threads. It usually takes a large chunk of memory when a large primitive cell was used in the simulation. Its actual size is reported as ``MEMORY increase XXX MB BsplineSetReader'' in the output file.
See details about how to reduce it in Section~\ref{sec:spo_spline}.

The other memory objects that are distinct for each walker during random walks need to be associated with individual walkers and cannot be shared. This part of memory grows linearly as the number of walkers per MPI rank. Those objects include wavefunction values (Slater determinants) at given electronic configurations and electron-related distance tables (electron-electron distance table). Those matrices dominate the $N^2$ scaling of the memory usage per walker.

\section{Running on GPU machines}
\label{sec:gpurunning}

The GPU version for the NVIDIA CUDA platform is fully incorporated into
the main source code. Commonly used functionalities for
solid-state and molecular systems using B-spline single-particle
orbitals are supported. Use of Gaussian basis sets, three-body
Jastrow functions, and many observables are not yet supported. A detailed description of the GPU
implementation can be found in Ref. \cite{EslerKimCeperleyShulenburger2012}.

The current GPU implementation assumes one MPI process per GPU. To use
nodes with multiple GPUs, use multiple MPI processes per node.
Vectorization is achieved over walkers, that is, all walkers are
propagated in parallel. In each GPU kernel, loops over electrons,
atomic cores, or orbitals are further vectorized to exploit an
additional level of parallelism and to allow coalesced memory access.

%----------------------------------------------------------------------------%

\subsection{Performance considerations}
\label{sec:gpu:performance}

To run with high performance on GPUs it is crucial to perform some
benchmarking runs: the optimum configuration is system size, walker
count, and GPU model dependent. The GPU implementation vectorizes
operations over multiple walkers, so generally the more walkers that
are placed on a GPU, the higher the performance that will be
obtained. Performance also increases with electron count, up until the
memory on the GPU is exhausted. A good strategy is to perform a short
series of VMC runs with walker count increasing in multiples of
two. For systems with 100s of electrons, typically 128--256 walkers per
GPU use a sufficient number of GPU threads to operate the GPU
efficiently and to hide memory-access latency. For smaller systems,
thousands of walkers might be required. For QMC algorithms where the number of
walkers is fixed such as VMC, choosing a walker count the is a multiple of the
number of streaming multiprocessors can be most efficient. For
variable population DMC runs, this exact match is not possible.

To achieve better performance, the current GPU implementation uses
single-precision operations for most of the calculations. Double
precision is used in matrix inversions and the Coulomb interaction to
retain high accuracy. The mixed-precision GPU code is as accurate as
the double-precision CPU code up to a certain system size. Cross
checking and verification of accuracy are encouraged for systems with
more than approximately 1,500 electrons. For typical calculations on
smaller electron counts, the statistical error bars are much larger
then the error introduced by mixed precision.

%------------------------------------------------------------------------------%

\subsection{Memory considerations}

In the GPU implementation, each walker has a buffer in the GPU's
global memory to store temporary data associated with the
wavefunctions. Therefore, the amount of memory available on a GPU
limits the number of walkers and eventually the system size that it
can process. Additionally, for calculations using B-splines, this data
is stored on the GPU in a shared read-only buffer. Often the size of the
B-spline data limits the calculations that can be run on the GPU.

If the GPU memory is exhausted, first try reducing the number of walkers per GPU.
Coarsening the grids of the B-splines representation (by decreasing
the value of the mesh factor in the input file) can also lower the memory
usage, at the expense (risk) of obtaining inaccurate results. Proceed
with caution if this option has to be considered.  It is also possible
to distribute the B-spline coefficients table between the host and GPU
memory, see option Spline\_Size\_Limit\_MB in
Section ~\ref{sec:spo_spline}.


\chapter{Units used in QMCPACK}
\label{sec:units}

Internally, QMCPACK uses atomic units throughout. Unless stated, all inputs and outputs are also in atomic units. For convenience the analysis tools offer conversions to eV, Ry, Angstrom, Bohr, etc.



\chapter{Input file overview}
\label{chap:input_overview}

This chapter introduces XML as it is used in the QMCPACK input file.  The focus is on the XML file format itself and the general structure of the input file rather than an exhaustive discussion of all keywords and structure elements.  

QMCPACK uses XML to represent structured data in its input file.  Instead of text blocks like

\begin{shade}
begin project
  id     = vmc
  series = 0
end project

begin vmc
  move     = pbyp
  blocks   = 200
  steps    =  10
  timestep = 0.4
end vmc
\end{shade} 
QMCPACK input looks like
\begin{lstlisting}[style=QMCPXML]
   <project id="vmc" series="0">
   </project>

   <qmc method="vmc" move="pbyp">
      <parameter name="blocks"  >  200 </parameter>
      <parameter name="steps"   >   10 </parameter>
      <parameter name="timestep">  0.4 </parameter>
   </qmc>
\end{lstlisting}
XML elements start with \ixml{<element\_name>}, end with \ixml{</element\_name>}, and can be nested within each other to denote substructure (the trial wavefunction is composed of a Slater determinant and a Jastrow factor, which are each further composed of \ldots).  \ixml{id} and \ixml{series} are attributes of the \ixml{<project/>} element.  XML attributes are generally used to represent simple values, like names, integers, or real values.  Similar functionality is also commonly provided by \ixml{<parameter/>} elements like those previously shown.

The overall structure of the input file reflects different aspects of the QMC simulation: the simulation cell, particles, trial wavefunction, Hamiltonian, and QMC run parameters.  A condensed version of the actual input file is shown as follows:
\begin{lstlisting}[style=QMCPXML]
<?xml version="1.0"?>
<simulation>

  <project id="vmc" series="0">
    ...
  </project>

  <qmcsystem>

    <simulationcell>
      ...
    </simulationcell>

    <particleset name="e">
      ...
    </particleset>

    <particleset name="ion0">
      ...
    </particleset>

    <wavefunction name="psi0" ... >
      ...
      <determinantset>
        <slaterdeterminant>
          ..
        </slaterdeterminant>
      </determinantset>
      <jastrow type="One-Body" ... >
         ...
      </jastrow>
      <jastrow type="Two-Body" ... >
        ...
      </jastrow>
    </wavefunction>

    <hamiltonian name="h0" ... >
      <pairpot type="coulomb" name="ElecElec" ... />
      <pairpot type="coulomb" name="IonIon"   ... />
      <pairpot type="pseudo" name="PseudoPot" ... >
        ...
      </pairpot>
    </hamiltonian>

   </qmcsystem>

   <qmc method="vmc" move="pbyp">
     <parameter name="warmupSteps">   20 </parameter>
     <parameter name="blocks"     >  200 </parameter>
     <parameter name="steps"      >   10 </parameter>
     <parameter name="timestep"   >  0.4 </parameter>
   </qmc>

</simulation>
\end{lstlisting}
The omitted portions (\texttt{...}) are more fine-grained inputs such as the axes of the simulation cell, the number of up and down electrons, positions of atomic species, external orbital files, starting Jastrow parameters, and external pseudopotential files.  


\section{Project}
The \ixml{<project>} tag uses the \ixml{id} and \ixml{series} attributes.
The value of \ixml{id} is the first part of the prefix for output file names.

Output file names also contain the series number, starting at the value given by the
\ixml{series} tag.  After every \ixml{<qmc>} section, the series value will increment, giving each section a unique prefix.

For the input file shown previously, the output files will start with \ishell{vmc.s000}, for example, \ishell{vmc.s000.scalar.dat}.
If there were another \ixml{<qmc>} section in the input file, the corresponding output files would use the prefix \ishell{vmc.s001}.



\section{Random number initialization}

The random number generator state is initialized from the \ixml{random} element using the \ixml{seed} attribute:
\begin{lstlisting}[style=QMCPXML]
<random seed="1000"/>
\end{lstlisting}

If the random element is not present, or the seed value is negative, the seed will be generated from the current time.

To initialize the many independent random number generators (one per thread and MPI process), the seed value is used (modulo 1024) as a starting index into a list of prime numbers.
Entries in this offset list of prime numbers are then used as the seed for the random generator on each thread and process.

If checkpointing is enabled, the random number state is written to an HDF file at the end of each block (suffix: \ishell{.random.h5}).
This file will be read if the \ixml{mcwalkerset} tag is present to perform a restart.
For more information, see the \ixml{checkpoint} element in the QMC methods Chapter \ref{chap:qmcmethods} and Section~\ref{sec:checkpoint_files} on checkpoint and restart files.


\chapter{Specifying the system to be simulated}
\section{Specifying the simulation cell}
\label{chap:simulationcell}

The \ixml{simulationcell} block specifies the geometry of the cell, how the boundary conditions should be handled, and how ewald summation should be broken up.

\begin{table}[h]
\begin{center}
\begin{tabularx}{\textwidth}{l l l l l X }
\hline
\multicolumn{6}{l}{\texttt{simulationcell} element} \\
\hline
\multicolumn{2}{l}{parent elements:} & \multicolumn{4}{l}{\texttt{qmcsystem}}\\
\multicolumn{2}{l}{child  elements:} & \multicolumn{4}{l}{None}\\
\multicolumn{2}{l}{attribute      :} & \multicolumn{4}{l}{}\\
   &   \bfseries parameter name            & \bfseries datatype & \bfseries values & \bfseries default   & \bfseries description \\
\hline
   &   \texttt{lattice}  & 9 floats & any float & Must be specified & Specification of \\
   &                     &        &             &                   & lattice vectors. \\
   &   \texttt{bconds}   & string & ``p'' or ``n''  & ``n n n'' & Boundary conditions \\
   &                     &        &             &           & for each axis. \\
   &   \texttt{vacuum} & float & $\ge 1.0$ & 1.0        & Vacuum scale. \\
   &   \texttt{LR\_dim\_cutoff} & float & float & 15        & Ewald breakup distance. \\
   &   \texttt{LR\_tol} & float & float & 3e-4        & Tolerance in Ha for Ewald ion-ion energy per atom. \\
\hline
\end{tabularx}
\end{center}
\end{table}

An example of a \ixml{simulationcell} block is given below:
\begin{lstlisting}[style=QMCPXML]
  <simulationcell>
    <parameter name="lattice">
      3.8       0.0       0.0
      0.0       3.8       0.0
      0.0       0.0       3.8
    </parameter>
    <parameter name="bconds">
       p p p
    </parameter>
    <parameter name="LR_dim_cutoff"> 20 </parameter>
  </simulationcell>
\end{lstlisting}

Here, a cubic cell 3.8 bohr on a side will be used.
This simulation will use periodic boundary conditions, and the maximum
$k$ vector will be $20/r_{wigner-seitz}$ of the cell.


\subsection{Lattice}
The cell is specified using 3 lattice vectors.


\subsection{Boundary conditions}
QMCPACK offers the capability to use a mixture of open and periodic boundary conditions.
The \ixml{bconds} parameter expects a single string of three characters separated by
spaces, \textit{e.g.} ``p p p'' for purely periodic boundary conditions. These characters control
the behavior of the $x$, $y$, and $z$, axes, respectively. Non periodic directions must be placed after the periodic ones.
Examples of valid \ixml{bconds} include:

\begin{description}
\item[``p p p''] Periodic boundary conditions. Corresponds to a 3D crystal.
\item[``p p n''] Slab geometry. Corresponds to a 2D crystal.
\item[``p n n''] Wire geometry. Corresponds to a 1D crystal.
\item[``n n n''] Open boundary conditions. Corresponds to an isolated molecule in a vacuum.
\end{description}

\subsection{Vacuum}
The vacuum option allows adding a vacuum region in slab or wire boundary conditions
(\ixml{bconds= p p n} or \ixml{bconds= p n n}, respectively). The main use is
to save memory with spline or plane-wave basis trial wavefunctions, because no basis
functions are required inside the vacuum region. For example, a large vacuum region
can be added above and below a graphene sheet without having to generate the trial
wavefunction in such a large box or to have as many splines as would otherwise
be required. Note that the trial wavefunction must still be generated in a
large enough box to sufficiently reduce periodic interactions in the underlying
electronic structure calculation.

With the vacuum option, the box used for Ewald summation increases along the axis labeled \ixml{n} by a factor of \ixml{vacuum}.
Note that all the particles remain in the original box without altering their positions. i.e. Bond lengths are not changed by this option.
The default value is 1, no change to the specified axes.

An example of a \ixml{simulationcell} block using \ixml{vacuum} is given below.
The size of the box along the z-axis increases from 12 to 18 by the vacuum scale of 1.5.
\begin{lstlisting}[style=QMCPXML]
  <simulationcell>
    <parameter name="lattice">
      3.8       0.0       0.0
      0.0       3.8       0.0
      0.0       0.0      12.0
    </parameter>
    <parameter name="bconds">
       p p n
    </parameter>
    <parameter name="vacuum"> 1.5 </parameter>
    <parameter name="LR_dim_cutoff"> 20 </parameter>
  </simulationcell>
\end{lstlisting}

\subsection{LR\_dim\_cutoff}
When using periodic boundary conditions direct calculation of the Coulomb energy is
not well behaved. As a result, QMCPACK uses an optimized Ewald summation technique
to compute the Coulomb interaction.\cite{Natoli1995}

In the Ewald summation, the energy is broken into short- and long-ranged terms.
The short-ranged term is computed directly in real space, while the long-ranged term is computed in reciprocal space.
\ixml{LR_dim_cutoff} controls where the short-ranged term ends and the long-ranged term begins.
The real-space cutoff, reciprocal-space cutoff, and \ixml{LR_dim_cutoff} are related via:
\begin{equation}
\mathrm{LR\_dim\_cutoff} = r_{c} \times k_{c}
\end{equation}
where $r_{c}$ is the Wigner-Seitz radius, and $k_{c}$ is the length of the maximum $k$-vector used in the long-ranged term. Larger
values of \ixml{LR_dim_cutoff} increase the accuracy of the evaluation. A value of 15 tends to be conservative.

\section{Specifying the particle set}
\label{sec:particleset}


The \ixml{particleset} blocks specify the particles in the QMC simulations: their types, attributes (mass, charge, valence), and positions.   

\subsection{Input specification}
\begin{table}[h]
\begin{center}
\begin{tabularx}{\textwidth}{l l l l l X }
\hline
\multicolumn{6}{l}{\texttt{particleset} element} \\
\hline
\multicolumn{2}{l}{Parent elements:} & \multicolumn{4}{l}{\texttt{simulation}}\\
\multicolumn{2}{l}{Child  elements:} & \multicolumn{4}{l}{\texttt{group, attrib}}\\
\multicolumn{2}{l}{Attribute:} & \multicolumn{4}{l}{}\\
   &   \bfseries Name            & \bfseries Datatype & \bfseries Values & \bfseries Default   & \bfseries Description \\
   &   \texttt{name}/\texttt{id}   &  Text              &  \textit{Any}    &  e                & Name of particle set  \\
   &   \texttt{size}$^o$           &  Integer           &  \textit{Any}    &  0                & Number of particles in set \\
   &   \texttt{random}$^o$         &  Text              &  Yes/no          &  No               & Randomize starting positions \\
   &   \texttt{randomsrc}/         &  Text     & \texttt{particleset.name} & \textit{None}     & Particle set to randomize  \\
   &   \texttt{random\_source}$^o$ &                    &                  &                   &                       \\
%   &   \texttt{role}     &  text              &  MC/none         &  none               & (obsolete)                       \\
  \hline
\end{tabularx}
\end{center}
\end{table}

\begin{table}[h]
\begin{center}
\begin{tabularx}{\textwidth}{l l l l l X }
\hline
\multicolumn{6}{l}{\texttt{group} element} \\
\hline
\multicolumn{2}{l}{Parent elements:} & \multicolumn{4}{l}{\texttt{particleset}}\\
\multicolumn{2}{l}{Child  elements:} & \multicolumn{4}{l}{\texttt{parameter, attrib}}\\
\multicolumn{2}{l}{Attribute:} & \multicolumn{4}{l}{}\\
   &   \bfseries Name            & \bfseries Datatype & \bfseries Values & \bfseries Default   & \bfseries Description \\
   &   \texttt{name}               &  Text              &  \textit{Any}    &  e                & Name of particle set  \\
   &   \texttt{size}$^o$           &  Integer           &  \textit{Any}    &  0                & Number of particles in set \\
   &   \texttt{mass}$^o$           &  Real              &  \textit{Any}    &  1                & Mass of particles in set \\
   &   \texttt{unit}$^o$          &  Text              &  au/amu          &  au               & Units for mass of particles \\
\multicolumn{2}{l}{parameters}  & \multicolumn{4}{l}{}\\
   &   \bfseries Name     & \bfseries Datatype & \bfseries Values & \bfseries Default   & \bfseries Description \\
   &   \texttt{charge}    &  Real              &  \textit{Any}    &  0                  & Charge of particles in set \\
   &   \texttt{valence}   &  Real              &  \textit{Any}    &  0                  & Valence charge of particles in set \\
   &   \texttt{atomicnumber} &  Integer        &  \textit{Any}    &  0                  & Atomic number of particles in set \\
  \hline
  \hline
\end{tabularx}
\end{center}
\end{table}

\begin{table}[h]
\begin{center}
\begin{tabularx}{\textwidth}{l l l l l X }
\hline
\multicolumn{6}{l}{\texttt{attrib} element} \\
\hline
\multicolumn{2}{l}{Parent elements:} & \multicolumn{4}{l}{\texttt{particleset,group}}\\
\multicolumn{2}{l}{Attribute:} & \multicolumn{4}{l}{}\\
   &   \bfseries Name            & \bfseries Datatype & \bfseries Values & \bfseries Default   & \bfseries Description \\
   &   \texttt{name}             &  String            &  \textit{Any}    &  \textit{None}    & Name of attrib              \\
   &   \texttt{datatype}         &  String            &  IntArray, realArray, &  \textit{None} & Type of data in attrib \\
   &                             &                    &  posArray, stringArray &             &                        \\
   &   \texttt{size}$^o$         &  String            &  \textit{Any}    &  \textit{None}    & Size of data in attrib \\
  \hline
  \hline
\end{tabularx}
\end{center}
\end{table}

\subsection{Detailed attribute description}

\subsubsection{Required particleset attributes}

\begin{itemize}
\item \ixml{name}/\ixml{id} \\
Unique name for the particle set. Default is ``e" for electrons. ``i" or ``ion0" is typically used for ions. 
\end{itemize}
% Line 192 in ParticleIO/XMLParticleIO.cpp
% Lines 144-145 in QMCApp/ParticleSetPool.cpp

\subsubsection{Optional particleset attributes}

\begin{itemize}
\item \ixml{size} \\
Number of particles in set.
\end{itemize}
% Line 191 in ParticleIO/XMLParticleIO.cpp

%\begin{itemize}
%\item \ixml{role} \\
%What the particles do in the simulation
%\end{itemize}
% Line 146 in QMCApp/ParticleSetPool.cpp

\begin{itemize}
\item \ixml{random} \\
Randomize starting positions of particles. Each component of each particle's position is randomized independently in the range of the simulation cell in that component's direction. 
\end{itemize}
% Line 190 in ParticleIO/XMLParticleIO.cpp
% Line 147 in QMCApp/ParticleSetPool.cpp

\begin{itemize}
\item \ixml{randomsrc}/\ixml{random_source} \\
Specify source particle set around which to randomize the initial positions of this particle set.
\end{itemize}
% Lines 148-149 in QMCApp/ParticleSetPool.cpp

\subsubsection{Required name attributes}

\begin{itemize}
\item \ixml{name}/\ixml{id} \\
Unique name for the particle set group. Typically, element symbols are used for ions and ``u" or ``d" for spin-up and spin-down electron groups, respectively. 
\end{itemize}
% Line 192 in ParticleIO/XMLParticleIO.cpp
% Lines 144-145 in QMCApp/ParticleSetPool.cpp

\subsubsection{Optional group attributes}

\begin{itemize}
\item \ixml{mass} \\
Mass of particles in set.
\end{itemize}
% Line 190 in Particle/ParticleSet.cpp

\begin{itemize}
\item \ixml{unit} \\
Units for mass of particles in set (au[$m_e$ = 1] or amu[$\frac{1}{12}m_{\rm ^{12}C}$ = 1]).
\end{itemize}
% Line 66 in ParticleIO/XMLParticleIO.cpp


%condition appears to be future functionality for different unit types on the position array
%condition must be an integer
% Line 407 in ParticleIO/XMLParticleIO.cpp (reads condition in)
% Line 402 in ParticleIO/XMLParticleIO.cpp (declares utype integer)

\subsection{Example use cases}
\begin{minipage}{\linewidth}
\begin{lstlisting}[style=QMCPXML,caption=Particleset elements for ions and electrons randomizing electron start positions.]
  <particleset name="i" size="2">
    <group name="Li">
      <parameter name="charge">3.000000</parameter>
      <parameter name="valence">3.000000</parameter>
      <parameter name="atomicnumber">3.000000</parameter>
    </group>
    <group name="H">
      <parameter name="charge">1.000000</parameter>
      <parameter name="valence">1.000000</parameter>
      <parameter name="atomicnumber">1.000000</parameter>
    </group>
    <attrib name="position" datatype="posArray" condition="1">
    0.0   0.0   0.0
    0.5   0.5   0.5
    </attrib>
    <attrib name="ionid" datatype="stringArray">
       Li H
    </attrib>
  </particleset>
  <particleset name="e" random="yes" randomsrc="i">
    <group name="u" size="2">
      <parameter name="charge">-1</parameter>
    </group>
    <group name="d" size="2">
      <parameter name="charge">-1</parameter>
    </group>
  </particleset>                 
\end{lstlisting}
\end{minipage}

\begin{minipage}{\linewidth}
\begin{lstlisting}[style=QMCPXML,caption=Particleset elements for ions and electrons specifying electron start positions.]
  <particleset name="e">
    <group name="u" size="4">
      <parameter name="charge">-1</parameter>
      <attrib name="position" datatype="posArray">
	2.9151687332e-01 -6.5123272502e-01 -1.2188463918e-01
	5.8423636048e-01  4.2730406357e-01 -4.5964306231e-03
	3.5228575807e-01 -3.5027014639e-01  5.2644808295e-01
       -5.1686250912e-01 -1.6648002292e+00  6.5837023441e-01
      </attrib>
    </group>
    <group name="d" size="4">
      <parameter name="charge">-1</parameter>
      <attrib name="position" datatype="posArray">
	3.1443445436e-01  6.5068682609e-01 -4.0983449009e-02
       -3.8686061749e-01 -9.3744432997e-02 -6.0456005388e-01
	2.4978241724e-02 -3.2862514649e-02 -7.2266047173e-01
       -4.0352404772e-01  1.1927734805e+00  5.5610824921e-01
      </attrib>
    </group>
  </particleset>
  <particleset name="ion0" size="3">
    <group name="O">
      <parameter name="charge">6</parameter>
      <parameter name="valence">4</parameter>
      <parameter name="atomicnumber">8</parameter>
    </group>
    <group name="H">
      <parameter name="charge">1</parameter>
      <parameter name="valence">1</parameter>
      <parameter name="atomicnumber">1</parameter>
    </group>
    <attrib name="position" datatype="posArray">
      0.0000000000e+00  0.0000000000e+00  0.0000000000e+00
      0.0000000000e+00 -1.4308249289e+00  1.1078707576e+00
      0.0000000000e+00  1.4308249289e+00  1.1078707576e+00
    </attrib>
    <attrib name="ionid" datatype="stringArray">
      O H H 
    </attrib>
  </particleset>
\end{lstlisting}
\end{minipage}

\begin{minipage}{\linewidth}
\begin{lstlisting}[style=QMCPXML,caption=Particleset elements for ions specifying positions by ion type.]
  <particleset name="ion0">
    <group name="O" size="1">
      <parameter name="charge">6</parameter>
      <parameter name="valence">4</parameter>
      <parameter name="atomicnumber">8</parameter>
      <attrib name="position" datatype="posArray">
        0.0000000000e+00  0.0000000000e+00  0.0000000000e+00
      </attrib>
    </group>
    <group name="H" size="2">
      <parameter name="charge">1</parameter>
      <parameter name="valence">1</parameter>
      <parameter name="atomicnumber">1</parameter>
      <attrib name="position" datatype="posArray">
        0.0000000000e+00 -1.4308249289e+00  1.1078707576e+00
        0.0000000000e+00  1.4308249289e+00  1.1078707576e+00
      </attrib>
    </group>
  </particleset>
\end{lstlisting}
\end{minipage}


\chapter{Trial wavefunction specification}
\section{Introduction}
\label{sec:intro_wavefunction}

This section describes the input blocks associated with the specification of the trial wavefunction in a QMCPACK calculation. These sections are contained within the \ixml{<wavefunction> ...  </wavefunction>} xml blocks. \textbf{Users are expected to rely on converters to generate the input blocks described in this section.} The converters and the workflows are designed such that input blocks require minimum modifications from users. Unless the workflow requires modification of wavefunction blocks (e.g., setting the cutoff in a multideterminant calculation), only expert users should directly alter them.
  
The trial wavefunction in QMCPACK has a general product form:
\begin{equation}
\Psi_T(\vec{r}) = \prod_k \Theta_k(\vec{r}) ,
\end{equation}
where each $\Theta_k(\vec{r})$ is a function of the electron coordinates (and possibly ionic coordinates and variational parameters). For problems involving electrons, the overall trial wavefunction must be antisymmetric with respect to electron exchange, so at least one of the functions in the product must be antisymmetric. Notice that, although QMCPACK allows for the construction of arbitrary trial wavefunctions based on the functions implemented in the code (e.g., slater determinants, jastrow functions), the user must make sure that a correct wavefunction is used for the problem at hand. From here on, we assume a standard trial wavefunction for an electronic structure problem 
\begin{equation}
\Psi_T(\vec{r}) =  \textit{A}(\vec{r}) \prod_k \textit{J}_k(\vec{r}),
\end{equation}
where $\textit{A}(\vec{r})$ is one of the antisymmetric functions: (1) slater determinant, (2) multislater determinant, or (3) pfaffian and $\textit{J}_k$ is any of the Jastrow functions (described in Section~\ref{sec:jastrow}).  The antisymmetric functions are built from a set of single particle orbitals (\texttt{sposet}). QMCPACK implements four different types of \texttt{sposet}, described in the following section. Each \texttt{sposet} is designed for a different type of calculation, so their definition and generation varies accordingly. 

\section{Single determinant wavefunctions}
\label{sec:singledeterminant}
Placing a single determinant for each spin is the most used ansatz for the antisymmetric part of a trial wavefunction.
The input xml block for \texttt{slaterdeterminant} is give in Listing~\ref{listing:singledet}. A list of options is given in
Table~\ref{table:singledet}.

\begin{table}[h]
\begin{center}
\begin{tabularx}{\textwidth}{l l l l l X }
\hline
\multicolumn{6}{l}{\texttt{slaterdeterminant} element} \\
\hline
\multicolumn{2}{l}{Parent elements:} & \multicolumn{4}{l}{\texttt{determinantset}}\\
\multicolumn{2}{l}{Child  elements:} & \multicolumn{4}{l}{\texttt{determinant}}\\
\multicolumn{2}{l}{Attribute:} & \multicolumn{4}{l}{}\\
   &   \bfseries Name       & \bfseries Datatype & \bfseries Values & \bfseries Default & \bfseries Description \\
   &   \texttt{delay\_rank} &  Integer           &  >0              & 1           &  Number of delayed updates. \\
   &   \texttt{optimize}    &  Text              &  Yes/no          & Yes         &  Enable orbital optimization. \\
  \hline
\end{tabularx}
\end{center}
\caption{Options for the \texttt{slaterdeterminant} xml-block.}
\label{table:singledet}
\end{table}

\begin{lstlisting}[style=QMCPXML,caption=Slaterdeterminant set XML element.\label{listing:singledet}]
  <slaterdeterminant delay_rank="32">
    <determinant id="updet" size="208">
      <occupation mode="ground" spindataset="0">
      </occupation>
    </determinant>
    <determinant id="downdet" size="208">
      <occupation mode="ground" spindataset="0">
      </occupation>
    </determinant>
  </slaterdeterminant>
\end{lstlisting}

Additional information:
\begin{itemize}
\item \ixml{delay\_rank}. This option enables delayed updates of the Slater matrix inverse when particle-by-particle move is used.
By default, \ixml{delay\_rank=1} uses the Fahy's variant~\cite{Fahy1990} of the Sherman-Morrison rank-1 update, which is mostly using memory bandwidth-bound BLAS-2 calls.
With \ixml{delay\_rank>1}, the delayed update algorithm~\cite{Luo2018delayedupdate,McDaniel2017} turns most of the computation to compute bound BLAS-3 calls.
Tuning this parameter is highly recommended to gain the best performance on medium-to-large problem sizes ($>200$ electrons).
We have seen up to an order of magnitude speedup on large problem sizes.
When studying the performance of QMCPACK, a scan of this parameter is required and we recommend starting from 32.
The best \ixml{delay\_rank} giving the maximal speedup depends on the problem size.
Usually the larger \ixml{delay\_rank} corresponds to a larger problem size.
On CPUs, \ixml{delay\_rank} must be chosen as a multiple of SIMD vector length for good performance of BLAS libraries.
The best \ixml{delay\_rank} depends on the processor microarchitecture.
GPU support is under development.
\end{itemize}


\input{spo}
\section{Jastrow Factors}
\label{sec:jastrow}

Jastrow factors are among the simplest and most effective ways of including
dynamical correlation in the trial many body wavefunction.  The resulting many body
wavefunction is expressed as the product of an antisymmetric (in the case
of Fermions) or symmetric (for Bosons) part and a correlating Jastrow factor
like so:
\begin{equation}
\Psi(\vec{R}) = \mathcal{A}(\vec{R}) \exp\left[J(\vec{R})\right]
\end{equation}

In this section we will detail the types and forms of Jastrow factor used 
in QMCPACK.  Note that each type of Jastrow factor needs to be specified using
its own individual \texttt{jastrow} XML element.  For this reason, we have repeated the
specification of the \texttt{jastrow} tag in each section, with specialization for the
options available for that given type of Jastrow.

\subsection{One-body Jastrow functions}
\label{sec:onebodyjastrow}
The one-body Jastrow factor is a form that allows for the direct inclusion
of correlations between particles that are included in the wavefunction with
particles that are not explicitly part of it.  The most common example of
this are correlations between electrons and ions.  

The Jastrow function is specified within a \texttt{wavefunction} element
and must contain one or more \texttt{correlation} elements specifying
additional parameters as well as the actual coefficients. Section
\ref{sec:1bjsplineexamples} gives examples of the typical nesting of
\texttt{jastrow}, \texttt{correlation}, and \texttt{coefficient} elements.

\subsubsection{Input Specification}

\begin{table}[h]
\begin{center}
\begin{tabular}{l c c c l }
\hline
\multicolumn{5}{l}{Jastrow element} \\
\hline
\bfseries name & \bfseries datatype & \bfseries values & \bfseries defaults  & \bfseries description \\
\hline
name & text &    & (required) & Unique name for this Jastrow function \\
type & text & One-body & (required) & Define a one-body function \\ 
function & text & Bspline & (required) & BSpline Jastrow \\
             & text & pade2 & & Pade form \\
             & text & \ldots & & \ldots \\
source & text & name & (required) & Name of attribute of classical particle set \\ 
print & text & yes / no & yes & Jastrow factor printed in external file?\\
  \hline
\multicolumn{5}{l}{elements}\\ \hline
& Correlation & & & \\ \hline
\multicolumn{5}{l}{Contents}\\ \hline
& (None)  & & &  \\ \hline
\end{tabular}
%\end{tabular*}
\end{center}
\end{table}

To be more concrete, the one-body Jastrow factors used to describe correlations
between electrons and ions take the form below
\begin{equation}
J1=\sum_I^{ion0}\sum_i^e u_{ab}(|r_i-R_I|)
\end{equation}
where I runs over all of the ions in the calculation, i runs over the electrons
and $u_{ab}$ describes the functional form of the correlation between them.
Many different forms of $u_{ab}$ are implemented in QMCPACK.  We will detail 
two of the most common ones below.
\subsubsection{Spline form}
\label{sec:onebodyjastrowspline}

The one-body spline Jastrow function is the most commonly used one-body Jastrow for solids. This form 
was first described and used in \cite{EslerKimCeperleyShulenburger2012}.  
Here $u_{ab}$ is an interpolating 1D B-spline (tricublc spline on a linear grid) between zero distance and $r_{cut}$. In 3D periodic systems 
the default cutoff distance is the Wigner Seitz cell radius. For other periodicities, including isolated 
molecules, the $r_{cut}$ must be specified. The cusp can be set.   $r_i$ 
and $R_I$ are most commonly the electron and ion positions, but any particlesets that can provide the 
needed centers can be used.

\paragraph{Input specification}
\begin{table}[h]
\begin{center}
\begin{tabular}{l c c c l }
\hline
\multicolumn{5}{l}{Correlation element} \\
\hline
\bfseries Name & \bfseries Datatype & \bfseries Values & \bfseries Defaults & \bfseries Description \\
\hline
ElementType & Text & Name & See below & Classical particle target  \\
SpeciesA & Text & Name & See below & Classical particle target \\
SpeciesB & Text & Name & See below & Quantum species target \\
Size & Integer & $> 0$ & (Required) & Number of coefficients \\
Rcut & Real & $> 0$ & See below & Distance at which the correlation goes to 0 \\
Cusp & Real & $\ge 0$ & 0 & Value for use in Kato cusp condition \\
Spin & Text & Yes or no & No & Spin-dependent Jastrow factor \\
\hline
\multicolumn{5}{l}{Elements}\\ \hline
& Coefficients & & & \\ \hline
\multicolumn{5}{l}{Contents}\\ \hline
& (None)  & & &  \\ \hline
\end{tabular}
%\end{tabular*}
\end{center}
\end{table}

Additional information:

 \begin{itemize}
 \item \ixml{elementType, speciesA, speciesB, spin}.  For a spin-independent Jastrow factor (spin = ``no''),
elementType should be the name of the group of ions in the classical particleset to which the quantum
particles should be correlated.  For a spin-dependent Jastrow factor (spin = ``yes''), set speciesA to the
group name in the classical particleset and speciesB to the group name in the quantum particleset.
 \item \ixml{rcut}. The cutoff distance for the function in atomic units (bohr). 
For 3D fully periodic systems, this parameter is optional, and a default of the Wigner 
Seitz cell radius is used. Otherwise this parameter is required.
 \item \ixml{cusp}. The one-body Jastrow factor can be used to make the wavefunction
satisfy the electron-ion cusp condition\cite{kato}.  In this case, the derivative of the Jastrow
factor as the electron approaches the nucleus will be given by
\begin{equation}
\left(\frac{\partial J}{\partial r_{iI}}\right)_{r_{iI} = 0} = -Z .
\end{equation}
Note that if the antisymmetric part of the wavefunction satisfies the electron-ion cusp
condition (for instance by using single-particle orbitals that respect the cusp condition)
or if a nondivergent pseudopotential is used, the Jastrow should be cuspless at the 
nucleus and this value should be kept at its default of 0.
 \end{itemize}


\begin{table}[h]
\begin{center}
\begin{tabular}{l c c c l }
\hline
\multicolumn{5}{l}{Coefficients element} \\
\hline
\bfseries Name & \bfseries Datatype & \bfseries Values & \bfseries Defaults & \bfseries Description \\
\hline
Id & Text & & (Required) & Unique identifier \\
Type & Text & Array & (Required) & \\
Optimize & Text & Yes or no & Yes & if no, values are fixed in optimizations \\
\hline
\multicolumn{5}{l}{Elements}\\ \hline
(None) & & & \\ \hline
\multicolumn{5}{l}{Contents}\\ \hline
 (No name) & Real array & & Zeros & Jastrow coefficients \\ \hline
\end{tabular}
%\end{tabular*}
\end{center}
\end{table}


\paragraph{Example use cases}
\label{sec:1bjsplineexamples}

Specify a spin-independent function with four parameters. Because rcut  is not 
specified, the default cutoff of the Wigner Seitz cell radius is used; this 
Jastrow must be used with a 3D periodic system such as a bulk solid. The name of 
the particleset holding the ionic positions is ``i."
\begin{lstlisting}[style=QMCPXML]
<jastrow name="J1" type="One-Body" function="Bspline" print="yes" source="i">
 <correlation elementType="C" cusp="0.0" size="4">
   <coefficients id="C" type="Array"> 0  0  0  0  </coefficients>
 </correlation>
</jastrow>
\end{lstlisting}

Specify a spin-dependent function with seven up-spin and seven down-spin parameters. 
The cutoff distance is set to 6 atomic units.  Note here that the particleset holding
the ions is labeled as ion0 rather than ``i,'' as in the other example.  Also in this case,
the ion is lithium with a coulomb potential, so the cusp condition is satisfied by 
setting cusp=``d."
\begin{lstlisting}[style=QMCPXML]
<jastrow name="J1" type="One-Body" function="Bspline" source="ion0" spin="yes">
  <correlation speciesA="Li" speciesB="u" size="7" rcut="6">
    <coefficients id="eLiu" cusp="3.0" type="Array"> 
    0.0 0.0 0.0 0.0 0.0 0.0 0.0
    </coefficients>
  </correlation>
  <correlation speciesA="C" speciesB="d" size="7" rcut="6">
    <coefficients id="eLid" cusp="3.0" type="Array"> 
    0.0 0.0 0.0 0.0 0.0 0.0 0.0
    </coefficients>
  </correlation>
</jastrow>
\end{lstlisting}


\subsubsection{Pade form}
\label{sec:onebodyjastrowpade}

Although the spline Jastrow factor is the most flexible and most commonly used form implemented in \qmcpack, 
there are times where its flexibility can make it difficult to optimize.  As an example, a spline Jastrow
with a very large cutoff can be difficult to optimize for isolated systems such as molecules because of the small
number of samples present in the tail of the function.  In such cases, a simpler functional
form might be advantageous.  The second-order Pade Jastrow factor, given in Equation~\ref{padeeqn}, is a good choice 
in such cases.  
\begin{equation}
\label{padeeqn}
u_{ab}(r) = \frac{a*r+c*r^2}{1+b*r} .
\end{equation}
Unlike the spline Jastrow factor, which includes a cutoff, this form has an infinite range and will be applied to every particle
pair (subject to the minimum image convention).  It also is a cuspless Jastrow factor,
so it should be used either in combination with a single particle basis set that contains the proper cusp or
with a smooth pseudopotential.

\paragraph{Input specification}
\begin{table}[h]
\begin{center}
\begin{tabular}{l c c c l }
\hline
\multicolumn{5}{l}{Correlation element} \\
\hline
\bfseries Name & \bfseries Datatype & \bfseries Values & \bfseries Defaults & \bfseries Description \\
\hline
ElementType & Text & Name & See below & Classical particle target  \\
\hline
\multicolumn{5}{l}{Elements}\\ \hline
& Coefficients & & & \\ \hline
\multicolumn{5}{l}{Contents}\\ \hline
& (None)  & & &  \\ \hline
\end{tabular}
%\end{tabular*}
\end{center}
\end{table}

\begin{table}[h]
\begin{center}
\begin{tabular}{l c c c l }
\hline
\multicolumn{5}{l}{Parameter element} \\
\hline
\bfseries Name & \bfseries Datatype & \bfseries Values & \bfseries Defaults & \bfseries Description \\
\hline
Id & String & Name & (Required) & Name for variable \\
Name & String & A or B or C & (Required) & See Equation~\ref{padeeqn}\\
Optimize & Text & Yes or no & Yes & If no, values are fixed in optimizations \\
\hline
\multicolumn{5}{l}{Elements}\\ \hline
(None) & & & \\ \hline
\multicolumn{5}{l}{Contents}\\ \hline
 (No name) & Real & Parameter value & (Required) & Jastrow coefficients \\ \hline
\end{tabular}
%\end{tabular*}
\end{center}
\end{table}

\paragraph{Example use case}
\label{sec:1bjpadeexamples}

Specify a spin-independent function with independent Jastrow factors for two different species (Li and H).
The name of the particleset holding the ionic positions is ``i."
\begin{lstlisting}[style=QMCPXML]
<jastrow name="J1" function="pade2" type="One-Body" print="yes" source="i">
  <correlation elementType="Li">
    <var id="LiA" name="A">  0.34 </var>
    <var id="LiB" name="B"> 12.78 </var>
    <var id="LiC" name="C">  1.62 </var>
  </correlation>
  <correlation elementType="H"">
    <var id="HA" name="A">  0.14 </var>
    <var id="HB" name="B"> 6.88 </var>
    <var id="HC" name="C"> 0.237 </var>
  </correlation>
</jastrow>
\end{lstlisting}

\subsubsection{Short Range Cusp Form}
\label{sec:onebodyjastrowsrcusp}

The idea behind this functor is to encode nuclear cusps and other details at very
short range around a nucleus in the region that the Gaussian orbitals of quantum
chemistry are not capable of describing correctly.
The functor is kept short ranged, because outside this small region, quantum chemistry
orbital expansions are already capable of taking on the correct shapes.
Unlike a pre-computed cusp correction, this optimizable functor can respond to
changes in the wave function during VMC optimization.
The functor's form is
\begin{equation}
\label{srcuspform}
u(r) = -\exp{\left(-r/R_0\right)} \left( A R_0 + \sum_{k=0}^{N-1} B_k \frac{ (r/R_0)^{k+2} }{ 1 + (r/R_0)^{k+2} } \right)
\end{equation}
in which $R_0$ acts as a soft cutoff radius ($u(r)$ decays to zero quickly beyond roughly this distance)
and $A$ determines the cusp condition.
\begin{equation}
\label{srcusplimit}
\lim_{r \to 0} \frac{\partial u}{\partial r} = A
\end{equation}
The simple exponential decay is modified by the $N$ coefficients $B_k$ that define
an expansion in sigmoidal functions, thus adding detailed structure in a short-ranged
region around a nucleus while maintaining the correct cusp condition at the nucleus.
Note that sigmoidal functions are used instead of, say, a bare polynomial expansion, as they
trend to unity past the soft cutoff radius and so interfere less with the exponential decay
that keeps the functor short ranged.
Although $A$, $R_0$, and the $B_k$ coefficients can all be optimized as variational
parameters, $A$ will typically be fixed as the desired cusp condition is known.

To specify this one-body Jastrow factor, use an input section like the following.

\begin{lstlisting}[style=QMCPXML]
<jastrow name="J1Cusps" type="One-Body" function="shortrangecusp" source="ion0" print="yes">
  <correlation rcut="6" cusp="3" elementType="Li">
    <var id="LiCuspR0" name="R0" optimize="yes"> 0.06 </var>
    <coefficients id="LiCuspB" type="Array" optimize="yes">
      0 0 0 0 0 0 0 0 0 0
    </coefficients>
  </correlation>
  <correlation rcut="6" cusp="1" elementType="H">
    <var id="HCuspR0" name="R0" optimize="yes"> 0.2 </var>
    <coefficients id="HCuspB" type="Array" optimize="yes">
      0 0 0 0 0 0 0 0 0 0
    </coefficients>
  </correlation>
</jastrow>
\end{lstlisting}

Here ``rcut'' is specified as the range beyond which the functor is assumed to be zero.
The value of $A$ can either be specified via the ``cusp'' option as shown above, in
which case its optimization is disabled, or through its own ``var'' line as
for $R_0$, in which case it can be specified as either optimizable (``yes'')
or not (``no'').
The coefficients $B_k$ are specified via the ``coefficients'' section,
with the length $N$ of the expansion determined automatically based on the length
of the array.

Note that this one-body Jastrow form can (and probably should) be used in conjunction
with a longer ranged one-body Jastrow, such as a spline form.
Be sure to set the longer-ranged Jastrow to be cusp-free!


\subsection{Two-body Jastrow functions}
The two-body Jastrow factor is a form that allows for the explicit inclusion
of dynamic correlation between two particles included in the wavefunction.  It
is almost always given in a spin dependent form so as to satisfy the Kato cusp
condition between electrons of different spins\cite{kato}.

 The two body Jastrow function is specified within a \texttt{wavefunction} element
and must contain one or more correlation elements specifying additional parameters
as well as the actual coefficients.  Section~\ref{sec:2bjsplineexamples} gives 
examples of the typical nesting of \texttt{jastrow}, \texttt{correlation} and
\texttt{coefficient} elements.

\subsubsection{Input Specification}

\begin{table}[h]
\begin{center}
\begin{tabular}{l c c c l }
\hline
\multicolumn{5}{l}{Jastrow element} \\
\hline
\bfseries name & \bfseries datatype & \bfseries values & \bfseries defaults  & \bfseries description \\
\hline
name & text &    & (required) & Unique name for this Jastrow function \\
type & text & Two-body & (required) & Define a one-body function \\ 
function & text & Bspline & (required) & BSpline Jastrow \\
print & text & yes / no & yes & Jastrow factor printed in external file?\\
  \hline
\multicolumn{5}{l}{elements}\\ \hline
& Correlation & & & \\ \hline
\multicolumn{5}{l}{Contents}\\ \hline
& (None)  & & &  \\ \hline
\end{tabular}
%\end{tabular*}
\end{center}
\end{table}

The two-body Jastrow factors used to describe correlations between electrons take the form
\begin{equation}
J2=\sum_i^{e}\sum_{j>i}^{e} u_{ab}(|r_i-r_j|)
\end{equation}

The most commonly used form of two body Jastrow factor supported by the code is a splined
Jastrow factor, with many similarities to the one body spline Jastrow.

\subsubsection{Spline form}
\label{sec:twobodyjastrowspline}

The two-body spline Jastrow function is the most commonly used two-body Jastrow for solids. This form 
was first described and used in \cite{EslerKimCeperleyShulenburger2012}.  
Here $u_{ab}$ is an interpolating 1D B-spline (tricublc spline on a linear grid) between 
zero distance and $r_{cut}$. In 3D periodic systems, 
the default cutoff distance is the Wigner Seitz cell radius. For other periodicities, including isolated 
molecules, the $r_{cut}$ must be specified.  $r_i$ and $r_j$ are typically electron positions.  The cusp 
condition as $r_i$ approaches $r_j$ is set by the relative spin of the electrons.

\FloatBarrier
\paragraph{Input specification}

\FloatBarrier
\begin{table}[h!]
\begin{center}
\begin{tabular}{l c c c l }
\hline
\multicolumn{5}{l}{Correlation element} \\
\hline
\bfseries Name & \bfseries Datatype & \bfseries Values & \bfseries Defaults & \bfseries Description \\
\hline
SpeciesA & Text & U or d & (Required) & Quantum species target \\
SpeciesB & Text & U or d & (Required) & Quantum species target \\
Size & Integer & $> 0$ & (Required) & Number of coefficients \\
Rcut & Real & $> 0$ & See below & Distance at which the correlation goes to 0 \\
Spin & Text & Yes or no & No & Spin-dependent Jastrow factor \\
\hline
\multicolumn{5}{l}{Elements}\\ \hline
& Coefficients & & & \\ \hline
\multicolumn{5}{l}{Contents}\\ \hline
& (None)  & & &  \\ \hline
\end{tabular}
%\end{tabular*}
\end{center}
\end{table}

\FloatBarrier

Additional information:
\begin{itemize}
\item \ishell{speciesA, speciesB} The scale function u(r) is defined for species pairs uu and ud.  
There is no need to define ud or dd since uu=dd and ud=du.  The cusp condition is computed internally 
based on the charge of the quantum particles.
\end{itemize}

\begin{table}[h]
\begin{center}
\begin{tabular}{l c c c l }
\hline
\multicolumn{5}{l}{Coefficients element} \\
\hline
\bfseries Name & \bfseries Datatype & \bfseries Values & \bfseries Defaults & \bfseries Description \\
\hline
Id & Text & & (Required) & Unique identifier \\
Type & Text & Array & (Required) & \\
Optimize & Text & Yes or no & Yes & If no, values are fixed in optimizations \\
\hline
\multicolumn{5}{l}{Elements}\\ \hline
(None) & & & \\ \hline
\multicolumn{5}{l}{Contents}\\ \hline
 (No name) & Real array & & Zeros & Jastrow coefficients \\ \hline
\end{tabular}
%\end{tabular*}
\end{center}
\end{table}

\paragraph{Example use cases}
\label{sec:2bjsplineexamples}

Specify a spin-dependent function with four parameters for each channel.  In this case, the cusp is set at 
a radius of 4.0 bohr (rather than to the default of the Wigner Seitz cell radius).  Also, in this example,
the coefficients are set to not be optimized during an optimization step.

\begin{lstlisting}[style=QMCPXML]
<jastrow name="J2" type="Two-Body" function="Bspline" print="yes">
  <correlation speciesA="u" speciesB="u" size="8" rcut="4.0">
    <coefficients id="uu" type="Array" optimize="no"> 0.2309049836 0.1312646071 0.05464141356 0.01306231516</coefficients>
  </correlation>
  <correlation speciesA="u" speciesB="d" size="8" rcut="4.0">
    <coefficients id="ud" type="Array" optimize="no"> 0.4351561096 0.2377951747 0.1129144262 0.0356789236</coefficients>
  </correlation>
</jastrow>
\end{lstlisting}


\subsection{User defined functional form}
\label{sec:jastrowuserform}

To aid in implementing different forms for $u_{ab}(r)$, there is a script that uses a symbolic expression to generate the appropriate code (with spatial and parameter derivatives).
The script is located in \texttt{src/QMCWaveFunctions/Jastrow/codegen/user\_jastrow.py}.
The script requires Sympy (\url{www.sympy.org}) for symbolic mathematics and code generation.

To use the script, modify it to specify the functional form and a list of variational parameters.
Optionally, there may be fixed parameters - ones that are specified in the input file, but are not part of the variational optimization.
Also one symbol may be specified that accepts a cusp value in order to satisfy the cusp condition.
There are several example forms in the script.  The default form is the simple Pad\'e.

Once the functional form and parameters are specified in the script, run the script from the \texttt{codegen} directory and recompile QMCPACK.
The main output of the script is the file \texttt{src/QMCWaveFunctions/Jastrow/UserFunctor.h}.
The script also prints information to the screen, and one section is a sample XML input block containing all the parameters.

There is a unit test in \texttt{src/QMCWaveFunctions/test/test\_user\_jastrow.cpp} to perform
some minimal testing of the Jastrow factor.   The unit test will need updating to properly test
new functional forms.   Most of the changes relate to the number and name of variational 
parameters.


\begin{table}[h]
\begin{center}
\begin{tabular}{l c c c l }
\hline
\multicolumn{5}{l}{Jastrow element} \\
\hline
\bfseries name & \bfseries datatype & \bfseries values & \bfseries defaults  & \bfseries description \\
\hline
name & text &    & (required) & Unique name for this Jastrow function \\
type & text & One-body & (required) & Define a one-body function \\
     &      & Two-body & (required) & Define a two-body function \\
function & text & user & (required) & User-defined functor \\
\multicolumn{5}{l}{See other parameters as approriate for one or two-body functions} \\
  \hline
\multicolumn{5}{l}{elements}\\ \hline
& Correlation & & & \\ \hline
\multicolumn{5}{l}{Contents}\\ \hline
& (None)  & & &  \\ \hline
\end{tabular}
%\end{tabular*}
\end{center}
\end{table}



\subsection{Long-ranged Jastrow factors}
While short-ranged Jastrow factors capture the majority of the benefit 
for minimizing the total energy and the energy variance, long-ranged 
Jastrow factors are important to accurately reproduce the short-ranged 
(long wavelength) behavior of quantities such as the static structure 
factor, and are therefore essential for modern accurate finite size 
corrections in periodic systems.

Below two types of long-ranged Jastrow factors are described.  The 
first (the k-space Jastrow) is simply an expansion of the one and/or 
two body correlation functions in plane waves, with the coefficients 
comprising the optimizable parameters.  The second type have few 
variational parameters and use the optimized breakup method of Natoli 
and Ceperley\cite{Natoli1995} (the Yukawa and Gaskell RPA Jastrows).


\subsubsection{Long-ranged Jastrow: k-space Jastrow}
The k-space Jastrow introduces explicit long-ranged dependence commensurate with the periodic supercell.  This Jastrow is to be used in periodic boundary conditions only.  

The input for the k-space Jastrow fuses both one and two-body forms into a single element and so they are discussed together here.  The one- and two-body terms in the k-Space Jastrow have the form:
\begin{align}
  J_1 &= \sum_{G\ne 0}b_G\rho_G^I\rho_{-G} \\
  J_2 &= \sum_{G\ne 0}a_G\rho_G\rho_{-G}
\end{align}
Here $\rho_G$ is the Fourier transform of the instantaneous electron density:
\begin{align}
  \rho_G=\sum_{n\in electrons}e^{iG\cdot r_n}
\end{align}
and $\rho_G^I$ has the same form, but for the fixed ions. In both cases the coefficients are restricted to be real, though in general the coefficients for the one-body term need not be.  See section~\ref{sec:feature_kspace_jastrow} for more detail.

Input for the k-space Jastrow follows the familar nesting of \texttt{jastrow-correlation-coefficients} elements, with attributes unique to the k-space Jastrow at the \texttt{correlation} input level.

\FloatBarrier
\begin{table}[h]
\begin{center}
\begin{tabularx}{\textwidth}{l l l l l X }
\hline
\multicolumn{6}{l}{\texttt{jastrow type=kSpace} element} \\
\hline
\multicolumn{2}{l}{parent elements:} & \multicolumn{4}{l}{\texttt{wavefunction}}\\
\multicolumn{2}{l}{child  elements:} & \multicolumn{4}{l}{\texttt{correlation}}\\
\multicolumn{2}{l}{attributes}  & \multicolumn{4}{l}{}\\
   &   \bfseries name     & \bfseries datatype & \bfseries values          & \bfseries default  & \bfseries description \\
   & \texttt{type}$^r$    &  text              & \textbf{kSpace}           &                    & Must be kSpace           \\
   & \texttt{name}$^r$    &  text              & \textit{anything}         & 0                  & Unique name for Jastrow \\
   & \texttt{source}$^r$  &  text              & \texttt{particleset.name} &                    & Ion particleset name\\
  \hline
\end{tabularx}
\end{center}
\end{table}
\FloatBarrier

\FloatBarrier
\begin{table}[h]
\begin{center}
\begin{tabularx}{\textwidth}{l l l l l X }
\hline
\multicolumn{6}{l}{\texttt{correlation} element} \\
\hline
\multicolumn{2}{l}{parent elements:} & \multicolumn{4}{l}{\texttt{jastrow type=kSpace}}\\
\multicolumn{2}{l}{child  elements:} & \multicolumn{4}{l}{\texttt{coefficients}}\\
\multicolumn{2}{l}{attributes}  & \multicolumn{4}{l}{}\\
   &   \bfseries name           & \bfseries datatype & \bfseries values  & \bfseries default  & \bfseries description \\
   & \texttt{type}$^r$          &  text              & \textbf{One-Body},\textbf{Two-Body}    &                     & Must be One-Body/Two-Body     \\
   & \texttt{kc}$^r$            &  real              & kc$\ge$ 0                                & 0.0                 & k-space cutoff in a.u. \\
   & \texttt{symmetry}$^o$      &  text              & crystal,isotropic,none                 & crystal             & Symmetry of coefficients\\
   & \texttt{spinDependent}$^o$ &  boolean           & yes,no                                 & no                  & \textit{No current function} \\
  \hline
\end{tabularx}
\end{center}
\end{table}
\FloatBarrier

\FloatBarrier
\begin{table}[h]
\begin{center}
\begin{tabularx}{\textwidth}{l l l l l X }
\hline
\multicolumn{6}{l}{\texttt{coefficients} element} \\
\hline
\multicolumn{2}{l}{parent elements:} & \multicolumn{4}{l}{\texttt{correlation}}\\
\multicolumn{2}{l}{child  elements:} & \multicolumn{4}{l}{\textit{None}}\\
\multicolumn{2}{l}{attributes}  & \multicolumn{4}{l}{}\\
   &   \bfseries name     & \bfseries datatype & \bfseries values  & \bfseries default   & \bfseries description \\
   & \texttt{id}$^r$      &  text              & \textit{anything} &     cG1/cG2         & Label for coeffs     \\
   & \texttt{type}$^r$    &  text              & \texttt{Array}    &   0                 & Must be Array \\
\multicolumn{2}{l}{body text}  & \multicolumn{4}{l}{}\\
   &                           & \multicolumn{4}{l}{The body text is a list of real values for the parameters.}     \\
  \hline
\end{tabularx}
\end{center}
\end{table}
\FloatBarrier


Additional information:
\begin{itemize}
  \item{It is normal to provide no coefficients as an initial guess.  The number of coefficients will be automatically calculated according to the k-space cutoff + symmetry and set to zero. }
  \item{Providing an incorrect number of parameters also results in all parameters being set to zero.}
  \item{There is currently no way to turn optimization on/off for the k-space Jastrow.  The coefficients are always optimized.}
  \item{Spin dependence is currently not implemented for this Jastrow.}
  \item{\texttt{kc}: Parameters with G vectors magnitudes less than \texttt{kc} are included in the Jastrow.  If \texttt{kc} is zero, it is the same as excluding the k-space term.}
  \item{\texttt{symmetry=crystal}: Impose crystal symmetry on coefficients according to the structure factor.}
  \item{\texttt{symmetry=isotropic}: Impose spherical symmetry on coefficients according to G-vector magnitude.}
  \item{\texttt{symmetry=none}: Impose no symmetry on the coefficients.}
\end{itemize}


\begin{lstlisting}[style=QMCPXML,caption=k-space Jastrow with one- and two-body terms.]
  <jastrow type="kSpace" name="Jk" source="ion0">
     <correlation kc="4.0" type="One-Body" symmetry="cystal">
        <coefficients id="cG1" type="Array">                  
        </coefficients>
     </correlation>
     <correlation kc="4.0" type="Two-Body" symmetry="crystal">
        <coefficients id="cG2" type="Array">                  
        </coefficients>
     </correlation>
  </jastrow>
\end{lstlisting}

\begin{lstlisting}[style=QMCPXML,caption=k-space Jastrow with one-body term only.]
  <jastrow type="kSpace" name="Jk" source="ion0">
     <correlation kc="4.0" type="One-Body" symmetry="cystal">
        <coefficients id="cG1" type="Array">                  
        </coefficients>
     </correlation>
  </jastrow>
\end{lstlisting}

\begin{lstlisting}[style=QMCPXML,caption=k-space Jastrow with two-body term only.]
  <jastrow type="kSpace" name="Jk" source="ion0">
     <correlation kc="4.0" type="Two-Body" symmetry="crystal">
        <coefficients id="cG2" type="Array">                  
        </coefficients>
     </correlation>
  </jastrow>
\end{lstlisting}




\subsubsection{Long-ranged Jastrows: Gaskell RPA and Yukawa forms}
\label{sec:twobodyjastrowlr}
\textbf{NOTE: The Yukawa and RPA Jastrows do not work at present 
and are currently being revived.  Please contact the developers if 
you are interested in using them.} 

The exact Jastrow correlation functions contain terms which have a 
form similar to the Coulomb pair potential.  In periodic systems 
the Coulomb potential is replaced by an Ewald summation of the 
bare potential over all periodic image cells.  This sum is often 
handled by the optimized breakup method\cite{Natoli1995} and this 
same approach is applied to the long-ranged Jastrow factors in QMCPACK.

There are two main long-ranged Jastrow factors of this type 
implemented in QMCPACK: the Gaskell RPA\cite{Gaskell1961,Gaskell1962} 
form and the Yukawa\cite{Ceperley1978} form.  Both of these forms 
were used by Ceperley in early studies of the electron gas\cite{Ceperley1978}, 
but they are also appropriate starting points for general solids. 

The Yukawa form is defined in real space.  It's long-range form is 
formally defined as
\begin{align}
  u_Y^{PBC}(r) = \sum_{L\ne 0}\sum_{i<j}u_Y(\abs{r_i-r_j+L})
\end{align}
with $u_Y(r)$ given by
\begin{align}
  u_Y(r) = \frac{a}{r}\left(1-e^{-r/b}\right)
\end{align}
In QMCPACK a slightly more restricted form is used:
\begin{align}
  u_Y(r) = \frac{r_s}{r}\left(1-e^{-r/\sqrt{r_s}}\right)
\end{align}
here ``$r_s$'' is understood to be a variational parameter.

The Gaskell RPA form--which contains correct short/long range limits 
and minimizes the total energy of the electron gas within the RPA--is 
defined directly in k-space:
\begin{align}
  u_{RPA}(k) = -\frac{1}{2S_0(k)}+\frac{1}{2}\left(\frac{1}{S_0(k)^2}+\frac{4m_ev_k}{\hbar^2k^2}\right)^{1/2}
\end{align}
where $v_k$ is the Fourier transform of the Coulomb potential and 
$S_0(k)$ is the static structure factor of the non-interacting 
electron gas:
\[
  S_0(k) = \left.
  \begin{cases}
    1 &  k>2k_F \\
    \frac{3k}{4k_F}-\frac{1}{2}\left(\frac{k}{2k_F}\right)^3 & k<2k_F
  \end{cases}
  \right.
\]
When written in atomic units, RPA Jastrow implemented in QMCPACK has the 
form
\begin{align}
  u_{RPA}(k) = \frac{1}{2N_e}\left(-\frac{1}{S_0(k)}+\left(\frac{1}{S_0(k)^2}+\frac{12}{r_s^3k^4}\right)^{1/2}\right)
\end{align}
Here ``$r_s$'' is again a variational parameter and $k_F\equiv(\tfrac{9\pi}{4r_s^3})^{1/3}$.

For both the Yukawa and Gaskell RPA Jastrows, the default value for $r_s$ is $r_s=(\tfrac{3\Omega}{4\pi N_e})^{1/3}$.


\FloatBarrier
\begin{table}[h]
\begin{center}
\begin{tabularx}{\textwidth}{l l l l l X }
\hline
\multicolumn{6}{l}{\texttt{jastrow type=Two-Body function=rpa/yukawa} element} \\
\hline
\multicolumn{2}{l}{parent elements:} & \multicolumn{4}{l}{\texttt{wavefunction}}\\
\multicolumn{2}{l}{child  elements:} & \multicolumn{4}{l}{\texttt{correlation}}\\
\multicolumn{2}{l}{attributes}  & \multicolumn{4}{l}{}\\
   &   \bfseries name        & \bfseries datatype & \bfseries values             & \bfseries default  & \bfseries description   \\
   & \texttt{type}$^r$       &  text              & \textbf{Two-Body}            &                    & Must be Two-Body   \\
   & \texttt{function}$^r$   &  text              & \textbf{rpa}/\textbf{yukawa} &                    & Must be rpa or yukawa   \\
   & \texttt{name}$^r$       &  text              & \textit{anything}            & RPA\_Jee            & Unique name for Jastrow \\
   & \texttt{longrange}$^o$  &  boolean           & yes/no                       & yes                & Use long-range part     \\
   & \texttt{shortrange}$^o$ &  boolean           & yes/no                       & yes                & Use short-range part    \\
\multicolumn{2}{l}{parameters}  & \multicolumn{4}{l}{}\\
   & \texttt{rs}$^o$         &  rs                & $r_s>0$                      & $\tfrac{3\Omega}{4\pi N_e}$ & Avg. elec-elec distance \\
   & \texttt{kc}$^o$         &  kc                & $k_c>0$                      & $2\left(\tfrac{9\pi}{4}\right)^{1/3}\tfrac{4\pi N_e}{3\Omega}$ & K-space cutoff\\
  \hline
\end{tabularx}
\end{center}
\end{table}
\FloatBarrier


\begin{lstlisting}[style=QMCPXML,caption=Two body RPA Jastrow with long- and short-ranged parts.]
<jastrow name=''Jee'' type=''Two-Body'' function=''rpa''>
</jastrow>
\end{lstlisting}



% J1 RPA (intended for electron-proton system)
%   source = particleset.name
%   function = RPA
%   name = anything [Jep]
%   rs = >0 [-1]
%   kc = >0 [-1]
% 
% J2 RPA
%   attributes
%     function = yukawa or rpa
%     name = anything [RPA_Jee]
%     longrange = yes/no [yes]
%     shortrange = yes/no [yes]
%   parameters
%     rs = >0 [-1]  3\Omega/4\pi N_e
%     kc = >0 [-1]  2 (9\pi/4)^1/3 * 4\pi N_e/3\Omega



\subsection{Three-body Jastrow functions}
Explicit three-body correlations can be included in the wavefunction via the three-body Jastrow factor.
The three-body electron-electron-ion correlation function ($u_{\sigma\sigma'I}$) currently used in \qmcpack is identical to the one proposed in \cite{Drummond2004}:
\begin{eqnarray}
u_{\sigma\sigma'I}(r_{\sigma I},r_{\sigma'I},r_{\sigma\sigma'}) &= \sum_{\ell=0}^{M_{eI}}\sum_{m=0}^{M_{eI}}\sum_{n=0}^{M_{ee}}\gamma_{\ell mn} r_{\sigma I}^\ell r_{\sigma'I}^m r_{\sigma\sigma'}^n \\
   &\times \left(r_{\sigma I}-\frac{r_c}{2}\right)^3 \Theta\left(r_{\sigma I}-\frac{r_c}{2}\right) \nonumber \\
   &\times \left(r_{\sigma' I}-\frac{r_c}{2}\right)^3 \Theta\left(r_{\sigma' I}-\frac{r_c}{2}\right) \nonumber
\end{eqnarray}
Here $M_{eI}$ and $M_{ee}$ are the maximum polynomial orders of the
electron-ion and electron-electron distances, respectively,
$\{\gamma_{\ell mn}\}$ are the optimizable parameters (modulo
constraints), $r_c$ is a cutoff radius, and $r_{ab}$ are the distances
between electrons or ions $a$ and $b$. i.e. The correlation function
is only a function of the interparticle distances and not a more
complex function of the particle positions, $\mathbf{r}$. As indicated by the
$\Theta$ functions, correlations are set to zero beyond a distance of
$r_c/2$ in either of the electron-ion distances and the largest
meaningful electron-electron distance is $r_c$.  This is the
highest-order Jastrow correlation function currently implemented.


Today, solid state applications of \qmcpack usually utilize one and
two-body B-spline Jastrow functions, with calculations on heavier
elements often also using the three-body term described above.

\paragraph{Example use case}
Here is an example of H2O molecule. After optimizing one and two body Jastrow factors, add the following block in the wavefunction.
The coefficients will be filled zero automatically if not given.
\begin{lstlisting}[style=QMCPXML]
<jastrow name="J3" type="eeI" function="polynomial" source="ion0" print="yes">
  <correlation ispecies="O" especies="u" isize="3" esize="3" rcut="10">
    <coefficients id="uuO" type="Array" optimize="yes"> </coefficients>
  </correlation>
  <correlation ispecies="O" especies1="u" especies2="d" isize="3" esize="3" rcut="10">
    <coefficients id="udO" type="Array" optimize="yes"> </coefficients>
  </correlation>
  <correlation ispecies="H" especies="u" isize="3" esize="3" rcut="10">
    <coefficients id="uuH" type="Array" optimize="yes"> </coefficients>
  </correlation>
  <correlation ispecies="H" especies1="u" especies2="d" isize="3" esize="3" rcut="10">
    <coefficients id="udH" type="Array" optimize="yes"> </coefficients>
  </correlation>
</jastrow>
\end{lstlisting}

\section{Multideterminant wavefunctions}
\label{sec:multideterminants}
Multiple schemes to generate a multideterminant wavefunction are
possible, from CASSF to full CI or selected CI. The \qmcpack converter can
convert MCSCF multideterminant wavefunctions from
GAMESS\cite{schmidt93} and CIPSI\cite{Caffarel2013} wavefunctions from
Quantum Package\cite{QP} (QP). Full details of how to run a CIPSI
calculation and convert the wavefunction for QMCPACK are given in 
Section~\ref{sec:cipsi}.

The script \ishell{utils/determinants_tools.py} can be used to generate
useful information about the multideterminant wavefunction. This script takes, as a required argument, the path of an h5 file corresponding to the wavefunction. Used without optional arguments, it prints the number of determinants, the number of CSFs, and a histogram of the excitation degree.

\begin{lstlisting}[style=SHELL]
> determinants_tools.py ./tests/molecules/C2_pp/C2.h5
Summary:
excitation degree 0 count: 1
excitation degree 1 count: 6
excitation degree 2 count: 148
excitation degree 3 count: 27
excitation degree 4 count: 20

n_det 202
n_csf 104
\end{lstlisting}

If the \ishell{--verbose} argument is used, the script will print each determinant,
the associated CSF, and the excitation degree relative to the first determinant.
\begin{lstlisting}[style=SHELL]
> determinants_tools.py -v ./tests/molecules/C2_pp/C2.h5 | head
1
alpha  1111000000000000000000000000000000000000000000000000000000
beta   1111000000000000000000000000000000000000000000000000000000
scf    2222000000000000000000000000000000000000000000000000000000
excitation degree  0

2
alpha  1011100000000000000000000000000000000000000000000000000000
beta   1011100000000000000000000000000000000000000000000000000000
scf    2022200000000000000000000000000000000000000000000000000000
excitation degree  2
\end{lstlisting}

\section{Backflow wavefunctions}

\label{sec:backflow}

One can perturb the nodal surface of a single-Slater/multi-Slater wavefunction through use of a backflow transformation.  Specifically, if we have an antisymmetric function $D(\mathbf{x}_{0\uparrow},\cdots,\mathbf{x}_{N\uparrow}, \mathbf{x}_{0\downarrow},\cdots,\mathbf{x}_{N\downarrow})$, and if $i_\alpha$ is the $i$-th particle of species type $\alpha$, then the backflow transformation works by making the coordinate transformation $\mathbf{x}_{i_\alpha} \to \mathbf{x}'_{i_\alpha}$ and evaluating $D$ at these new ``quasiparticle" coordinates.  QMCPACK currently supports quasiparticle transformations given by

\begin{equation}\label{backflowdef}
\mathbf{x}'_{i_\alpha}=\mathbf{x}_{i_\alpha}+\sum_{\alpha \leq \beta} \sum_{i_\alpha \neq j_\beta} \eta^{\alpha\beta}(|\mathbf{x}_{i_\alpha}-\mathbf{x}_{j_\beta}|)(\mathbf{x}_{i_\alpha}-\mathbf{x}_{j_\beta})\:.
\end{equation}

Here, $\eta^{\alpha\beta}(|\mathbf{x}_{i_\alpha}-\mathbf{x}_{j_\beta}|)$ is a radially symmetric backflow transformation between species $\alpha$ and $\beta$.  In QMCPACK, particle $i_\alpha$ is known as the ``target" particle and $j_\beta$ is known as the ``source."  The main types of transformations are so-called one-body terms, which are between an electron and an ion $\eta^{eI}(|\mathbf{x}_{i_e}-\mathbf{x}_{j_I}|)$ and two-body terms.  Two-body terms are distinguished as those between like and opposite spin electrons:  $\eta^{e(\uparrow)e(\uparrow)}(|\mathbf{x}_{i_e(\uparrow)}-\mathbf{x}_{j_e(\uparrow)}|)$ and  $\eta^{e(\uparrow)e(\downarrow)}(|\mathbf{x}_{i_e(\uparrow)}-\mathbf{x}_{j_e(\downarrow)}|)$.  Henceforth, we will assume that $\eta^{e(\uparrow)e(\uparrow)}=\eta^{e(\downarrow)e(\downarrow)}$.

In the following, we explain how to describe general terms such as Eq. \ref{backflowdef} in a QMCPACK XML file.  For specificity, we will consider a particle set consisting of H and He (in that order).  This ordering will be important when we build the XML file, so you can find this out either through your specific declaration of <particleset>, by looking at the hdf5 file in the case of plane waves, or by looking at the QMCPACK output file in the section labeled ``Summary of QMC systems."  
\subsection{Input Specifications}
All backflow declarations occur within a single \ixml{<backflow> ... </backflow>} block.  Backflow transformations occur in \ixml{<transformation>} blocks and have the following input parameters:

\begin{table}[h]
\begin{center}
\begin{tabular}{l c c c l }
\hline
\multicolumn{5}{l}{Transformation element} \\
\hline
\bfseries Name & \bfseries Datatype & \bfseries Values & \bfseries Defaults  & \bfseries Description \\
\hline
name & Text &  & (Required) & Unique name for this Jastrow function. \\
type & Text & ``e-I" & (Required) & Define a one-body backflow transformation. \\ 
        &          & ``e-e" & & Define a two-body backflow transformation. \\
function & Text & B-spline & (Required) & B-spline type transformation (no other types supported). \\
source & Text &  &  & ``e" if two body, ion particle set if one body.\\ 
  \hline
\end{tabular}
%\end{tabular*}
\end{center}
\end{table}

Just like one- and two-body jastrows, parameterization of the backflow transformations are specified within the \ixml{<transformation>} blocks by  \ixml{<correlation>} blocks.  Please refer to Section~\ref{sec:onebodyjastrowspline} for more information.

\subsection{Example Use Case}
Having specified the general form, we present a general example of one-body and two-body backflow transformations in a hydrogen-helium mixture.  The hydrogen and helium ions have independent backflow transformations, as do the like and unlike-spin two-body terms.  One caveat is in order:  ionic backflow transformations must be listed in the order they appear in the particle set.  If in our example, helium is listed first and hydrogen is listed second, the following example would be correct.  However, switching backflow declaration to hydrogen first then helium, will result in an error.  Outside of this, declaration of one-body blocks and two-body blocks are not sensitive to ordering.  

\begin{lstlisting}[style=QMCPXML]
<backflow>
<!--The One-Body term with independent e-He and e-H terms. IN THAT ORDER -->
<transformation name="eIonB" type="e-I" function="Bspline" source="ion0">
    <correlation cusp="0.0" size="8" type="shortrange" init="no" elementType="He" rcut="3.0">
        <coefficients id="eHeC" type="Array" optimize="yes"> 
            0 0 0 0 0 0 0 0
        </coefficients>
    </correlation>
    <correlation cusp="0.0" size="8" type="shortrange" init="no" elementType="H" rcut="3.0">
        <coefficients id="eHC" type="Array" optimize="yes"> 
            0 0 0 0 0 0 0 0
        </coefficients>
    </correlation>
</transformation>

<!--The Two-Body Term with Like and Unlike Spins -->
<transformation name="eeB" type="e-e" function="Bspline" >
    <correlation cusp="0.0" size="7" type="shortrange" init="no" speciesA="u" speciesB="u" rcut="1.2">
        <coefficients id="uuB1" type="Array" optimize="yes"> 
            0 0 0 0 0 0 0
        </coefficients>
    </correlation>
    <correlation cusp="0.0" size="7" type="shortrange" init="no" speciesA="d" speciesB="u" rcut="1.2">
        <coefficients id="udB1" type="Array" optimize="yes"> 
            0 0 0 0 0 0 0
        </coefficients>
    </correlation>
</transformation>
</backflow>
\end{lstlisting}  

Currently, backflow works only with single-Slater determinant wavefunctions.  When a backflow transformation has been declared, it should be placed within the \ixml{<determinantset>} block, but outside of the \ixml{<slaterdeterminant>} blocks, like so:

\begin{lstlisting}[style=QMCPXML]
<determinantset ... >
    <!--basis set declarations go here, if there are any -->
    
    <backflow>
        <transformation ...>
          <!--Here is where one and two-body terms are defined -->
         </transformation>
     </backflow>
     
     <slaterdeterminant>
         <!--Usual determinant definitions -->
     </slaterdeterminant>
 </determinantset>
\end{lstlisting}

\subsection{Optimization Tips}
Backflow is notoriously difficult to optimize---it is extremely nonlinear in the variational parameters and moves the nodal surface around.  As such, it is likely that a full Jastrow+Backflow optimization with all parameters initialized to zero might not converge in a reasonable time.  If you are experiencing this problem, the following pointers are suggested (in no particular order).

\subsubsection{Get a good starting guess for $\Psi_T$:}

\begin{enumerate}
\item Try optimizing the Jastrow first without backflow.
\item Freeze the Jastrow parameters, introduce only the e-e terms in the backflow transformation, and optimize these parameters.
\item Freeze the e-e backflow parameters, and then optimize the e-I terms.
  \begin{itemize}
    \item If difficulty is encountered here, try optimizing each species independently.
  \end{itemize}
\item Unfreeze all Jastrow, e-e backflow, and e-I backflow parameters, and reoptimize.  

\end{enumerate}


\subsubsection{Optimizing Backflow Terms}

It is possible that the previous prescription might grind to a halt in steps 2 or 3 with the inability to optimize the e-e or e-I backflow transformation independently, especially if it is initialized to zero.  One way to get around this is to build a good starting guess for the e-e or e-I backflow terms iteratively as follows:  

\begin{enumerate}
\item Start off with a small number of knots initialized to zero.  Set $r_{cut}$ to be small (much smaller than an interatomic distance).
\item Optimize the backflow function.
\item If this works, slowly increase $r_{cut}$ and/or the number of knots.
\item Repeat steps 2 and 3 until there is no noticeable change in energy or variance of $\Psi_T$.
\end{enumerate}

\subsubsection{Tweaking the Optimization Run}

The following modifications are worth a try in the optimization block:

\begin{itemize}
\item Try setting ``useDrift" to ``no."  This eliminates the use of wavefunction gradients and force biasing in the VMC algorithm.  This could be an issue for poorly optimized wavefunctions with pathological gradients.  
\item Try increasing ``exp0" in the optimization block.  Larger values of exp0 cause the search directions to more closely follow those predicted by steepest-descent than those by the linear method.
\end{itemize}

Note that the new adaptive shift optimizer has not yet been tried with backflow wavefunctions. It should perform better than the older optimizers, but a considered optimization process is still recommended.



\section{Finite-difference linear response wave functions}

\label{sec:fdlr}

The finite-difference linear response wavefunction (FDLR) is an
experimental wavefunction type described in detail in
Ref.\cite{blunt_charge-transfer_2017}. In this method, the wavefunction is formed as the linear response of some existing trial wavefunction in QMCPACK. This derivatives of this linear response are
approximated by a simple finite difference.

Forming a wavefunction within the linear response space of an existing ansatz can be very powerful. For example, a configuration interaction singles (CIS) wavefunction can be formed as a linear combination of the first derivatives of a Slater determinant (with respect to its orbital rotation parameters). Thus, in this sense, CIS is the linear response of Hartree--Fock theory.

Forming a CIS wavefunction as the linear response of an optimizable Slater determinant is where all testing of this wavefunction has been performed. In theory, the implementation is flexible and can be used with other trial wavefunctions in QMCPACK, but this has not been tested; the FDLR trial wavefunction is experimental.

Mathematically, the FDLR wavefunction has the form
\begin{equation}
\Psi_{\textrm{FDLR}} (\mathbf{\mu}, \mathbf{X}) = \Psi (\mathbf{X} + \mathbf{\mu}) - \Psi (\mathbf{X} - \mathbf{\mu})\: ,
\end{equation}
where $\Psi(\mathbf{P})$ is some trial wavefunction in QMCPACK, and $\mathbf{P}$ is its optimizable parameters. $\mathbf{X}$ is the ``base'' parameters about which the finite difference is performed (for example, an overall orbital rotation). $\mathbf{\mu}$ is the ``finite-difference'' parameters, which define the direction of the derivative, and whose magnitude determines the magnitude of the finite difference. In the limit that the magnitude of $\mathbf{mu}$ goes to $0$, the $\Psi_{\textrm{FDLR}}$ object just defined becomes equivalent to
\begin{equation}
\Psi_{\textrm{FDLR}} (\mathbf{\mu}, \mathbf{X}) = \sum_{pq} \mu_{pq} \: \frac{\partial \Psi_{\textrm{det}} (\mathbf{X}) }{\partial X_{pq}}\: ,
\end{equation}
which is the desired linear response wavefunction we are approximating. In the case that $\Psi(\mathbf{P})$ is a determinant with orbital rotation parameters $\mathbf{P}$, the previous equation is a CIS wavefunction with CIS expansion coefficients $\mathbf{\mu}$ and orbital rotation $\mathbf{X}$.

\subsection{Input Specifications}
An FDLR wavefunction is specified within a \texttt{<fdlr> ... </fdlr>} block.

To fully specify an FDLR wavefunction as done previously, we require the initial parameters for both $\mathbf{X}$ and $\mathbf{\mu}$ to be input. This therefore requires two trial wavefunctions to be provided on input. Each of these is best specified in its own XML file. The names of these two files are provided in an \texttt{<include>} tag via \texttt{<include wfn\_x\_href=`` ... '' wfn\_d\_href=`` ... ''>}. \texttt{wfn\_x\_href} specifies the file that will hold the $\mathbf{X}$ parameters. \texttt{wfn\_d\_href} specifies the file that will hold the $\mathbf{\mu}$ parameters.

Other options inside the \texttt{<include>} tag are \texttt{opt\_x} and \texttt{opt\_d}, which specify whether or not $\mathbf{X}$ and $\mathbf{\mu}$ parameters are optimizable, respectively.

\subsection{Example Use Case}

\begin{lstlisting}[style=QMCPXML]
<fdlrwfn name="FDLR">
  <include wfn_x_href="h2.wfn_x.xml" wfn_d_href="h2.wfn_d.xml" opt_x="yes" opt_d="yes"/>
</fdlrwfn>
\end{lstlisting}

with the \texttt{h2.wfn\_x.xml} file containing one of the wavefunctions and corresponding set of $\mathbf{X}$ parameters, such as:

\begin{lstlisting}[style=QMCPXML]
<?xml version="1.0"?>
<wfn_x>
    <determinantset name="LCAOBSet" type="MolecularOrbital" transform="yes" source="ion0">
      <basisset name="LCAOBSet">
        <atomicBasisSet name="Gaussian-G2" angular="cartesian" type="Gaussian" elementType="H" normalized="no">
          <grid type="log" ri="1.e-6" rf="1.e2" npts="1001"/>
          <basisGroup rid="H00" n="0" l="0" type="Gaussian">
            <radfunc exponent="1.923840000000e+01" contraction="3.282799101900e-02"/>
            <radfunc exponent="2.898720000000e+00" contraction="2.312039367510e-01"/>
            <radfunc exponent="6.534720000000e-01" contraction="8.172257764360e-01"/>
          </basisGroup>
          <basisGroup rid="H10" n="1" l="0" type="Gaussian">
            <radfunc exponent="1.630642000000e-01" contraction="1.000000000000e+00"/>
          </basisGroup>
        </atomicBasisSet>
      </basisset>

    <slaterdeterminant optimize="yes">
      <determinant id="det_up" sposet="spo-up">
        <opt_vars size="3">
          0.0 0.0 0.0
        </opt_vars>
      </determinant>

      <determinant id="det_down" sposet="spo-dn">
        <opt_vars size="3">
          0.0 0.0 0.0
        </opt_vars>
      </determinant>
    </slaterdeterminant>

      <sposet basisset="LCAOBSet" name="spo-up" size="4" optimize="yes">
        <occupation mode="ground"/>
        <coefficient size="4" id="updetC">
  2.83630000000000e-01  3.35683000000000e-01  2.83630000000000e-01  3.35683000000000e-01
  1.66206000000000e-01  1.22367400000000e+00 -1.66206000000000e-01 -1.22367400000000e+00
  8.68279000000000e-01 -6.95081000000000e-01  8.68279000000000e-01 -6.95081000000000e-01
 -9.77898000000000e-01  1.19682400000000e+00  9.77898000000000e-01 -1.19682400000000e+00
</coefficient>
      </sposet>
      <sposet basisset="LCAOBSet" name="spo-dn" size="4" optimize="yes">
        <occupation mode="ground"/>
        <coefficient size="4" id="downdetC">
  2.83630000000000e-01  3.35683000000000e-01  2.83630000000000e-01  3.35683000000000e-01
  1.66206000000000e-01  1.22367400000000e+00 -1.66206000000000e-01 -1.22367400000000e+00
  8.68279000000000e-01 -6.95081000000000e-01  8.68279000000000e-01 -6.95081000000000e-01
 -9.77898000000000e-01  1.19682400000000e+00  9.77898000000000e-01 -1.19682400000000e+00
</coefficient>
      </sposet>

    </determinantset>
</wfn_x>
\end{lstlisting}
and similarly for the \texttt{h2.wfn\_d.xml} file, which will hold the initial $\mathbf{\mu}$ parameters.

This use case is a wavefunction file for an optimizable determinant wavefunction for H$_2$, in a double zeta valence basis set. Thus, the FDLR wavefunction here would perform CIS on H$_2$ in a double zeta basis set.


\section{Gaussian Product Wavefunction}
\label{sec:ionwf}

The Gaussian Product wavefunction implements Equation~\ref{eq:gauss_prod_wf}
\begin{equation}
\Psi(\vec{R}) = \prod_{i=1}^N \exp\left[ -\frac{(\vec{R}_i-\vec{R}_i^o)^2}{2\sigma_i^2} \right]
\label{eq:gauss_prod_wf},
\end{equation}
where $\vec{R}_i$ is the position of the $i^{\text{th}}$ quantum particle and $\vec{R}_i^o$ is its center. $\sigma_i$ is the width of the Gaussian orbital around center $i$.

This variational wavefunction enhances single-particle density at chosen spatial locations with adjustable strengths. It is useful whenever such localization is physically relevant yet not captured by other parts of the trial wavefunction. For example, in an electron-ion simulation of a solid, the ions are localized around their crystal lattice sites. This single-particle localization is not captured by the ion-ion Jastrow. Therefore, the addition of this localization term will improve the wavefunction. The simplest use case of this wavefunction is perhaps the quantum harmonic oscillator (please see the ``tests/models/sho'' folder for examples).

\subsubsection{Input Specification}

\begin{table}[h]
\begin{center}
\begin{tabular}{l c c c l }
\hline
\multicolumn{5}{l}{Gaussian Product Wavefunction (ionwf)} \\
\hline
\bfseries Name & \bfseries Datatype & \bfseries Values & \bfseries Defaults  & \bfseries Description \\
\hline
Name & Text & ionwf & (Required) & Unique name for this wavefunction \\
Width & Floats & 1.0 -1 & (Required) & Widths of Gaussian orbitals\\ 
Source & Text & ion0 & (Required) & Name of classical particle set\\ 
\hline
\end{tabular}
\end{center}
\end{table}

\FloatBarrier

Additional information:
\begin{itemize}
\item \texttt{width} There must be one width provided for each quantum particle. If a negative width is given, then its corresponding Gaussian orbital is removed. Negative width is useful if one wants to use Gaussian wavefunction for a subset of the quantum particles.
\item \texttt{source} The Gaussian centers must be specified in the form of a classical particle set. This classical particle set is likely the ion positions ``ion0,'' hence the name ``ionwf.'' However, arbitrary centers can be defined using a different particle set. Please refer to the examples in ``tests/models/sho.''
\end{itemize}

\subsection{Example Use Case}
\begin{lstlisting}[style=QMCPXML]
  <qmcsystem>
    <simulationcell>
      <parameter name="bconds">
            n n n
      </parameter>
    </simulationcell>
    <particleset name="e">
      <group name="u" size="1">
        <parameter name="mass">5.0</parameter>
        <attrib name="position" datatype="posArray" condition="0">
          0.0001 -0.0001 0.0002
        </attrib>
      </group>
    </particleset>
    <particleset name="ion0" size="1">
      <group name="H">
        <attrib name="position" datatype="posArray" condition="0">
          0 0 0
        </attrib>
      </group>
    </particleset>
    <wavefunction target="e" id="psi0">
      <ionwf name="iwf" source="ion0" width="0.8165"/>
    </wavefunction>
    <hamiltonian name="h0" type="generic" target="e">
      <extpot type="HarmonicExt" mass="5.0" energy="0.3"/>
      <estimator type="latticedeviation" name="latdev" 
        target="e"    tgroup="u" 
        source="ion0" sgroup="H"/>
    </hamiltonian>
  </qmcsystem>
\end{lstlisting}



\chapter{Hamiltonian and Observables}
\label{chap:hamiltobs}


\qmcpack is capable of the simultaneous measurement of the Hamiltonian and many other quantum operators.  The Hamiltonian attains a special status among the available operators (also referred to as observables) because it ultimately generates all available information regarding the quantum system.  This is evident from an algorithmic standpoint as well since the Hamiltonian (embodied in the projector) generates the imaginary time dynamics of the walkers in DMC and reptation Monte Carlo (RMC). 

This section covers how the Hamiltonian can be specified, component by component, by the user in the XML format native to \qmcpack. It also covers the input structure of statistical estimators corresponding to quantum observables such as the density, static structure factor, and forces.



\section{The Hamiltonian}

The many-body Hamiltonian in Hartree units is given by
\begin{align}
  \hat{H} = -\sum_i\frac{1}{2m_i}\nabla_i^2 + \sum_iv^{ext}(r_i) + \sum_{i<j}v^{qq}(r_i,r_j)   + \sum_{i\ell}v^{qc}(r_i,r_\ell)   + \sum_{\ell<m}v^{cc}(r_\ell,r_m)\:.  
\end{align}
Here, the sums indexed by $i/j$ are over quantum particles, while $\ell/m$ are reserved for classical particles.  Often the quantum particles are electrons, and the classical particles are ions, though \qmcpack is not limited in this way.  The mass of each quantum particle is denoted $m_i$, $v^{qq}/v^{qc}/v^{cc}$ are pair potentials between quantum-quantum/quantum-classical/classical-classical particles, and $v^{ext}$ denotes a purely external potential.

\qmcpack is designed modularly so that any potential can be supported with minimal additions to the code base.  Potentials currently supported include Coulomb interactions in open and periodic boundary conditions, the MPC potential, nonlocal pseudopotentials, helium pair potentials, and various model potentials such as hard sphere, Gaussian, and modified Poschl-Teller.

Reference information and examples for the \texttt{<hamiltonian/>} XML element are provided subsequently.  Detailed descriptions of the input for individual potentials is given in the sections that follow.  

% hamiltonian element
%  dev notes
%    Hamiltonian element read
%      HamiltonianPool::put
%        reads attributes: id name role target 
%          id/name is passed to QMCHamiltonian
%          role selects the primary hamiltonian
%          target associates to quantum particleset
%      HamiltonianFactory::build
%        reads attributes: type source default
%    HamiltonianFactory cloning may be flawed for non-electron systems
%      see HamiltonianFactory::clone
%        aCopy->renameProperty(``e'',qp->getName());
%        aCopy->renameProperty(psiName,psi->getName());
%      the renaming may not work if dynamic particleset.name!=''e''
%   lots of xml inputs are simply ignored if do not explicitly match (fix! here and elsewhere in the build tree)

\FloatBarrier
\begin{table}[h]
\begin{center}
\begin{tabularx}{\textwidth}{l l l l l X }
\hline
\multicolumn{6}{l}{\texttt{hamiltonian} element} \\
\hline
\multicolumn{2}{l}{parent elements:} & \multicolumn{4}{l}{\texttt{simulation, qmcsystem}}\\
\multicolumn{2}{l}{child  elements:} & \multicolumn{4}{l}{\texttt{pairpot extpot estimator constant}(deprecated)}\\
\multicolumn{2}{l}{attributes}  & \multicolumn{4}{l}{}\\
   &   \bfseries name     & \bfseries datatype & \bfseries values & \bfseries default & \bfseries description \\
   & \texttt{name/id}$^o$ &  text              & \textit{anything}& h0                & Unique id for this Hamiltonian instance  \\
   & \texttt{type}$^o$    &  text              &                  & generic           & \textit{No current function}             \\
   & \texttt{role}$^o$    &  text              & primary/extra    & extra             & Designate as primary Hamiltonian or not  \\
   & \texttt{source}$^o$  &  text              & \texttt{particleset.name} & i        & Identify classical \texttt{particleset}           \\
   & \texttt{target}$^o$  &  text              & \texttt{particleset.name} & e        & Identify quantum \texttt{particleset}             \\
   & \texttt{default}$^o$ &  boolean           & yes/no           & yes               & Include kinetic energy term implicitly   \\
  \hline
\end{tabularx}
\end{center}
\end{table}
\FloatBarrier

Additional information:
\begin{itemize}
  \item{\textbf{target:} Must be set to the name of the quantum \texttt{particleset}.  The default value is typically sufficient.  In normal usage, no other attributes are provided.}
\end{itemize}

% All-electron hamiltonian element
\begin{lstlisting}[style=QMCPXML,caption=All electron Hamiltonian XML element.]
<hamiltonian target="e">
  <pairpot name="ElecElec" type="coulomb" source="e" target="e"/>
  <pairpot name="ElecIon"  type="coulomb" source="i" target="e"/>
  <pairpot name="IonIon"   type="coulomb" source="i" target="i"/>
</hamiltonian>
\end{lstlisting}


% Pseudopotential hamiltonian element
\begin{lstlisting}[style=QMCPXML,caption=Pseudopotential Hamiltonian XML element.]
<hamiltonian target="e">
  <pairpot name="ElecElec"  type="coulomb" source="e" target="e"/>
  <pairpot name="PseudoPot" type="pseudo"  source="i" wavefunction="psi0" format="xml">
    <pseudo elementType="Li" href="Li.xml"/>
    <pseudo elementType="H" href="H.xml"/>
  </pairpot>
  <pairpot name="IonIon"    type="coulomb" source="i" target="i"/>
</hamiltonian>
\end{lstlisting}


\section{Pair potentials}

Many pair potentials are supported.  Though only the most commonly used pair potentials are covered in detail in this section, all currently available potentials are listed subsequently.  If a potential you desire is not listed, or is not present at all, feel free to contact the developers.

% pairpot element
\FloatBarrier
\begin{table}[h]
\begin{center}
\begin{tabularx}{\textwidth}{l l l l l X }
\hline
\multicolumn{6}{l}{\texttt{pairpot} factory element} \\
\hline
\multicolumn{2}{l}{parent elements:} & \multicolumn{4}{l}{\texttt{hamiltonian}}\\
\multicolumn{2}{l}{type   selector:} & \multicolumn{4}{l}{\texttt{type} attribute}\\
\multicolumn{2}{l}{type   options: } & \multicolumn{2}{l}{coulomb           } & \multicolumn{2}{l}{Coulomb/Ewald potential}\\
\multicolumn{2}{l}{                } & \multicolumn{2}{l}{pseudo            } & \multicolumn{2}{l}{Semilocal pseudopotential}\\
\multicolumn{2}{l}{                } & \multicolumn{2}{l}{mpc               } & \multicolumn{2}{l}{Model periodic Coulomb interaction/correction}\\
\multicolumn{2}{l}{                } & \multicolumn{2}{l}{cpp               } & \multicolumn{2}{l}{Core polarization potential}\\
\multicolumn{2}{l}{                } & \multicolumn{2}{l}{skpot             } & \multicolumn{2}{l}{\textit{Unknown}}\\
\multicolumn{2}{l}{shared attributes:} & \multicolumn{4}{l}{}\\
   &   \bfseries name     & \bfseries datatype & \bfseries values & \bfseries default   & \bfseries description \\
   &   \texttt{type}$^r$      &  text              & \textit{See above}        & 0                   & Select pairpot type         \\
   &   \texttt{name}$^r$      &  text              & \textit{anything}         & any                 & Unique name for this pairpot\\
   &   \texttt{source}$^r$    &  text              & \texttt{particleset.name} &\texttt{hamiltonian.target}& Identify interacting particles\\
   &   \texttt{target}$^r$    &  text              & \texttt{particleset.name} &\texttt{hamiltonian.target}& Identify interacting particles  \\
   &   \texttt{units}$^o$     &  text              &                           & hartree             & \textit{No current function}  \\
\hline
\end{tabularx}
\end{center}
\end{table}
\FloatBarrier

Additional information:
\begin{itemize}
  \item{\textbf{type:} Used to select the desired pair potential.  Must be selected from the list of type options.}
  \item{\textbf{name:} A unique name used to identify this pair potential.  Block averaged output data will appear under this name in \texttt{scalar.dat} and/or \texttt{stat.h5} files.}
  \item{\textbf{source/target:}  These specify the particles involved in a pair interaction.  If an interaction is between classical (e.g., ions) and quantum (e.g., electrons), \texttt{source}/\texttt{target} should be the name of the classical/quantum \texttt{particleset}.}
  \item{Only \texttt{Coulomb, pseudo}, and \texttt{mpc} are described in detail in the following subsections.  The older or less-used types (\texttt{cpp, skpot}) are not covered.}
  \dev{
  \item{Available only if \texttt{QMC\_CUDA} is not defined: \texttt{skpot}.}
  \item{Available only if \texttt{OHMMS\_DIM==3}: \texttt{mpc, vhxc, pseudo}.}
  \item{Available only if \texttt{OHMMS\_DIM==3} and \texttt{QMC\_CUDA} is not defined: \texttt{cpp}.}
  }
\end{itemize}


% physical read by coulomb potentials
% potential is only for pressure estimator



% pairpot instances

%   do coulomb, pseudo, mpc

\subsection{Coulomb potentials}

The bare Coulomb potential is used in open boundary conditions:
\begin{align}
  V_c^{open} = \sum_{i<j}\frac{q_iq_j}{\abs{r_i-r_j}}\:.
\end{align}

When periodic boundary conditions are selected, Ewald summation is used automatically:
\begin{align}\label{eq:ewald}
  V_c^{pbc} = \sum_{i<j}\frac{q_iq_j}{\abs{r_i-r_j}} + \frac{1}{2}\sum_{L\ne0}\sum_{i,j}\frac{q_iq_j}{\abs{r_i-r_j+L}}\:.
\end{align}
The sum indexed by $L$ is over all nonzero simulation cell lattice vectors.  In practice, the Ewald sum is broken into short- and long-range parts in a manner optimized for efficiency (see Ref. \cite{Natoli1995}) for details. 

For information on how to set the boundary conditions, consult Section~\ref{chap:simulationcell}.


\FloatBarrier
\begin{table}[h]
\begin{center}
\begin{tabularx}{\textwidth}{l l l l l X }
\hline
\multicolumn{6}{l}{\texttt{pairpot type=coulomb} element} \\
\hline
\multicolumn{2}{l}{parent elements:} & \multicolumn{4}{l}{\texttt{hamiltonian}}\\
\multicolumn{2}{l}{child  elements:} & \multicolumn{4}{l}{\textit{None}}\\
\multicolumn{2}{l}{attributes}  & \multicolumn{4}{l}{}\\
   &   \bfseries name     & \bfseries datatype & \bfseries values & \bfseries default   & \bfseries description \\
   & \texttt{type}$^r$    &  text              & \textbf{coulomb} &                     & Must be coulomb     \\
   & \texttt{name/id}$^r$ &  text              & \textit{anything}&  ElecElec           & Unique name for interaction\\
   & \texttt{source}$^r$  &  text              & \texttt{particleset.name} &\texttt{hamiltonian.target}& Identify interacting particles\\
   & \texttt{target}$^r$  &  text              & \texttt{particleset.name} &\texttt{hamiltonian.target}& Identify interacting particles\\
   & \texttt{pbc}$^o$     &  boolean           & yes/no           & yes                 & Use Ewald summation  \\
   & \texttt{physical}$^o$&  boolean           & yes/no           & yes                 & Hamiltonian(yes)/observable(no) \\
\dev{& \texttt{forces}      &  boolean           & yes/no           & no                  & \textit{Deprecated}             \\ }
  \hline
\end{tabularx}
\end{center}
\end{table}
\FloatBarrier

Additional information
\begin{itemize}
  \item{\textbf{type/source/target:} See description for the previous generic \texttt{pairpot} factory element.}
  \item{\textbf{name:} Traditional user-specified names for electron-electron, electron-ion, and ion-ion terms are \texttt{ElecElec}, \texttt{ElecIon}, and \texttt{IonIon}, respectively.  Although any choice can be used, the data analysis tools expect to find columns in \texttt{*.scalar.dat} with these names.}
  \item{\textbf{pbc}: Ewald summation will not be performed if \texttt{simulationcell.bconds== n n n}, regardless of the value of \texttt{pbc}.  Similarly, the \texttt{pbc} attribute can only be used to turn off Ewald summation if \texttt{simulationcell.bconds!= n n n}.  The default value is recommended.}
  \item{\textbf{physical}: If \texttt{physical==yes}, this pair potential is included in the Hamiltonian and will factor into the \texttt{LocalEnergy} reported by QMCPACK and also in the DMC branching weight.  If \texttt{physical==no}, then the pair potential is treated as a passive observable but not as part of the Hamiltonian itself.  As such it does not contribute to the outputted \texttt{LocalEnergy}.  Regardless of the value of \texttt{physical} output data will appear in \texttt{scalar.dat} in a column headed by \texttt{name}.}
\end{itemize}


\begin{lstlisting}[style=QMCPXML,caption=QMCPXML element for Coulomb interaction between electrons.]
  <pairpot name="ElecElec" type="coulomb" source="e" target="e"/>
\end{lstlisting}

\begin{lstlisting}[style=QMCPXML,caption=QMCPXML element for Coulomb interaction between electrons and ions (all-electron only).]
  <pairpot name="ElecIon"  type="coulomb" source="i" target="e"/>
\end{lstlisting}

\begin{lstlisting}[style=QMCPXML,caption=QMCPXML element for Coulomb interaction between ions.]
  <pairpot name="IonIon"   type="coulomb" source="i" target="i"/>
\end{lstlisting}


\subsection{Pseudopotentials}
\label{sec:nlpp}
\qmcpack supports pseudopotentials in semilocal form, which is local in the radial coordinate and nonlocal in angular coordinates.  When all angular momentum channels above a certain threshold ($\ell_{max}$) are well approximated by the same potential ($V_{\bar{\ell}}\equiv V_{loc}$), the pseudpotential separates into a fully local channel and an angularly nonlocal component:
\begin{align}
  V^{PP} = \sum_{ij}\Big(V_{\bar{\ell}}(\abs{r_i-\tilde{r}_j}) + \sum_{\ell\ne\bar{\ell}}^{\ell_{max}}\sum_{m=-\ell}^\ell \operator{Y_{\ell m}}{\big[V_\ell(\abs{r_i-\tilde{r}_j}) - V_{\bar{\ell}}(\abs{r_i-\tilde{r}_j}) \big]}{Y_{\ell m}} \Big)\:.
\end{align}  
Here the electron/ion index is $i/j$, and only one type of ion is shown for simplicity.

Evaluation of the localized pseudopotential energy $\Psi_T^{-1}V^{PP}\Psi_T$ requires additional angular integrals.  These integrals are evaluated on a randomly shifted angular grid.  The size of this grid is determined by $\ell_{max}$.  See Ref. \cite{Mitas1991} for further detail. 

\qmcpack uses the FSAtom pseudopotential file format associated with the ``Free Software Project for Atomic-scale Simulations'' initiated in 2002  (see \url{http://www.tddft.org/fsatom/manifest.php} for general information).  The FSAtom format uses XML for structured data.  Files in this format do not use a specific identifying file extension; instead they are simply suffixed with ``\texttt{.xml}.''  The tabular data format of CASINO is also supported.


% FSAtom format links
%   unfortunately none of the surviving links detail the format itself
% http://www.tddft.org/fsatom/index.php
% http://www.tddft.org/fsatom/programs.php
% http://www.tddft.org/fsatom/manifest.php
% http://163.13.111.58/cchu/reference/web/PseudoPotentials%20-%20FSAtom%20Wiki.htm
% http://departments.icmab.es/leem/alberto/xml/pseudo/index.html



% pseudopotential element
%   dev notes
%     attributes name, source, wavefunction, format are read in CoulombFactory.cpp  HamiltonianFactory::addPseudoPotential
%     format==''old'' refers to an old table format that is no longer supported
%     read continues in ECPotentialBuilder::put()
%       if format!=xml/old (i.e. table) qmcpack will attempt to read from *.psf files
%         in this case, <pairpot type=''pseudo'' format=''table''/>, ie there are no elements
%         if particlset groups are Li H (in order), then it looks for Li.psf and H.psf
%         what is the psf format?
%       if format==xml, normal read continues, i.e. <pseudo/> child elements are expected
%         read is not sensitive to particleset group/species ordering
%         child elements not named <pseudo/> are simply ignored (FIX!)
\FloatBarrier
\begin{table}[h]
\begin{center}
\begin{tabularx}{\textwidth}{l l l l l X }
\hline
\multicolumn{6}{l}{\texttt{pairpot type=pseudo} element} \\
\hline
\multicolumn{2}{l}{parent elements:} & \multicolumn{4}{l}{\texttt{hamiltonian}}\\
\multicolumn{2}{l}{child  elements:} & \multicolumn{4}{l}{\texttt{pseudo}}\\
\multicolumn{2}{l}{attributes}  & \multicolumn{4}{l}{}\\
   &   \bfseries name     & \bfseries datatype & \bfseries values & \bfseries default   & \bfseries description \\
   & \texttt{type}$^r$    &  text              & \textbf{pseudo} &                      & Must be pseudo         \\
   & \texttt{name/id}$^r$ &  text              & \textit{anything}&  PseudoPot          & \textit{No current function}\\
   & \texttt{source}$^r$  &  text              & \texttt{particleset.name} &  i                  & Ion \texttt{particleset} name\\
   & \texttt{target}$^r$  &  text              & \texttt{particleset.name} &\texttt{hamiltonian.target}& Electron \texttt{particleset} name  \\
   & \texttt{pbc}$^o$     &  boolean           & yes/no           & yes$^*$             & Use Ewald summation  \\
   & \texttt{forces}      &  boolean           & yes/no           & no                  & \textit{Deprecated}             \\
   &\texttt{wavefunction}$^r$ &  text          & \texttt{wavefunction.name}& invalid    & Identify wavefunction \\
   &   \texttt{format}$^r$    &  text          & xml/table        & table               & Select file format   \\
   & \texttt{algorithm}$^o$   &  text          & batched/default  & default             & Choose NLPP algorithm \\
  \hline
\end{tabularx}
\end{center}
\end{table}
\FloatBarrier

Additional information:
\begin{itemize}
  \item{\textbf{type/source/target} See description for the generic \texttt{pairpot} factory element.}
  \item{\textbf{name:} Ignored.  Instead, default names will be present in \texttt{*scalar.dat} output files when pseudopotentials are used.  The field \texttt{LocalECP} refers to the local part of the pseudopotential.  If nonlocal channels are present, a \texttt{NonLocalECP} field will be added that contains the nonlocal energy summed over all angular momentum channels.}
  \item{\textbf{pbc:} Ewald summation will not be performed if \texttt{simulationcell.bconds== n n n}, regardless of the value of \texttt{pbc}.  Similarly, the \texttt{pbc} attribute can only be used to turn off Ewald summation if \texttt{simulationcell.bconds!= n n n}.}
  \item{\textbf{format:}  If \texttt{format}==table, QMCPACK looks for \texttt{*.psf} files containing pseudopotential data in a tabular format.  The files must be named after the ionic species provided in \texttt{particleset} (e.g., \texttt{Li.psf} and \texttt{H.psf}). If \texttt{format}==xml, additional \texttt{pseudo} child XML elements must be provided (see the following).  These elements specify individual file names and formats (both the FSAtom XML and CASINO tabular data formats are supported). }
  \item{\textbf{algorithm} The default algorithm evaluates the ratios of wavefunction components together for each qudarature point and then one point after another. The batched algorithm evaluates the ratios of qudarature points together for each wavefunction component and then one component after another. Internally, it uses \texttt{VirtualParticleSet} for quadrature points. Hybrid orbital representation has an extra optimization enabled when using the batched algorithm.}
\end{itemize}


\begin{lstlisting}[style=QMCPXML,caption=QMCPXML element for pseudopotential electron-ion interaction (psf files).]
  <pairpot name="PseudoPot" type="pseudo"  source="i" wavefunction="psi0" format="psf"/>
\end{lstlisting}

\begin{lstlisting}[style=QMCPXML,caption=QMCPXML element for pseudopotential electron-ion interaction (xml files).]
  <pairpot name="PseudoPot" type="pseudo"  source="i" wavefunction="psi0" format="xml">
    <pseudo elementType="Li" href="Li.xml"/>
    <pseudo elementType="H" href="H.xml"/>
  </pairpot>
\end{lstlisting}

%\begin{lstlisting}[caption=QMCPXML element for pseudopotential electron-ion interaction (CASINO files).]
%  <pairpot name="PseudoPot" type="pseudo"  source="i" wavefunction="psi0" format="xml">
%    <pseudo elementType="Li" href="Li.data"/>
%    <pseudo elementType="H" href="H.data"/>
%  </pairpot>
%\end{lstlisting}


Details of \texttt{<pseudo/>} input elements are shown in the following.  It is possible to include (or construct) a full pseudopotential directly in the input file without providing an external file via \texttt{href}.  The full XML format for pseudopotentials is not yet covered.

% pseudo element
%   dev notes
%     initial read of href elementType/symbol attributes at ECPotentialBuilder::useXmlFormat()
%     read continues in ECPComponentBuilder
%       format==xml and href==none (not provided) => ECPComponentBuilder::put(cur)
%       format==xml and href==a file => ECPComponentBuilder::parse(href,cur)
%       format==casino => ECPComponentBuilder::parseCasino(href,cur)
%         this reader is tucked away in ECPComponentBuilder.2.cpp
%         nice demonstration of OhmmsAsciiParser here
%         maximum cutoff defined by a 1.e-5 (Ha?) spread in the nonlocal potentials
%     quadrature rules (1-7) set as in J. Chem. Phys. 95 (3467) (1991), see below
%       Rule     # points     lexact
%        1           1          0
%        2           4          2
%        3           6          3
%        4          12          5
%        5          18          5
%        6          26          7
%        7          50         11
%     looks like channels only go from s-g (see ECPComponentBuilder constructor)
%       perhaps not, quadrature rules really do go up to 7 (lexact==11), see SetQuadratureRule()
\FloatBarrier
\begin{table}[h]
\begin{center}
\begin{tabularx}{\textwidth}{l l l l l X }
\hline
\multicolumn{6}{l}{\texttt{pseudo} element} \\
\hline
\multicolumn{2}{l}{parent elements:} & \multicolumn{4}{l}{\texttt{pairpot type=pseudo}}\\
\multicolumn{2}{l}{child  elements:} & \multicolumn{4}{l}{\texttt{header local grid}}\\
\multicolumn{2}{l}{attributes}  & \multicolumn{4}{l}{}\\
   &   \bfseries name     & \bfseries datatype & \bfseries values & \bfseries default   & \bfseries description \\
   & \texttt{elementType/symbol}$^r$&  text    &\texttt{group.name}& none               & Identify ionic species   \\
   & \texttt{href}$^r$    &  text              & \textit{filepath}& none                & Pseudopotential file path\\
   & \texttt{format}$^r$  &  text              & xml/casino       & xml                 & Specify file format\\
   & \texttt{cutoff}$^o$  &  real              &                  &                     & Nonlocal cutoff radius  \\
   & \texttt{lmax}$^o$    &  integer           &                  &                     & Largest angular momentum  \\
   & \texttt{nrule}$^o$   &  integer           &                  &                     & Integration grid order             \\
  \hline
\end{tabularx}
\end{center}
\end{table}
\FloatBarrier


\begin{lstlisting}[style=QMCPXML,caption=QMCPXML element for pseudopotential of single ionic species.]
  <pseudo elementType="Li" href="Li.xml"/>
\end{lstlisting}



\subsection{MPC Interaction/correction}

The MPC interaction is an alternative to direct Ewald summation.  The MPC corrects the exchange correlation hole to more closely match its thermodynamic limit.  Because of this, the MPC exhibits smaller finite-size errors than the bare Ewald interaction, though a few alternative and competitive finite-size correction schemes now exist.  The MPC is itself often used just as a finite-size correction in postprocessing (set \texttt{physical=false} in the input).  


% mpc element
%  dev notes
%    most attributes are read in CoulombPotentialFactory.cpp  HamiltonianFactory::addMPCPotential()
%    user input for the name attribute is ignored and the name is always MPC
%    density G-vectors are stored in ParticleSet: Density_G and DensityReducedGvecs members
%    check the Linear Extrap and Quadratic Extrap output in some real examples (see MPC::init_f_G())
%      what are acceptable values for the discrepancies?
%      check that these decrease as cutoff is increased 
%    commented out code for MPC.dat creation in MPC::initBreakup()
%    short range part is 1/r, MPC::evalSR()
%    long range part is on a spline (VlongSpline), MPC::evalLR()
\FloatBarrier
\begin{table}[h]
\begin{center}
\begin{tabularx}{\textwidth}{l l l l l X }
\hline
\multicolumn{6}{l}{\texttt{pairpot type=mpc} element} \\
\hline
\multicolumn{2}{l}{parent elements:} & \multicolumn{4}{l}{\texttt{hamiltonian}}\\
\multicolumn{2}{l}{child  elements:} & \multicolumn{4}{l}{\textit{None}}\\
\multicolumn{2}{l}{attributes}  & \multicolumn{4}{l}{}\\
   &   \bfseries name     & \bfseries datatype & \bfseries values & \bfseries default   & \bfseries description \\
   & \texttt{type}$^r$    &  text              & \textbf{mpc}     &                     & Must be mpc         \\
   & \texttt{name/id}$^r$ &  text              & \textit{anything}&  MPC                & Unique name for interaction \\
   & \texttt{source}$^r$  &  text              & \texttt{particleset.name} &\texttt{hamiltonian.target}& Identify interacting particles\\
   & \texttt{target}$^r$  &  text              & \texttt{particleset.name} &\texttt{hamiltonian.target}& Identify interacting particles  \\
   & \texttt{physical}$^o$&  boolean           & yes/no           & no                  & Hamiltonian(yes)/observable(no) \\
   &  \texttt{cutoff}     &  real              & $>0$             & 30.0                & Kinetic energy cutoff \\
  \hline
\end{tabularx}
\end{center}
\end{table}
\FloatBarrier
Remarks
\begin{itemize}
  \item{\texttt{physical}:  Typically set to \texttt{no}, meaning the standard Ewald interaction will be used during sampling and MPC will be measured as an observable for finite-size post-correction.  If \texttt{physical} is \texttt{yes}, the MPC interaction will be used during sampling.  In this case an electron-electron Coulomb \texttt{pairpot} element should not be supplied.}
  \item{\textbf{Developer note:} Currently the \texttt{name} attribute for the MPC interaction is ignored.  The name is always reset to \texttt{MPC}.}
\end{itemize}

% MPC correction
\begin{lstlisting}[style=QMCPXML,caption=MPC for finite-size postcorrection.]
  <pairpot type="MPC" name="MPC" source="e" target="e" ecut="60.0" physical="no"/>
\end{lstlisting}



% estimator element
\section{General estimators}

A broad range of estimators for physical observables are available in \qmcpack.  The following sections contain input details for the total number density (\texttt{density}), number density resolved by particle spin (\texttt{spindensity}), spherically averaged pair correlation function (\texttt{gofr}), static structure factor (\texttt{sk}), energy density (\texttt{energydensity}), one body reduced density matrix (\texttt{dm1b}), $S(k)$ based kinetic energy correction (\texttt{chiesa}), forward walking (\texttt{ForwardWalking}), and force (\texttt{Force}) estimators.  Other estimators are not yet covered.

When an \texttt{<estimator/>} element appears in \texttt{<hamiltonian/>}, it is evaluated for all applicable chained QMC runs ({e.g.,} VMC$\rightarrow$DMC$\rightarrow$DMC).  Estimators are generally not accumulated during wavefunction optimization sections.    If an \texttt{<estimator/>} element is instead provided in a particular \texttt{<qmc/>} element, that estimator is only evaluated for that specific section ({e.g.,} during VMC only).


\FloatBarrier
\begin{table}[h]
\begin{center}
\begin{tabularx}{\textwidth}{l l l l l X }
\hline
\multicolumn{6}{l}{\texttt{estimator} factory element} \\
\hline
\multicolumn{2}{l}{parent elements:} & \multicolumn{4}{l}{\texttt{hamiltonian, qmc}}\\
\multicolumn{2}{l}{type   selector:} & \multicolumn{4}{l}{\texttt{type} attribute}\\
\multicolumn{2}{l}{type   options: } & \multicolumn{2}{l}{density           } & \multicolumn{2}{l}{Density on a grid}\\
\multicolumn{2}{l}{                } & \multicolumn{2}{l}{spindensity       } & \multicolumn{2}{l}{Spin density on a grid}\\
\multicolumn{2}{l}{                } & \multicolumn{2}{l}{gofr              } & \multicolumn{2}{l}{Pair correlation function (quantum species)}\\
\multicolumn{2}{l}{                } & \multicolumn{2}{l}{sk                } & \multicolumn{2}{l}{Static structure factor}\\
\multicolumn{2}{l}{                } & \multicolumn{2}{l}{structurefactor   } & \multicolumn{2}{l}{Species resolved structure factor}\\
\multicolumn{2}{l}{                } & \multicolumn{2}{l}{specieskinetic    } & \multicolumn{2}{l}{Species resolved kinetic energy}\\
\multicolumn{2}{l}{                } & \multicolumn{2}{l}{latticedeviation  } & \multicolumn{2}{l}{Spatial deviation between two particlesets}\\
\multicolumn{2}{l}{                } & \multicolumn{2}{l}{momentum          } & \multicolumn{2}{l}{Momentum distribution}\\
\multicolumn{2}{l}{                } & \multicolumn{2}{l}{energydensity     } & \multicolumn{2}{l}{Energy density on uniform or Voronoi grid}\\
\multicolumn{2}{l}{                } & \multicolumn{2}{l}{dm1b              } & \multicolumn{2}{l}{One body density matrix in arbitrary basis}\\
\multicolumn{2}{l}{                } & \multicolumn{2}{l}{chiesa            } & \multicolumn{2}{l}{Chiesa-Ceperley-Martin-Holzmann kinetic energy correction}\\
\multicolumn{2}{l}{                } & \multicolumn{2}{l}{Force             } & \multicolumn{2}{l}{Family of ``force'' estimators (see~\ref{sec:force_est})}\\
\multicolumn{2}{l}{                } & \multicolumn{2}{l}{ForwardWalking    } & \multicolumn{2}{l}{Forward walking values for existing estimators}\\
\multicolumn{2}{l}{                } & \multicolumn{2}{l}{orbitalimages     } & \multicolumn{2}{l}{Create image files for orbitals, then exit}\\
\multicolumn{2}{l}{                } & \multicolumn{2}{l}{flux              } & \multicolumn{2}{l}{Checks sampling of kinetic energy}\\
\multicolumn{2}{l}{                } & \multicolumn{2}{l}{localmoment       } & \multicolumn{2}{l}{Atomic spin polarization within cutoff radius}\\
\dev{
\multicolumn{2}{l}{                } & \multicolumn{2}{l}{Pressure          } & \multicolumn{2}{l}{\textit{No current function}}\\
\multicolumn{2}{l}{shared attributes:} & \multicolumn{4}{l}{}\\
}
   &   \bfseries name     & \bfseries datatype & \bfseries values & \bfseries default   & \bfseries description \\
   &   \texttt{type}$^r$      &  text              & \textit{See above}        & 0                   & Select estimator type         \\
   &   \texttt{name}$^r$      &  text              & \textit{anything}         & any                 & Unique name for this estimator\\
   %&   \texttt{source}$^r$    &  text              & \texttt{particleset.name} &\texttt{hamiltonian.target}& Identify interacting particles\\
   %&   \texttt{target}$^r$    &  text              & \texttt{particleset.name} &\texttt{hamiltonian.target}& Identify interacting particles  \\
   %&   \texttt{units}$^o$     &  text              &                           & hartree             & \textit{No current function}  \\
\hline
\end{tabularx}
\end{center}
\end{table}
\FloatBarrier


%  <estimator type="structurefactor" name="StructureFactor" report="yes"/>
%  <estimator type="nofk" name="nofk" wavefunction="psi0"/>


%\dev{
%\FloatBarrier
%\begin{table}[h]
%\begin{center}
%\begin{tabularx}{\textwidth}{l l l l l X }
%\hline
%\multicolumn{6}{l}{\texttt{estimator type=X} element} \\
%\hline
%\multicolumn{2}{l}{parent elements:} & \multicolumn{4}{l}{\texttt{hamiltonian, qmc}}\\
%\multicolumn{2}{l}{child  elements:} & \multicolumn{4}{l}{\textit{None}}\\
%\multicolumn{2}{l}{attributes}  & \multicolumn{4}{l}{}\\
%   &   \bfseries name     & \bfseries datatype & \bfseries values & \bfseries default   & \bfseries description \\
%   & \texttt{type}$^r$    &  text              & \textbf{X}     &                     & Must be X         \\
%   & \texttt{name}$^r$    &  text              & \textit{anything}&                  & Unique name for estimator \\
%   & \texttt{source}$^o$  &  text              & \texttt{particleset.name} &\texttt{hamiltonian.target}& Identify particles\\
%   & \texttt{target}$^o$  &  text              & \texttt{particleset.name} &\texttt{hamiltonian.target}& Identify particles  \\
%  \hline
%\end{tabularx}
%\end{center}
%\end{table}
%\FloatBarrier
%}


\subsection{Chiesa-Ceperley-Martin-Holzmann kinetic energy correction}

This estimator calculates a finite-size correction to the kinetic energy following the formalism laid out in Ref. \cite{Chiesa2006}.  The total energy can be corrected for finite-size effects by using this estimator in conjunction with the MPC correction.

\FloatBarrier
\begin{table}[h]
\begin{center}
\begin{tabularx}{\textwidth}{l l l l l X }
\hline
\multicolumn{6}{l}{\texttt{estimator type=chiesa} element} \\
\hline
\multicolumn{2}{l}{parent elements:} & \multicolumn{4}{l}{\texttt{hamiltonian, qmc}}\\
\multicolumn{2}{l}{child  elements:} & \multicolumn{4}{l}{\textit{None}}\\
\multicolumn{2}{l}{attributes}  & \multicolumn{4}{l}{}\\
   &   \bfseries name     & \bfseries datatype & \bfseries values & \bfseries default   & \bfseries description \\
   & \texttt{type}$^r$    &  text              & \textbf{chiesa}            &        & Must be chiesa         \\
   & \texttt{name}$^o$    &  text              & \textit{anything}          & KEcorr & Always reset to KEcorr \\
   & \texttt{source}$^o$  &  text              & \texttt{particleset.name}  & e      & Identify quantum particles\\
   & \texttt{psi}$^o$     &  text              & \texttt{wavefunction.name} & psi0   & Identify wavefunction  \\
  \hline
\end{tabularx}
\end{center}
\end{table}
\FloatBarrier

% kinetic energy correction
\begin{lstlisting}[style=QMCPXML,caption=``Chiesa'' kinetic energy finite-size postcorrection.]
   <estimator name="KEcorr" type="chiesa" source="e" psi="psi0"/>
\end{lstlisting}




\subsection{Density estimator}
The particle number density operator is given by
\begin{align}
  \hat{n}_r = \sum_i\delta(r-r_i)\:.
\end{align}
The \texttt{density} estimator accumulates the number density on a uniform histogram grid over the simulation cell.  The value obtained for a grid cell $c$ with volume $\Omega_c$ is then the average number of particles in that cell:
\begin{align}
  n_c = \int dR \abs{\Psi}^2 \int_{\Omega_c}dr \sum_i\delta(r-r_i)\:.
\end{align}  


\FloatBarrier
\begin{table}[h]
\begin{center}
\begin{tabularx}{\textwidth}{l l l l l X }
\hline
\multicolumn{6}{l}{\texttt{estimator type=density} element} \\
\hline
\multicolumn{2}{l}{parent elements:} & \multicolumn{4}{l}{\texttt{hamiltonian, qmc}}\\
\multicolumn{2}{l}{child  elements:} & \multicolumn{4}{l}{\textit{None}}\\
\multicolumn{2}{l}{attributes}  & \multicolumn{4}{l}{}\\
   &   \bfseries name     & \bfseries datatype & \bfseries values  & \bfseries default   & \bfseries description \\
   & \texttt{type}$^r$      &  text              & \textbf{density}      &                     & Must be density         \\
   & \texttt{name}$^r$      &  text              & \textit{anything}     & any                 & Unique name for estimator \\
   & \texttt{delta}$^o$     &  real array(3)     & $0\le v_i \le 1$      & 0.1 0.1 0.1         & Grid cell spacing, unit coords\\
   & \texttt{x\_min}$^o$    &  real              & $>0$                  & 0                   & Grid starting point in x (Bohr)\\
   & \texttt{x\_max}$^o$    &  real              & $>0$                  &$|\texttt{lattice[0]}|$& Grid ending point in x (Bohr)\\
   & \texttt{y\_min}$^o$    &  real              & $>0$                  & 0                   & Grid starting point in y (Bohr)\\
   & \texttt{y\_max}$^o$    &  real              & $>0$                  &$|\texttt{lattice[1]}|$& Grid ending point in y (Bohr)\\
   & \texttt{z\_min}$^o$    &  real              & $>0$                  & 0                   & Grid starting point in z (Bohr)\\
   & \texttt{z\_max}$^o$    &  real              & $>0$                  &$|\texttt{lattice[2]}|$& Grid ending point in z (Bohr)\\
   & \texttt{potential}$^o$ &  boolean           & yes/no                & no                  & Accumulate local potential, \textit{Deprecated}\\
   & \texttt{debug}$^o$     &  boolean           & yes/no                & no                  & \textit{No current function}\\
  \hline
\end{tabularx}
\end{center}
\end{table}
\FloatBarrier


Additional information:
\begin{itemize}
  \item{\texttt{name}: The name provided will be used as a label in the \texttt{stat.h5} file for the blocked output data.  Postprocessing tools expect \texttt{name="Density."}}
  \item{\texttt{delta}:  This sets the histogram grid size used to accumulate the density: \texttt{delta="0.1 0.1 0.05"}$\rightarrow 10\times 10\times 20$ grid, \texttt{delta="0.01 0.01 0.01"}$\rightarrow 100\times 100\times 100$ grid.  The density grid is written to a \texttt{stat.h5} file at the end of each MC block.  If you request many $blocks$ in a \texttt{<qmc/>} element, or select a large grid, the resulting \texttt{stat.h5} file could be many gigabytes in size.}
  \item{\texttt{*\_min/*\_max}: Can be used to select a subset of the simulation cell for the density histogram grid.  For example if a (cubic) simulation cell is 20 Bohr on a side, setting \texttt{*\_min=5.0} and \texttt{*\_max=15.0} will result in a density histogram grid spanning a $10\times 10\times 10$ Bohr cube about the center of the box.  Use of \texttt{x\_min, x\_max, y\_min, y\_max, z\_min, z\_max} is only appropriate for orthorhombic simulation cells with open boundary conditions.}
  \item{When open boundary conditions are used, a \texttt{<simulationcell/>} element must be explicitly provided as the first subelement of \texttt{<qmcsystem/>} for the density estimator to work.  In this case the molecule should be centered around the middle of the simulation cell ($L/2$) and not the origin ($0$} since the space within the cell, and hence the density grid, is defined from $0$ to $L$).
\end{itemize}


% density estimator
\begin{lstlisting}[style=QMCPXML,caption=Density estimator (uniform grid).]
   <estimator name="Density" type="density" delta="0.05 0.05 0.05"/>
\end{lstlisting}

\subsection{Spin density estimator}

The spin density is similar to the total density described previously.  In this case, the sum over particles is performed independently for each spin component.

\FloatBarrier
\begin{table}[h]
\begin{center}
\begin{tabularx}{\textwidth}{l l l l l X }
\hline
\multicolumn{6}{l}{\texttt{estimator type=spindensity} element} \\
\hline
\multicolumn{2}{l}{parent elements:} & \multicolumn{4}{l}{\texttt{hamiltonian, qmc}}\\
\multicolumn{2}{l}{child  elements:} & \multicolumn{4}{l}{\textit{None}}\\
\multicolumn{2}{l}{attributes}  & \multicolumn{4}{l}{}\\
   & \bfseries name       & \bfseries datatype & \bfseries values  & \bfseries default   & \bfseries description \\
   & \texttt{type}$^r$    &  text              & \textbf{spindensity} &                  & Must be spindensity       \\
   & \texttt{name}$^r$    &  text              & \textit{anything}    & any              & Unique name for estimator \\
   & \texttt{report}$^o$  &  boolean           & yes/no               & no               & Write setup details to stdout \\
\multicolumn{2}{l}{parameters}  & \multicolumn{4}{l}{}\\
   & \bfseries name       & \bfseries datatype & \bfseries values  & \bfseries default   & \bfseries description \\
   & \texttt{grid}$^o$      & integer array(3) & $v_i>0$           &                     & Grid cell count       \\
   & \texttt{dr}$^o$        & real array(3)    & $v_i>0$           &                     & Grid cell spacing (Bohr) \\
   & \texttt{cell}$^o$      & real array(3,3)  & \textit{anything} &                     & Volume grid exists in           \\
   & \texttt{corner}$^o$    & real array(3)    & \textit{anything} &                     & Volume corner location  \\
   & \texttt{center}$^o$    & real array(3)    & \textit{anything} &                     & Volume center/origin location \\
   & \texttt{voronoi}$^o$   & text             &\texttt{particleset.name}&               & \textit{Under development}\\%Ion particleset for Voronoi centers\\
   & \texttt{test\_moves}$^o$& integer         & $>=0$             & 0                   & Test estimator with random moves  \\
  \hline
\end{tabularx}
\end{center}
\end{table}
\FloatBarrier

Additional information:
\begin{itemize}
  \item{\texttt{name}: The name provided will be used as a label in the \texttt{stat.h5} file for the blocked output data.  Postprocessing tools expect \texttt{name="SpinDensity."}}
  \item{\texttt{grid}: The grid sets the dimension of the histogram grid.  Input like \texttt{<parameter name="grid"> 40 40 40 </parameter>} requests a $40 \times 40\times 40$ grid.  The shape of individual grid cells is commensurate with the supercell shape.}
  \item{\texttt{dr}: The {\texttt{dr}} sets the real-space dimensions of grid cell edges (Bohr units).  Input like \texttt{<parameter name="dr"> 0.5 0.5 0.5 </parameter>} in a supercell with axes of length 10 Bohr each (but of arbitrary shape) will produce a $20\times 20\times 20$ grid. The inputted \texttt{dr} values are rounded to produce an integer number of grid cells along each supercell axis.  Either \texttt{grid} or \texttt{dr} must be provided, but not both.}
  \item{\texttt{cell}: When \texttt{cell} is provided, a user-defined grid volume is used instead of the global supercell.  This must be provided if open boundary conditions are used.  Additionally, if \texttt{cell} is provided, the user must specify where the volume is located in space in addition to its size/shape (\texttt{cell}) using either the \texttt{corner} or \texttt{center} parameters.}
  \item{\texttt{corner}: The grid volume is defined as $corner+\sum_{d=1}^3u_dcell_d$ with $0<u_d<1$ (``cell'' refers to either the supercell or user-provided cell).}
  \item{\texttt{center}: The grid volume is defined as $center+\sum_{d=1}^3u_dcell_d$ with $-1/2<u_d<1/2$ (``cell'' refers to either the supercell or user-provided cell).  \texttt{corner/center} can be used to shift the grid even if \texttt{cell} is not specified.  Simultaneous use of \texttt{corner} and \texttt{center} will cause QMCPACK to abort.}
\end{itemize}

% spin density estimators
\begin{lstlisting}[style=QMCPXML,caption=Spin density estimator (uniform grid).]
  <estimator type="spindensity" name="SpinDensity" report="yes">
    <parameter name="grid"> 40 40 40 </parameter>
  </estimator>
\end{lstlisting}

\begin{lstlisting}[style=QMCPXML,caption=Spin density estimator (uniform grid centered about origin).]
  <estimator type="spindensity" name="SpinDensity" report="yes">
    <parameter name="grid">
      20 20 20
    </parameter>
    <parameter name="center">
      0.0 0.0 0.0
    </parameter>
    <parameter name="cell">
      10.0  0.0  0.0
       0.0 10.0  0.0
       0.0  0.0 10.0
    </parameter>
  </estimator>
\end{lstlisting}
   


\subsection{Pair correlation function, $g(r)$}

The functional form of the species-resolved radial pair correlation function operator is
\begin{align}
  g_{ss'}(r) = \frac{V}{4\pi r^2N_sN_{s'}}\sum_{i_s=1}^{N_s}\sum_{j_{s'}=1}^{N_{s'}}\delta(r-|r_{i_s}-r_{j_{s'}}|)\:,
\end{align}
where $N_s$ is the number of particles of species $s$ and $V$ is the supercell volume.  If $s=s'$, then the sum is restricted so that $i_s\ne j_s$.

In QMCPACK, an estimate of $g_{ss'}(r)$ is obtained as a radial histogram with a set of $N_b$ uniform bins of width $\delta r$.  This can be expressed analytically as
\begin{align}
  \tilde{g}_{ss'}(r) = \frac{V}{4\pi r^2N_sN_{s'}}\sum_{i=1}^{N_s}\sum_{j=1}^{N_{s'}}\frac{1}{\delta r}\int_{r-\delta r/2}^{r+\delta r/2}dr'\delta(r'-|r_{si}-r_{s'j}|)\:,
\end{align}
where the radial coordinate $r$ is restricted to reside at the bin centers, $\delta r/2, 3 \delta r/2, 5 \delta r/2, \ldots$.

\FloatBarrier
\begin{table}[h]
\begin{center}
\begin{tabularx}{\linewidth}{l l l l l X }
\hline
\multicolumn{6}{l}{\texttt{estimator type=gofr} element} \\
\hline
\multicolumn{2}{l}{parent elements:} & \multicolumn{4}{l}{\texttt{hamiltonian, qmc}}\\
\multicolumn{2}{l}{child  elements:} & \multicolumn{4}{l}{\textit{None}}\\
\multicolumn{2}{l}{attributes}  & \multicolumn{4}{l}{}\\
   & \bfseries name       & \bfseries datatype & \bfseries values  & \bfseries default   & \bfseries description \\
   & \texttt{type}$^r$    &  text              & \textbf{gofr}     &                     & Must be gofr       \\
   & \texttt{name}$^o$    &  text              & \textit{anything} & any                 & \textit{No current function} \\
   & \texttt{num\_bin}$^r$&  integer           & $>1$              & 20                  & \# of histogram bins \\
   & \texttt{rmax}$^o$    &  real              & $>0$              & 10                  & Histogram extent (Bohr) \\
   & \texttt{dr}$^o$      &  real              & $>0$              & 0.5                 & \textit{No current function} \\%Histogram bin width (Bohr) \\
   & \texttt{debug}$^o$   &  boolean           & yes/no            & no                  & \textit{No current function} \\
   & \texttt{target}$^o$  &  text              &\texttt{particleset.name}&\texttt{hamiltonian.target}& Quantum particles \\   
   & \texttt{source/sources}$^o$&  text array  &\texttt{particleset.name}&\texttt{hamiltonian.target}& Classical particles\\
  \hline
\end{tabularx}
\end{center}
\end{table}
\FloatBarrier

Additional information:
\begin{itemize}
  \item{\texttt{num\_bin:} This is the number of bins in each species pair radial histogram.}
  \item{\texttt{rmax:} This is the maximum pair distance included in the histogram.  The uniform bin width is $\delta r=\texttt{rmax/num\_bin}$.  If periodic boundary conditions are used for any dimension of the simulation cell, then the default value of \texttt{rmax} is the simulation cell radius instead of 10 Bohr.  For open boundary conditions, the volume ($V$) used is 1.0 Bohr$^3$.}
  \item{\texttt{source/sources:} If unspecified, only pair correlations between each species of quantum particle will be measured.  For each classical particleset specified by \texttt{source/sources}, additional pair correlations between each quantum and classical species will be measured.  Typically there is only one classical particleset (e.g., \texttt{source="ion0"}), but there can be several in principle (e.g., \texttt{sources="ion0 ion1 ion2"}).}
  \item{\texttt{target:} The default value is the preferred usage (i.e., \texttt{target} does not need to be provided).}
  \item{Data is output to the \texttt{stat.h5} for each QMC subrun.  Individual histograms are named according to the quantum particleset and index of the pair.  For example, if the quantum particleset is named ``e" and there are two species (up and down electrons, say), then there will be three sets of histogram data in each \texttt{stat.h5} file named \texttt{gofr\_e\_0\_0},  \texttt{gofr\_e\_0\_1}, and  \texttt{gofr\_e\_1\_1} for up-up, up-down, and down-down correlations, respectively.}
\end{itemize}

\begin{lstlisting}[style=QMCPXML,caption=Pair correlation function estimator element.]
  <estimator type="gofr" name="gofr" num_bin="200" rmax="3.0" />
\end{lstlisting}
\begin{lstlisting}[style=QMCPXML,caption=Pair correlation function estimator element with additional electron-ion correlations.]
  <estimator type="gofr" name="gofr" num_bin="200" rmax="3.0" source="ion0" />
\end{lstlisting}


\subsection{Static structure factor, $S(k)$}

Let $\rho^e_{\mathbf{k}}=\sum_j e^{i \mathbf{k}\cdot\mathbf{r}_j^e}$ be the Fourier space electron density, with $\mathbf{r}^e_j$ being the coordinate of the j-th electron.  $\mathbf{k}$ is a wavevector commensurate with the simulation cell.  QMCPACK allows the user to accumulate the static electron structure factor $S(\mathbf{k})$ at all commensurate $\mathbf{k}$ such that $|\mathbf{k}| \leq (LR\_DIM\_CUTOFF) r_c$.  $N^e$ is the number of electrons, \texttt{LR\_DIM\_CUTOFF} is the optimized breakup parameter, and $r_c$ is the Wigner-Seitz radius.  It is defined as follows:
\begin{equation}
S(\mathbf{k}) = \frac{1}{N^e}\langle \rho^e_{-\mathbf{k}} \rho^e_{\mathbf{k}} \rangle\:.
\end{equation}

% has a CUDA counterpart, may be useful to understand difference between cpu and gpu estimators
% see HamiltonianFactory.cpp
%    SkEstimator_CUDA* apot=new SkEstimator_CUDA(*targetPtcl);

\FloatBarrier
\begin{table}[h]
\begin{center}
\begin{tabularx}{\textwidth}{l l l l l X }
\hline
\multicolumn{6}{l}{\texttt{estimator type=sk} element} \\
\hline
\multicolumn{2}{l}{parent elements:} & \multicolumn{4}{l}{\texttt{hamiltonian, qmc}}\\
\multicolumn{2}{l}{child  elements:} & \multicolumn{4}{l}{\textit{None}}\\
\multicolumn{2}{l}{attributes}  & \multicolumn{4}{l}{}\\
   & \bfseries name       & \bfseries datatype & \bfseries values  & \bfseries default   & \bfseries description \\
   & \texttt{type}$^r$    &  text              & sk      &                     & Must be sk       \\
   & \texttt{name}$^r$    &  text              & \textit{anything} & any                 & Unique name for estimator \\
   & \texttt{hdf5}$^o$    &  boolean           & yes/no            & no                  & Output to \texttt{stat.h5} (yes) or \texttt{scalar.dat} (no) \\
  \hline
\end{tabularx}
\end{center}
\end{table}
\FloatBarrier

Additional information:
\begin{itemize}
  \item{\texttt{name:} This is the unique name for estimator instance.  A data structure of the same name will appear in \texttt{stat.h5} output files.}
  \item{\texttt{hdf5:} If \texttt{hdf5==yes}, output data for $S(k)$ is directed to the \texttt{stat.h5} file (recommended usage).  If \texttt{hdf5==no}, the data is instead routed to the \texttt{scalar.dat} file, resulting in many columns of data with headings prefixed by \texttt{name} and postfixed by the k-point index (e.g., \texttt{sk\_0 sk\_1 \ldots sk\_1037 \ldots}).}
  \item{This estimator only works in periodic boundary conditions.  Its presence in the input file is ignored otherwise.}
  \item{This is not a species-resolved structure factor.  Additionally, for $\mathbf{k}$ vectors commensurate with the unit cell, $S(\mathbf{k})$ will include contributions from the static electronic density, thus meaning it wil not accurately measure the electron-electron density response.  }
\end{itemize}

\begin{lstlisting}[style=QMCPXML,caption=Static structure factor estimator element.]
  <estimator type="sk" name="sk" hdf5="yes"/>
\end{lstlisting}



\subsection{Species kinetic energy}
Record species-resolved kinetic energy instead of the total kinetic energy in the \verb|Kinetic| column of scalar.dat. \verb|SpeciesKineticEnergy| is arguably the simplest estimator in QMCPACK. The implementation of this estimator is detailed in \verb|manual/estimator/estimator_implementation.pdf|.

\FloatBarrier
\begin{table}[h]
\begin{center}
\begin{tabularx}{\textwidth}{l l l l l X }
\hline
\multicolumn{6}{l}{\texttt{estimator type=specieskinetic} element} \\
\hline
\multicolumn{2}{l}{parent elements:} & \multicolumn{4}{l}{\texttt{hamiltonian, qmc}}\\
\multicolumn{2}{l}{child  elements:} & \multicolumn{4}{l}{\textit{None}}\\
\multicolumn{2}{l}{attributes}  & \multicolumn{4}{l}{}\\
   & \bfseries name       & \bfseries datatype & \bfseries values  & \bfseries default   & \bfseries description \\
   & \texttt{type}$^r$    &  text              & specieskinetic      &                     & Must be specieskinetic       \\
   & \texttt{name}$^r$    &  text              & \textit{anything} & any                 & Unique name for estimator \\
   & \texttt{hdf5}$^o$    &  boolean           & yes/no            & no                  & Output to \texttt{stat.h5} (yes) \\
  \hline
\end{tabularx}
\end{center}
\end{table}
\FloatBarrier


\begin{lstlisting}[style=QMCPXML,caption=Species kinetic energy estimator element.]
  <estimator type="specieskinetic" name="skinetic" hdf5="no"/>
\end{lstlisting}



\subsection{Lattice deviation estimator}
Record deviation of a group of particles in one particle set (target) from a group of particles in another particle set (source).

\FloatBarrier
\begin{table}[h]
\begin{center}
\begin{tabularx}{\textwidth}{l l l l l X }
\hline
\multicolumn{6}{l}{\texttt{estimator type=latticedeviation} element} \\
\hline
\multicolumn{2}{l}{parent elements:} & \multicolumn{4}{l}{\texttt{hamiltonian, qmc}}\\
\multicolumn{2}{l}{child  elements:} & \multicolumn{4}{l}{\textit{None}}\\
\multicolumn{2}{l}{attributes}  & \multicolumn{4}{l}{}\\
   & \bfseries name       & \bfseries datatype & \bfseries values  & \bfseries default   & \bfseries description \\
   & \texttt{type}$^r$    &  text              & latticedeviation      &                     & Must be latticedeviation       \\
   & \texttt{name}$^r$    &  text              & \textit{anything} & any                 & Unique name for estimator \\
   & \texttt{hdf5}$^o$    &  boolean           & yes/no            & no                  & Output to \texttt{stat.h5} (yes) \\
   & \texttt{per\_xyz}$^o$    &  boolean           & yes/no            & no                  & Directionally resolved (yes) \\
   & \texttt{source}$^r$    &  text           & e/ion0/\dots         & no                  & source particleset \\
   & \texttt{sgroup}$^r$    &  text           & u/d/\dots         & no                  & source particle group \\
   & \texttt{target}$^r$    &  text           & e/ion0/\dots         & no                  & target particleset \\
   & \texttt{tgroup}$^r$    &  text           & u/d/\dots         & no                  & target particle group \\
  \hline
\end{tabularx}
\end{center}
\end{table}
\FloatBarrier

Additional information:
\begin{itemize}
  \item{\texttt{source}: The ``reference'' particleset to measure distances from; actual reference points are determined together with \verb|sgroup|.}
  \item{\texttt{sgroup}: The ``reference'' particle group to measure distances from.} 
  \item{\texttt{source}: The ``target'' particleset to measure distances to.}
  \item{\texttt{sgroup}: The ``target'' particle group to measure distances to. For example, in Listing~\ref{lst:latdev}, the distance from the up electron (``u'') to the origin of the coordinate system is recorded.}
  \item{\texttt{per\_xyz}: Used to record direction-resolved distance. In Listing~\ref{lst:latdev}, the x,y,z coordinates of the up electron will be recorded separately if \texttt{per\_xyz=yes}.}
  \item{\texttt{hdf5}: Used to record particle-resolved distances in the h5 file if \texttt{gdf5=yes}.}
\end{itemize}

\begin{lstlisting}[style=QMCPXML,caption={Lattice deviation estimator element.},label={lst:latdev}]
  <particleset name="e" random="yes">
    <group name="u" size="1" mass="1.0">
       <parameter name="charge"              >    -1                    </parameter>
       <parameter name="mass"                >    1.0                   </parameter>
    </group>
    <group name="d" size="1" mass="1.0">
       <parameter name="charge"              >    -1                    </parameter>
       <parameter name="mass"                >    1.0                   </parameter>
    </group>
  </particleset>
  
  <particleset name="wf_center">
    <group name="origin" size="1">
      <attrib name="position" datatype="posArray" condition="0">
               0.00000000        0.00000000        0.00000000
      </attrib>
    </group>
  </particleset>
  
  <estimator type="latticedeviation" name="latdev" hdf5="yes" per_xyz="yes"
    source="wf_center" sgroup="origin" target="e" tgroup="u"/>
\end{lstlisting}




\subsection{Energy density estimator}
An energy density operator, $\hat{\mathcal{E}}_r$,  satisfies
\begin{align}
  \int dr \hat{\mathcal{E}}_r = \hat{H},
\end{align}
where the integral is over all space and $\hat{H}$ is the Hamiltonian.  In \qmcpack, the energy density is split into kinetic and potential components
\begin{align}
  \hat{\mathcal{E}}_r = \hat{\mathcal{T}}_r + \hat{\mathcal{V}}_r\:, 
\end{align}
with each component given by
\begin{align}
   \hat{\mathcal{T}}_r &=  \frac{1}{2}\sum_i\delta(r-r_i)\hat{p}_i^2 \\  
   \hat{\mathcal{V}}_r &=  \sum_{i<j}\frac{\delta(r-r_i)+\delta(r-r_j)}{2}\hat{v}^{ee}(r_i,r_j)
              + \sum_{i\ell}\frac{\delta(r-r_i)+\delta(r-\tilde{r}_\ell)}{2}\hat{v}^{eI}(r_i,\tilde{r}_\ell) \nonumber\\ 
    &\qquad   + \sum_{\ell< m}\frac{\delta(r-\tilde{r}_\ell)+\delta(r-\tilde{r}_m)}{2}\hat{v}^{II}(\tilde{r}_\ell,\tilde{r}_m)\:.\nonumber
\end{align}
Here, $r_i$ and $\tilde{r}_\ell$ represent electron and ion positions, respectively; $\hat{p}_i$ is a single electron momentum operator; and $\hat{v}^{ee}(r_i,r_j)$, $\hat{v}^{eI}(r_i,\tilde{r}_\ell)$, and $\hat{v}^{II}(\tilde{r}_\ell,\tilde{r}_m)$ are the electron-electron, electron-ion, and ion-ion pair potential operators (including nonlocal pseudopotentials, if present).  This form of the energy density is size consistent; that is, the partially integrated energy density operators of well-separated atoms gives the isolated Hamiltonians of the respective atoms.  For periodic systems with twist-averaged boundary conditions, the energy density is formally correct only for either a set of supercell k-points that correspond to real-valued wavefunctions or a k-point set that has inversion symmetry around a k-point having a real-valued wavefunction.  For more information about the energy density, see Ref. \cite{Krogel2013}.

In \qmcpack, the energy density can be accumulated on piecewise uniform 3D grids in generalized Cartesian, cylindrical, or spherical coordinates.  The energy density integrated within Voronoi volumes centered on ion positions is also available.  The total particle number density is also accumulated on the same grids by the energy density estimator for convenience so that related quantities, such as the regional energy per particle, can be computed easily.


\FloatBarrier
\begin{table}[h]
\begin{center}
\begin{tabularx}{\textwidth}{l l l l l X }
\hline
\multicolumn{6}{l}{\texttt{estimator type=EnergyDensity} element} \\
\hline
\multicolumn{2}{l}{parent elements:} & \multicolumn{4}{l}{\texttt{hamiltonian, qmc}}\\
\multicolumn{2}{l}{child  elements:} & \multicolumn{4}{l}{\texttt{reference\_points, spacegrid}}\\
\multicolumn{2}{l}{attributes}  & \multicolumn{4}{l}{}\\
   &   \bfseries name     & \bfseries datatype & \bfseries values & \bfseries default   & \bfseries description \\
   & \texttt{type}$^r$    &  text              & \textbf{EnergyDensity}    &                  & Must be EnergyDensity     \\
   & \texttt{name}$^r$    &  text              & \textit{anything}         &                  & Unique name for estimator \\
   & \texttt{dynamic}$^r$ &  text              & \texttt{particleset.name} &                  & Identify electrons \\
   & \texttt{static}$^o$  &  text              & \texttt{particleset.name} &                  & Identify ions  \\
   
  \hline
\end{tabularx}
\end{center}
\end{table}
\FloatBarrier

Additional information:
\begin{itemize}
  \item{\texttt{name:}  Must be unique.  A dataset with blocked statistical data for the energy density will appear in the \texttt{stat.h5} files labeled as \texttt{name}.}
\end{itemize}


\begin{lstlisting}[style=QMCPXML,caption=Energy density estimator accumulated on a $20 \times  10 \times 10$ grid over the simulation cell.]
  <estimator type="EnergyDensity" name="EDcell" dynamic="e" static="ion0">
    <spacegrid coord="cartesian">
      <origin p1="zero"/>
      <axis p1="a1" scale=".5" label="x" grid="-1 (.05) 1"/>
      <axis p1="a2" scale=".5" label="y" grid="-1 (.1) 1"/>
      <axis p1="a3" scale=".5" label="z" grid="-1 (.1) 1"/>
    </spacegrid>
  </estimator>
\end{lstlisting}


\begin{lstlisting}[style=QMCPXML,caption=Energy density estimator accumulated within spheres of radius 6.9 Bohr centered on the first and second atoms in the ion0 particleset.]
  <estimator type="EnergyDensity" name="EDatom" dynamic="e" static="ion0">
    <reference_points coord="cartesian">
      r1 1 0 0 
      r2 0 1 0
      r3 0 0 1
    </reference_points>
    <spacegrid coord="spherical">
      <origin p1="ion01"/>
      <axis p1="r1" scale="6.9" label="r"     grid="0 1"/>
      <axis p1="r2" scale="6.9" label="phi"   grid="0 1"/>
      <axis p1="r3" scale="6.9" label="theta" grid="0 1"/>
    </spacegrid>
    <spacegrid coord="spherical">
      <origin p1="ion02"/>
      <axis p1="r1" scale="6.9" label="r"     grid="0 1"/>
      <axis p1="r2" scale="6.9" label="phi"   grid="0 1"/>
      <axis p1="r3" scale="6.9" label="theta" grid="0 1"/>
    </spacegrid>
  </estimator>
\end{lstlisting}


\begin{lstlisting}[style=QMCPXML,caption=Energy density estimator accumulated within Voronoi polyhedra centered on the ions.]
  <estimator type="EnergyDensity" name="EDvoronoi" dynamic="e" static="ion0">
    <spacegrid coord="voronoi"/>
  </estimator>
\end{lstlisting}



The \texttt{<reference\_points/>} element provides a set of points for later use in specifying the origin and coordinate axes needed to construct a spatial histogramming grid.  Several reference points on the surface of the simulation cell (see Table~\ref{tab:ref_points}), as well as the positions of the ions (see the \texttt{energydensity.static} attribute), are made available by default.  The reference points can be used, for example, to construct a cylindrical grid along a bond with the origin on the bond center. 

\FloatBarrier
\begin{table}[h]
\begin{center}
\begin{tabularx}{\textwidth}{l l l l l X }
\hline
\multicolumn{6}{l}{\texttt{reference\_points} element} \\
\hline
\multicolumn{2}{l}{parent elements:} & \multicolumn{4}{l}{\texttt{estimator type=EnergyDensity}}\\
\multicolumn{2}{l}{child  elements:} & \multicolumn{4}{l}{\textit{None}}\\
\multicolumn{2}{l}{attributes}  & \multicolumn{4}{l}{}\\
   &   \bfseries name     & \bfseries datatype & \bfseries values & \bfseries default   & \bfseries description \\
   &   \texttt{coord}$^r$ &  text              & Cartesian/cell   &                     & Specify coordinate system \\
\multicolumn{2}{l}{body text}  & \multicolumn{4}{l}{}\\
   &                           & \multicolumn{4}{l}{The body text is a line formatted list of points with labels}     \\
  \hline
\end{tabularx}
\end{center}
\end{table}
\FloatBarrier

Additional information
\begin{itemize}
  \item{\texttt{coord:} If \texttt{coord=cartesian}, labeled points are in Cartesian (x,y,z) format in units of Bohr.  If \texttt{coord=cell}, then labeled points are in units of the simulation cell axes.}
  \item{\texttt{body text:}  The list of points provided in the body text are line formatted, with four entries per line (\textit{label} \textit{coor1} \textit{coor2} \textit{coor3}}).  A set of points referenced to the simulation cell is available by default (see Table~\ref{tab:ref_points}).  If \texttt{energydensity.static} is provided, the location of each individual ion is also available (e.g., if \texttt{energydensity.static=ion0}, then the location of the first atom is available with label ion01, the second with ion02, etc.). All points can be used by label when constructing spatial histogramming grids (see the following \texttt{spacegrid} element) used to collect energy densities.    
\end{itemize}


\FloatBarrier
\begin{table}[h]
\begin{center}
\caption{Reference points available by default.  Vectors $a_1$, $a_2$, and $a_3$ refer to the simulation cell axes.  The representation of the cell is centered around \texttt{zero}.\label{tab:ref_points}}
\begin{tabular}{l l l}
\hline
\texttt{label} & \texttt{point} & \texttt{description} \\
\hline
\texttt{zero} & 0 0 0    & Cell center  \\
\texttt{a1}   &  $a_1$   & Cell axis 1  \\
\texttt{a2}   &  $a_2$   & Cell axis 2  \\
\texttt{a3}   &  $a_3$   & Cell axis 3  \\
\texttt{f1p}  &  $a_1$/2 & Cell face 1+ \\
\texttt{f1m}  & -$a_1$/2 & Cell face 1- \\
\texttt{f2p}  &  $a_2$/2 & Cell face 2+ \\
\texttt{f2m}  & -$a_2$/2 & Cell face 2- \\
\texttt{f3p}  &  $a_3$/2 & Cell face 3+ \\
\texttt{f3m}  & -$a_3$/2 & Cell face 3- \\
\texttt{cppp} & $(a_1+a_2+a_3)/2$  & Cell corner +,+,+ \\
\texttt{cppm} & $(a_1+a_2-a_3)/2$  & Cell corner +,+,- \\
\texttt{cpmp} & $(a_1-a_2+a_3)/2$  & Cell corner +,-,+ \\
\texttt{cmpp} & $(-a_1+a_2+a_3)/2$ & Cell corner -,+,+ \\
\texttt{cpmm} & $(a_1-a_2-a_3)/2$  & Cell corner +,-,- \\
\texttt{cmpm} & $(-a_1+a_2-a_3)/2$ & Cell corner -,+,- \\
\texttt{cmmp} & $(-a_1-a_2+a_3)/2$ & Cell corner -,-,+ \\
\texttt{cmmm} & $(-a_1-a_2-a_3)/2$ & Cell corner -,-,- \\
\hline
\end{tabular}
\end{center}
\end{table}
\FloatBarrier



The \texttt{<spacegrid/>} element is used to specify a spatial histogramming grid for the energy density.  Grids are constructed based on a set of, potentially nonorthogonal, user-provided coordinate axes.  The axes are based on information available from \texttt{reference\_points}.  Voronoi grids are based only on nearest neighbor distances between electrons and ions.  Any number of space grids can be provided to a single energy density estimator.


\FloatBarrier
\begin{table}[h]
\begin{center}
\begin{tabularx}{\textwidth}{l l l l l X }
\hline
\multicolumn{6}{l}{\texttt{spacegrid} element} \\
\hline
\multicolumn{2}{l}{parent elements:} & \multicolumn{4}{l}{\texttt{estimator type=EnergyDensity}}\\
\multicolumn{2}{l}{child  elements:} & \multicolumn{4}{l}{\texttt{origin, axis}}\\
\multicolumn{2}{l}{attributes}  & \multicolumn{4}{l}{}\\
   &   \bfseries name     & \bfseries datatype & \bfseries values & \bfseries default   & \bfseries description \\
   &   \texttt{coord}$^r$ &  text              & Cartesian        &                     & Specify coordinate system \\
   &                      &                    & cylindrical      &                     &                           \\
   &                      &                    & spherical        &                     &                           \\
   &                      &                    & Voronoi          &                     &                           \\
  \hline
\end{tabularx}
\end{center}
\end{table}
\FloatBarrier


The \texttt{<origin/>} element gives the location of the origin for a non-Voronoi grid.\\

\FloatBarrier
\begin{table}[h]
\begin{center}
\begin{tabularx}{\textwidth}{l l l l l X }
\hline
\multicolumn{6}{l}{\texttt{origin} element} \\
\hline
\multicolumn{2}{l}{parent elements:} & \multicolumn{4}{l}{\texttt{spacegrid}}\\
\multicolumn{2}{l}{child  elements:} & \multicolumn{4}{l}{\textit{None}}\\
\multicolumn{2}{l}{attributes}  & \multicolumn{4}{l}{}\\
   &   \bfseries name     & \bfseries datatype & \bfseries values & \bfseries default   & \bfseries description \\
   &   \texttt{p1}$^r$      &  text              & \texttt{reference\_point.label}   &    &  Select end point       \\
   &   \texttt{p2}$^o$      &  text              & \texttt{reference\_point.label}   &    &  Select end point       \\
   &   \texttt{fraction}$^o$&  real              &                  &  0                  &  Interpolation fraction \\
  \hline
\end{tabularx}
\end{center}
\end{table}

Additional information:
\begin{itemize}
  \item{\texttt{p1/p2/fraction:} The location of the origin is set to \texttt{p1+fraction*(p2-p1)}.  If only \texttt{p1} is provided, the origin is at \texttt{p1}.}
\end{itemize}
\FloatBarrier


The \texttt{<axis/>} element represents a coordinate axis used to construct the, possibly curved, coordinate system for the histogramming grid.  Three \texttt{<axis/>} elements must be provided to a non-Voronoi \texttt{<spacegrid/>} element.

\FloatBarrier
\begin{table}[h]
\begin{center}
\begin{tabularx}{\textwidth}{l l l l l X }
\hline
\multicolumn{6}{l}{\texttt{axis} element} \\
\hline
\multicolumn{2}{l}{parent elements:} & \multicolumn{4}{l}{\texttt{spacegrid}}\\
\multicolumn{2}{l}{child  elements:} & \multicolumn{4}{l}{\textit{None}}\\
\multicolumn{2}{l}{attributes}  & \multicolumn{4}{l}{}\\
   &   \bfseries name        & \bfseries datatype & \bfseries values & \bfseries default   & \bfseries description \\
   &   \texttt{label}$^r$    &  text              & \textit{See below}&                    &  Axis/dimension label \\
   &   \texttt{grid}$^r$     &  text              &                  & ``0 1"             &  Grid ranges/intervals \\
   &   \texttt{p1}$^r$       &  text              & \texttt{reference\_point.label}   &    &  Select end point     \\
   &   \texttt{p2}$^o$       &  text              & \texttt{reference\_point.label}   &    &  Select end point     \\
   &   \texttt{scale}$^o$    &  real              &                  &                     &  Interpolation fraction\\
  \hline
\end{tabularx}
\end{center}
\end{table}
\FloatBarrier
Additional information:
\begin{itemize}
  \item{\texttt{label:} The allowed set of axis labels depends on the coordinate system (i.e., \texttt{spacegrid.coord}).  Labels are \texttt{x/y/z} for \texttt{coord=cartesian}, \texttt{r/phi/z} for \texttt{coord=cylindrical}, \texttt{r/phi/theta} for \texttt{coord=spherical}.}
  \item{\texttt{p1/p2/scale:} The axis vector is set to \texttt{p1+scale*(p2-p1)}.  If only \texttt{p1} is provided, the axis vector is \texttt{p1}.}
  \item{\texttt{grid:} The grid specifies the histogram grid along the direction specified by \texttt{label}.  The allowed grid points fall in the range [-1,1] for \texttt{label=x/y/z} or [0,1] for \texttt{r/phi/theta}.  A grid of 10 evenly spaced points between 0 and 1 can be requested equivalently by \texttt{grid="0 (0.1) 1"} or  \texttt{grid="0 (10) 1."}  Piecewise uniform grids covering portions of the range are supported, e.g., \texttt{grid="-0.7 (10) 0.0 (20) 0.5."}  }
  \item{Note that \texttt{grid} specifies the histogram grid along the (curved) coordinate given by \texttt{label}.  The axis specified by \texttt{p1/p2/scale} does not correspond one-to-one with \texttt{label} unless \texttt{label=x/y/z}, but the full set of axes provided defines the (sheared) space on top of which the curved (e.g., spherical) coordinate system is built. }
\end{itemize}






\subsection{One body density matrix}
The N-body density matrix in DMC is $\hat{\rho}_N=\operator{\Psi_{T}}{}{\Psi_{FN}}$ (for VMC, substitute $\Psi_T$ for $\Psi_{FN}$).  The one body reduced density matrix (1RDM) is obtained by tracing out all particle coordinates but one:
\begin{align}
  \hat{n}_1 &= \sum_nTr_{R_n}\operator{\Psi_{T}}{}{\Psi_{FN}}\:.
\end{align}
In this formula, the sum is over all electron indices and $Tr_{R_n}(*)\equiv\int dR_n\expval{R_n}{*}{R_n}$ with $R_n=[r_1,...,r_{n-1},r_{n+1},...,r_N]$.  When the sum is restricted over spin-up or spin-down electrons, one obtains a density matrix for each spin species.  The 1RDM computed by \qmcpack is partitioned in this way.

In real space, the matrix elements of the 1RDM are
\begin{align}
  n_1(r,r') &= \expval{r}{\hat{n}_1}{r'} = \sum_n\int dR_n \Psi_T(r,R_n)\Psi_{FN}^*(r',R_n)\:. 
\end{align}

A more efficient and compact representation of the 1RDM is obtained by expanding in the SPOs obtained from a Hartree-Fock or DFT calculation, $\{\phi_i\}$:
\begin{align}\label{eq:dm1b_direct}
  n_1(i,j) &= \expval{\phi_i}{\hat{n}_1}{\phi_j} \nonumber \\
           &= \int dR \Psi_{FN}^*(R)\Psi_{T}(R) \sum_n\int dr'_n \frac{\Psi_T(r_n',R_n)}{\Psi_T(r_n,R_n)}\phi_i(r_n')^* \phi_j(r_n)\:.
\end{align} 

The integration over $r'$ in Eq. \ref{eq:dm1b_direct} is inefficient when one is also interested in obtaining matrices involving energetic quantities, such as the energy density matrix of Ref. \cite{Krogel2014} or the related (and more well known) generalized Fock matrix.  For this reason, an approximation is introduced as follows:
\begin{align}
    n_1(i,j) \approx \int dR \Psi_{FN}(R)^*\Psi_T(R)  \sum_n \int dr_n' \frac{\Psi_T(r_n',R_n)^*}{\Psi_T(r_n,R_n)^*}\phi_i(r_n)^* \phi_j(r_n')\:. 
\end{align}
For VMC, FN-DMC, FP-DMC, and RN-DMC this formula represents an exact sampling of the 1RDM corresponding to $\hat{\rho}_N^\dagger$ (see appendix A of Ref. \cite{Krogel2014} for more detail).




\FloatBarrier
\begin{table}[h]
\begin{center}
\begin{tabularx}{\textwidth}{l l l l l X }
\hline
\multicolumn{6}{l}{\texttt{estimator type=dm1b} element} \\
\hline
\multicolumn{2}{l}{parent elements:} & \multicolumn{4}{l}{\texttt{hamiltonian, qmc}}\\
\multicolumn{2}{l}{child  elements:} & \multicolumn{4}{l}{\textit{none}}\\
\multicolumn{2}{l}{attributes}  & \multicolumn{4}{l}{}\\
   &   \bfseries name     & \bfseries datatype & \bfseries values & \bfseries default   & \bfseries description \\
   & \texttt{type}$^r$    &  text              & \textbf{dm1b}    &                     & Must be dm1b     \\
   & \texttt{name}$^r$    &  text              & \textit{anything}&                     & Unique name for estimator \\
\multicolumn{2}{l}{parameters}  & \multicolumn{4}{l}{}\\
   &   \bfseries name     & \bfseries datatype & \bfseries values & \bfseries default   & \bfseries description \\
   &\texttt{basis}$^r$         &  text array   & sposet.name(s)   &                     & Orbital basis         \\
   &\texttt{integrator}$^o$    &  text         & uniform\_grid    & uniform\_grid       & Integration method    \\
   &                           &               & uniform          &                     &                       \\
   &                           &               & density          &                     &                       \\
   &\texttt{evaluator}$^o$     &  text         & loop/matrix      & loop                & Evaluation method     \\
   &\texttt{scale}$^o$         &  real         & $0<scale<1$      & 1.0                 & Scale integration cell\\
   &\texttt{center}$^o$        &  real array(3)&\textit{any point}&                     & Center of cell        \\
   &\texttt{points}$^o$        &  integer      & $>0$             & 10                  & Grid points in each dim\\
   &\texttt{samples}$^o$       &  integer      & $>0$             & 10                  & MC samples            \\
   &\texttt{warmup}$^o$        &  integer      & $>0$             & 30                  & MC warmup             \\
   &\texttt{timestep}$^o$      &  real         & $>0$             & 0.5                 & MC time step          \\
   &\texttt{use\_drift}$^o$    &  boolean      &  yes/no          & no                  & Use drift in VMC      \\
   &\texttt{check\_overlap}$^o$&  boolean      &  yes/no          & no                  & Print overlap matrix  \\
   &\texttt{check\_derivatives}$^o$& boolean   &  yes/no          & no                  & Check density derivatives \\
   &\texttt{acceptance\_ratio}$^o$&  boolean   &  yes/no          & no                  & Print accept ratio    \\
   &\texttt{rstats}$^o$        &  boolean      &  yes/no          & no                  & Print spatial stats   \\
   &\texttt{normalized}$^o$    &  boolean      &  yes/no          & yes                 & \texttt{basis} comes norm'ed \\
   &\texttt{volume\_normed}$^o$&  boolean      &  yes/no          & yes                 & \texttt{basis} norm is volume \\
   &\texttt{energy\_matrix}$^o$&  boolean      & yes/no           & no                  & Energy density matrix \\
  \hline
\end{tabularx}
\end{center}
\end{table}
\FloatBarrier

Additional information:
\begin{itemize}
  \item{\texttt{name:} Density matrix results appear in \texttt{stat.h5} files labeled according to \texttt{name}.}
  \item{\texttt{basis:} List \texttt{sposet.name}'s.  The total set of orbitals contained in all \texttt{sposet}'s comprises the basis (subspace) onto which the one body density matrix is projected.  This set of orbitals generally includes many virtual orbitals that are not occupied in a single reference Slater determinant.}
  \item{\texttt{integrator:} Select the method used to perform the additional single particle integration.  Options are \texttt{uniform\_grid} (uniform grid of points over the cell), \texttt{uniform} (uniform random sampling over the cell), and \texttt{density} (Metropolis sampling of approximate density, $\sum_{b\in \texttt{basis}}\abs{\phi_b}^2$, is not well tested, please check results carefully!)}.  Depending on the integrator selected, different subsets of the other input parameters are active.
  \item{\texttt{evaluator:} Select for-loop or matrix multiply implementations.  Matrix is preferred for speed.  Both implementations should give the same results, but please check as this has not been exhaustively tested.}
  \item{\texttt{scale:} Resize the simulation cell by scale for use as an integration volume (active for \texttt{integrator=uniform/uniform\_grid}).}
  \item{\texttt{center:} Translate the integration volume to center at this point (active for \texttt{integrator=uniform/\\uniform\_grid}). If \texttt{center} is not provided, the scaled simulation cell is used as is. }
  \item{\texttt{points:} Number of grid points in each dimension for \texttt{integrator=uniform\_grid}.  For example, \texttt{points=10} results in a uniform $10 \times 10 \times 10$ grid over the cell.}
  \item{\texttt{samples:} Sets the number of MC samples collected for each step (active for \texttt{integrator=uniform/\\density}).  }
  \item{\texttt{warmup:} Number of warmup Metropolis steps at the start of the run before data collection (active for \texttt{integrator=density}). }
  \item{\texttt{timestep:} Drift-diffusion time step used in Metropolis sampling (active for \texttt{integrator=density}).}
  \item{\texttt{use\_drift:} Enable drift in Metropolis sampling  (active for \texttt{integrator=density}).}
  \item{\texttt{check\_overlap:} Print the overlap matrix (computed via simple Riemann sums) to the log, then abort.  Note that subsequent analysis based on the 1RDM is simplest if the input orbitals are orthogonal.}
  \item{\texttt{check\_derivatives:} Print analytic and numerical derivatives of the approximate (sampled) density for several sample points, then abort. }
  \item{\texttt{acceptance\_ratio:} Print the acceptance ratio of the density sampling to the log for each step.}
  \item{\texttt{rstats:} Print statistical information about the spatial motion of the sampled points to the log for each step.}
  \item{\texttt{normalized:} Declare whether the inputted orbitals are normalized or not.  If \texttt{normalized=no}, direct Riemann integration over a $200 \times 200 \times 200$ grid will be used to compute the normalizations before use.}
  \item{\texttt{volume\_normed:} Declare whether the inputted orbitals are normalized to the cell volume (default) or not (a norm of 1.0 is assumed in this case).  Currently, B-spline orbitals coming from QE and HEG planewave orbitals native to QMCPACK are known to be volume normalized.}
  \item{\texttt{energy\_matrix:} Accumulate the one body reduced energy density matrix, and write it to \texttt{stat.h5}.  This matrix is not covered in any detail here; the interested reader is referred to Ref. \cite{Krogel2014}.}
\end{itemize}


\begin{lstlisting}[style=QMCPXML,caption=One body density matrix with uniform grid integration.]
  <estimator type="dm1b" name="DensityMatrices">
    <parameter name="basis"        >  spo_u spo_uv  </parameter>
    <parameter name="evaluator"    >  matrix        </parameter>
    <parameter name="integrator"   >  uniform_grid  </parameter>
    <parameter name="points"       >  4             </parameter>
    <parameter name="scale"        >  1.0           </parameter>
    <parameter name="center"       >  0 0 0         </parameter>
  </estimator>
\end{lstlisting}


\begin{lstlisting}[style=QMCPXML,caption=One body density matrix with uniform sampling.]
  <estimator type="dm1b" name="DensityMatrices">
    <parameter name="basis"        >  spo_u spo_uv  </parameter>
    <parameter name="evaluator"    >  matrix        </parameter>
    <parameter name="integrator"   >  uniform       </parameter>
    <parameter name="samples"      >  64            </parameter>
    <parameter name="scale"        >  1.0           </parameter>
    <parameter name="center"       >  0 0 0         </parameter>
  </estimator>
\end{lstlisting}


\begin{lstlisting}[style=QMCPXML,caption=One body density matrix with density sampling.]
  <estimator type="dm1b" name="DensityMatrices">
    <parameter name="basis"        >  spo_u spo_uv  </parameter>
    <parameter name="evaluator"    >  matrix        </parameter>
    <parameter name="integrator"   >  density       </parameter>
    <parameter name="samples"      >  64            </parameter>
    <parameter name="timestep"     >  0.5           </parameter>
    <parameter name="use_drift"    >  no            </parameter>
  </estimator>
\end{lstlisting}


\begin{lstlisting}[style=QMCPXML,caption={Example \texttt{sposet} initialization for density matrix use.  Occupied and virtual orbital sets are created separately, then joined (\texttt{basis="spo\_u spo\_uv"}).}]
  <sposet_builder type="bspline" href="../dft/pwscf_output/pwscf.pwscf.h5" tilematrix="1 0 0 0 1 0 0 0 1" twistnum="0" meshfactor="1.0" gpu="no" precision="single">
    <sposet type="bspline" name="spo_u"  group="0" size="4"/>
    <sposet type="bspline" name="spo_d"  group="0" size="2"/>
    <sposet type="bspline" name="spo_uv" group="0" index_min="4" index_max="10"/>
  </sposet_builder>
\end{lstlisting}


\begin{lstlisting}[style=QMCPXML,caption={Example \texttt{sposet} initialization for density matrix use.  Density matrix orbital basis created separately (\texttt{basis="dm\_basis"}).}]
  <sposet_builder type="bspline" href="../dft/pwscf_output/pwscf.pwscf.h5" tilematrix="1 0 0 0 1 0 0 0 1" twistnum="0" meshfactor="1.0" gpu="no" precision="single">
    <sposet type="bspline" name="spo_u"  group="0" size="4"/>
    <sposet type="bspline" name="spo_d"  group="0" size="2"/>
    <sposet type="bspline" name="dm_basis" size="50" spindataset="0"/>
  </sposet_builder>
\end{lstlisting}


%  <estimator type="dm1b" name="DensityMatrices">
%     <parameter name="energy_matrix"       >    yes                   </parameter>
%     <parameter name="integrator"          >    uniform_grid            </parameter>
%     <parameter name="points"              >    6                     </parameter>
%     <parameter name="scale"               >    1.0                   </parameter>
%     <parameter name="basis"               >
%        spo_dm
%     </parameter>
%     <parameter name="evaluator"           >    matrix                </parameter>
%     <parameter name="center">
%        0 0 0
%     </parameter>
%     <parameter name="check_overlap"       >    no                    </parameter>
%  </estimator>
%
%  <sposet_builder type="bspline" href="./dft/pwscf_output/pwscf.pwscf.h5" tilematrix="1 0 0 0 1 0 0 0 1" twistnum="0" meshfactor="1.0" gpu="no" precision="single" sort="0">
%    <sposet type="bspline" name="spo_u" size="4" spindataset="0"/>
%    <sposet type="bspline" name="spo_d" size="2" spindataset="1"/>
%    <sposet type="bspline" name="dm_basis" size="50" spindataset="0"/>
%  </sposet_builder>
%  <estimator type="dm1b" name="DensityMatrices">
%    <parameter name="energy_matrix"       >    yes                   </parameter>
%    <parameter name="integrator"          >    uniform_grid          </parameter>
%    <parameter name="points"              >    10                    </parameter>
%    <parameter name="scale"               >    1.0                   </parameter>
%    <parameter name="basis"               >    dm_basis              </parameter>
%    <parameter name="normalized"          >    no                    </parameter>
%    <parameter name="evaluator"           >    matrix                </parameter>
%    <parameter name="center"              >    0 0 0                 </parameter>
%    <parameter name="check_overlap"       >    no                    </parameter>
%    <parameter name="rstats"              >    no                    </parameter>
%  </estimator>
%
%
% found at /psi2/home/development/qmcpack/energy_density_matrix/tests/r6080_edm/02_atoms/runs/O/qmc/vmc.in.xml

%
%  <sposet_builder type="bspline" href="../dft/pwscf_output/pwscf.pwscf.h5" tilematrix="1 0 0 0 1 0 0 0 1" twistnum="0" meshfactor="1.0" gpu="no" precision="single">
%    <sposet type="bspline" name="spo_u"  group="0" size="4"/>
%    <sposet type="bspline" name="spo_d"  group="0" size="2"/>
%    <sposet type="bspline" name="spo_uv" group="0" index_min="4" index_max="10"/>
%  </sposet_builder>
%  <estimator type="dm1b" name="DensityMatrices">
%    <parameter name="basis"        >  spo_u spo_uv  </parameter>
%    <parameter name="energy_matrix">  yes           </parameter>
%    <parameter name="evaluator"    >  matrix        </parameter>
%    <parameter name="center"       >  0 0 0         </parameter>
%    <parameter name="rstats"           >  no        </parameter>
%    <parameter name="acceptance_ratio" >  no        </parameter>
%    <parameter name="check_overlap"    >  no        </parameter>
%    <parameter name="check_derivatives">  no        </parameter>
%    
%    <parameter name="integrator"   >  uniform_grid  </parameter>
%    <parameter name="points"       >  20            </parameter>
%    <parameter name="scale"        >  1.0           </parameter>
% 
%    <!--
%    <parameter name="integrator"   >  uniform       </parameter>
%    <parameter name="samples"      >  14          </parameter>
%    <parameter name="scale"        >  1.0           </parameter>
%    -->
%    
%    <!--
%    <parameter name="integrator"   >  density       </parameter>
%    <parameter name="timestep"     >  1.0           </parameter>
%    <parameter name="use_drift"    >  no            </parameter>
%    <parameter name="samples"      >  1000          </parameter>
%    -->
%    
%    <!--
%    <parameter name="integrator"   >  density       </parameter>
%    <parameter name="timestep"     >  1.0           </parameter>
%    <parameter name="use_drift"    >  yes           </parameter>
%    <parameter name="samples"      >  1000          </parameter>
%    -->
%  </estimator>


\section{Forward-Walking Estimators} \label{sec:forward_walking}
Forward walking is a method for sampling the pure fixed-node distribution $\langle \Phi_0 | \Phi_0\rangle$.  Specifically, one multiplies each walker's DMC mixed estimate for the observable $\mathcal{O}$, $\frac{\mathcal{O}(\mathbf{R})\Psi_T(\mathbf{R})}{\Psi_T(\mathbf{R})}$, by the weighting factor $\frac{\Phi_0(\mathbf{R})}{\Psi_T(\mathbf{R})}$.  As it turns out, this weighting factor for any walker $\mathbf{R}$ is proportional to the total number of descendants the walker will have after a sufficiently long projection time $\beta$.  

To forward walk on an observable, declare a generic forward-walking estimator within a \texttt{<hamiltonian>} block, and then specify the observables to forward walk on and the forward-walking parameters.  Here is a summary.\\  

\begin{table}[h]
\begin{center}
\begin{tabularx}{\textwidth}{l l l l l X }
\hline
\multicolumn{6}{l}{\texttt{estimator type=ForwardWalking} element} \\
\hline
\multicolumn{2}{l}{parent elements:} & \multicolumn{4}{l}{\texttt{hamiltonian, qmc}}\\
\multicolumn{2}{l}{child  elements:} & \multicolumn{4}{l}{\texttt{Observable}}\\
\multicolumn{2}{l}{attributes}  & \multicolumn{4}{l}{}\\
   & \bfseries name       & \bfseries datatype & \bfseries values  & \bfseries default   & \bfseries description \\
   & \texttt{type}$^r$    &  text              & \textbf{ForwardWalking}&                & Must be ``ForwardWalking" \\
   & \texttt{name}$^r$    &  text              & \textit{anything} & any                 & Unique name for estimator \\
  \hline
\end{tabularx}
\end{center}
\end{table}

\begin{table}[h]
\begin{center}
\begin{tabularx}{\textwidth}{l l l l l X }
\hline
\multicolumn{6}{l}{\texttt{Observable} element} \\
\hline
\multicolumn{2}{l}{parent elements:} & \multicolumn{4}{l}{\texttt{estimator, hamiltonian, qmc}}\\
\multicolumn{2}{l}{child  elements:} & \multicolumn{4}{l}{\textit{None}}\\
\multicolumn{2}{l}{attributes}  & \multicolumn{4}{l}{}\\
   & \bfseries name       & \bfseries datatype & \bfseries values  & \bfseries default   & \bfseries description \\
   & \texttt{name}$^r$    &  text              & \textit{anything} & any                 & Registered name of existing estimator on which to forward walk \\
   & \texttt{max}$^r$     &  integer             & $ > 0$     &                     & Maximum projection time in steps (\texttt{max}$=\beta/\tau$)      \\
   & \texttt{frequency}$^r$     &  text              & $\geq 1$      &                & Dump data only for every \texttt{frequency}-th  \\
      &  &               &     &                & to \texttt{scalar.dat} file   \\

  \hline
\end{tabularx}
\end{center}
\end{table}

Additional information:
\begin{itemize}
  \item{\textbf{Cost}:  Because histories of observables up to \texttt{max} time steps have to be stored, the memory cost of storing the nonforward-walked observables variables should be multiplied by $\texttt{max}$.  Although this is not an issue for items such as potential energy, it could be prohibitive for observables such as density, forces, etc.  }
  \item{\textbf{Naming Convention}: Forward-walked observables are automatically named \texttt{FWE\_name\_i}, where \texttt{i} is the forward-walked expectation value at time step \texttt{i}, and \texttt{name} is whatever name appears in the \texttt{<Observable>} block.  This is also how it will appear in the \texttt{scalar.dat} file.  }
\end{itemize}

In the following example case, QMCPACK forward walks on the potential energy for 300 time steps and dumps the forward-walked value at every time step.  

\begin{lstlisting}[style=QMCPXML,caption=Forward-walking estimator element.]
  <estimator name="fw" type="ForwardWalking">
      <Observable name="LocalPotential" max="300" frequency="1"/>
       <!--- Additional Observable blocks go here -->
   </estimator>
\end{lstlisting}






\section{``Force'' estimators} \label{sec:force_est}

% Force estimators added in CoulombPotentialFactory.cpp, HamiltonianFactory::addForceHam
QMCPACK supports force estimation by use of the Chiesa-Ceperly-Zhang (CCZ) estimator.  Currently, open and periodic boundary conditions are supported but for all-electron calculations only.  

Without loss of generality, the CCZ estimator for the z-component of the force on an ion centered at the origin is given by the following expression:
\begin{equation}
F_z = -Z \sum_{i=1}^{N_e}\frac{z_i}{r_i^3}[\theta(r_i-\mathcal{R}) + \theta(\mathcal{R}-r_i)\sum_{\ell=1}^{M}c_\ell r_i^\ell]\:.
\end{equation}

Z is the ionic charge, $M$ is the degree of the smoothing polynomial, $\mathcal{R}$ is a real-space cutoff of the sphere within which the bare-force estimator is smoothed, and $c_\ell$ are predetermined coefficients.  These coefficients are chosen to minimize the weighted mean square error between the bare force estimate and the s-wave filtered estimator.  Specifically, 
\begin{equation}
\chi^2 = \int_0^\mathcal{R}dr\,r^m\,[f_z(r) - \tilde{f}_z(r)]^2\:.
\end{equation}
Here, $m$ is the weighting exponent, $f_z(r)$ is the unfiltered radial force density for the z force component, and $\tilde{f}_z(r)$ is the smoothed polynomial function for the same force density.  The reader is invited to refer to the original paper for a more thorough explanation of the methodology, but with the notation in hand, QMCPACK takes the following parameters.
\FloatBarrier
\begin{table}[h]
\begin{center}
\begin{tabularx}{\textwidth}{l l l l l X }
\hline
\multicolumn{6}{l}{\texttt{estimator type=Force} element} \\
\hline
\multicolumn{2}{l}{parent elements:} & \multicolumn{4}{l}{\texttt{hamiltonian, qmc}}\\
\multicolumn{2}{l}{child  elements:} & \multicolumn{4}{l}{\texttt{parameter}}\\
\multicolumn{2}{l}{attributes}  & \multicolumn{4}{l}{}\\
   & \bfseries name       & \bfseries datatype & \bfseries values  & \bfseries default   & \bfseries description \\
   &   \texttt{mode}$^o$      &  text              & \textit{See above}        & bare          & Select estimator type\\
   &   \texttt{type}$^r$      &  text              &  Force            &               & Must be ``Force"         \\
   &   \texttt{name}$^o$      &  text              & \textit{anything}         & ForceBase     & Unique name for this estimator\\
%   &   \texttt{psi}$^o$       &  text              & \texttt{wavefunction.name}& psi0          & Identify wavefunction\\
   &   \texttt{pbc}$^o$       &  boolean           & yes/no                    & yes           & Using periodic BC's or not\\
   &   \texttt{addionion}$^o$       &  boolean           & yes/no                    & no           & Add the ion-ion force contribution to output force estimate   \\
   \multicolumn{2}{l}{parameters}  & \multicolumn{4}{l}{}\\
   & \bfseries name       & \bfseries datatype & \bfseries values  & \bfseries default   & \bfseries description \\
   &   \texttt{rcut}$^o$      &  real             & $> 0$        & 1.0         & Real-space cutoff $\mathcal{R}$ in bohr\\
   &   \texttt{nbasis}$^o$      &  integer              & $> 0 $           &  2            & Degree of smoothing polynomial $M$ \\
   &   \texttt{weightexp}$^o$      &  integer              &$ > 0$         & 2    & $\chi^2$ weighting exponent $m$\\  
  \hline
\end{tabularx}
\end{center}
\end{table}
\FloatBarrier

Additional information:
\begin{itemize}
  \item{\textbf{Naming Convention}:  The unique identifier \texttt{name} is appended with \texttt{name\_X\_Y} in the \texttt{scalar.dat} file, where \texttt{X} is the ion ID number and \texttt{Y} is the component ID (an integer with x=0, y=1, z=2).  All force components for all ions are computed and dumped to the \texttt{scalar.dat} file.}
  \item{\textbf{Miscellaneous}: Usually, the default choice of \texttt{weightexp} is sufficient.  Different combinations of \texttt{rcut} and  \texttt{nbasis} should be tested though to minimize variance and bias.  There is, of course, a tradeoff, with larger \texttt{nbasis} and smaller \texttt{rcut} leading to smaller biases and larger variances.  }
\end{itemize}

The following is an example use case.  
\begin{lstlisting}[style=QMCPXML]
<estimator name="myforce" type="Force" mode="cep" addionion="yes">
    <parameter name="rcut">0.1</parameter>
    <parameter name="nbasis">4</parameter>
    <parameter name="weightexp">2</parameter>
</estimator>
\end{lstlisting}




\chapter{Quantum Monte Carlo Methods}
\label{chap:qmcmethods}

\begin{table}[h]
\begin{center}
\begin{tabularx}{\linewidth}{l l l l l X }
\hline
\multicolumn{6}{l}{\texttt{qmc} factory element} \\
\hline
\multicolumn{2}{l}{parent elements:} & \multicolumn{4}{l}{\texttt{simulation, loop}}\\
\multicolumn{2}{l}{type   selector:} & \multicolumn{4}{l}{\texttt{method} attribute}\\
\multicolumn{2}{l}{type   options: } & vmc           & \multicolumn{3}{l}{Variational Monte Carlo}\\
%\multicolumn{2}{l}{                } & opt           & \multicolumn{3}{l}{}\\
\multicolumn{2}{l}{                } & linear        & \multicolumn{3}{l}{Wavefunction optimization with linear method}\\
%\multicolumn{2}{l}{                } & cslinear      & \multicolumn{3}{l}{}\\
\multicolumn{2}{l}{                } & dmc           & \multicolumn{3}{l}{Diffusion Monte Carlo}\\
\multicolumn{2}{l}{                } & rmc           & \multicolumn{3}{l}{Reptation Monte Carlo}\\
%\multicolumn{2}{l}{                } & ptcl          & \multicolumn{3}{l}{}\\
%\multicolumn{2}{l}{                } & mul           & \multicolumn{3}{l}{}\\
%\multicolumn{2}{l}{                } & warp          & \multicolumn{3}{l}{}\\
\multicolumn{2}{l}{shared attributes:} & \multicolumn{4}{l}{}\\
   &   \bfseries name         & \bfseries datatype & \bfseries values & \bfseries default & \bfseries description \\
   &   \texttt{method}        &  text              &   listed above   & invalid           & QMC driver            \\
   &   \texttt{move}          &  text              &   pbyp, alle     & pbyp              & Method used to move electrons \\
   &   \texttt{gpu}           &  text              &   yes, no        & dep.              & Use the GPU\\
   &   \texttt{trace}         &  text              &                  & no                & ???                      \\
   &   \texttt{checkpoint}   &  integer           &   -1, 0, n       & -1                & Checkpoint frequency \\
   &   \texttt{record}      &  integer           &   n              & 0                & Save configuration every n steps  \\
   &   \texttt{target}        &  text              &                  &                   & ???  \\
   &   \texttt{completed}     &  text              &                  &                   & ???  \\
   &   \texttt{append}        &  text              &   yes, no        & yes               & ???  \\
%   &   \texttt{multiple}      &  text              &   yes, no        & no                & ???  \\
%   &   \texttt{warp}          &  text              &   yes, no        & no                & ???  \\
\hline

\end{tabularx}
\end{center}
\end{table}

Additional information:
\begin{itemize}
\item \texttt{move}: There are two ways to move electrons. The more used method is the particle-by-particle move. In this method, only one electron is moved for acceptance or rejection. The other method is the all-electron move; namely, all the electrons are moved once for testing acceptance or rejection.

\item \texttt{gpu}: When the executable is compiled with CUDA, the target computing device can be chosen by this switch. With a regular CPU-only compilation, this option is not effective.

\item \texttt{checkpoint}:
This enables and disables checkpointing and specifying the frequency of output.  Possible values are:
\begin{description}
\item [-1] No checkpoint (default setting).
\item [0] Dump after the completion of a QMC section.
\item [n] Dump after every $n$ blocks.  Also dump at the end of the run.
\end{description}
%\item  ``-1'' No checkpoint (default setting).
%\item ``0'' Dump after the completion of a qmc section.
%\item ``n'' Dump after every $n$ blocks.  Also dump at the end of the run.
%\end{itemize}
%If Checkpoint=``-1'' no checkpoint will be done (default setting). If Checkpoint=``0'' dump after the completion of a qmc section. When dumconfig=``n'' is present with Checkpoint=``0'', the configurations will be dumped at the end of the run (due to Checkpoint=``0''), but also at every n block. If Checkpoint=``n'', configurations will be dump every n block. 

% TODO: Fill in more information about checkpoint/restart

The particle configurations are written to a \texttt{.config.h5} file.

%Other sections of data needed for
%restart include the random generator state, written to a \texttt{.random.xml} file.




% I think this description refers to a previous version of the config.h5 file.  See HDFWalkerInput0.cpp.
%All the dumped data will be written in a \texttt{*.config.h5} file. The \texttt{config.h5} file will contain the state of a population to continue a run. The list of what is included in the .config.h5 is: number of walkers, status of the run, branch mode, energy dataset, ratio to accepted moves, ratio to proposed moves, variance dataset, vParam\{tau, taueff. E\_trial, E\_ref, Branch\_Max, BranchCutOff, BranchFilter, Sigma, Accepted\_Energy, Accepted\_Samples\}, IParam{warmumSteps, Energy\_Update\_Interval, Counter, targetwalkers, Maxwalkers, MinWalkers, Branching Interval}, Walker coordinates, Random number size, Random number sequence, version of the code.

\begin{lstlisting}[style=QMCPXML,caption=The following is an example of running a simulation that can be restarted. ]
  <qmc method="dmc" move="pbyp"  checkpoint="0">
    <parameter name="timestep">         0.004  </parameter>
    <parameter name="blocks">           100   </parameter>
    <parameter name="steps">            400    </parameter>
  </qmc>
\end{lstlisting}

The checkpoint flag instructs QMCPACK to output walker configurations.  This also
works in VMC.  This outputs an h5 file with the name \texttt{projectid.run-number.config.h5}.
Check that this file exists before attempting a restart.

To continue a run, specify the \texttt{mcwalkerset} element before your VMC/DMC block:
\begin{lstlisting}[style=QMCPXML,caption=Restart (read walkers from previous run) ]
 <mcwalkerset fileroot="BH.s002" version="0 6" collected="yes"/>
  <qmc method="dmc" move="pbyp"  checkpoint="0">
    <parameter name="timestep">         0.004  </parameter>
    <parameter name="blocks">           100   </parameter>
    <parameter name="steps">            400    </parameter>
  </qmc>
\end{lstlisting}
\texttt{BH} is the project id, and \texttt{s002} is the calculation number to read in the walkers from the previous run.

In the project id section, make sure that the series number is different from any existing ones to avoid overwriting them. 

\end{itemize}

\section{Variational Monte Carlo}
\label{sec:vmc}

\begin{table}[h]
\begin{tabularx}{\textwidth}{l l l l l X }
\hline
\multicolumn{6}{l}{\texttt{vmc} method} \\
\hline
\multicolumn{2}{l}{parameters}  & \multicolumn{4}{l}{}\\
   &   \bfseries name     & \bfseries datatype & \bfseries values & \bfseries default   & \bfseries description \\
   &   \texttt{walkers             } &  integer  & $> 0$   & dep.& Number of walkers per MPI task  \\
   &   \texttt{blocks              } &  integer  & $\ge 0$ & 1   & Number of blocks            \\
   &   \texttt{steps               } &  integer  & $\ge 0$ & 1   & Number of steps per block   \\
   &   \texttt{warmupsteps         } &  integer  & $\ge 0$ & 0   & Number of steps for warming up\\
   &   \texttt{substeps            } &  integer  & $\ge 0$ & 1   & Number of substeps per step \\
   &   \texttt{usedrift            } &  text     & yes, no & yes  & Use the algorithm with drift\\
   &   \texttt{timestep            } &  real     & $> 0$   & 0.1 & Time step for each electron move \\
   &   \texttt{samples             } &  integer  & $\ge 0$ & 0   & Number of walker samples for DMC/optimization\\
   &   \texttt{stepsbetweensamples } &  integer  & $> 0$   & 1   & Period of sample accumulation\\
   &   \texttt{samplesperthread    } &  integer  & $\ge 0$ & 0   & Number of samples per thread  \\
   &   \texttt{storeconfigs        } &  integer  & all values & 0   & Store configurations o  \\
   &   \texttt{blocks\_between\_recompute} &  integer  & $\ge 0$ & dep.  & Wavefunction recompute frequency  \\
  \hline
\end{tabularx}
\end{table}

Additional information:
\begin{itemize}
\item \ixml{walkers}: The initial default number of \ixml{walkers} is one per OpenMP thread or per MPI task if threading is disabled, with a minimum of one per thread. One walker per thread is created in the event fewer \ixml{walkers} than threads are requested. 

\item \ixml{blocks}: This parameter is universal for all the QMC
  methods. The MC processes are divided into a number of
  \ixml{blocks}, each containing a number of steps. At the end of each block,
  the statistics accumulated in the block are dumped into files,
  e.g., \ixml{scalar.dat}. Typically, each block should have a sufficient number of steps that the I/O at the end of each block is negligible
  compared with the computational cost. Each block should not take so
  long that monitoring its progress is difficult. There should be a
  sufficient number of \ixml{blocks} to perform statistical analysis.

\item \ixml{warmupsteps}: \ixml{warmupsteps} are used only for
  equilibration. Property measurements are not performed during
  warm-up steps.

\item \ixml{steps}: \ixml{steps} are the number of energy and other property measurements to perform per block.
  
\item \ixml{substeps}: For each substep, an attempt is made to move each of the electrons once only by either particle-by-particle or an
  all-electron move.  Because the local energy is evaluated only at
  each full step and not each substep, \ixml{substeps} are computationally cheaper
  and can be used to reduce the correlation between property measurements
  at a lower cost.
  
\item \ixml{usedrift}: The VMC is implemented in two algorithms with
  or without drift. In the no-drift algorithm, the move of each
  electron is proposed with a Gaussian distribution. The standard
  deviation is chosen as the time step input. In the drift algorithm,
  electrons are moved by Langevin dynamics.

\item \ixml{timestep}: The meaning of time step depends on whether or not
  the drift is used. In general, larger time steps reduce the
  time correlation but might also reduce the acceptance ratio,
  reducing overall statistical efficiency. For VMC, typically the
  acceptance ratio should be close to 50\% for an efficient
  simulation.

\item \ixml{samples}: Seperate from conventional energy and other
  property measurements, samples refers to storing whole electron
  configurations in memory (``walker samples'') as would be needed by subsequent
  wavefunction optimization or DMC steps. \textit{A standard VMC run to
  measure the energy does not need samples to be set.}

\[
\texttt{samples}=
\frac{\texttt{blocks}\cdot\texttt{steps}\cdot\texttt{walkers}}{\texttt{stepsbetweensamples}}\cdot\texttt{number of MPI tasks}
\]

\item \ixml{samplesperthread}: This is an alternative way to set the target amount of samples and can be useful when preparing a stored population for a subsequent DMC calculation.
\[
\texttt{samplesperthread}=
\frac{\texttt{blocks}\cdot\texttt{steps}}{\texttt{stepsbetweensamples}}
\]

\item \ixml{stepsbetweensamples}: Because samples generated by consecutive steps are correlated, having \ixml{stepsbetweensamples} larger than 1 can be used to reduces that correlation. In practice, using larger substeps is cheaper than using \ixml{stepsbetweensamples} to decorrelate samples. 

\item \ixml{storeconfigs}: If \ixml{storeconfigs} is set to a nonzero value, then electron configurations during the VMC run are saved to files.

\item \ixml{blocks_between_recompute}: Recompute the accuracy critical determinant part of the wavefunction
  from scratch: =1 by default when using mixed precision. =0 (no
  recompute) by default when not using mixed precision. Recomputing
  introduces a performance penalty dependent on system size.
\end{itemize}

An example VMC section for a simple VMC run:
\begin{lstlisting}[style=QMCPXML]
  <qmc method="vmc" move="pbyp">
    <estimator name="LocalEnergy" hdf5="no"/>
    <parameter name="walkers">    256 </parameter>
    <parameter name="warmupSteps">  100 </parameter>
    <parameter name="substeps">  5 </parameter>
    <parameter name="blocks">  20 </parameter>
    <parameter name="steps">  100 </parameter>
    <parameter name="timestep">  1.0 </parameter>
    <parameter name="usedrift">   yes </parameter>
  </qmc>
\end{lstlisting}
Here we set 256 \ixml{walkers} per MPI, have a brief initial equilibration of 100 \ixml{steps}, and then have 20 \ixml{blocks} of 100 \ixml{steps} with 5 \ixml{substeps} each.

The following is an example of VMC section storing configurations (walker samples) for optimization.
\begin{lstlisting}[style=QMCPXML]
  <qmc method="vmc" move="pbyp" gpu="yes">
    <estimator name="LocalEnergy" hdf5="no"/>
    <parameter name="walkers">    256 </parameter>
    <parameter name="samples">    2867200 </parameter>
    <parameter name="stepsbetweensamples">    1 </parameter>
    <parameter name="substeps">  5 </parameter>
    <parameter name="warmupSteps">  5 </parameter>
    <parameter name="blocks">  70 </parameter>
    <parameter name="timestep">  1.0 </parameter>
    <parameter name="usedrift">   no </parameter>
  </qmc>
\end{lstlisting}




\section{Wavefunction Optimization}
\label{sec:optimization}

Optimizing wavefunction is critical in all kinds of real-space quantum Monte Carlo calculations
because it significantly improves both the accuracy and efficiency of computation.
However, it is very difficult to directly adopt deterministic minimization approaches due to the stochastic nature of evaluating quantities with Monte Carlo.
Thanks to the algorithmic breakthrough during the first decade of this century and the tremendous computer power available, 
it becomes feasible to optimize tens of thousands of parameters in a wavefunction for a solid or molecule.
QMCPACK has multiple optimizers implemented based on the state-of-the-art linear method.
We are continually improving our optimizers for the robustness and friendliness and trying to provide a single solution.
Due to the large variation of wavefunction types carrying distinct characteristics, using several optimizer may be needed in some cases.
It is highly suggested to read the recommendation from the experts maintaining these optimizers.

A typical optimization block looks like the following. It starts with method=``linear" and contains three blocks of parameters.
\begin{lstlisting}[style=QMCPXML]
 <loop max="10">
  <qmc method="linear" move="pbyp" gpu="yes">
    <!-- Specify the VMC options -->
    <parameter name="walkers">              256 </parameter>
    <parameter name="samples">          2867200 </parameter>
    <parameter name="stepsbetweensamples">    1 </parameter>
    <parameter name="substeps">               5 </parameter>
    <parameter name="warmupSteps">            5 </parameter>
    <parameter name="blocks">                70 </parameter>
    <parameter name="timestep">             1.0 </parameter>
    <parameter name="usedrift">              no </parameter>
    <estimator name="LocalEnergy" hdf5="no"/>
    ...
    <!-- Specify the correlated sampling options and define the cost function -->
    <parameter name="minwalkers">            0.3 </parameter>
         <cost name="energy">               0.95 </cost>
         <cost name="unreweightedvariance"> 0.00 </cost>
         <cost name="reweightedvariance">   0.05 </cost>
    ...
    <!-- Specify the optimizer options -->
    <parameter name="MinMethod">    OneShiftOnly </parameter>
    ...
  </qmc>
 </loop>
\end{lstlisting}
\begin{itemize}
\item loop is helpful to execute identical optimization blocks repeatedly.
\item The first part is highly identical to a regular VMC block.
\item The second part is to specify the correlated sampling options and define the cost function.
\item The last part is used to specify the options of different optimizers. They can be very distinct from one optimizer to another.
\end{itemize}

\subsection{VMC run for the optimization}
The VMC calculation for the wavefunction optimization has a strict requirement 
that \ixml{samples} or \ixml{samplesperthread} must be specified because of the optimizer needs for the stored samples.
The input parameters of this part are identical to the VMC method.

Recommendations:
\begin{itemize}
\item Run the inclusive VMC calculation correctly and efficiently, because the this part takes significant amount of time in optimization.
For example, make sure the derived steps per block is 1 and use larger substeps to control the correlation between samples.
\item A reasonable starting wavefunction is necessary. A lot of optimization fails because of a bad wavefunction starting point.
The sign of a bad initial wavefunction includes but not limited to very long equilibration time, low acceptance ratio and huge variance.
The first thing to do after a failed optimization is to check the information provided by the VMC calculation via *.scalar.dat files.
\end{itemize}

\subsection{Correlated sampling and Cost function}
After generating the samples with VMC, the derivatives of the wavefunction with respect to the parameters are computed for proposing a new set of parameters by optimizers.
And later, a correlated sampling calculation is performed to quickly evaluate values of the cost function on the old set of parameters and the new set for the further decisions.
The input parameters are listed in the following table.
\begin{table}[h]
\begin{center}
\begin{tabularx}{\textwidth}{l l l l l X }
\hline
\multicolumn{6}{l}{\texttt{linear} method} \\
\hline
\multicolumn{2}{l}{parameters}  & \multicolumn{4}{l}{}\\
   &   \bfseries name     & \bfseries datatype & \bfseries values & \bfseries default   & \bfseries description \\
   &   \texttt{nonlocalpp} &  text     & yes, no & no  & include non-local PP energy in the cost function\\
%   &   \texttt{GEVMethod} &  text     & mixed, H2 & mixed  & methods of generalized eigenvalue problem\\
%   &   \texttt{beta} &  real     & any value & 0.0  & a parameter for GEVMethod\\
%   &   \texttt{use\_nonlocalpp\_deriv} &  text     & yes, no & no  & include the derivatives of non-local PP\\
   &   \texttt{minwalkers} &  real     & 0--1   & 0.3 & lower bound of the effective weight\\
   &   \texttt{maxWeight} &  real     & $>1$   & 1e6 & Maximum weight allowed in reweighting\\
  \hline
\end{tabularx}
\end{center}
\end{table}

Additional information:
\begin{itemize}
\item \ixml{maxWeight}. The default should be good.
\item \ixml{nonlocalpp}. Non-local PP contribution to the local energy depends on the wavefunction.
When a new set of parameter is proposed, this contribution needs to be updated if the cost function consists of local energy.
Fortunately, non-local contribution is chosen small when making a PP for small locality error.
We can ignore its change and avoid the expensive computational cost.
GPU code has a implementation issue that large amount of memory is consumed with this option.
\item \ixml{minwalkers}. A CRITICAL parameter. When the ratio of effective samples to actual number of samples in a reweighting step goes lower than \ixml{minwalkers},
the proposed set of parameters is invalid. % The last set of acceptable parameters is kept.
\end{itemize}

The cost function consists of three components: energy, unreweighted variance and reweighted variance.
\begin{lstlisting}[style=QMCPXML]
     <cost name="energy">                   0.95 </cost>
     <cost name="unreweightedvariance">     0.00 </cost>
     <cost name="reweightedvariance">       0.05 </cost>
\end{lstlisting}

\subsection{Optimizers}
QMCPACK implements a few optimizers having different preference aiming for different priorities.
They can be switched among  `OneShiftOnly' (default), `adaptive' and `quartic' (old) by the following line in the optimization block.
\begin{lstlisting}
<parameter name="MinMethod"> THE METHOD YOU LIKE </parameter>
\end{lstlisting}

\subsubsection{OneShiftOnly}
The OneShiftOnly optimizer targets a fast optimization by moving parameters more aggressively. It works with OpenMP and GPU and can be considered for large systems.
This method relies on the effective weight of correlated sampling rather than the cost function value to justify a new set of parameters.
If effective weight is larger than \ixml{minwalkers}, the new set is taken no matter the cost function value decreases or not.
If a proposed set is rejected, the standard output prints the measured ratio of effective samples to the total number of samples
and adjustment on \ixml{minwalkers} can be made if needed.

\begin{table}[h]
\begin{center}
\begin{tabularx}{\textwidth}{l l l l l X }
\hline
\multicolumn{6}{l}{\texttt{linear} method} \\
\hline
\multicolumn{2}{l}{parameters}  & \multicolumn{4}{l}{}\\
   &   \bfseries name     & \bfseries datatype & \bfseries values & \bfseries default   & \bfseries description \\
   &   \texttt{shift\_i} &  real     & $>0$ & 0.01 & Direct stabilizer added to the Hamiltonian matrix\\
   &   \texttt{shift\_s} &  real     & $>0$ & 1.00 & Initial stabilizer based on the overlap matrix\\
  \hline
\end{tabularx}
\end{center}
\end{table}

Additional information:
\begin{itemize}
\item \ixml{shift_i}. This is the direct term added to the diagonal of the Hamiltonian matrix.
                         More stable but slower optimization with a large value.
\item \ixml{shift_s}. This is the initial value of the stabilizer based on the overlap matrix added to the Hamiltonian matrix.
                         More stable but slower optimization with a large value. The used value is auto-adjusted by the optimizer.
\end{itemize}


Recommendations:
\begin{itemize}
  \item Default \ixml{shift_i}, \ixml{shift_s} should be fine.
  \item For hard cases, increasing \ixml{shift_i} (factor of 5 or 10) can significantly stabilize the optimization by reducing the pace towards the optimal parameter set.
  \item If the VMC energy of the last optimization iterations grows significantly, increase \ixml{minwalkers} closer to 1 and make the optimization stable.
  \item If the first iterations of optimization are rejected on a reasonable initial wavefunction, 
        lower the \ixml{minwalkers} value based on the measured value printed in the standard output to accept the move.
\end{itemize}

It is recommended to use this optimizer in two sections with a very small \ixml{minwalkers} in the first and a large value in the second like the following.
In the very beginning, parameters are far away form optimal values and large changes are proposed by the optimizer.
Having a small \ixml{minwalkers} allows accepting these changes much easier.
When the energy gradually converges, we can have a large \ixml{minwalkers} to avoid risky parameter sets.
\begin{lstlisting}[style=QMCPXML]
 <loop max="6">
  <qmc method="linear" move="pbyp" gpu="yes">
    <!-- Specify the VMC options -->
    <parameter name="walkers">                1 </parameter>
    <parameter name="samples">            10000 </parameter>
    <parameter name="stepsbetweensamples">    1 </parameter>
    <parameter name="substeps">               5 </parameter>
    <parameter name="warmupSteps">            5 </parameter>
    <parameter name="blocks">                25 </parameter>
    <parameter name="timestep">             1.0 </parameter>
    <parameter name="usedrift">              no </parameter>
    <estimator name="LocalEnergy" hdf5="no"/>
    <!-- Specify the optimizer options -->
    <parameter name="MinMethod">    OneShiftOnly </parameter>
    <parameter name="minwalkers">           1e-4 </parameter>
  </qmc>
 </loop>
 <loop max="12">
  <qmc method="linear" move="pbyp" gpu="yes">
    <!-- Specify the VMC options -->
    <parameter name="walkers">                1 </parameter>
    <parameter name="samples">            20000 </parameter>
    <parameter name="stepsbetweensamples">    1 </parameter>
    <parameter name="substeps">               5 </parameter>
    <parameter name="warmupSteps">            2 </parameter>
    <parameter name="blocks">                50 </parameter>
    <parameter name="timestep">             1.0 </parameter>
    <parameter name="usedrift">              no </parameter>
    <estimator name="LocalEnergy" hdf5="no"/>
    <!-- Specify the optimizer options -->
    <parameter name="MinMethod">    OneShiftOnly </parameter>
    <parameter name="minwalkers">            0.5 </parameter>
  </qmc>
 </loop>
\end{lstlisting}

For each optimization step, you will see
\begin{lstlisting}
The new set of parameters is valid. Updating the trial wave function!
\end{lstlisting}
or
\begin{lstlisting}
The new set of parameters is not valid. Revert to the old set!
\end{lstlisting}
Occasional rejection is fine. Frequent rejection indicates potential problems and users should inspect the VMC calculation or change optimization strategy.
To track the progress of optimization, using command ``qmca -q ev *.scalar.dat'' to look at the VMC energy and variance for each optimization step.

\subsubsection{adaptive}

The default setting of the adaptive optimizer is to construct the linear method Hamiltonian and overlap matrices explicitly and add different shifts to the Hamiltonian matrix 
as ``stabilizers''.
The generalized eigenvalue problem is solved for each shift to obtain updates to the wave function parameters.
Then a correlated sampling is performed for each shift's updated wave function and the initial trial wave function
using the middle shift's updated wave function as the guiding function.
The cost function for these wave functions is compared, and the update corresponding to the best cost function is selected.
In the next iteration, the median magnitude of the stabilizers is set to that that generated the best update in the current iteration, thus adapting the magnitude of
the stabilizers automatically.

When the trial wave function contains more than ten thousand parameters, constructing and storing the linear method matrices may become a memory bottleneck. 
To avoid explicit construction of these matrices, the adaptive optimizer implements the block linear method (BLM) approach. \cite{Zhao:2017:blocked_lm}
The BLM tries to find an approximate 
solution $\vec{c}_{opt}$ to the standard LM generalized eigenvalue problem by dividing the variable space into a number of blocks
and making intelligent estimates for which directions within those blocks will be most important for constructing $\vec{c}_{opt}$.
which is then obtained by solving a smaller, more memory-efficient 
eigenproblem in the basis of these supposedly important block-wise directions. 

\begin{table}[h]
\begin{center}
\begin{tabularx}{\textwidth}{l l l l l X }
\hline
\multicolumn{6}{l}{\texttt{linear} method} \\
\hline
\multicolumn{2}{l}{parameters}  & \multicolumn{4}{l}{}\\
   &   \bfseries name     & \bfseries datatype & \bfseries values & \bfseries default   & \bfseries description \\
   %&   \texttt{stepsize} &  real     & 0--1 & 0.25  & Step size for moving parameters\\
   &   \texttt{max\_relative\_change} &  real     & $>0$ & 10.0 & Allowed change in cost function\\
   &   \texttt{max\_param\_change} &  real     & $>0$ & 0.3 & Allowed change in wave function parameter\\
   &   \texttt{shift\_i} &  real     & $>0$ & 0.01 & Initial diagonal      stabilizer added to the Hamiltonian matrix\\
   &   \texttt{shift\_s} &  real     & $>0$ & 1.00 & Initial overlap-based stabilizer added to the Hamiltonian matrix\\
   &   \texttt{target\_shift\_i} &  real     & any & -1.0 & Diagonal stabilizer value aimed for during adaptive method (disabled if $\leq$ 0)\\
   &   \texttt{cost\_increase\_tol} &  real     & $\geq 0$ & 0.0 & Tolerance for cost function increases\\
   &   \texttt{chase\_lowest} &  text   & yes, no & yes & Chase the lowest eigenvector in iterative solver\\
   &   \texttt{chase\_closest} &  text   & yes, no & no & Chase the eigenvector closest to initial guess\\
   &   \texttt{block\_lm} &  text   & yes, no & no & Use block linear method\\
   &   \texttt{nblocks} &  integer   & $>0$ &  & \# of blocks in BLM\\
   &   \texttt{nolds} &  integer   & $>0$ &  & \# of old update vectors used in BLM\\
   &   \texttt{nkept} &  integer   & $>0$ &  & \# of eigenvectors to keep per block in BLM\\
  \hline
\end{tabularx}
\end{center}
\end{table}

Additional information:
\begin{itemize}
  \item \ixml{shift_i}.  This is the initial coefficient used to scale the diagonal stabilizer.
                            More stable but slower optimization is expected with a large value.
                            The adaptive method will automatically adjust this value after each linear method iteration.
  \item \ixml{shift_s}.  This is the initial coefficient used to scale the overlap-based stabilizer.
                            More stable but slower optimization is expected with a large value.
                            The adaptive method will automatically adjust this value after each linear method iteration.
  \item \ixml{target_shift_i}.  If set greater than zero, the adaptive method will choose the update whose shift\_i value is closest to
                              this target value so long as the associated cost is within cost\_increase\_tol of the lowest cost.
                              Disable this behavior by setting target\_shift\_i to a negative number.
  \item \ixml{cost_increase_tol}.  Tolerance for cost function increases when selecting the best shift.
  \item \ixml{nblocks}.   This is the number of blocks used in block LM. The amount of memory required to store LM matrices decreases
                            with increased number of blocks. But the error introduced by BLM would increase with number of blocks.  
  \item \ixml{nolds}.     In BLM, the inter-block correlation is accounted for by including a small number of wave function update vectors
                            outside the block. Larger \ixml{nolds} would include more inter-block correlation and more accurate results, but 
                            also higher memory requirements. 
  \item \ixml{nkept}.     This is the number of update directions retained from each block in the BLM. If all directions are retained in each block, 
                            then the BLM becomes equivalent to the standard LM.  Retaining 5 or fewer directions per block is often sufficient.
\end{itemize}

Recommendations:
\begin{itemize}
  \item Default \ixml{shift_i}, \ixml{shift_s} should be fine. 
  \item When there are fewer than about 5,000 variables being optimized, the traditional LM is preferred as it has a lower overhead than the BLM when the number of variables is small.
  \item Initial experience with the BLM suggests that a few hundred blocks and a handful of \ixml{nolds} and \ixml{nkept}
        often provide a good balance between memory use and accuracy.  In general, using fewer blocks should be more accurate but will require more memory.
\end{itemize}

\begin{lstlisting}[style=QMCPXML]
 <loop max="15">
  <qmc method="linear" move="pbyp">
    <!-- Specify the VMC options -->
    <parameter name="walkers">                1 </parameter>
    <parameter name="samples">            20000 </parameter>
    <parameter name="stepsbetweensamples">    1 </parameter>
    <parameter name="substeps">               5 </parameter>
    <parameter name="warmupSteps">            5 </parameter>
    <parameter name="blocks">                50 </parameter>
    <parameter name="timestep">             1.0 </parameter>
    <parameter name="usedrift">              no </parameter>
    <estimator name="LocalEnergy" hdf5="no"/>
    <!-- Specify the correlated sampling options and define the cost function -->
         <cost name="energy">               1.00 </cost>
         <cost name="unreweightedvariance"> 0.00 </cost>
         <cost name="reweightedvariance">   0.00 </cost>
    <!-- Specify the optimizer options -->
    <parameter name="MinMethod">adaptive</parameter>
    <parameter name="max_relative_cost_change">10.0</parameter>
    <parameter name="shift_i"> 1.00 </parameter>
    <parameter name="shift_s"> 1.00 </parameter>
    <parameter name="max_param_change"> 0.3 </parameter>
    <parameter name="chase_lowest"> yes </parameter>
    <parameter name="chase_closest"> yes </parameter>
    <parameter name="block_lm"> no </parameter>
    <!-- Specify the BLM specific options if needed
      <parameter name="nblocks"> 100 </parameter>
      <parameter name="nolds"> 5 </parameter>
      <parameter name="nkept"> 3 </parameter>
    -->
  </qmc>
 </loop>
\end{lstlisting}
%To activate this optimizer, add ``-D BUILD\_LMYENGINE\_INTERFACE=1'' in the CMake command line.

The adaptive optimizer is also able to optimize individual excited states directly. \cite{Zhao:2016:dir_tar}
In this case, it tries to minimize the following function: 
\begin{equation*}
\Omega[\Psi]=\frac{\left<\Psi|\omega-H|\Psi\right>}{\left<\Psi|{\left(\omega-H\right)}^2|\Psi\right>}
\end{equation*}
The global minimum of this function corresponds to the state whose energy lies immediately above the shift parameter $\omega$ in the energy spectrum.
For example, if $\omega$ were placed in between the ground state energy and the first excited state energy and the wave function ansatz was capable of a good
description for the first excited state, then the wave function would be optimized for the first excited state.
It is important to note that, if the ansatz is not capable of a good description of the excited state in question, the optimization may converge to a different
state, as is known to occur in some circumstances for traditional ground state optimizations.
Note also that the ground state can be targeted by this method by choosing $\omega$ to be below the ground state energy, although we should stress that this
is not the same thing as a traditional ground state optimization and will in general give a slightly different wave function.
Excited state targeting requires two additional parameters, as shown in this table.

\begin{table}[h]
\begin{center}
\begin{tabularx}{\textwidth}{l l l l l X }
\hline
\multicolumn{6}{l}{Excited State Targeting} \\
\hline
\multicolumn{2}{l}{parameters}  & \multicolumn{4}{l}{}\\
   &   \bfseries name     & \bfseries datatype & \bfseries values & \bfseries default   & \bfseries description \\
   %&   \texttt{stepsize} &  real     & 0--1 & 0.25  & Step size for moving parameters\\
   &   \texttt{targetExcited} &  text   & yes, no      & no   & Whether to use the excited state targeting optimization\\
   &   \texttt{omega}         &  real   & real numbers & none & Energy shift used to target different excited states\\
  \hline
\end{tabularx}
\end{center}
\end{table}

Excited state recommendations:
\begin{itemize}
  \item Due to the finite variance in any approximate wave function, it is recommended to set $\omega=\omega_0-\sigma$, where $\omega_0$ is placed just
        below the energy of the targeted state and $\sigma^2$ is the energy variance.
  \item In order to obtain an unbiased excitation energy, one should optimize the ground state with the excited state variational principle as well by setting
        \ixml{omega} below the ground state energy.  Note that using the ground state variational principle for the ground state and the excited state variational
        principle for the excited state creates a bias in favor of the ground state. 
\end{itemize}

\subsubsection{quartic}
\textbf{This is an older optimizer method retained for compatibility. We recommend starting with the newest OneShiftOnly or adaptive optimizers.}
The quartic optimizer fits a quartic polynomial to 7 values of the cost function obtained using reweighting along chosen direction and determines the optimal move.
This optimizer is very robust but a bit conservative to accept new steps especially when large parameters changes are proposed.
\begin{table}[h]
\begin{center}
\begin{tabularx}{\textwidth}{l l l l l X }
\hline
\multicolumn{6}{l}{\texttt{linear} method} \\
\hline
\multicolumn{2}{l}{parameters}  & \multicolumn{4}{l}{}\\
   &   \bfseries name     & \bfseries datatype & \bfseries values & \bfseries default   & \bfseries description \\
   %&   \texttt{stepsize} &  real     & 0--1 & 0.25  & Step size for moving parameters\\
   &   \texttt{bigchange} &  real     & $>0$ & 50.0  & Largest parameter change allowed\\
   &   \texttt{alloweddifference} &  real     & $>0$ & 1e-4 & Allowed increased in energy\\
   &   \texttt{exp0} &  real     & any value & -16.0 & Initial value for stabilizer\\
   &   \texttt{stabilizerscale} &  real     & $>0$ & 2.0 & Increase in value of exp0 between iterations\\
   &   \texttt{nstabilizers} &  integer     & $>0$ & 3 & Number of stabilizers to try\\
   &   \texttt{max\_its} &  integer   & $>0$ & 1 & Number of inner loops with same samples\\
  \hline
\end{tabularx}
\end{center}
\end{table}

Additional information:
\begin{itemize}
\item \ixml{exp0}. It is the initial value for stabilizer (shift to diagonal of H). The actual value of stabilizer is $10^{\textrm{exp0}}$.
\end{itemize}

Recommendations:
\begin{itemize}
  \item{For hard cases (e.g. simultaneous optimization of long MSD and 3-Body J), set exp0
to 0 and do a single inner iteration (max its=1) per sample of configurations.}
\end{itemize}

\begin{lstlisting}[style=QMCPXML]
    <!-- Specify the optimizer options -->
    <parameter name="MinMethod">quartic</parameter>
    <parameter name="exp0">-6</parameter>
    <parameter name="alloweddifference"> 1.0e-4 </parameter>
    <parameter name="nstabilizers"> 1 </parameter>
    <parameter name="bigchange">15.0</parameter>
\end{lstlisting}

\subsection{General recommendations}
Here are a few recommendations to make wavefunction optimization easier.
\begin{itemize}
\item All electron wavefunctions are typically more difficult to optimize than pseudopotential ones due to the importance of the wavefunction near the nucleus.
\item Two body Jastrow contributes the largest portion of correlation energy from bare Slater determinants. For this reason, the recommended order for optimizing wavefunction components is two-body, one-body, three-body Jastrow factors and MSD coefficients.
\item For two-body spline Jastrows, always start from a reasonable one. The lack of physically-motivated constraints in the functional form at large distances can cause slow convergence if starting from zero. 
\item One-body spline Jastrow from old calculations can be a good starting point.
\item Three-body polynomial Jastrow can start from zero. It is beneficial to first optimize one-body and two-body Jastrow factors without adding three-body terms in the calculation and then add the three-body Jastrow and optimize all the three components together.
\end{itemize}
\subsubsection{Optimization of CI coefficients}
When storing a CI wavefunction in HDF5 format, the CI coefficients and the $\alpha$ and $\beta$ components of each CI are not in the XML input file. When optimizing the CI coefficients, they will be stored in HDF5 format. 
The optimization header block will have to specify that the new CI coefficients will be saved to HDF5 format. If the tag is not added coefficients will not be saved. 
\begin{lstlisting}[style=QMCPXML]
  <qmc method="linear" move="pbyp" gpu="no" hdf5="yes">
\end{lstlisting}

The rest of the optimization block remains the same. 

When running the optimization, the new coefficients will be stored in a *.sXXX.opt.h5 file,  where XXX coressponds to the series number. The H5 file contains only the optimized coefficients. The corresponding *.sXXX.opt.xml  will be updated for each optimization block as follow: 
\begin{lstlisting}[style=QMCPXML]
<detlist size="1487" type="DETS" nca="0" ncb="0" nea="2" neb="2" nstates="85" cutoff="1e-2" href="../LiH.orbs.h5" opt_coeffs="LiH.s001.opt.h5"/>
\end{lstlisting}

The opt\_coeffs tag will then reference where the new CI coefficients are stored.\\

When restarting the run with the new optimized coeffs, you need to specify the previous hdf5 containing the basis set, orbitals, and MSD, as well as the new optimized coefficients. The code will read the previous data but will rewrite the coefficients that were optimized with the values found in the *.sXXX.opt.h5 file. 
\subsection{General recommendations}
Here are a few recommendations to avoid bad calculations
\begin{itemize}
\item CI coefficients are optimized with a set of Jastrows. Make sure you maintain the pair (Jastrows, CI-coefficients) as computed. 
\end{itemize}



\section{Diffusion Monte Carlo}
\label{sec:dmc}
\pagebreak
\begin{table}[h]
\begin{center}
\begin{tabularx}{\textwidth}{l l l l l X }
\hline
\multicolumn{6}{l}{\texttt{dmc} method} \\
\hline
\multicolumn{2}{l}{parameters}  & \multicolumn{4}{l}{}\\
   &   \bfseries name     & \bfseries datatype & \bfseries values & \bfseries default   & \bfseries description \\
   &   \texttt{targetwalkers             } &  integer  & $> 0$ & dep.   & Overall total number of walkers \\
   &   \texttt{blocks              } &  integer  & $\ge 0$ & 1   & Number of blocks            \\
   &   \texttt{steps               } &  integer  & $\ge 0$ & 1   & Number of steps per block   \\
   &   \texttt{warmupsteps         } &  integer  & $\ge 0$ & 0   & Number of steps for warming up\\
   &   \texttt{timestep            } &  real     & $> 0$ & 0.1 & Time step for each electron move \\
  % &   \texttt{samples             } &  real  & $\ge 0$ & 0   & total number of samples \\
%   &   \texttt{stepsbetweensamples } &  integer  & $> 0$ & 1   & period of the sample accumulation\\
%   &   \texttt{samplesperthread    } &  real  & $\ge 0$ & 0   & number of samples per thread  \\
%   &   \texttt{rewind              } &  integer  & $\ge 0$ & 0   & number of blocks to roll back   \\
%   &   \texttt{storeconfigs        } &  integer  & $\ge 0$ & 0   & whether to store samples  \\
   &   \texttt{checkproperties     } &  integer  & $\ge 0$ & 100   & Number of steps between walker updates  \\
%  &   \texttt{recordwalkers       } &  integer  & $\ge 0$ & 0   & number of steps between saving a sample configuration. (only for VMC)  \\
%   &   \texttt{recordconfigs       } &  integer  & $\ge 0$ & 0   & number of steps between dumping a configuration to h5  \\
%   &   \texttt{current             } &  integer  & $\ge 0$ & 0   & current step (only used in optimization runs)  \\
%   &   \texttt{dmcwalkersperthread } &  real  & $\ge 0$ & 0   & number of samples per thread  \\
   &   \texttt{maxcpusecs          } &  real  & $\ge 0$ & 3.6e5   & Maximum allowed walltime in seconds \\
   &   \texttt{energyUpdateInterval} &  integer  & $\ge 0$ & 0   & Trial energy update interval \\
   &   \texttt{refEnergy           } &  AU  & all values & dep.   & Reference energy  \\
   &   \texttt{feedback            } &  double  & $\ge 0$ & 1.0   & Population feedback on the trial energy \\
   &   \texttt{warmupByReconfiguration} &  option  & yes,no & 0   & Warm up with a fixed population  \\
 %  &   \texttt{energyBound         } &  double  & $\ge 0$ & 0   & number of samples per thread  \\
   &   \texttt{sigmaBound          } &  double  & $\ge 0$  & 10   & Parameter to cutoff large weights  \\
   &   \texttt{killnode            } &  string  & yes/other & no   & Kill or reject walkers that cross nodes  \\
  % &   \texttt{benchmark           } &  string  & $\ge 0$ & 0   & number of sample \\
   &   \texttt{reconfiguration     } &  string  & yes/pure/other & no   & Fixed population technique  \\
   &   \texttt{branchInterval      } &  integer  & $\ge 0$ & 1   & Branching interval \\
   &   \texttt{substeps            } &  integer  & $\ge 0$ & 1   & Branching interval \\
   &   \texttt{nonlocalmoves       } &  string  & yes,no,v0,v1,v3 & no   & Run with T-moves  \\
   &   \texttt{scaleweight         } &  string  & yes/other & yes   & Scale weights (CUDA only)  \\
   &   \texttt{MaxAge              } &  double  & $\ge 0$ & 10   & Kill persistent walkers  \\
    &   \texttt{MaxCopy             } &  double  & $\ge 0$ &2   & Limit population growth \\
   &   \texttt{fastgrad            } &  text  & yes/other & yes   & Fast gradients  \\
 %  &   \texttt{printderivs         } &  text  & $\ge 0$ & 0   & number of samples per  thread  \\
 %  &   \texttt{wlen                } &  integer  & $\ge 0$ & 0   & number of samples per  thread  \\
   &   \texttt{maxDisplSq      } &  real  & all values & -1   & Maximum particle move  \\
   &   \texttt{storeconfigs        } &  integer  & all values & 0   & Store configurations  \\
   &   \texttt{use\_nonblocking    } &  string  & yes/no & yes   & Using nonblocking send/recv \\
   &   \texttt{blocks\_between\_recompute} &  integer  & $\ge 0$ & dep.  & Wavefunction recompute frequency  \\
  \hline
\end{tabularx}
\end{center}
\end{table}

Additional information:
\begin{itemize}
\item \texttt{targetwalkers}:  A DMC run can be considered a restart run or a new run.  A restart run is considered to be any method block beyond the first one, such as when a DMC method block follows a VMC block.  Alternatively,  a user reading in configurations from disk would also considered a restart run.  In the case of a restart run, the DMC driver will use the configurations from the previous run, and this variable will not be used.  For a new run, if the number of walkers is less than the number of threads, then the number of walkers will be set equal to the number of threads.  

\item \texttt{blocks}: This is the number of blocks run during a DMC method block.  A block consists of a number of DMC steps (steps), after which all the statistics accumulated in the block are written to disk.

\item \texttt{steps}: This is the number of DMC steps in a block.

\item \texttt{warmupsteps}: These are the steps at the beginning of a DMC run in which the 
instantaneous average energy is used to update the trial energy.  During regular steps, E$_{ref}$ is used.

\item \texttt{timestep}: The \texttt{timestep} determines the accuracy of the imaginary time propagator.  Generally, multiple time steps are used to extrapolate to the infinite time step limit.   A good range of time steps  in which to perform time step extrapolation will typically have a minimum of 99\% acceptance probability for each step.

\item \texttt{checkproperties}:  When using a particle-by-particle driver, this variable specifies how often to reset all the variables kept in the buffer.

\item \texttt{maxcpusecs}: The default is 100 hours. Once the specified time has elapsed, the program will finalize the simulation even if all blocks are not  completed.

\item \texttt{energyUpdateInterval}: The default is to update the trial energy at every step. Otherwise the trial energy is updated every \texttt{energyUpdateInterval} step.

\[
E_{\text{trial}}=
\textrm{refEnergy}+\textrm{feedback}\cdot(\ln\texttt{targetWalkers}-\ln N)\:,
\]
where $N$ is the current population.

\item \texttt{refEnergy}: The default reference energy is taken from the VMC run that precedes the DMC run. This value is updated to the current mean whenever branching happens.

\item \texttt{feedback}: This variable is used to determine how strong to react to population fluctuations when doing population control.  See the equation in energyUpdateInterval for more details.

\item \texttt{useBareTau}: The same time step is used whether or not a move is rejected. The default is to use an effective time step when a move is rejected.

\item \texttt{warmupByReconfiguration}:  Warmup DMC is done with a fixed population.

\item \texttt{sigmaBound}:  This determines the branch cutoff to limit wild weights based on the sigma and \texttt{sigmaBound}.

\item \texttt{killnode}:  When running fixed-node, if a walker attempts to cross a node, the move will normally be rejected.  If \texttt{killnode} = ``yes," then walkers are destroyed when they cross a node.

%\item \texttt{benchmark}. 

\item \texttt{reconfiguration}:  If \texttt{reconfiguration} is ``yes," then run with a fixed walker population using the reconfiguration technique.  

\item \texttt{branchInterval}: This is the number of steps between branching.  The total number of DMC steps in a block will be \texttt{BranchInterval}*Steps.   

\item \texttt{substeps}:  This is the same as \texttt{BranchInterval}.


\item \texttt{nonlocalmoves}: Evaluate pseudopotentials using one of the nonlocal move algorithms such as T-moves.
\begin{itemize}
\item no(default): Imposes the locality approximation.
\item yes/v0: Implements the algorithm in the 2006 Casula paper~\cite{Casula2006}
\item v1: Implements the v1 algorithm in the 2010 Casula paper~\cite{Casula2010}.
\item v2: Is \textbf{not implemented} and is \textbf{skipped} baususe of the existence of the v2 algorithm in the 2010 Casula paper~\cite{Casula2010}.
\item v3: (Experimental) Implements an algorithm similar to v1 but that is much faster. v1 computes the transition probability before each single electron T-move selection because of the acceptance of previous T-moves. v3 mostly reuses the transition probability computed during the evaluation of nonlocal pseudopotentials for the local energy, namely before accepting any T-moves, and only recomputes the transition probability of the electrons within the same pseudopotential region of any electrons touched by T-moves. This is an approximation to v1 and results in a slightly different time step error, but it significantly reduces the computational cost. v1 and v3 agree at zero time step. This faster algorithm is the topic of a paper in preparation.
\end{itemize}
The v1 and v3 algorithms are size-consistent and are important advances over the previous v0 non-size-consistent algorithm. We highly recommend investigating the importance of size-consistency.

\item \texttt{scaleweight}: This is the scaling weight per Umrigar/Nightengale.  CUDA only.

\item \texttt{MaxAge}: Set the weight of a walker to min(currentweight,0.5) after a walker has not moved for \texttt{MaxAge} steps.  Needed if persistent walkers appear during the course of a run.

\item \texttt{MaxCopy}: When determining the number of copies of a walker to branch, set the number of copies equal to min(Multiplicity,MaxCopy).

\item \texttt{fastgrad}: This calculates gradients with either the fast version or the full-ratio version.

\item \texttt{maxDisplSq}:  When running a DMC calculation with particle by particle, this sets the maximum displacement allowed for a single particle move.  All distance displacements larger than the max are rejected.  If initialized to a negative value, it becomes equal to Lattice(LR/rc).

\item \texttt{sigmaBound}:  This determines the branch cutoff to limit wild weights based on the sigma and \texttt{sigmaBound}.

%\item \texttt{rewind}. \textit{This input is recorded by QMCDriver.cpp, but is never used anywhere else.}

\item \texttt{storeconfigs}: If \texttt{storeconfigs} is set to a nonzero value, then electron configurations during the DMC run will be saved. This option is disabled for the OpenMP version of DMC.

\item \texttt{blocks\_between\_recompute}: See details in Section~\ref{sec:vmc}, Variational Monte Carlo.

%\item \texttt{recordwalkers}. In VMC this is equivalent for \texttt{stepsbetweensamples}. \textit{This input is not used in DMC.}

%\item \texttt{recordconfigs}. \textit{This input is recorded by QMCDriver.cpp, but is never used anywhere else.}

%\item \texttt{current}. \textit{Only used in QMCLinearOptimize.cpp and QMCOptimize.cpp
%}
%\item \texttt{dmcwalkersperthread}. \textit{This input is only used in VMC.} It is equivalent to \texttt{samplesperthread}.

%\item \texttt{usedrift}. The VMC is implemented in two algorithms with or without drift. In the no-drift algorithm, the move of each electron is proposed with a Gaussian distribution. The standard deviation is chosen as the timestep input. In the drift algorithm, electrons are moved by langevin dynamics.




%\item \texttt{stepsbetweensamples}. Due to the fact that samples generated by consecutive steps might be still correlated. Having stepsbetweensamples larger than 1 reduces that correlation. In practice, using larger substeps is cheaper than using stepsbetweensamples to decorrelate samples.

%\item \texttt{samples}. This is the total amount of samples generated in the current VMC session. This parameter is not important for VMC only calculation but necessary if optimization or DMC follows.
%\[
%\textrm{samples}=
%\frac{\textrm{blocks}\cdot\textrm{steps}\cdot\textrm{walkers}}{\textrm{stepsbetweensamples}}\cdot\textrm{number of MPI tasks}
%\]

%\item \texttt{samplesperthread}. This is an alternative way to set the target amount of samples. More useful in the VMC session preparing the population for the following DMC calculation.
%\[
%\textrm{samplesperthread}=
%\frac{\textrm{blocks}\cdot\textrm{steps}}{\textrm{stepsbetweensamples}}
%\]

\end{itemize}

\begin{lstlisting}[style=QMCPXML,caption=The following is an example of a very simple DMC section. ]
  <qmc method="dmc" move="pbyp" target="e">
    <parameter name="blocks">100</parameter>
    <parameter name="steps">400</parameter>
    <parameter name="timestep">0.010</parameter>
    <parameter name="warmupsteps">100</parameter>
  </qmc>
\end{lstlisting}
The time step should be individually adjusted for each problem.  Please refer to the theory section
on diffusion Monte Carlo.


\begin{lstlisting}[style=QMCPXML,caption=The following is an example of running a simulation that can be restarted. ]
  <qmc method="dmc" move="pbyp"  checkpoint="0">
    <parameter name="timestep">         0.004  </parameter>
    <parameter name="blocks">           100   </parameter>
    <parameter name="steps">            400    </parameter>
  </qmc>
\end{lstlisting}
The checkpoint flag instructs QMCPACK to output walker configurations.  This also
works in VMC.  This will output an h5 file with the name \texttt{projectid.run-number.config.h5}.
Check that this file exists before attempting a restart.
To read in this file for a continuation run, specify the following:
\begin{lstlisting}[caption=Restart (read walkers from previous run) ]
 <mcwalkerset fileroot="BH.s002" version="0 6" collected="yes"/>
\end{lstlisting}
BH is the project id, and s002 is the calculation number to read in the walkers from the previous run.\\

Combining VMC and DMC in a single run (wavefunction optimization can be combined in this way too) is the standard way in which QMCPACK is typically run.   There is no need to run two separate jobs since method sections can be stacked and walkers are transferred between them.

\begin{lstlisting}[style=QMCPXML,caption=Combined VMC and DMC run. ]
  <qmc method="vmc" move="pbyp" target="e">
    <parameter name="blocks">100</parameter>
    <parameter name="steps">4000</parameter>
    <parameter name="warmupsteps">100</parameter>
    <parameter name="samples">1920</parameter>
    <parameter name="walkers">1</parameter>
    <parameter name="timestep">0.5</parameter>
  </qmc>
  <qmc method="dmc" move="pbyp" target="e">
    <parameter name="blocks">100</parameter>
    <parameter name="steps">400</parameter>
    <parameter name="timestep">0.010</parameter>
    <parameter name="warmupsteps">100</parameter>
  </qmc>
  <qmc method="dmc" move="pbyp" target="e">
    <parameter name="warmupsteps">500</parameter>
    <parameter name="blocks">50</parameter>
    <parameter name="steps">100</parameter>
    <parameter name="timestep">0.005</parameter>
  </qmc>
\end{lstlisting}





\section{Reptation Monte Carlo}
\label{sec:rmc}
Like DMC, RMC is a projector-based method that allows sampling of the fixed-node wavefunciton.  However, by exploiting the path-integral formulation of Schr\"{o}dinger's equation, the RMC algorithm can offer some advantages over traditional DMC, such as sampling both the mixed and pure fixed-node distributions in polynomial time, as well as not having population fluctuations and biases.  The current implementation does not work with T-moves.

There are two adjustable parameters that affect the quality of the RMC projection:  imaginary projection time $\beta$ of the sampling path (commonly called a ``reptile") and the Trotter time step $\tau$.  $\beta$ must be chosen to be large enough such that $e^{-\beta \hat{H}}|\Psi_T\rangle \approx |\Phi_0\rangle$ for mixed observables, and $e^{-\frac{\beta}{2} \hat{H}}|\Psi_T\rangle \approx |\Phi_0\rangle$ for pure observables.  The reptile is discretized into $M=\beta/\tau$ beads at the cost of an $\mathcal{O}(\tau)$ time-step error for observables arising from the Trotter-Suzuki breakup of the short-time propagator.  

The following table lists some of the more practical 
\begin{table}[h]
\begin{center}
\begin{tabularx}{\textwidth}{l l l l l X }
\hline
\multicolumn{6}{l}{\texttt{vmc} method} \\
\hline
\multicolumn{2}{l}{parameters}  & \multicolumn{4}{l}{}\\
   &   \bfseries name     & \bfseries datatype & \bfseries values & \bfseries default   & \bfseries description \\
   &   \texttt{beta            } &  real  & $> 0$ & dep.   & Reptile projection time $\beta$  \\
   &   \texttt{timestep            } &  real     & $> 0$ & 0.1 & Trotter time step $\tau$ for each electron move \\
   &   \texttt{beads           } &  int     & $> 0$ & 1 & Number of reptile beads $M=\beta/\tau$ \\
   &   \texttt{blocks              } &  integer  & $\ge 0$ & 1   & Number of blocks            \\
   &   \texttt{steps               } &  integer  & $\ge 0$ & 1   & Number of steps per block   \\
   &   \texttt{vmcpresteps        } &  integer  & $\ge 0$ & 0   & Propagates reptile using VMC for given number of steps\\
   &   \texttt{warmupsteps         } &  integer  & $\ge 0$ & 0   & Number of steps for warming up\\
   &   \texttt{MaxAge              }   & integer & $\ge 0 $   & 0   & Force accept for stuck reptile if age exceeds \texttt{MaxAge} \\
  \hline
\end{tabularx}
\end{center}
\end{table}

Additional information:

Because of the sampling differences between DMC ensembles of walkers and RMC reptiles, the RMC block should contain the following estimator declaration to ensure correct sampling:  \texttt{ <estimator name="RMC" hdf5="no">}. 
  
\begin{itemize}
\item \texttt{beta} or \texttt{beads}?  One or the other can be specified, and from the Trotter time step, the code will construct an appropriately sized reptile.  If both are given, \texttt{beta} overrides \texttt{beads}.  

\item \textbf{Mixed vs. pure observables?}  Configurations sampled by the endpoints of the reptile are distributed according to the mixed distribution $f(\mathbf{R})=\Psi_T(\mathbf{R})\Phi_0(\mathbf{R})$.  Any observable that is computable within DMC and is dumped to the \texttt{scalar.dat} file will likewise be found in the \texttt{scalar.dat} file generated by RMC, except there will be an appended \texttt{\_m} to alert the user that the observable was computed on the mixed distribution.  For pure observables, care must be taken in the interpretation.  If the observable is diagonal in the position basis (in layman's terms, if it is entirely computable from a single electron configuration $\mathbf{R}$, like the potential energy), and if the observable does not have an explicit dependence on the trial wavefunction (e.g., the local energy has an explicit dependence on the trial wavefunction from the kinetic energy term), then pure estimates will be correctly computed.  These observables will be found in either the \texttt{scalar.dat}, where they will be appended with a \texttt{\_p} suffix, or in the \texttt{stat.h5} file.  No mixed estimators will be dumped to the h5 file. 

\item \textbf{Sampling}:  For pure estimators, the traces of both pure and mixed estimates should be checked.  Ergodicity is a known problem in RMC.  Because we use the bounce algorithm, it is possible for the reptile to bounce back and forth without changing the electron coordinates of the central beads.  This might not easily show up with mixed estimators, since these are accumulated at constantly regrown ends, but pure estimates are accumulated on these central beads and so can exhibit strong autocorrelations in pure estimate traces.  

\item \textbf{Propagator}:  Our implementation of RMC uses Moroni's DMC link action (symmetrized), with Umrigar's scaled drift near nodes.  In this regard, the propagator is identical to the one QMCPACK uses in DMC.  

\item \textbf{Sampling}:  We use Ceperley's bounce algorithm.  \texttt{MaxAge} is used in case the reptile gets stuck, at which point the code forces move acceptance, stops accumulating statistics, and requilibrates the reptile.  Very rarely will this be required.  For move proposals, we use particle-by-particle VMC a total of $N_e$ times to generate a new all-electron configuration, at which point the action is computed and the move is either accepted or rejected.  
\end{itemize}






\chapter{Output overview}
\label{chap:output_overview}

%% Detail contents of output files.
QMCPACK writes several output files that report information about the simulation (e.g., the physical properties such as the energy), as well as information about the computational aspects of the simulation, checkpoints, and restarts.
The types of output files generated depend on the details of a calculation. The following list is not meant to be exhaustive but rather to highlight some salient features of the more common file types. Further details can be found in the description of the estimator of interest.


\section{The .scalar.dat file}
\label{sec:scalardat_file}
The most important output file is the \ishell{.scalar.dat} file. This file contains the output of block-averaged properties of the system such as the local energy and other estimators.
Each line corresponds to an average over $N_{walkers}*N_{steps}$ samples.
By default, the quantities reported in the \ishell{.scalar.dat} file include the following:

\begin{description}
\item[LocalEnergy] The local energy.
\item[LocalEnergy\_sq] The local energy squared.
\item[LocalPotential] The local potential energy.
\item[Kinetic] The kinetic energy.
\item[ElecElec] The electron-electron potential energy.
\item[IonIon] The ion-ion potential energy.
\item[LocalECP] The energy due to the pseudopotential/effective core potential.
\item[NonLocalECP] The nonlocal energy due to the pseudopotential/effective core potential.
\item[MPC] The modified periodic Coulomb potential energy.
\item[BlockWeight] The number of MC samples in the block.
\item[BlockCPU] The number of seconds to compute the block.
\item[AcceptRatio] The acceptance ratio.
\end{description}

QMCPACK includes a python utility, \ishell{qmca}, that can be used to process these files. Details and examples are given in Chapter~\ref{chap:analyzing}.
\section{The .opt.xml file}
\label{sec:optxml_file}
This file is generated after a VMC wavefunction optimization and contains the part of the input file that lists the optimized Jastrow factors.
Conveniently, this file is already formatted such that it can easily be incorporated into a DMC input file.

\section{The .qmc.xml file}
\label{sec:qmc_file}
This file contains information about the computational aspects of the simulation, for example, which parts of the code are being executed when. This file is generated only during an ensemble run in which QMCPACK runs multiple input files.

\section{The .dmc.dat file}
\label{sec:dmc_file}
This file contains information similar to the \ishell{.scalar.dat} file but also includes extra information about the details of a DMC calculation, for example, information about the walker population.

\begin{description}
\item[Index] The block number.
\item[LocalEnergy] The local energy.
\item[Variance] The variance.
\item[Weight] The number of samples in the block.
\item[NumOfWalkers] The number of walkers times the number of steps.
\item[AvgSentWalkers] The average number of walkers sent. During a DMC simulation, walkers might be created or destroyed. At every step, QMCPACK will do some load balancing to ensure that the walkers are evenly distributed across nodes.
\item[TrialEnergy] The trial energy. See Section~\ref{sec:dmc} for an explanation of trial energy.
\item[DiffEff] The diffusion efficiency.
\item[LivingFraction] The fraction of the walker population from the previous step that survived to the current step.
\end{description}


\section{The .bandinfo.dat file}
\label{sec:bandinfo_file}
This file contains information from the trial wavefunction about the band structure of the system,
including the available $k$-points. This can
be helpful in constructing trial wavefunctions.


\section{Checkpoint and restart files}
\label{sec:checkpoint_files}
\subsection{The .cont.xml file}
This file enables continuation of the run.  It is mostly a copy of the input XML file with the series number incremented and the \ishell{mcwalkerset} element added to read the walkers from a config file.   The \ishell{.cont.xml} file is always created, but other files it depends on are  present only if checkpointing is enabled.

\subsection{The .config.h5 file}
This file contains stored walker configurations.

\subsection{The .random.h5 file}
This file contains the state of the random number generator to allow restarts.
(Older versions used an XML file with a suffix of \ishell{.random.xml}).


\chapter{Analyzing QMCPACK data}
\label{chap:analyzing}

\section{Using the qmca tool to obtain total energies and related quantities}
\label{sec:qmca}

The \texttt{qmca} tool is the primary means of analyzing scalar-valued 
data generated by QMCPACK.  Output files that contain scalar-valued data 
are \texttt{*.scalar.dat} and \texttt{*.dmc.dat} (see Chapter 
\ref{chap:output_overview} for a detailed description of these files).
Quantities that are available for analysis in \texttt{*.scalar.dat} files 
include the local energy and its variance, kinetic energy, 
potential energy and its components,
acceptance ratio, and the average CPU time spent per block, among 
others.  The \texttt{*.dmc.dat} files provide information regarding 
the DMC walker population in addition to the local energy.  

Basic capabilities of \texttt{qmca} include calculating mean values 
and associated error bars, processing multiple files at once in batched 
fashion, performing twist averaging, plotting mean values by series, 
and plotting traces (per block or step) of the underlying data.  
These capabilities are explained with accompanying examples in the 
following subsections.

To use \texttt{qmca}, installations of Python and NumPy must be 
present on the local machine.  For graphical plotting, the matplotlib module 
must also be available.

An overview of all supported input flags to \texttt{qmca} can be 
obtained by typing \texttt{qmca} at the command line with no 
other inputs (also try \texttt{qmca -x} for a short list of examples):

\begin{shade}
>qmca
  no files provided, please see help info below 
  
  Usage: qmca [options] [file(s)]
  
  Options:
    --version             show program's version number and exit
    -v, --verbose         Print detailed information (default=False).
    -q QUANTITIES, --quantities=QUANTITIES
                          Quantity or list of quantities to analyze.  See names
                          and abbreviations below (default=all).
    -u UNITS, --units=UNITS
                          Desired energy units.  Can be Ha (Hartree), Ry
                          (Rydberg), eV (electron volts), kJ_mol (k.
                          joule/mole), K (Kelvin), J (Joules) (default=Ha).
    -e EQUILIBRATION, --equilibration=EQUILIBRATION
                          Equilibration length in blocks (default=auto).
    -a, --average         Average over files in each series (default=False).
    -w WEIGHTS, --weights=WEIGHTS
                          List of weights for averaging (default=None).
    -b, --reblock         (pending) Use reblocking to calculate statistics
                          (default=False).
    -p, --plot            Plot quantities vs. series (default=False).
    -t, --trace           Plot a trace of quantities (default=False).
    -h, --histogram       (pending) Plot a histogram of quantities
                          (default=False).
    -o, --overlay         Overlay plots (default=False).
    --legend=LEGEND       Placement of legend.  None for no legend, outside for
                          outside legend (default=upper right).
    --noautocorr          Do not calculate autocorrelation. Warning: error bars
                          are no longer valid! (default=False).
    --noac                Alias for --noautocorr (default=False).
    --sac                 Show autocorrelation of sample data (default=False).
    --sv                  Show variance of sample data (default=False).
    -i, --image           (pending) Save image files (default=False).
    -r, --report          (pending) Write a report (default=False).
    -s, --show_options    Print user provided options (default=False).
    -x, --examples        Print examples and exit (default=False).
    --help                Print help information and exit (default=False).
    -d DESIRED_ERROR, --desired_error=DESIRED_ERROR
                          Show number of samples needed for desired error bar
                          (default=none).
    -n PARTICLE_NUMBER, --enlarge_system=PARTICLE_NUMBER
                          Show number of samples needed to maintain error bar on
                          larger system: desired particle number first, current
                          particle number second (default=none) 
\end{shade}

\subsection{Obtaining a statistically correct mean and error bar}
\label{sec:qmca_mean_error}
A rough guess at the mean and error bar of the local energy can be 
obtained in the following way with \texttt{qmca}:
\begin{shade}
>qmca -q e qmc.s000.scalar.dat 
qmc  series 0  LocalEnergy           =  -45.876150 +/- 0.017688 
\end{shade}
\noindent
In this case the VMC energy of an 8-atom cell of diamond is estimated 
to be $-45.876(2)$ Hartrees (Ha).  This rough guess should not be used 
for production-level or publication-quality estimates.

To obtain production-level results, the underlying data should first be 
inspected visually to ensure that all data included in the averaging 
can be attributed to a distribution sharing the same mean.  The first 
steps of essentially any MC calculation (the ``equilibration 
phase'') do not belong to the equilibrium distribution and should be 
excluded from estimates of the mean and its error bar.

We can plot a data trace (\texttt{-t}) of the local energy in the 
following way:
\begin{shade}
>qmca -t -q e -e 0 qmc.s000.scalar.dat
\end{shade}
\noindent
The \texttt{-e 0} part indicates that we do not want any data 
to be initially excluded from the calculation of averages.  The resulting 
plot is shown in Figure~\ref{fig:qmca_mean_error_trace}.  The unphysical 
equilibration period is visible on the left side of the plot.

Most of the data fluctuates around a well-defined mean (consistent 
variations around a flat line).  This property is important to verify  
by plotting the trace for each QMC run. 

\begin{figure}
\begin{center}
\ifdefined\HCode
\includegraphics[trim = 0mm 0mm 0mm 0mm,clip,width=0.75\textwidth]{./figures/qmca_mean_error_trace.dmn}
\else
\includegraphics[trim = 0mm 0mm 0mm 0mm,clip,width=0.75\textwidth]{./figures/qmca_mean_error_trace.pdf}
\fi
\caption{Trace of the VMC local energy for an 8-atom cell of diamond generated with \texttt{qmca}.  The x-axis (``samples'') refers to the VMC block index in this case.}
\label{fig:qmca_mean_error_trace}
\end{center}
\end{figure}

If we exclude none of the equilibration data points, we get an 
erroneous estimate of $-45.870(2)$~Ha for the local energy:
\begin{shade}
>qmca -q e -e 0 qmc.s000.scalar.dat 
qmc  series 0  LocalEnergy           =  -45.870071 +/- 0.018072
\end{shade}
\noindent
The equilibration period is typically estimated by eye, though a few conservative values should be checked to ensure that the mean remains 
unaffected.  In this dataset, the equilibration appears to have been 
reached after 100 or so samples.  After excluding the first 100 
VMC blocks from the analysis we get
\begin{shade}
>qmca -q e -e 100 qmc.s000.scalar.dat 
qmc  series 0  LocalEnergy           =  -45.877363 +/- 0.017432
\end{shade}
\noindent
This estimate ($-45.877(2)$ Ha) differs significantly from the 
$-45.870(2)$ Ha figure obtained from the full set of data, but it 
agrees with the rough estimate of $-45.876(2)$ Ha obtained 
with the abbreviated command (\texttt{qmca -q e qmc.s000.scalar.dat}).
This is because \texttt{qmca} makes a heuristic guess at the 
equilibration period and got it reasonably correct in this case. 
In many cases, the heuristic guess fails and should not 
be relied on for quality results.

We have so far obtained a statistically correct mean.  To obtain 
a statistically correct error bar, it is best to include $\sim$100 or more 
statistically independent samples.  An estimate of the number 
of independent samples can be obtained by considering the 
autocorrelation time, which is essentially a measure of the number of 
samples that must be traversed before an uncorrelated/independent sample 
is reached.  We can get an estimate of the autocorrelation time 
in the following way:
\begin{shade}
>qmca -q e -e 100 qmc.s000.scalar.dat --sac
qmc  series 0  LocalEnergy           =  -45.877363 +/- 0.017432    4.8 
\end{shade}
\noindent
The flag \texttt{--sac} stands for (s)how (a)uto(c)orrelation.  
In this case, the autocorrelation estimate is $4.8\approx 5$ samples. 
Since the total run contained 800 samples and we have excluded 100 of 
them, we can estimate the number of independent samples as 
$(800-100)/5=140$.  In this case, the error bar is expected to be 
estimated reasonably well.

Keep in mind that the error bar represents the expected range
of the mean with a certainty of only $\sim 70\%$; i.e., it is a one
sigma error bar.  The actual mean value will lie outside the range
indicated by the error bar in 1 out of every 3 runs, and in a set
of 20 runs 1 value can be expected to deviate from its estimate by
twice the error bar.


\subsection{Judging wavefunction optimization}
\label{sec:qmca_judge_opt}
Wavefunction optimization is a highly nonlinear and sometimes 
sensitive process.  As such, there is a risk that systematic 
errors encountered at this stage of the QMC process can be propagated 
into subsequent (expensive) DMC runs unless they are guarded against 
with vigilance.

In this section we again consider an 8-atom cell of diamond but 
now in the context of Jastrow optimization (one- and two-body terms). 
In optimization runs it is often preferable to use a large number 
of \texttt{warmupsteps} ($\sim 100$) so that equilibration bias does 
not propagate into the optimization process.  We can check that 
the added warm-up has had its intended effect by again checking the 
local energy trace:
\begin{shade}
>qmca -t -q e *scalar*
\end{shade}
\noindent
The resulting plot can be found in Figure~\ref{fig:qmca_judge_opt}. 
In this case sufficient \texttt{warmupsteps} were used to exit 
the equilibration period before samples were collected and we can 
proceed without using the \texttt{-e} option with \texttt{qmca}.

\begin{figure}
\begin{center}
\ifdefined\HCode  
\includegraphics[trim = 0mm 0mm 0mm 0mm, clip,width=0.9\textwidth]{./figures/qmca_judge_opt.dmn}
\else
\includegraphics[trim = 0mm 0mm 0mm 0mm, clip,width=0.9\columnwidth]{./figures/qmca_judge_opt.pdf}
\fi
\end{center}
\caption{Trace of the local energy during one- and two-body Jastrow optimizations for an 8-atom cell of diamond generated with \texttt{qmca}.  Data for each optimization cycle (QMCPACK series) is separated by a vertical black line.
}
\label{fig:qmca_judge_opt}
\end{figure}

After inspecting the trace, we should inspect the text output 
from \texttt{qmca}, now including the total energy and its variance:
\begin{shade}
>qmca -q ev opt*scalar.dat
                            LocalEnergy               Variance           ratio 
opt  series 0  -44.823616 +/- 0.007430   7.054219 +/- 0.041998   0.1574 
opt  series 1  -45.877643 +/- 0.003329   1.095362 +/- 0.041154   0.0239 
opt  series 2  -45.883191 +/- 0.004149   1.077942 +/- 0.021555   0.0235 
opt  series 3  -45.877524 +/- 0.003094   1.074047 +/- 0.010491   0.0234 
opt  series 4  -45.886062 +/- 0.003750   1.061707 +/- 0.014459   0.0231 
opt  series 5  -45.877668 +/- 0.003475   1.091585 +/- 0.021637   0.0238 
opt  series 6  -45.877109 +/- 0.003586   1.069205 +/- 0.009387   0.0233 
opt  series 7  -45.882563 +/- 0.004324   1.058771 +/- 0.008651   0.0231 
\end{shade}
\noindent
The flags \texttt{-q ev} requested the energy (\texttt{e}) and 
the variance (\texttt{v}).  For this combination of quantities, a 
third column (\texttt{ratio}) is printed containing the ratio 
of the variance and the absolute value of the local energy.
The variance/energy ratio is an intensive quantity and is useful  
to inspect regardless of the system under study.  Successful 
optimization of molecules and solids of any size generally result 
in comparable values for the variance/energy ratio. 

The first line of 
the output (\texttt{series 0}) corresponds to the local energy 
and variance of the system without a Jastrow factor (all Jastrow 
coefficients were initialized to zero in this case), reflecting the 
quality of the orbitals alone. For pseudopotential systems, a 
variance/energy ratio $>0.20$ Ha generally indicates there is a problem 
with the input orbitals that needs to be resolved before 
performing wavefunction optimization.  

The subsequent lines correspond to energies and variances of 
intermediate parameterizations of the trial wavefunction during 
the optimization process.  The output line containing 
\texttt{opt  series 1}, for example, corresponds to the trial 
wavefunction parameterized during the \texttt{series 0} step 
(the parameters of this wavefunction would be found in an output 
file matching \texttt{*s000*opt.xml}).  The first thing to check 
about the resulting optimization is again the variance/energy ratio. 
For pseudopotential systems, a variance/energy ratio $<0.03$ Ha is 
consistent with a trial wavefunction of production quality, and values 
of $0.01$ Ha are rarely obtainable for standard Slater-Jastrow 
wavefunctions.  By this metric, all parameterizations obtained for 
optimizations performed in series 0-6 are of comparable quality 
(note that the quality of the wavefunction obtained during optimization 
series 7 is effectively unknown).

A good way to further discriminate among the parameterizations is to 
plot the energy and variance as a function of series with \texttt{qmca}:
\begin{shade}
>qmca -p -q ev opt*scalar.dat
\end{shade}
\noindent
The \texttt{-p} option results in plots of means plus error bars 
vs. series for all requested quantities.
The resulting plots for the local energy and variance are shown 
in Figure~\ref{fig:qmca_opt_ev}.  In this case, the resulting energies 
and variances are statistically indistinguishable for all optimization 
cycles.  

\begin{figure}
  \centering
  \ifdefined\HCode%
  \begin{tabularx}{1024pt}{X X}
    \includegraphics[trim=0mm 0mm 4mm 0mm,clip,width=512pt]{./figures/qmca_opt_energy.dmn}&
    \includegraphics[trim=2mm 0mm 4mm 0mm,clip,width=512pt]{./figures/qmca_opt_variance.png}\\
  \end{tabularx}
\else%
  \begin{tabularx}{\textwidth}{X X}
    \includegraphics[trim=0mm 0mm 4mm 0mm,clip,width=0.47\textwidth]{./figures/qmca_opt_energy.pdf}&
    \includegraphics[trim=2mm 0mm 4mm 0mm,clip,width=0.47\textwidth]{./figures/qmca_opt_variance.png}\\
    \end{tabularx}
\fi%
  \caption{Energy and variance vs. optimization series for an 8-atom cell of diamond as plotted by \texttt{qmca}.}%
  \label{fig:qmca_opt_ev}%
\end{figure}

A good way to choose the optimal wavefunction for use in DMC is to select 
the one with the lowest statistically significant energy within the set of 
optimized wavefunctions with reasonable variance (e.g., among 
those with a variance/energy ratio $<0.03$ Ha).  For pseudopotential 
calculations, minimizing according to the total energy is recommended 
to reduce locality errors in DMC.


\subsection{Judging diffusion Monte Carlo runs}
\label{sec:qmca_judge_dmc}
Judging the quality of the DMC projection process requires more 
care than is needed in VMC. To reduce bias, a small 
time step is required in the approximate projector but this also 
leads to slow equilibration and long autocorrelation times.  
Systematic errors in the projection process can also arise from 
statistical fluctuations due to pseudopotentials or from trial 
wavefunctions with larger-than-necessary variance.

\begin{figure}
\begin{center}
  \ifdefined\HCode  
\includegraphics[trim = 0mm 0mm 0mm 0mm,clip,width=0.75\columnwidth]{./figures/qmca_short_dmc.dmn}
\else
\includegraphics[trim = 0mm 0mm 0mm 0mm,clip,width=0.75\columnwidth]{./figures/qmca_short_dmc.pdf}
\fi
\end{center}
\caption{Trace of the local energy for VMC followed by DMC with a small time step ($0.002$ Ha$^{-1}$) for an 8-atom cell of diamond generated with \texttt{qmca}.}
\label{fig:qmca_short_dmc}
\end{figure}

To illustrate the problems that can arise with respect to slow 
equilibration and long autocorrelation times, we consider the 
8-atom diamond system with VMC ($200$ blocks of $160$ steps) followed 
by DMC ($400$ blocks of $5$ steps) with a small time step ($0.002$ Ha$^{-1}$).
A good first step in assessing the quality of any DMC run is 
to plot the trace of the local energy:
\begin{shade}
>qmca -t -q e -e 0 *scalar*
\end{shade}
\noindent
The resulting trace plot is shown in Figure~\ref{fig:qmca_short_dmc}.  
As always, the DMC local energy decreases exponentially away from 
the VMC value, but in this case it takes a long time to do so.  
At least half of the DMC run is inefficiently consumed by equilibration.
If we are not careful to inspect and remove the transient, the estimated 
DMC energy will be strongly biased by the transient as shown by the 
horizontal red line (estimated mean) in the figure.  The autocorrelation 
time is also large ($\sim 12$ blocks):
\begin{shade}
>qmca -q e -e 200 --sac *s001.scalar*
qmc  series 1  LocalEnergy           =  -46.045720 +/- 0.004813   11.6
\end{shade}
\noindent
Of the included 200 blocks, fewer than 20 contribute to the estimated error 
bar, indicating that we cannot trust the reported error bar.  
This can also be demonstrated directly from the data.  If we halve the number 
of included samples to 100, we expect from Gaussian statistics 
that the error bar will grow by a factor of $\sqrt{2}$, but instead we 
get
\begin{shade}
>qmca -q e -e 300 *s001.scalar*
qmc  series 1  LocalEnergy           =  -46.048537 +/- 0.009280
\end{shade}
\noindent
which erroneously shows an estimated increase in the error bar by a factor 
of about 2.  Overall, this run is simply too short to gain meaningful 
information.  

Consider the case in which we are interested in the cohesive energy of 
diamond, and, after having performed a time step study of the cohesive 
energy, we have found that the energy difference between bulk diamond 
and atomic carbon converges to our required accuracy with a larger 
time step of $0.01$ Ha$^{-1}$.  In a production setting, a small cell 
could be used to determine  the appropriate time step, while a larger 
cell would subsequently be used to obtain a converged cohesive energy, 
though for purposes of demonstration we still proceed here with the 8-atom 
cell.  The new time step of $0.01$~Ha$^{-1}$ will result in a shorter 
autocorrelation time than the smaller time step used previously, but 
we would like to shorten the equilibration time further still.  This 
can be achieved by using a larger time step (say $0.02$ Ha$^{-1}$) in a 
short intermediate DMC run used to walk down the transient.  The 
rapidly achieved equilibrium with the $0.02$ Ha$^{-1}$ time step 
projector will be much nearer to the $0.01$ Ha$^{-1}$ time step 
we seek than the original VMC equilibrium, so we can expect 
a shortened secondary equilibration time in the production 
$0.01$ Ha$^{-1}$ time step run. Note that this procedure is fully 
general, even if having to deal with an even shorter 
time step (e.g., $0.002$ Ha$^{-1}$) for a particular problem.

\begin{figure}
\begin{center}
\ifdefined\HCode  
\includegraphics[trim = 0mm 0mm 0mm 0mm, clip,width=0.75\columnwidth]{./figures/qmca_accel_dmc.dmn}
\else
\includegraphics[trim = 0mm 0mm 0mm 0mm, clip,width=0.75\columnwidth]{./figures/qmca_accel_dmc.pdf}
\fi
\end{center}
\caption{Trace of the local energy for VMC followed by a short intermediate DMC with a large time step ($0.02$ Ha$^{-1}$) and finally a production DMC run with a time step of $0.01$ Ha$^{-1}$.  Calculations were performed in an 8-atom cell of diamond.} 
\label{fig:qmca_accel_dmc}
\end{figure}

We now rerun the previous example but with an intermediate DMC 
calculation using $40$ blocks of $5$ steps with a time step of 
$0.02$ Ha$^{-1}$, followed by a production DMC calculation 
using $400$ blocks of $10$ steps with a time step of $0.01$ Ha$^{-1}$.
We again plot the local energy trace using \texttt{qmca}:
\begin{shade}
>qmca -t -q e -e 0 *scalar* 
\end{shade}
\noindent
with the result shown in Figure~\ref{fig:qmca_accel_dmc}.
The projection transient has been effectively contained in the 
short DMC run with a larger time step.  As expected, the 
production run contains only a short equilibration period.
Removing the first 20 blocks as a precaution, we obtain an estimate 
of the total energy in VMC and DMC:
\begin{shade}
>qmca -q ev -e 20 --sac qmc.*.scalar.dat 
                            LocalEnergy               Variance           ratio 
qmc  series 0  -45.881042 +/- 0.001283    1.0   1.076726 +/- 0.007013    1.0   0.0235 
qmc  series 1  -46.040814 +/- 0.005046    3.9   1.011303 +/- 0.016807    1.1   0.0220 
qmc  series 2  -46.032960 +/- 0.002077    5.2   1.014940 +/- 0.002547    1.0   0.0220 
\end{shade}
\noindent
Notice that the variance/energy ratio in DMC ($0.220$ Ha) is similar to but 
slightly smaller than that obtained with VMC ($0.235$ Ha).  If the DMC 
variance/energy ratio is ever significantly larger than with VMC, this is 
cause to be concerned about the correctness of the DMC run.  Also notice 
the estimated autocorrelation time ($\sim 5$ blocks).  This leaves us with 
an estimated $\sim 76$ independent samples, though we should recall that 
the autocorrelation time is also a statistical estimate that can be improved 
with more data.  We can gain a better estimate of the autocorrelation 
time by using the \texttt{*.dmc.dat} files, which contain output data resolved 
per step rather than per block (there are $10\times$ more steps than blocks 
in this example case):
\begin{shade}
>qmca -q ev -e 200 --sac qmc.s002.dmc.dat 
                            LocalEnergy               Variance           ratio 
qmc  series 2  -46.032909 +/- 0.002068   31.2   1.015781 +/- 0.002536    1.4   0.0221 
\end{shade}
\noindent
This results in an estimated autocorrelation time of $\sim 31$ steps, or 
$\sim 3$ blocks, indicating that we actually have $\sim 122$ independent 
samples, which should be sufficient to obtain a trustworthy error bar.
Our final DMC total energy is estimated to be $-46.0329(2)$ Ha.

\begin{figure}
\begin{center}
  \ifdefined\HCode
\includegraphics[trim = 0mm 0mm 0mm 0mm, clip,width=0.75\columnwidth]{./figures/qmca_pop_trace.dmn}
\else
\includegraphics[trim = 0mm 0mm 0mm 0mm, clip,width=0.75\columnwidth]{./figures/qmca_pop_trace.pdf}
\fi
\end{center}
\caption{Trace of the DMC walker population for an 8-atom cell of diamond obtained with \texttt{qmca}.}
\label{fig:qmca_pop_trace}
\end{figure}

Another simulation property that should be explicitly monitored  
is the behavior of the DMC walker population.  Data regarding the 
walker population is contained in the \texttt{*.dmc.dat} files.
In Figure~\ref{fig:qmca_pop_trace} we show the trace of the DMC 
walker population for the current run:
\begin{shade}
>qmca -t -q nw *dmc.dat
qmc  series 1  NumOfWalkers          =  2056.905405 +/- 8.775527 
qmc  series 2  NumOfWalkers          =  2050.164160 +/- 4.954850 
\end{shade}
\noindent
Following a DMC run, the walker population should be checked for 
two qualities: (1) that the population is sufficiently large (a number 
$>2,000$ is generally sufficient to reduce population control bias) and  
(2) that the population fluctuates benignly around its intended target 
value. In this case the target walker count (provided in the input file)
was $2,048$ and we can confirm from the plot that the population is simply 
fluctuating around this value.  Also, from the text output we have a dynamic 
population estimate of 2,050(5) walkers.  Rapid population reductions or 
increases---population explosions---are indicative of problems with a run.  
These issues sometimes result from using a considerably poor wavefunction 
(see comments regarding variance/energy ratio in the preceding 
subsections).  QMCPACK has internal guards in place that prevent 
the population from exceeding certain maximum and minimum bounds, so 
in particularly faulty runs one might see the population ``stabilize'' 
to a constant value much larger or smaller than the target.  In such 
cases the cause(s) for the divergent population behavior needs to 
be investigated and resolved before proceeding further.



\subsection{Obtaining other quantities}
\label{sec:qmca_other_quantities}
A number of other scalar-valued quantities are available with 
\texttt{qmca}.  To obtain text output for all quantities 
available, simply exclude the \texttt{-q} option used in 
previous examples.  The following example shows output for a DMC calculation 
of the 8-atom diamond system from the \texttt{scalar.dat} file:
\begin{shade}
>qmca -e 20 qmc.s002.scalar.dat 
qmc  series 2 
  LocalEnergy           =          -46.0330 +/-           0.0021 
  Variance              =            1.0149 +/-           0.0025 
  Kinetic               =            33.851 +/-            0.019 
  LocalPotential        =           -79.884 +/-            0.020 
  ElecElec              =          -11.4483 +/-           0.0083 
  LocalECP              =           -22.615 +/-            0.029 
  NonLocalECP           =            5.2815 +/-           0.0079 
  IonIon                =            -51.10 +/-             0.00 
  LocalEnergy_sq        =           2120.05 +/-             0.19 
  BlockWeight           =          20514.27 +/-            48.38 
  BlockCPU              =            1.4890 +/-           0.0038 
  AcceptRatio           =         0.9963954 +/-        0.0000055 
  Efficiency            =             71.88 +/-             0.00 
  TotalTime             =            565.80 +/-             0.00 
  TotalSamples          =           7795421 +/-                0 
\end{shade}
\noindent
Similarly, for the \texttt{dmc.dat} file we get
\begin{shade}
>qmca -e 20 qmc.s002.dmc.dat 
qmc  series 2 
  LocalEnergy           =          -46.0329 +/-           0.0020 
  Variance              =            1.0162 +/-           0.0025 
  TotalSamples          =           8201275 +/-                0 
  TrialEnergy           =          -46.0343 +/-           0.0023 
  DiffEff               =         0.9939150 +/-        0.0000088 
  Weight                =           2050.23 +/-             4.82 
  NumOfWalkers          =              2050 +/-                5 
  LivingFraction        =          0.996427 +/-         0.000021 
  AvgSentWalkers        =            0.2625 +/-           0.0011 
\end{shade}

Any subset of desired quantities can be obtained by using the 
\texttt{-q} option with either the full names of the quantities 
just listed 
\begin{shade}
>qmca -q 'LocalEnergy Kinetic LocalPotential' -e 20 qmc.s002.scalar.dat 
qmc  series 2 
  LocalEnergy           =          -46.0330 +/-           0.0021 
  Kinetic               =            33.851 +/-            0.019 
  LocalPotential        =           -79.884 +/-            0.020 
\end{shade}
\noindent
or with their corresponding abbreviations.
\begin{shade}
>qmca -q ekp -e 20 qmc.s002.scalar.dat 
qmc  series 2 
  LocalEnergy           =          -46.0330 +/-           0.0021 
  Kinetic               =            33.851 +/-            0.019 
  LocalPotential        =           -79.884 +/-            0.020 
\end{shade}
\noindent
Abbreviations for each quantity can be found by typing \texttt{qmca}
at the command line with no other input.  This following is a current list:
\begin{shade}
  Abbreviations and full names for quantities:
    ar              = AcceptRatio
    bc              = BlockCPU
    bw              = BlockWeight
    ce              = CorrectedEnergy
    de              = DiffEff
    e               = LocalEnergy
    ee              = ElecElec
    eff             = Efficiency
    ii              = IonIon
    k               = Kinetic
    kc              = KEcorr
    l               = LocalECP
    le2             = LocalEnergy_sq
    mpc             = MPC
    n               = NonLocalECP
    nw              = NumOfWalkers
    p               = LocalPotential
    sw              = AvgSentWalkers
    te              = TrialEnergy
    ts              = TotalSamples
    tt              = TotalTime
    v               = Variance
    w               = Weight
\end{shade}
\noindent
See the output overview for \texttt{scalar.dat} 
(Section~\ref{sec:scalardat_file}) and \texttt{dmc.dat} 
(Section~\ref{sec:dmc_file}) for more information about 
these quantities.  The data analysis aspects for these 
quantities are essentially the same as for the local 
energy as covered in the preceding subsections. 
Quantities that do not belong to an equilibrium distribution 
(e.g., \texttt{BlockCPU}) are somewhat different, though they 
still exhibit statistical fluctuations.


\subsection{Processing multiple files}
\label{sec:qmca_multiple_files}
Batch file processing is a common use case for \texttt{qmca}. 
If we consider an ``equation-of-state'' calculation involving 
the 8-atom diamond cell we have used so far, we might be interested 
in the total energy for the various supercell volumes along the 
trajectory from compression to expansion.  After checking 
the traces (\texttt{qmca -t -q e scale\_*/vmc/*scalar*}) 
to settle on a sensible equilibration cutoff as discussed in 
the preceding subsections, we can obtain the total energies 
all at once:
\begin{shade}
>qmca -q ev -e 40 scale_*/vmc/*scalar*
                            LocalEnergy               Variance           ratio 
scale_0.80/vmc/qmc  series 0 -44.670984 +/- 0.006051  2.542384 +/- 0.019902  0.0569 
scale_0.82/vmc/qmc  series 0 -44.982818 +/- 0.005757  2.413011 +/- 0.022626  0.0536 
scale_0.84/vmc/qmc  series 0 -45.228257 +/- 0.005374  2.258577 +/- 0.019322  0.0499 
scale_0.86/vmc/qmc  series 0 -45.415842 +/- 0.005532  2.204980 +/- 0.052978  0.0486 
scale_0.88/vmc/qmc  series 0 -45.570215 +/- 0.004651  2.061374 +/- 0.014359  0.0452 
scale_0.90/vmc/qmc  series 0 -45.683684 +/- 0.005009  1.988539 +/- 0.018267  0.0435 
scale_0.92/vmc/qmc  series 0 -45.751359 +/- 0.004928  1.913282 +/- 0.013998  0.0418 
scale_0.94/vmc/qmc  series 0 -45.791622 +/- 0.005026  1.843704 +/- 0.014460  0.0403 
scale_0.96/vmc/qmc  series 0 -45.809256 +/- 0.005053  1.829103 +/- 0.014536  0.0399 
scale_0.98/vmc/qmc  series 0 -45.806235 +/- 0.004963  1.775391 +/- 0.015199  0.0388 
scale_1.00/vmc/qmc  series 0 -45.783481 +/- 0.005293  1.726869 +/- 0.012001  0.0377 
scale_1.02/vmc/qmc  series 0 -45.741655 +/- 0.005627  1.681776 +/- 0.011496  0.0368 
scale_1.04/vmc/qmc  series 0 -45.685101 +/- 0.005353  1.682608 +/- 0.015423  0.0368 
scale_1.06/vmc/qmc  series 0 -45.615164 +/- 0.005978  1.652155 +/- 0.010945  0.0362 
scale_1.08/vmc/qmc  series 0 -45.543037 +/- 0.005191  1.646375 +/- 0.013446  0.0361 
scale_1.10/vmc/qmc  series 0 -45.450976 +/- 0.004794  1.707649 +/- 0.048186  0.0376 
scale_1.12/vmc/qmc  series 0 -45.371851 +/- 0.005103  1.686997 +/- 0.035920  0.0372 
scale_1.14/vmc/qmc  series 0 -45.265490 +/- 0.005311  1.631614 +/- 0.012381  0.0360 
scale_1.16/vmc/qmc  series 0 -45.161961 +/- 0.004868  1.656586 +/- 0.014788  0.0367 
scale_1.18/vmc/qmc  series 0 -45.062579 +/- 0.005971  1.671998 +/- 0.019942  0.0371 
scale_1.20/vmc/qmc  series 0 -44.960477 +/- 0.004888  1.651864 +/- 0.009756  0.0367 
\end{shade}
\noindent

In this case, we are using a Jastrow factor optimized only at the 
equilibrium geometry (\texttt{scale\_1.00}) but with radial 
cutoffs restricted to the Wigner-Seitz radius of the most compressed 
supercell (\texttt{scale\_0.80}) to avoid introducing wavefunction 
cusps at the cell boundary (had we tried, QMCPACK would have aborted with a warning in 
this case).  It is clear that this restricted Jastrow factor 
is not an optimal choice because it yields variance/energy ratios between $0.036$ 
and $0.057$ Ha.  This issue is largely a result of our undersized (8-atom) 
supercell; larger cells should always be used in real production 
calculations.

Batch processing is also possible for multiple quantities.  If multiple 
quantities are requested, an additional line is inserted to separate 
results from different runs:
\begin{shade}
>qmca -q 'e bc eff' -e 40 scale_*/vmc/*scalar*
scale_0.80/vmc/qmc  series 0 
  LocalEnergy           =          -44.6710 +/-           0.0061 
  BlockCPU              =           0.02986 +/-          0.00038 
  Efficiency            =          38104.00 +/-             0.00 

scale_0.82/vmc/qmc  series 0 
  LocalEnergy           =          -44.9828 +/-           0.0058 
  BlockCPU              =           0.02826 +/-          0.00013 
  Efficiency            =          44483.91 +/-             0.00 

scale_0.84/vmc/qmc  series 0 
  LocalEnergy           =          -45.2283 +/-           0.0054 
  BlockCPU              =           0.02747 +/-          0.00030 
  Efficiency            =          52525.12 +/-             0.00 

scale_0.86/vmc/qmc  series 0 
  LocalEnergy           =          -45.4158 +/-           0.0055 
  BlockCPU              =           0.02679 +/-          0.00013 
  Efficiency            =          50811.55 +/-             0.00 

scale_0.88/vmc/qmc  series 0 
  LocalEnergy           =          -45.5702 +/-           0.0047 
  BlockCPU              =           0.02598 +/-          0.00015 
  Efficiency            =          74148.79 +/-             0.00 

scale_0.90/vmc/qmc  series 0 
  LocalEnergy           =          -45.6837 +/-           0.0050 
  BlockCPU              =           0.02527 +/-          0.00011 
  Efficiency            =          65714.98 +/-             0.00 

...
\end{shade}



\subsection{Twist averaging}
\label{sec:qmca_twist_average}
Twist averaging can be performed straightforwardly for any 
output quantity listed in Section~\ref{sec:qmca_other_quantities} 
with \texttt{qmca}.  We illustrate these capabilities by 
repeating the 8-atom diamond DMC runs performed in Section 
\ref{sec:qmca_judge_dmc} at 8 real-valued supercell twist 
angles (a $2\times 2\times 2$ Monkhorst-Pack grid centered at 
the $\Gamma$-point).  Data traces for each twist can be overlapped 
on the same plot:
\begin{shade}
>qmca -to -q e -e '30 20 30' *scalar* --legend outside
\end{shade}
\noindent
The \texttt{-o} option requests the plots to be overlapped; otherwise,
8 separate plots would be generated.  The 
equilibration input \texttt{-e '30 20 30'} cuts out from 
the analyzed data the first 30 blocks for series 0 (VMC), 
20 blocks for series 1 (intermediate DMC), and 30 blocks for 
series 2 (production DMC).  The resulting plot is shown in 
Figure~\ref{fig:qmca_twist_overlap}.

\begin{figure}
\begin{center}
\ifdefined\HCode
\includegraphics[trim = 0mm 0mm 0mm 0mm, clip,width=0.9\columnwidth]{./figures/qmca_twist_trace_overlap.dmn}
\else
\includegraphics[trim = 0mm 0mm 0mm 0mm, clip,width=0.9\columnwidth]{./figures/qmca_twist_trace_overlap.pdf}
\fi
\end{center}
\caption{Overlapped energy traces from VMC to DMC for an 8-supercell diamond obtained with \texttt{qmca}.  Data for each twist appears in a different color.}
\label{fig:qmca_twist_overlap}
\end{figure}

Twist averaging is performed by providing the \texttt{-a} 
option.  If provided on its own, uniform weights are applied 
to each twist angle.  To obtain a trace plot with twist averaging 
enforced, use a command similar to the following:
\begin{shade}
>qmca -a -t -q e -e '30 20 30' *scalar*
\end{shade}
\noindent
The resulting plot is shown in Figure~\ref{fig:qmca_twist_average}.
As can be seen from the trace plot, the chosen equilibration lengths 
are appropriate, and we proceed to obtain the twist-averaged total energy
from the \texttt{scalar.dat} files
\begin{shade}
>qmca -a -q ev -e 30 --sac *s002.scalar*
                            LocalEnergy               Variance           ratio 
avg  series 2  -45.873369 +/- 0.000753    5.3   1.028751 +/- 0.001056    1.3   0.0224 
\end{shade}
\noindent
and also from the \texttt{dmc.dat} files
\begin{shade}
>qmca -a -q ev -e 300 --sac *s002.dmc*
                            LocalEnergy               Variance           ratio 
avg  series 2  -45.873371 +/- 0.000741   30.5   1.028843 +/- 0.000972    1.6   0.0224 
\end{shade}
\noindent
yielding a twist-averaged total energy of $-45.8733(8)$ Ha. 

\begin{figure}
\begin{center}
\ifdefined\HCode
\includegraphics[trim = 0mm 0mm 0mm 0mm, clip,width=0.75\columnwidth]{./figures/qmca_twist_average_trace.dmn}
\else
\includegraphics[trim = 0mm 0mm 0mm 0mm, clip,width=0.75\columnwidth]{./figures/qmca_twist_average_trace.pdf}
\fi
\end{center}
\caption{Twist-averaged energy trace from VMC to DMC for an 8-supercell diamond obtained with \texttt{qmca}.}
\label{fig:qmca_twist_average}
\end{figure}

As can be seen from the Figure~\ref{fig:qmca_twist_overlap}, some of the twist 
angles are degenerate. This is seen more clearly in the text output:
\begin{shade}
>qmca -q ev -e 30 *s002.scalar*
                            LocalEnergy               Variance           ratio 
qmc.g000  series 2  -45.264510 +/- 0.001942   1.057065 +/- 0.002318   0.0234 
qmc.g001  series 2  -46.035511 +/- 0.001806   1.015992 +/- 0.002836   0.0221 
qmc.g002  series 2  -46.035410 +/- 0.001538   1.015039 +/- 0.002661   0.0220 
qmc.g003  series 2  -46.047285 +/- 0.001898   1.018219 +/- 0.002588   0.0221 
qmc.g004  series 2  -46.034225 +/- 0.002539   1.013420 +/- 0.002835   0.0220 
qmc.g005  series 2  -46.046731 +/- 0.002963   1.018337 +/- 0.004109   0.0221 
qmc.g006  series 2  -46.047133 +/- 0.001958   1.021483 +/- 0.003082   0.0222 
qmc.g007  series 2  -45.476146 +/- 0.002065   1.070456 +/- 0.003133   0.0235 
\end{shade}
\noindent
The degenerate twists grouped by set are $\{0\}$, $\{1,2,4\}$, $\{3,5,6\}$, and
$\{7\}$.

Alternatively, the run could have been performed at the four 
unique (irreducible) twist angles \emph{only}.  We will emulate this situation by 
analyzing data for twists 0, 1, 3, and 7 only.  In a production setting 
with irreducibly weighted twists, the run would be performed on these twists 
alone; we reuse the uniform twist data for illustration purposes only.  

We can use \texttt{qmca} to perform twist averaging with different 
weights applied to each twist:
\begin{shade}
>qmca -a -w '1 3 3 1' -q ev -e 30 *g000*2*sc* *g001*2*sc* *g003*2*sc* *g007*2*sc*
                            LocalEnergy               Variance           ratio 
avg  series 2  -45.873631 +/- 0.001044   1.028769 +/- 0.001520   0.0224 
\end{shade}
\noindent
yielding a total energy value of $-45.874(1)$ Ha, in agreement with the 
uniform weighted twist average performed previously.  

The decision of whether or not to perform irreducible weighted twist 
averaging should be made on the basis of efficiency.  The relative 
efficiency of irreducible vs. uniform weighted twist averaging 
depends on the irreducible weights and the ratio of the lengths of 
the available sampling and equilibration periods.  A formula for 
the relative efficiency of these two cases is derived and discussed 
in more detail in Appendix~\ref{sec:app_ta_efficiency}.


\subsection{Setting output units}
\label{sec:qmca_output_units}
Estimates outputted by \texttt{qmca} are in Hartree units by 
default.  The output units for energetic quantities can be 
changed by using the \texttt{-u} option.  

\vspace{3mm}
\noindent
Energy in Hartrees:
\begin{shade}
>qmca -q e -u Ha -e 20 qmc.s002.scalar.dat
qmc  series 2  LocalEnergy           =  -46.032960 +/- 0.002077
\end{shade}

\noindent
Energy in electron volts:
\begin{shade}
>qmca -q e -u eV -e 20 qmc.s002.scalar.dat
qmc  series 2  LocalEnergy           =  -1252.620565 +/- 0.056521 
\end{shade}

\noindent
Energy in Rydbergs:
\begin{shade}
>qmca -q e -u rydberg -e 20 qmc.s002.scalar.dat
qmc  series 2  LocalEnergy           =  -92.065919 +/- 0.004154   
\end{shade}

\noindent
Energy in kilojoules per mole:
\begin{shade}
>qmca -q e -u kj_mol -e 20 qmc.s002.scalar.dat
qmc  series 2  LocalEnergy           =  -120859.512998 +/- 5.453431   
\end{shade}


\subsection{Speeding up trace plotting}
\label{sec:qmca_fast_trace_plot}
When working with many files or files with many entries, 
\texttt{qmca} might take a long time to produce plots.  The time 
delay is actually due to the autocorrelation time estimate 
used to calculate error bars.  The calculation time for 
the autocorrelation scales as $\mathcal{O}(M^2)$, with $M$ being 
the number of statistical samples.  If you are interested only 
in plotting traces and not in the estimated error bars, the 
autocorrelation time estimation can be turned off with the 
\texttt{--noac} option:
\begin{shade}
>qmca -t -q e -e 20 --noac qmc.s002.scalar.dat
\end{shade}
\noindent
Note that the resulting error bars printed to the console 
will be underestimated and are not meaningful.  Do \emph{not} 
use \texttt{--noac} in conjunction with the \texttt{-p} 
plotting option as these plots are of no use without meaningful 
error bars.


\subsection{Short usage examples}
\label{sec:qmca_short_examples}
\noindent
Plotting a trace of the local energy:
\begin{shade}
>qmca -t -q e *scalar*
\end{shade}
\noindent
Applying an equilibration cutoff to VMC data (series 0):
\begin{shade}
>qmca -q e -e 30 *s000.scalar*
\end{shade}
\noindent
Applying the same equilibration cutoff to VMC and DMC data (series 0, 1, 2):
\begin{shade}
>qmca -q e -e 20 *scalar*
\end{shade}
\noindent
Applying different equilibration cutoffs to VMC and DMC data (series 0, 1, 2):
\begin{shade}
>qmca -q e -e '30 20 40' *scalar*
\end{shade}
\noindent
Obtaining the energy, variance, and variance/energy ratio for all series:
\begin{shade}
>qmca -q ev -e 30 *scalar*
\end{shade}
\noindent
Overlaying plots of mean + error bar for energy and variance for separate 
two- and three-body Jastrow optimization runs:
\begin{shade}
>qmca -po -q ev ./optJ2/*scalar* ./optJ3/*scalar*
\end{shade}
\noindent
Obtaining the acceptance ratio:
\begin{shade}
>qmca -q ar -e 30 *scalar*
\end{shade}
\noindent
Obtaining the average DMC walker population:
\begin{shade}
>qmca -q nw -e 400 *s002.dmc.dat
\end{shade}
\noindent
Obtaining the MC efficiency:
\begin{shade}
>qmca -q eff -e 30 *scalar*
\end{shade}
\noindent
Obtaining the total wall clock time per series:
\begin{shade}
>qmca -q tt -e 0 *scalar*
\end{shade}
\noindent
Obtaining the average wall clock time spent per block:
\begin{shade}
>qmca -q bc -e 0 *scalar*
\end{shade}
\noindent
Obtaining a subset of desired quantities:
\begin{shade}
>qmca -q 'e v ar eff' -e 30 *scalar*
\end{shade}
\noindent
Obtaining all available quantities:
\begin{shade}
>qmca -e 30 *scalar*
\end{shade}
\noindent
Obtaining the twist-averaged total energy with uniform weights:
\begin{shade}
>qmca -a -q e -e 40 *g*s002.scalar.dat
\end{shade}
\noindent
Obtaining the twist-averaged total energy with specific weights:
\begin{shade}
>qmca -a -w '1 3 3 1' -q e -e 40 *g*s002.scalar.dat
\end{shade}
\noindent
Obtaining the local, kinetic, and potential energies in eV:
\begin{shade}
>qmca -q ekp -e 30 -u eV *scalar*
\end{shade}



\subsection{Production quality checklist}
\label{sec:qmca_production_checklist}

\begin{enumerate}
  \item{Inspect the trace plots (\texttt{-t} option) for any 
    oddities in the data.  Typical behavior is a short equilibration 
    period followed by benign fluctuations around a clear mean value.  
    There should not be any large spikes in the data. This applies 
    to \emph{all} runs (VMC, optimization, DMC, etc.).}

  \item{Remove all equilibration steps (\texttt{-e} option) from 
    the data by inspecting the trace plot.}

  \item{Check the quality of the orbitals (standalone Jastrow-less 
    VMC or sometimes the first \texttt{scalar} file produced during 
    optimization) by inspecting the variance/energy ratio 
    \texttt{qmca -q ev *scalar*}.  For pseudopotential systems 
    without a Jastrow, the variance/energy ratio should not exceed 
    $0.2$ Ha; otherwise, there is a problem with the orbitals.}

  \item{Check the quality of the optimized Jastrow factor by inspecting 
    the variance/energy ratio.  For pseudopotential systems with a 
    Jastrow, the variance/energy ratio should not exceed $0.04$ Ha 
    for pseudopotential systems.  A good Jastrow is indicated by a 
    variance/energy ratio in the range of $0.01-0.03$ Ha.  A value less 
    than $0.01$ Ha is difficult to achieve.}

  \item{Confirm that the optimization has converged by plotting the 
    energy and variance vs. optimization series 
    (\texttt{qmca -p -q ev *scalar*}).  Do not assume that 
    optimization has converged in only a few cycles.  Use at least 
    10 cycles with about 100,000 samples unless you already have 
    experience with the system in question.}

  \item{Optimize Jastrow factors according to energy minimization to 
    reduce locality errors arising from the use of nonlocal 
    pseudopotentials in DMC.  A good approach is to optimize with a 
    few cycles of variance minimization followed by several cycles of 
    energy minimization.}

  \item{Occasionally try optimizing with more samples and/or cycles 
    to see if improved results are obtained.}

  \item{If using a B-spline representation of the orbitals, converge 
    the VMC energy and variance with respect to the mesh size (controlled 
    via meshfactor).  This is best done in the presence of any 
    Jastrow factor to reduce noise.  Consider using the hybrid LMTO 
    representation of the orbitals as this can reduce both the VMC/DMC 
    variance and the DMC time step error, in addition to saving memory.}

  \item{Check the variance/energy ratio of all production VMC and DMC 
    calculations.  In all cases, the DMC ratio should be slightly 
    less than the VMC ratio and both should abide the preceding guidelines, 
    i.e., the ratio should be less than $0.04$ Ha for 
    pseudopotential systems.  The production ratio should also be 
    consistent with what is observed during wavefunction optimization.}

  \item{Be aware of population control bias in DMC.  Run with a 
    population of $\sim 2,000$ or greater.  Occasionally repeat a run 
    using a larger population to explicitly confirm that population 
    control bias is small.}

  \item{Check the stability of the DMC walker population by plotting 
    the trace of the population size (\texttt{qmca -t -q nw *dmc.dat}).  
    Verify that the average walker population is consistent with 
    the requested value provided in the input.}

  \item{In DMC, perform a time step study to obtain either (1) 
    extrapolated results or (2) a time step for future 
    production where an energy difference shows convergence 
    (e.g., a band gap or defect formation energy).  For 
    pseudopotential systems, converged time steps for many systems 
    are in the range of $0.002-0.01$ Ha$^{-1}$, but the actual converged 
    time step must be explicitly checked.}

  \item{In periodic systems, converge the total energy with respect to 
    the size of the twist/k-point grid.  Results for smaller systems 
    can easily be transferred to larger ones (e.g., a $2 \times 2 \times 2$ twist 
    grid in a $2 \times 2 \times 2$ tiled cell is equivalent to a $1 \times 1 \times 1$ twist grid in a 
    $4 \times 4 \times 4$ tiled cell)}.

  \item{In periodic systems, perform finite-size extrapolation 
    including two body corrections (needed for cohesive energy/phase 
    stability studies) unless it can be shown that finite-size effects 
    cancel for the energy difference in question (e.g., some 
    defect formation energies).}

\end{enumerate}


\section{Using the qmc-fit tool for statistical time step extrapolation and curve fitting}
\label{sec:qmcfit}

The \texttt{qmc-fit} tool is used to provide statistical estimates of
curve-fitting parameters based on QMCPACK data.  Although \texttt{qmc-fit}
will eventually support many types of fitted curves (e.g., Morse
potential binding curves and various equation-of-state fitting curves),
it is currently limited to estimating fitting parameters related to
time step extrapolation.

\subsection{The jackknife statistical technique}
The \texttt{qmc-fit} tool obtains estimates of fitting parameter
means and associated error bars via the ``jack-knife''
technique.  This technique is a powerful and general tool
to obtain meaningful error bars for any quantity that is related
in a nonlinear fashion to an underlying set of statistical data.
For this reason, we give a brief overview of the jackknife
technique before proceeding with usage instructions for the
\texttt{qmc-fit} tool.

Consider $N$ statistical variables $\{x_n\}_{n=1}^N$ that have
been outputted by one or more simulation runs.  If we have
$M$ samples of each of the $N$ variables, then the mean values
of each these variables can be estimated in the standard way,
that is, $\bar{x}_n\approx \tfrac{1}{M}\sum_{m=1}^Mx_{nm}$.

Suppose we are interested in $P$ statistical quantities
$\{y_p\}_{p=1}^P$ that are related to the original $N$ variables
by a known multidimensional function $F$:
\begin{align}
  y_1,y_2,\ldots,y_P &= F(x_1,x_2,\ldots,x_N)\quad \textrm{or} \nonumber \\
  \vec{y} &= F(\vec{x})\:.
\end{align}
The relationship implied by $F$ is completely general. 
For example, the $\{x_n\}$ might be elements of a matrix
with $\{y_p\}$ being the eigenvalues, or $F$ might be
a fitting procedure for $N$ energies at different time steps
with $P$ fitting parameters.  An approximate guess at the mean
value of $\vec{y}$ can be obtained by evaluating $F$ at the mean
value of $\vec{x}$ (i.e. $F(\bar{x}_1\ldots\bar{x}_N)$), but with
this approach we have no way to estimate the statistical error 
bar of any $\bar{y}_p$.

In the jackknife procedure, the statistical variability intrinsic
to the underlying data $\{x_n\}$ is used to obtain estimates of the
mean and error bar of $\{y_p\}$.  We first construct a new set of $x$
statistical data by taking the average over all samples but one:
\begin{align}
  \tilde{x}_{nm} = \frac{1}{N-1}(N\bar{x}_n-x_{nm})\qquad m\in [1,M]\:.
\end{align}
The result is a distribution of approximate $x$ mean values.  These
are used to construct a distribution of approximate means for $y$:
\begin{align}
  \tilde{y}_{1m},\ldots,\tilde{y}_{Pm} = F(\tilde{x}_{1m},\ldots,\tilde{x}_{Nm}) \qquad m\in [1,M]\:.
\end{align}
Estimates for the mean and error bar of the quantities of
interest can finally be obtained using the following formulas:
\begin{align}
  \bar{y}_p &= \frac{1}{M}\sum_{m=1}^M\tilde{y}_{pm}\:.  \\
  \sigma_{y_p} &= \sqrt{\frac{M-1}{M}\left(\sum_{m=1}^M\tilde{y}_{pm}^2-M\bar{y}_p^2\right)}\:.
\end{align}


\subsection{Performing time step extrapolation}
In this section, we use a 32-atom supercell of MnO as an example
system for time step extrapolation.  Data for this system has been
collected in DMC using the following sequence of time steps:
$0.04,~0.02,~0.01,~0.005,~0.0025,~0.00125$ Ha$^{-1}$.  For a typical
production pseudopotential study, time steps in the range of
$0.02-0.002$ Ha$^{-1}$ are usually sufficient and it is recommended
to increase the number of steps/blocks by a factor of two when
the time step is halved.  To perform accurate statistical
fitting, we must first understand the equilibration and autocorrelation
properties of the inputted local energy data.  After plotting the
local energy traces (\texttt{qmca -t -q e -e 0 ./qmc*/*scalar*}),
it is clear that an equilibration period of $30$ blocks is reasonable.
Approximate autocorrelation lengths are also obtained with \texttt{qmca}:
\begin{shade}
>qmca -e 30 -q e --sac ./qmc*/qmc.g000.s002.scalar.dat
./qmc_tm_0.00125/qmc.g000 series 2 LocalEnergy = -3848.234513 +/- 0.055754  1.7 
./qmc_tm_0.00250/qmc.g000 series 2 LocalEnergy = -3848.237614 +/- 0.055432  2.2 
./qmc_tm_0.00500/qmc.g000 series 2 LocalEnergy = -3848.349741 +/- 0.069729  2.8 
./qmc_tm_0.01000/qmc.g000 series 2 LocalEnergy = -3848.274596 +/- 0.126407  3.9 
./qmc_tm_0.02000/qmc.g000 series 2 LocalEnergy = -3848.539017 +/- 0.075740  2.4 
./qmc_tm_0.04000/qmc.g000 series 2 LocalEnergy = -3848.976424 +/- 0.075305  1.8 
\end{shade}
\noindent
The autocorrelation must be removed from the data before jackknifing,
so we will reblock the data by a factor of 4.

The \texttt{qmc-fit} tool can be used in the following way to obtain
a linear time step fit of the data:
\begin{shade}
>qmc-fit ts -e 30 -b 4 -s 2 -t '0.00125 0.0025 0.005 0.01 0.02 0.04' ./qmc*/*scalar*
fit function  : linear
fitted formula: (-3848.193 +/- 0.037) + (-18.95 +/- 1.95)*t
intercept     : -3848.193 +/- 0.037  Ha
\end{shade}
The input arguments are as follows: \texttt{ts} indicates we are
performing a time step fit, \texttt{-e 30} is the equilibration period
removed from each set of scalar data, \texttt{-b 4} indicates the data
will be reblocked by a factor of 4 (e.g., a file containing 400 
entries will be block averaged into a new set of 100 before jackknife
fitting), \texttt{-s 2} indicates that the time step data begins with
series 2 (scalar files matching \texttt{*s000*} or \texttt{*s001*} are
to be excluded), and \texttt{-t } `0.00125 0.0025 0.005 0.01 0.02 0.04' provides a list of time step values corresponding to the inputted scalar
files.  The \texttt{-e} and \texttt{-b} options can receive a
list of file-specific values (same format as \texttt{-t}) if desired.
As can be seen from the text output, the parameters for the linear fit
are printed with error bars obtained with jackknife resampling and
the zero time step ``intercept'' is $-3848.19(4)$ Ha.  In addition to
text output, the previous command will result in a plot of the fit with
the zero time step value shown as a red dot, as shown in the left
panel of Figure~\ref{fig:qmcfit_timestep}.

\begin{figure}
  \centering
\ifdefined\HCode%
  \begin{tabularx}{1024pt}{X X}
    \includegraphics[trim=0mm 0mm 4mm 0mm,clip,width=512pt]{./figures/qmcfit_timestep_linear.dmn}&
    \includegraphics[trim=2mm 0mm 4mm 0mm,clip,width=512pt]{./figures/qmcfit_timestep_quadratic.dmn}\\
  \end{tabularx}
\else%
  \begin{tabularx}{\textwidth}{X X}
    \includegraphics[trim=0mm 0mm 4mm 0mm,clip,width=0.47\textwidth]{./figures/qmcfit_timestep_linear.pdf}&
    \includegraphics[trim=2mm 0mm 4mm 0mm,clip,width=0.47\textwidth]{./figures/qmcfit_timestep_quadratic.pdf}\\
    \end{tabularx}
\fi%
\caption{Linear (left) and quadratic (right) time step fits to DMC data for a 32-atom supercell of MnO obtained with \texttt{qmc-fit}.  Zero time step estimates are indicated by the red data point on the left side of either panel.}
  \label{fig:qmcfit_timestep}
\end{figure}

Different fitting functions are supported via the \texttt{-f} option.
Currently supported options include \texttt{linear} ($a+bt$),
\texttt{quadratic} ($a+bt+ct^2$), and \texttt{sqrt} ($a+b\sqrt{t}+ct$).
Results for a quadratic fit are shown subsequently and in the right
panel of Figure\ref{fig:qmcfit_timestep}.
\begin{shade}
>qmc-fit ts -f quadratic -e30 -b4 -s2 -t '0.00125 0.0025 0.005 0.01 0.02 0.04' ./qmc*/*scalar*
fit function  : quadratic
fitted formula: (-3848.245 +/- 0.047) + (-7.25 +/- 8.33)*t + (-285.00 +/- 202.39)*t^2
intercept     : -3848.245 +/- 0.047  Ha
\end{shade}
In this case, we find a zero time step estimate of $-3848.25(5)$ Ha$^{-1}$.
A time step of $0.04$ Ha$^{-1}$ might be on the large side to include in
time step extrapolation, and it is likely to have an outsize influence
in the case of linear extrapolation.  Upon excluding this point, linear
extrapolation yields a zero timestep value of $-3848.22(4)$ Ha$^{-1}$.
Note that quadratic extrapolation can result in intrinsically
larger uncertainty in the extrapolated value.  For example, when the $0.04$
Ha$^{-1}$ point is excluded, the uncertainty grows by 50\% and we obtain an
estimated value of $-3848.28(7)$ instead.


\section{Using the qdens tool to obtain electron densities}
\label{sec:qdens}

The \texttt{qdens} tool is provided to post-process the heavy density data 
produced by QMCPACK and output the mean density (with and without errorbars) 
in file formats viewable with, e.g.,  XCrysDen or VESTA.  The tool currently 
works only with the \texttt{SpinDensity} estimator in QMCPACK.

Note: this tool is provisional and may be changed or replaced at any time.  
The planned successor to this tool (\texttt{qstat}) will expand access to 
other observables and will retain at least the non-plotting capabilities of 
\texttt{qdens}.

To use \texttt{qdens}, Nexus must be installed along with NumPy and H5Py. 
A short list of example use cases are covered in the next section.  Current 
input flags are:

\begin{shade}
>qdens

Usage: qdens [options] [file(s)]

Options:
  --version             show program's version number and exit
  -h, --help            Print help information and exit (default=False).
  -v, --verbose         Print detailed information (default=False).
  -f FORMATS, --formats=FORMATS
                        Format or list of formats for density file output.
                        Options: dat, xsf, chgcar (default=None).
  -e EQUILIBRATION, --equilibration=EQUILIBRATION
                        Equilibration length in blocks (default=0).
  -r REBLOCK, --reblock=REBLOCK
                        Block coarsening factor; use estimated autocorrelation
                        length (default=None).
  -a, --average         Average over files in each series (default=False).
  -w WEIGHTS, --weights=WEIGHTS
                        List of weights for averaging (default=None).
  -i INPUT, --input=INPUT
                        QMCPACK input file containing structure and grid
                        information (default=None).
  -s STRUCTURE, --structure=STRUCTURE
                        File containing atomic structure (default=None).
  -g GRID, --grid=GRID  Density grid dimensions (default=None).
  -c CELL, --cell=CELL  Simulation cell axes (default=None).
  --lineplot=LINEPLOT   Produce a line plot along the selected dimension: 0,
                        1, or 2 (default=None).
  --noplot              Do not show plots interactively (default=False).

\end{shade}


\subsection{Usage examples}

Process a single file, excluding the first 40 blocks, and produce XSF files:

\begin{shade}
  qdens -v -e 40 -f xsf -i qmc.in.xml qmc.s000.stat.h5
\end{shade}

Process files for all available series:

\begin{shade}
  qdens -v -e 40 -f xsf -i qmc.in.xml *stat.h5
\end{shade}

Combine groups of 10 adjacent statistical blocks together (appropriate if the 
estimated autocorrelation time is about 10 blocks): 

\begin{shade}
  qdens -v -e 40 -r 10 -f xsf -i qmc.in.xml qmc.s000.stat.h5
\end{shade}

Apply different equilibration lengths and reblocking factors to each series 
(below is appropriate if there are three series, e.g. \texttt{s000}, \texttt{s001}, and \texttt{s002}):

\begin{shade}
  qdens -v -e '20 20 40' -r '4 4 8' -f xsf -i qmc.in.xml *stat.h5
\end{shade}

Produce twist averaged densities (also works with multiple series and reblocking):

\begin{shade}
  qdens -v -a -e 40 -f xsf -i qmc.g000.twistnum_0.in.xml qmc.g*.s000.stat.h5
\end{shade}

Twist averaging with arbitrary weights can be performed via the \texttt{-w} option in a fashion identical to \texttt{qmca}.


\subsection{Files produced}

Look for files with names and extensions similar to:

\begin{shade}
  qmc.s000.SpinDensity_u.xsf      
  qmc.s000.SpinDensity_u-err.xsf  
  qmc.s000.SpinDensity_u+err.xsf  

  qmc.s000.SpinDensity_d.xsf        
  qmc.s000.SpinDensity_d-err.xsf    
  qmc.s000.SpinDensity_d+err.xsf    

  qmc.s000.SpinDensity_u+d.xsf    
  qmc.s000.SpinDensity_u+d-err.xsf  
  qmc.s000.SpinDensity_u+d+err.xsf

  qmc.s000.SpinDensity_u-d.xsf    
  qmc.s000.SpinDensity_u-d-err.xsf  
  qmc.s000.SpinDensity_u-d+err.xsf  
\end{shade}

Files postfixed with \texttt{u} relate to the up electron density, \texttt{d} to down, \texttt{u+d} to the total charge density, and \texttt{u-d} to the difference between up and down electron densities.

Files without \texttt{err} in the name contain only the mean, whereas files with \texttt{+err}/\texttt{-err} in the name contain the mean plus/minus the estimated error bar.  Please use caution in interpreting the error bars as their accuracy depends crucially on a correct estimation of the autocorrelation time by the user (see \texttt{-r} option) and having a sufficient number of blocks remaining following any reblocking.

When twist averaging, the group tag (e.g. \texttt{g000} or similar) will be replaced with \texttt{avg} in the names of the outputted files.

\chapter{Periodic LCAO for solids}
\label{chap:LCAO}

\section{Introduction}

QMCPACK implements the linear combination of atomic orbitals (LCAO) and Gaussian
basis sets in periodic boundary conditions. This method uses orders of
magnitude less memory than the real-space spline wavefunction. Although
the spline scheme enables very fast evaluation of the wavefunction, it might
require too much on-node memory for a large complex cell. The periodic
Gaussian evaluation provides a fallback that will definitely fit in
available memory but at significantly increased computational
expense. Well-designed Gaussian basis sets should be used to accurately
represent the wavefunction, typically
including both diffuse and high angular momentum functions.

The current implementation is not highly optimized for efficiency but can handle real and complex trial wavefunctions generated by PySCF\cite{Sun2018}, but other codes such as
Crystal can be interfaced on request. Supercell tiling is handled outside QMCPACK through a proper PySCF input generated by Nexus and the Supercell geometry and coefficients of the molecular orbotals are constructed in the converter provided by QMCPACK. This is different from the plane wave/spline route where the tiling is provided in QMCPACK.   

%\subsection{Single Particle Orbitals}
%
%In QMC the many-body trial wavefunction is expressed as the product of an antisymmetric part and a correlating Jastrow factor:
% \begin{equation}
%\Psi_T(\vec{R}) = \mathcal{A}(\vec{R}) \exp\left[\sum_i J_i(\vec{R})\right]
%\end{equation}
%
%Where $\Psi_T(\vec{R})$ is the trial wave function, $\vec{R}$ is a space spin coordinates, $J(\vec{R})$ the jastrow function and $\mathcal{A}(\vec{R})$  the antisymmetric wavefunction. $\mathcal{A}(\vec{R})$  is traditionally obtained from methods such as DFT, Hartree Fock, MCSCF or CI expansion.  Many trial-wavefunctions forms have been explored, but the most popular and effective general form remains the Slater Jastrow form
% \begin{equation}
%\Psi_T(\vec{R}) = \exp\left[\sum_i J_i(\vec{R})\right]\sum_k^M C_kD_k^{\uparrow}(\varphi)D_k^{\downarrow}(\varphi)
%\end{equation}
%Where $D_k^{\downarrow}(\varphi)$ is a slater determinant expressed in terms of single particle orbitals (SPO) $\varphi_i=\sum^{N_b}_l C_l ^i \Phi_l$ . The choice of SPO representation is crucial for QMC as the cost of computing $\Phi_l$ scales linearly with the number of basis functions evaluation.  The scaling grows with the system size and the total evaluation of the N SPOs scales as  $ \mathcal{O}(N)^3$ per Monte Carlo step. In the QMCPACK parallelization scheme, SPOs are stored in read only memory replicated on each node or GPU, limiting the size of the systems to the available memory per node. 
%
%In real space QMC methods it is standard to use a real space b-spline scheme or a closely related method, due to the considerable speedup over plane-waves while retaining simple convergence properties. Use of atomic orbitals and Gaussians that include more physics or chemistry results in much more efficient basis sets, but gives up easy convergence properties. 
%
%\subsubsection{B-splines}
%  3D tricubic B-splines provide a basis in which only
%64 elements are nonzero at any given point in space.
%The one-dimensional cubic B-spline is given by,
%\begin{equation}
%f(x) = \sum_{i'=i-1}^{i+2} b^{i'\!,3}(x)\,\,  p_{i'},
%\label{eq:SplineFunc}
%\end{equation}
%where $b^{i}(x)$ are $p_i$ the piecewise cubic polynomial basis functions
%and $i = \text{floor}(\Delta^{-1} x)$ is the index of
%the first grid point $\le x$.  Constructing a tensor product in each Cartesian
%direction, we can represent a 3D orbital as
%\begin{equation}
%  \phi_n(x,y,z) =
%  \!\!\!\!\sum_{i'=i-1}^{i+2} \!\! b_x^{i'\!,3}(x)
%  \!\!\!\!\sum_{j'=j-1}^{j+2} \!\! b_y^{j'\!,3}(y)
%  \!\!\!\!\sum_{k'=k-1}^{k+2} \!\! b_z^{k'\!,3}(z) \,\, p_{i', j', k',n}.
%\label{eq:TricubicValue}
%\end{equation}
%This allows the rapid evaluation of each orbital in constant time.
%Furthermore, this basis is systematically improvable with a single spacing
%parameter, so that accuracy is not compromised and convergence checks are simple.
%
%Trial wavefunctions for materials are commonly produced using plane wave codes such as Quantum Espresso. The conversion to real space b-splines is straightforward. Compared to directly evaluating Fourier series, b-splines are approximately one order of magnitude faster, with the speedup increasing with system size.
%
%\subsubsection{Linear Combination of Atomic Orbitals (LCAO)}

LCAO schemes use physical considerations to construct a highly
efficient basis set compared with plane waves. Typically only a few tens
of basis functions per atom are required compared with thousands of
plane waves. Many forms of LCAO schemes exist and are being
implemented in QMCPACK. The details of the already-implemented methods
are described in the following section.

\noindent \textbf{GTOs:}
 The Gaussian basis functions follow a radial-angular decomposition of
\begin{equation}
     \phi (\mathbf{r} )=R_{l}(r)Y_{lm}(\theta ,\phi )\:,
\end{equation}
where $ Y_{{lm}}(\theta ,\phi )$ is a spherical harmonic, $l$ and $m$
are the angular momentum and its $z$ component, and $r, \theta, \phi$
are spherical coordinates. In practice, they are atom centered and the
$l$ expansion typically includes 1--3 additional channels compared with
the formally occupied states of the atom (e.g., 4--6 for a nickel atom with
occupied $s$, $p$, and $d$ electron shells.

The evaluation of GTOs within PBC differs slightly from evaluating
GTOs in open boundary conditions (OBCs).  The orbitals are evaluated at
a distance $r$ in the primitive cell (similar to OBC), and then the
contributions of the periodic images are added by evaluating the
orbital at a distance $r+T$, where T is a translation of the cell
lattice vector. This requires loops over the periodic images until the
contributions are orbitals $\Phi$. In the current implementation, the
number of periodic images is an input parameter named
\texttt{PBCimages}, which takes three integers corresponding to the
number of periodic images along the supercell axes (X, Y and Z axes
for a cubic cell). By default these parameters are set to
\texttt{PBCimages= 5 5 5}, but they \textbf{require manual convergence
  checks}. Convergence checks can be performed by checking the total
energy convergence with respect to \texttt{PBCimages}, similar to checks
performed for plane wave cutoff energy and B-spline grids. Use of
diffuse Gaussians might require these parameters to be increased, while
sharply localized Gaussians might permit a decrease. The cost of
evaluating the wavefunction increases sharply as \texttt{PBCimages} is
increased. This input parameter will be replaced by a tolerance
factor and numerical screening in the future.

\section{Generating and using periodic Gaussian-type wavefunctions
  using PySCF}

Similar to any QMC calculation, using periodic GTOs requires the
generation of a periodic trial wavefunction. QMCPACK is currently
interfaced to PySCF, which is a multipurpose electronic structure
written mainly in Python with key numerical functionality implemented
via optimized C and C++ libraries\cite{Sun2018}. Such a wavefunction
can be generated according to the following example for a $2 \times 1 \times 1$ supercell using tiling (kpoints) and a supertwist shifted away from $\Gamma$, leading to a complex wavefunction.  
%Note that the current implementation and examples cover only
%the use of k-points where symmetry allows real coefficients to be
%used.  This allows calculation at $\Gamma$) and, e.g., some high
%symmetry k-points at the Brillouin zone edges.  More general k-points
%requiring complex coefficients will be supported in future releases.

\begin{lstlisting}[style=Python,caption=Example PySCF input for single k-point calculation for a $2 \times 1 \times 1$ carbon supercell.]
#!/usr/bin/env python
import numpy
import h5py
from pyscf.pbc import gto, scf, dft, df
from pyscf.pbc import df

cell = gto.Cell()
cell.a             = '''
         3.37316115       3.37316115       0.00000000
         0.00000000       3.37316115       3.37316115
         3.37316115       0.00000000       3.37316115'''
cell.atom = '''  
   C        0.00000000       0.00000000       0.00000000
   C        1.686580575      1.686580575      1.686580575 
            '''
cell.basis         = 'bfd-vdz'
cell.ecp           = 'bfd'
cell.unit          = 'B'
cell.drop_exponent = 0.1
cell.verbose       = 5
cell.charge        = 0
cell.spin          = 0
cell.build()


sp_twist=[0.07761248, 0.07761248, -0.07761248]

kmesh=[2,1,1]
kpts=[[ 0.07761248,  0.07761248, -0.07761248],[ 0.54328733,  0.54328733, -0.54328733]]


mf = scf.KRHF(cell,kpts)
mf.exxdiv = 'ewald'
mf.max_cycle = 200

e_scf=mf.kernel()

ener = open('e_scf','w')
ener.write('%s\n' % (e_scf))
print('e_scf',e_scf)
ener.close()

title="C_diamond-tiled-cplx"
from PyscfToQmcpack import savetoqmcpack
savetoqmcpack(cell,mf,title=title,kmesh=kmesh,kpts=kpts,sp_twist=sp_twist)

\end{lstlisting}

Note that the last three lines of the file
\begin{lstlisting}[style=Python]
title="C_diamond-tiled-cplx"
from PyscfToQmcpack import savetoqmcpack
savetoqmcpack(cell,mf,title=title,kmesh=kmesh,kpts=kpts,sp_twist=sp_twist)
\end{lstlisting}

contain the title (name of the HDF5 to be used in QMCPACK) and the call to the converter. The title variable will be the name of the
HDF5 file where all the data needed by QMCPACK will be stored.  The
function \textit{savetoqmcpack} will be called at the end of the
calculation and will generate the HDF5 similarly to the nonperiodic
PySCF calculation in Section~\ref{sec:convert4qmc} (convert4qmc). The
function is distributed with QMCPACK and is located in the
qmcpack/src/QMCTools directory under the name
\textit{PyscfToQmcpack.py}. Note that you need to specify the supertwist coordinates that was used with the provided kpoints. The supertwist must match the coordinates of the K-points otherwise the phase factor for the atomic orbital will be incorrect and incorrect results will be obtained. (For more details on how to generate tiling with PySCF and Nexus,  refer to the Nexus guide or the 2019 QMCPACK Workshop material available on github: \url{https://github.com/QMCPACK/qmcpack_workshop_2019} under \textbf{qmcpack\_workshop\_2019/day2\_nexus/pyscf/04\_pyscf\_diamond\_hf\_qmc/}

For the converter in the script to be called properly, you need
to specify the path to the file in your PYTHONPATH such as


\begin{lstlisting}[style=SHELL]
export PYTHONPATH=QMCPACK_PATH/src/QMCTools:$PYTHONPATH
\end{lstlisting}

%When using multiple k-points, it is necessary to expand the k-points into the equivalent supercell, adjust for the phase factor in the coefficient's value due to the translation by the lattice vector, and order the molecular coefficients from each k-point according to their occupation. These operations are all automated in the \textit{savetoqmcpack()} function.\\

%The following example corresponds to the same carbon system ($2 \times 1 \times 1$); however, in this case, we use a primitive simulation cell and a $2 \times 1 \times 1$ k-point mesh.   

%\begin{lstlisting}[style=Python,caption=Example PySCF input for single k-point calculation for a $2 \times 1 \times 1$ carbon supercell.]
%#!/usr/bin/env python

%import numpy
%from pyscf.pbc import gto, scf, dft,df
%kmesh = [2, 1, 1]

%cell = gto.Cell()
%cell.a = '''
%         3.37316115       3.37316115       0.00000000
%         0.00000000       3.37316115       3.37316115
%         3.37316115       0.00000000       3.37316115'''
%cell.atom = '''  
%   C        0.00000000       0.00000000       0.00000000
%   C        1.686580575      1.686580575      1.686580575 
%            '''
%cell.basis='bfd-vtz'
%cell.ecp = 'bfd'%

%cell.unit='B'
%cell.drop_exponent=0.1
%
%cell.verbose = 5
%
%cell.build()

%kpts = cell.make_kpts(kmesh)
%kpts -= kpts[0]

%mydf = df.GDF(cell,kpts)
%mydf.auxbasis = 'weigend'
%mf = scf.KRHF(supcell,kpts).density_fit()

%mf.exxdiv = 'ewald'
%mf.with_df = mydf
%e_scf=mf.kernel()


%title="C_Diamond-211"

%from PyscfToQmcpack import savetoqmcpack
%savetoqmcpack(supcell,mf,title=title,kpts=kpts,kmesh=kmesh)

%\end{lstlisting}



%Note the difference between the 2 input files where\\
%\begin{lstlisting}
%kmesh=[2,1,1]  #k-point mesh
%\end{lstlisting}%

%\begin{lstlisting}
%kpts = cell.make_kpts(kmesh)
%kpts -= kpts[0]
%\end{lstlisting}
%Will generate k-points centered around the $\Gamma$-point and will ensure that the molecular coefficients are %real.\\

%\begin{lstlisting}[style=Python]
%mf = scf.KRHF(supcell,kpts).density_fit()
%\end{lstlisting}
%The computational algorithm chosen in PySCF is \textit{KRHF} instead of \textit{RHF}.

%Finally, to generate the HDF5 file needed by \qmcpack we call the \textit{savetoqmcpack} function.\\
%\begin{lstlisting}[style=Python]
%from PyscfToQmcpack import savetoqmcpack
%savetoqmcpack(supcell,mf,title=title,kpts=kpts,kmesh=kmesh)
%\end{lstlisting}
%In this call, we simply specify the k-point mesh used to force the converter to generate the desired cell. Note that if the parameter \textit{kmesh} is omitted, the converter will still try to ``guess'' it.



To generate QMCPACK input files, you will need to run  \textit{convert4qmc} exactly as specified in Section ~\ref{sec:convert4qmc} for both cases;
\begin{lstlisting}[style=SHELL]
convert4qmc -pyscf C_diamond-tiled-cplx
\end{lstlisting}

This tool can be used with any option described in convert4qmc. Since
the HDF5 contains all the information needed, there is no need to
specify any other specific tag for periodicity. A supercell at
$\Gamma$-point or using multiple k-points will work without further
modification.

Running convert4qmc will generate 3 input files:\\
\begin{lstlisting}[style=QMCPXML,caption=C\_diamond-tiled-cplx.structure.xml. This file contains the geometry of the system.]
<?xml version="1.0"?>
<qmcsystem>
  <simulationcell>
    <parameter name="lattice">
  6.74632230000000e+00  6.74632230000000e+00  0.00000000000000e+00
  0.00000000000000e+00  3.37316115000000e+00  3.37316115000000e+00
  3.37316115000000e+00  0.00000000000000e+00  3.37316115000000e+00
</parameter>
    <parameter name="bconds">p p p</parameter>
    <parameter name="LR_dim_cutoff">15</parameter>
  </simulationcell>
  <particleset name="ion0" size="4">
    <group name="C">
      <parameter name="charge">4</parameter>
      <parameter name="valence">4</parameter>
      <parameter name="atomicnumber">6</parameter>
    </group>
    <attrib name="position" datatype="posArray">
  0.0000000000e+00  0.0000000000e+00  0.0000000000e+00
  1.6865805750e+00  1.6865805750e+00  1.6865805750e+00
  3.3731611500e+00  3.3731611500e+00  0.0000000000e+00
  5.0597417250e+00  5.0597417250e+00  1.6865805750e+00
</attrib>
    <attrib name="ionid" datatype="stringArray">
 C C C C
</attrib>
  </particleset>
  <particleset name="e" random="yes" randomsrc="ion0">
    <group name="u" size="8">
      <parameter name="charge">-1</parameter>
    </group>
    <group name="d" size="8">
      <parameter name="charge">-1</parameter>
    </group>
  </particleset>
</qmcsystem>
  \end{lstlisting}

  As one can see, for both examples, the two-atom primitive cell has been expanded to contain four atoms in a $2 \times 1 \times 1$ carbon cell.
\begin{lstlisting}[style=QMCPXML,caption=C\_diamond-tiled-cplx.wfj.xml. This file contains the trial wavefunction.]
<?xml version="1.0"?>
<qmcsystem>
  <wavefunction name="psi0" target="e">
    <determinantset type="MolecularOrbital" name="LCAOBSet" source="ion0" transform="yes" twist="0.07761248  0.07761248  -0.07761248" href="C_diamond-tiled-cplx.h5" PBCimages="8  8  8">
      <slaterdeterminant>
        <determinant id="updet" size="8">
          <occupation mode="ground"/>
          <coefficient size="52" spindataset="0"/>
        </determinant>
        <determinant id="downdet" size="8">
          <occupation mode="ground"/>
          <coefficient size="52" spindataset="0"/>
        </determinant>

      </slaterdeterminant>
    </determinantset>
    <jastrow name="J2" type="Two-Body" function="Bspline" print="yes">
      <correlation size="10" speciesA="u" speciesB="u">
        <coefficients id="uu" type="Array"> 0 0 0 0 0 0 0 0 0 0</coefficients>
      </correlation>
      <correlation size="10" speciesA="u" speciesB="d">
        <coefficients id="ud" type="Array"> 0 0 0 0 0 0 0 0 0 0</coefficients>
      </correlation>
    </jastrow>
    <jastrow name="J1" type="One-Body" function="Bspline" source="ion0" print="yes">
      <correlation size="10" cusp="0" elementType="C">
        <coefficients id="eC" type="Array"> 0 0 0 0 0 0 0 0 0 0</coefficients>
      </correlation>
    </jastrow>
  </wavefunction>
</qmcsystem>
 \end{lstlisting}
This file contains information related to the trial wavefunction. It is identical to the input file from an OBC calculation to the exception of the following tags:\\
\begin{table}[h]
\begin{center}
\begin{tabularx}{\textwidth}{l l l l l }
\hline
\multicolumn{5}{l}{*.wfj.xml specific tags} \\
\hline
%\multicolumn{2}{l}{Outputfiles}  & \multicolumn{3}{l}{}\\
   &   \bfseries tag     & \bfseries tag type & \bfseries default   & \bfseries description \\
   &   \texttt{twist             } &  3 doubles  &  ( 0 0 0)& Coordinate of the twist to compute\\
   &   \texttt{href             } &  string  & default& Name of the HDF5 file generated by\\ 
   &                              &          &        &  PySCF and used for convert4qmc\\  
   &   \texttt{PBCimages            } &  3 Integer   & 8 8 8  & Number of periodic images to evaluate the orbitals\\
    \hline
    \end{tabularx}
\end{center}
\end{table}

Other files containing QMC methods (such as optimization, VMC, and DMC blocks) will be generated and will behave in a similar fashion regardless of the type of SPO in the trial wavefunction. 




 


\chapter{Selected Configuration Interaction}
\label{chap:sCI}
A direct path towards improving the accuracy of a QMC calculation is
through a better trial wavefunction.  While using a multireference
wavefunction can be straightforward in theory, in actual practice
methods such as CASSCF are not always intuitive and often require
being an expert in either the method or the code generating the
wavefunction.  An alternative is to use a Selected Configuration of
Interaction method (selected CI) such as CIPSI (Configuration
Interaction using a Perturbative Selection done Iteratively). This
provides a direct route to systematically improving the wavefunction.

\section{Theoretical Background}

The principle behind selected CI is rather simple and was first published in 1955 by R.K. Nesbet\cite{Nesbet1955}.
The first calculations on atoms were performed by Diner, Malrieu and Claverie\cite{Diner1967} in 1967, and it become computationally viable for larger molecules in 2013 by Caffarel \textit{et al.}\cite{Caffarel2013}  
%  \textbf{To Paul: I do not recall who added this section (maybe me?)
% but it is word for word the paper by caffarel in ref
% \cite{Caffarel2013}. It needs either to be removed or strongly
% rewritten. see bellow for attempt to simplify in my own words; }\\

As described by Caffarel et al in Ref.~\cite{Caffarel2013},
multi-determinantal expansions of the ground-state wavefunction
$\Psi_T$ are written as a linear combination of Slater determinants
\begin{equation}
\sum_k c_k \sum_q d_{k,q}D_{k,q\uparrow } (r^{\uparrow})D_{k,q\downarrow}(r^{\downarrow}) %$\ket{D_i}$
\end{equation}
  where each determinant corresponds to a given occupation by the $N_{\alpha}$ and $N_{\beta}$ electrons of $N=N_{\alpha}+N_{\beta}$ orbitals among a set of M spin-orbitals $\{\phi_1,.,\phi_M\}$ (restricted case). When no symmetries are
considered, the maximum number of such determinants is
\begin{eqnarray}
\label{eqn:Det-Permutations}
\left(
\begin{array}{c} M \hspace{1.5mm} \\ N_{\alpha}  \end{array}
\right).
\left(
\begin{array}{c} M \hspace{1.5mm} \\ N_{\beta}  \end{array}
\right).
\end{eqnarray}
a number that grows factorially with M and N. The best representation of the exact wavefunction in the determinantal basis is the full configuration interaction (FCI) wave function written as 
\begin{equation}
\ket{\Psi_0}=\sum_{i} c_{i} \ket{D_i}
\end{equation}
where $c_i$ are the ground-state coefficients obtained by
diagonalizing the matrix, $H_{ij}=\bra{D_i}H\ket{D_j}$, within the
full orthonormalized set $\bra{D_i}\ket{D_j}=\delta_{ij}$, of
determinants $\ket{D_i}$. CIPSI provides a convenient method to build up to this full wavefunction with a single criteria.


A CIPSI wavefunction is built iteratively starting from a reference
wavefunction, usually Hartree-Fock or CASSCF, by adding all single and
double excitations and then iteratively selecting relevant
determinants according to some criteria. Detailed iterative steps can
be found in the reference by Caffarel \textit{et al.} and references
within\cite{Caffarel2013, Scemama2016,Scemama2018,Garniron2017-2} but
are summarized below:

\begin{itemize}
\item Step 1: Define a reference wavefunction:

    \begin{equation}
     \begin{gathered}
       \begin{aligned}
         \ket{\Psi}&=\sum_{i\in D} c_i\ket{i} \,         \,
         &E_{var}&= \frac{\bra{\Psi}\hat{H}\ket{\Psi}}{\bra{\Psi}\ket{\Psi}} \\[12pt]
       \end{aligned} \\[12pt]
     \end{gathered}
   \end{equation}

 
 \item Step 2: Generate external determinants $\ket{\alpha}$:\\
New determinants are added by generating all single and double excitations from determinants $i \in D$ such as:\\ 
\begin{equation}
\bra{\Psi_0^{(n)}}H\ket{D_{i_c}}\neq 0
\end{equation}

\item Step 3: Evaluate second order perturbative contribution to each determinant $\ket{\alpha}$:
\begin{equation}
\Delta E=\frac{\bra{\Psi}\hat{H}\ket{\alpha}\bra{\alpha}\hat{H}\ket{\Psi}}{E_{var}-\bra{\alpha}\hat{H}\ket{\alpha}}
\end{equation}

\item Step 4: Select the determinants with the largest contributions and add them to the Hamiltonian.
\item Step 5: Diagonalize the Hamiltonian within the new added determinants and update the wavefunction and the the value of $E_{var}$.
\item Step 6: Iterate until reaching convergence.\\
\end{itemize}
Repeating this process leads to a multi-reference trial wavefunction of high quality that can be used in QMC. 

\begin{equation}
\Psi_T(r)=e^{J(r)}\sum_k c_k \sum_q d_{k,q}D_{k,q\uparrow } (r^{\uparrow})D_{k,q\downarrow}(r^{\downarrow})
\end{equation}
The linear coefficients $c_k$ are then optimized with the presence of the Jastrow function. 

It is important to note that:
\begin{itemize}
\item When all determinants $\ket{\alpha}$ are selected, the full configuration interaction result is obtained.\\
\item CIPSI can be seen as a deterministic counter part of FCIQMC. \\
\item In practice, any wavefunction method can be made multireference with CIPSI. For instance, a multireference Coupled Cluster (MRCC) with CIPSI is implemented in QP.\cite{Garniron2017-1}\\
\item At any time, with CIPSI selection, $E_{PT_2}=\sum_\alpha \Delta E_\alpha$ estimates the distance to the FCI solution.
\end{itemize}


\subsection{CIPSI wavefunction interface}
\label{sec:cipsi}


\begin{figure}
\begin{center}
\includegraphics[trim = 0mm 0mm 0mm 0mm, clip,width=0.3\columnwidth]{./figures/Reactant.jpg}
\end{center}
\caption{$C_2O_2H_3N$ molecule.}
\protect{\label{fig:C2O2H3N}}
\end{figure}
The CIPSI method
%\cite{XXXrecentCIPSIpaper}
is implemented in the \textit{Quantum Package} (QP) code\cite{QP} developed by the Caffarel group. Once the trial wavefunction is generated, QP is able to produce output readable by the QMCPACK converter as described in section~\ref{sec:convert4qmc}.\\
QP can be installed with multiple plugins for different levels of theory in quantum chemistry. When installing the "QMC" plugin, QP can save the wavefunction in a format readable by the QMCPACK converter. \\

In the following we use the $C_2O_2H_3N$ molecule (Fig~\ref{fig:C2O2H3N}) as an example of how to run a multireference calculation with CIPSI as a trial wavefunction for \qmcpack. The choice of this molecule is motivated by its multireference nature. While the molecule remains small enough for CCSD(T) calculations with aug-cc-pVTZ basis set, the D1 diagnostic shows a very high value for  $C_2O_2H_3N$, suggesting a multireference character.  Therefore, an accurate reference for the system is not available and it becomes difficult to trust the quality of a single-determinant wavefunction, even when using the DFT-B3LYP exchange and correlation functional. Therefore, in the following, we show an example of how to systematically improve the nodal surface by increasing the number of determinants in the trial wavefunction.\\

The following steps show how to run from Hartree-Fock to selected CI using QP, convert the wavefunction to a QMCPACK trial wavefunction and finally how to analyze the result.

\begin{itemize}
\item Step 1: Generate the QP input file:\\
QP takes for input an XYZ file containing the geometry of the molecule such as:

\begin{center}
\begin{tabular}{ l c c c }
8\\
C2O2H3N\\
C &       1.067070 &  -0.370798 &   0.020324\\
C &      -1.115770 &  -0.239135 &   0.081860\\
O &      -0.537581 &   1.047619 &  -0.091020\\
N &       0.879629 &   0.882518 &   0.046830\\
H &      -1.525096 &  -0.354103 &   1.092299\\
H &      -1.868807 &  -0.416543 &  -0.683862\\
H &       2.035229 &  -0.841662 &   0.053363\\
O &      -0.025736 &  -1.160835 &  -0.084319   
\end{tabular}
\end{center}

The input file is generated through the following command line:\\

\begin{shade}
qp_create_ezfio_from_xyz C2O2H3N.xyz -b cc-pvtz 
\end{shade}

 
This means that we will be simulating the molecule in all-electrons within the cc-pVTZ basis set. Other options are of course possible such as using ECPs, different spin multiplicities etc. For more details, to the Quantum Package tutorial \url{https://github.com/LCPQ/quantum_package/wiki/Tutorial}.\\
A directory called \textit{C2O2H3N.ezfio} is created and contains all the relevant data to run the SCF Hartree-Fock calculation. Note that due to the large size of molecular orbitals (220), it is preferable to run QP in parallel. QP parallelization is based on a Master/Slave process allowing a master node to manage the work load between multiple MPI processes through the LibZMQ library. In practice one submits the run to one master node, then submits as many nodes as necessary to speed up the calculations. If a "slave" node dies before the end of its task, the master node will resubmit the workload to another available node. If more nodes are added at any time during the simulation, the master node will use them to reduce the time to solution.\\
\item Step 2: Running Hartree-Fock:\\
To save the integrals on disk and avoid recomputing them later, edit the ezfio directory with the following command:\\
\begin{shade}
qp_edit C2O2H3N.ezfio 
\end{shade}

This will generate a temporary file showing all the contents of the simulation and opens an editor to allow modification of their values. Look for \textit{disk\_access\_ao\_integrals} and modify its value from \textit{None} to \textit{Write}\\

To run a simulation with QP, use the binary \textit{qp\_run} with the desired level of theory, in this case Hartree-Fock (SCF). \\
\begin{shade}
mpirun -np 1 qp_run SCF C2O2H3N.ezfio &> C2O2H3N-SCF.out 
\end{shade}

If run in serial, the evaluation of the integrals and the Hamiltonian diagonalization would take a substantial amount of computer time. It is recommended to add a few more \textit{slave-nodes} to help speed up the calculation.\\

\begin{shade}
mpirun -np 20 qp_run -slave qp_ao_ints C2O2H3N.ezfio &> C2O2H3N-SCF-Slave.out 
\end{shade}
The total Hartree-Fock energy of the system in cc-pVTZ is \textit{$E_{HF}=-283.0992$}Ha.\\ 
\item Step 2: Freeze Core electrons:\\
In order to avoid making excitation from the core electrons, freeze the core electrons and only do the excitations from the valence electrons.\\  
\begin{shade}
qp_set_frozen_core.py C2O2H3N.ezfio
\end{shade}
This will will automatically freeze the orbitals from 1 to 5, and leave the remaining active. \\
\item Step 3: Atomic orbitals to Molecular Orbital transformation\\
This step, transforming the atomic orbitals to molecular orbitals, is the most costly, especially given that its implementation in Quantum Package is serial. It is recommended to do it in a separate run and on one node.\\
\begin{shade}
qp_run four_idx_transform C2O2H3N.ezfio
\end{shade}

The MO integrals are now saved on disk and unless the orbitals are changed, they will not be recomputed.\\
\item Step 4: CIPSI \\
At this point the wavefunction is ready for the selected CI. By default QP has 2 convergence criteria. The number of determinants (set by default to 1M) or the value of PT2 (set by default to $1.10^{-4}$Ha). For this molecule, the total number of determinants in the FCI space is $2.07e+88$ determinants. While his number is completely out of range of what is possible to compute, we will set the limit of determinants in QP to 5M determinants and see if the nodal surface of the wavefunction is converged enough for the DMC. It is important to remember at this point that the main value of CIPSI compared to other selected CI method, is that the value of PT2 is evaluated directly at each step giving a good estimated of the error to the FCI energy. This allows us to conclude that when the E+PT2 energy is converged, the nodal surface is probably also converged.\\
Similar to the SCF runs, FCI runs have to be submitted in parallel with a \textit{Master/Slave} process:\\

\begin{shade}
mpirun -np 1 qp_run fci_zmq C2O2H3N.ezfio &> C2O2H3N-FCI.out 
mpirun -np 199 qp_run -slave selection_davidson_slave C2O2H3N.ezfio\\
&> C2O2H3N-FCI-Slave.out 
\end{shade}

\item Step 5 (optional): Natural orbitals \\
While this step is optional, it is important to note that using natural orbitals instead of Hartree Fock orbitals will always improve the quality of the wavefunction and improve the quality of the nodal surface by reducing the number of needed determinants for the same accuracy. When a full convergence to the FCI limit is attainable, this step will not lead to any change in the energy but will only reduce the total number of determinants. However, if a full convergence is not possible, this step can increase significantly the accuracy of the calculation at the same number of determinants. \\

\begin{shade}
qp_run save_natorb C2O2H3N.ezfio  
\end{shade}
\hfill

At this point, the orbitals are modified, a new AO$\rightarrow$MO transformation is required and Steps 3 and 4 need to be run again.\\

\item Step 6: Analysis of the CIPSI results\\
\begin{figure}
\begin{center}
\includegraphics[trim = 2mm 2mm 2mm 2mm, clip,width=0.95\columnwidth]{./figures/CIPSI.jpg}
\end{center}
\caption{Evolution of the variational energy and the Energy + PT2 as a function of the number of determinants for the $C_2O_2H_3N$ molecule.}
\protect{\label{fig:CIPSI}}
\end{figure}
Figure~\ref{fig:CIPSI} shows the evolution of the variational energy and the energy corrected with PT2 as a function of the number of determinants up to 4M determinants. While it is clear that the raw variational energy is far from being converged, the Energy + PT2 appears converged around 0.4M determinants.\\



\begin{table}[t]
\centering
\caption{Energies of $C_2O_2H_3N$ using orbitals from Hartree-Fock, natural orbitals, 0.4M and 4M determinants}
\label{TAB:CIPSI}
\begin{tabular}{l|c|c}
\hline \hline
Method & N\_det & Energy\\
\hline
Hartree-Fock &    1    & -281.6729\\
Natural Orbitals & 1 & -281.6735\\
E\_Variational &  438,753 & -282.2951 \\
E\_Variational &  4,068,271   & -282.4882 \\
E+PT2 & 438,753& -282.6809 \\
E+PT2 & 4,068,271 & -282.6805  \\ \hline \hline
\end{tabular}
\end{table}


\item Step 7: Truncation of the number of determinants.\\ While using
  all the 4M determinants from CIPSI always guarantees that all
  important determinants are kept in the wavefunction, practically,
  such a large number of determinants would make any QMC calculation
  prohibitively expensive, as the cost of evaluating a determinant in
  DMC grows as $\sqrt[]{N_{det}}$ where $N_{det}$ is the number of
  determinants in the trial wavefunction. To truncate the number of
  determinants, we follow the method described by Scemama
  \textit{et. al}~\cite{Scemama2018} where the wavefunction is
  truncated by removing independently spin-up and spin-down
  determinants whose contribution to the norm of the wavefunction is
  below a user-defined threshold, $\epsilon$. For this step, we choose
  to truncate the determinants whose coefficients are below,
  $1.10^{-3}$, $1.10^{-4}$, $1.10^{-5}$ and $1.10^{-6}$, translating
  to 239, 44539, 541380 and 908128 determinants, respectively.

To  truncate the determinants in QP, edit the ezfio file as follows:

\begin{shade}
qp_edit C2O2H3N.ezfio  
\end{shade}

then look for \textit{ci\_threshold} and modify the value according to the desired threshold. Use the following run to truncate the determinants:

\begin{shade}
qp_run truncate_wf_spin C2O2H3N.ezfio  
\end{shade}

\item Step 7: Save the wavefunction for \qmcpack \\
The wavefunction in QP is now ready to be converted to \qmcpack format.
Save the wavefunction into \qmcpack format and then convert the wavefunction using the \textit{convert4qmc} tool\\

\begin{shade}
qp_run save_for_qmcpack C2O2H3N.ezfio &> C2O2H3N.dump  
convert4qmc -QP C2O2H3N.dump -addCusp -production
\end{shade}

Since we are running all-electron calculations, orbitals in QMC need
to be corrected for the electron-nuclearcusp condition..  This is done
by adding the option \textit{-addCusp} to \textit{convert4qmc}, which
adds a tag forcing \qmcpack to run the correction or read them from a
file if pre-computed. When running multiple DMC runs with different
truncation thresholds, only the number of determinants is varied and
the orbitals remain unchanged from one calculation to another and the
cusp correction needs only be run once.

\item Step 7: Running \qmcpack \\
At this point, running a multideterminant DMC becomes identical to running a regular DMC with \qmcpack; 
After correcting the orbitals for the cusp, optimize the Jastrow functions and then run the DMC. 
It is however important to mention a few remarks;\\

(1) \qmcpack allows reoptimization of the coefficients of the
determinants during the Jastrow optimization step. While this has
proven to lower the energy significantly when the number of
determinants is below 10k, a large number of determinants from CIPSI
is often too large to optimize conveniently. Keeping the coefficients
of the determinants from CIPSI unoptimized is an alternative strategy.\\

(2) The large determinant expansion and the Jastrows are both trying
to recover the missing correlations from the system. When optimizing
the Jastrows, we recommend to first optimize J1 and J2 without the J3,
and then with the added J3. Trying to initially optimze J1, J2 and J3
at the same time may lead to numerical instabilities.\\

(3) The parameters of the Jastrow function will need to be optimized
for each truncation scheme and usually cannot be reused efficiently
from one truncation scheme to another.

\item Step 8: Analyzing DMC results from \qmcpack \\

From Table~\ref{TAB:CIPSI-DMC}, we can see that increasing the number
of determinants from 0.5M to almost 1M determinant keeps the energy
within error bars and does not improve the quality of the nodal
surface. We can conclude that the DMC energy is converged at 0.54M
determinants. It is important to note that this number of determinants
also corresponds to the convergence of E+PT2 in CIPSI calculations,
confirming for this case that the convergence of the nodal surface can
follow the convergence of E+PT2 instead of the more difficult
variational energy.


\begin{table}[t]
\centering
\caption{DMC Energies and CIPSI(E+PT2) of $C_2O_2H_3N$ in function of the number of determinants in the trial wavefunction.}
\label{TAB:CIPSI-DMC}
\begin{tabular}{l|c|c}
\hline 
N\_det & DMC& CISPI\\
\hline
1 & -283.0696 (6)&-283.0063\\
239 & -283.0730 (9)&-282.9063\\
44,539 & -283.078 (1)&-282.7339\\
541,380 & -283.088 (1)&-282.6772\\
908,128& -283.089  (1)&-282.6775\end{tabular}
\end{table}

\end{itemize}

As mentioned in previous sections, DMC is variational relative to the
exact nodal surface. A nodal surface is "better" if it lowers the DMC
energy. To assess the quality of the nodal surface from CIPSI, we
compare these DMC results to other single determinant calculations
from multiple nodal surfaces and theories. Figure~\ref{fig:CIPSI-DMC}
shows the energy of the $C_2O_2H_3N$ molecule as a function of
different single-determinant (SD) trial wavefunctions with an
aug-cc-pVTZ basis set, including Hartree-Fock (HF), DFT-PBE and hybrid
functionals B3LYP and PBE0. The last 4 points in the plot show the
systematic improvement of the nodal surface as a function of the
number of determinants. 

When the DMC-CIPSI energy is converged with respect to the number of
determinants, its nodal surface is still lower than the best SD-DMC
(B3LYP) by 6(1)mHa. When compared to CCSD(T) with the same basis set,
$E_{CCSD(T)}$ is 4mHa higher than DMC-CIPSI and 2mHa lower than
DMC-B3LYP. While 6 (1) mHa can seem very small, it is however
important to remember that CCSD(T) cannot describe correctly
multireference systems and therefore it is impossible to assess the
correctness of the SD-DMC result, making CIPSI-DMC calculations an
ideal benchmark tool for multireference systems.

\begin{figure}
\begin{center}
\includegraphics[trim = 2mm 2mm 2mm 2mm, clip,width=0.9
\columnwidth]{./figures/DMC-Multidet.jpg}
\end{center}
\caption{DMC energy of $C_2O_2H_3N$ molecule as a function of different single determinant trial wavefunctions with aug-ccp-VTZ basis set using nodal surfaces from Hartree-Fock (HF), DFT-PBE and DFT with hybrid functionals PBE0 and P3LYP. The CIPSI trial wavefunction contains as indicated 239, 44539, 514380 and 908128 determinants (D). }
\protect{\label{fig:CIPSI-DMC}}

\end{figure}


\chapter{Auxiliary-Field Quantum Monte Carlo}
\label{chap:afqmc}
The Auxiliary-Field Quantum Monte Carlo (AFQMC) method is an orbital-space formulation of the imaginary-time propagation algorithm. We refer the reader to one of the review articles on the method \cite{AFQMC_review,PhysRevLett.90.136401,PhysRevE.70.056702}, for a detailed description of the algorithm. It uses the Hubbard-Stratonovich transformation to express the imaginary-time propagator, which is inherently a 2-body operator, as an integral over 1-body propagators which can be efficiently applied to an arbitrary Slater determinant. This transformation allows us to represent the interacting many-body system as an average over a non-interacting system (e.g. Slater determinants) in a time-dependent fluctuating external field (the Auxiliary fields). The walkers in this case represent non-orthogonal Slater determinants, whose time average represent the desired quantum state. QMCPACK currently implements the phaseless AFQMC algorithm of Zhang and Krakauer \cite{PhysRevLett.90.136401}, where a trial wave-function is used to project the simulation to the real axis, controlling the fermionic sign problem at the expense of a bias. This approximation is similar in spirit to the fixed-node approximation in real-space DMC, but applied in the Hilbert space where the AFQMC random walk occurs.     

\section{Theoretical Background}
... Coming Soon ...

\section{Input}

The input for an AFQMC calculation is fundamentally different to the input for other real-space algorithms in QMCPACK. The main source of input comes from the Hamiltonian matrix elements in an appropriate single particle basis. This must be evaluated by an external code and saved in a format that QMCPACK can read. More details about file formats are found below. The input file has six basic xml-blocks: \texttt{AFQMCInfo}, \texttt{Hamiltonian}, \texttt{Wavefunction}, \texttt{WalkerSet}, \texttt{Propagator}, and \texttt{execute}. The first five define input structures required for various types of calculations. The \texttt{execute} block represents actual calculations and takes as input the other blocks. 
Non-execution blocks are parsed first, followed by a second pass where execution blocks are parsed (and executed) in order. Listing 13.1 shows an example of a minimal input file for an AFQMC calculation. Table \ref{table:afqmc_basic} shows a brief description of the most important parameters in the calculation. All xml sections contain a ``name'' argument used to identify the resulting object within QMCPACK. For example, in the example, multiple Hamiltonian objects with different names can be defined. The one actually used in the calculation is the one passed to ``execute'' as ham.

\begin{lstlisting}[style=QMCPXML,caption=Sample input file for AFQMC.]
<?xml version="1.0"?>
<simulation method="afqmc">
  <project id="Carbon" series="0"/>

  <AFQMCInfo name="info0">
    <parameter name="NMO">32</parameter>
    <parameter name="NAEA">16</parameter>
    <parameter name="NAEB">16</parameter>
  </AFQMCInfo>

  <Hamiltonian name="ham0" info="info0">
    <parameter name="filename">../fcidump.h5</parameter>
  </Hamiltonian>

  <Wavefunction name="wfn0" type="MSD" info="info0">
    <parameter name="filetype">ascii</parameter>
    <parameter name="filename">wfn.dat</parameter>
  </Wavefunction>

  <WalkerSet name="wset0">
    <parameter name="walker_type">closed</parameter> 
  </WalkerSet>

  <Propagator name="prop0" info="info0">
  </Propagator>

  <execute wset="wset0" ham="ham0" wfn="wfn0" prop="prop0" info="info0">
    <parameter name="timestep">0.005</parameter>
    <parameter name="blocks">10000</parameter>
    <parameter name="nWalkers">20</parameter>
  </execute>

</simulation>
\end{lstlisting}

%The following table lists some of the most practical parameters in the \texttt{execute} block
%The following table lists some of the practical parameters
\begin{table}[h]
\begin{center}
\begin{tabularx}{\textwidth}{l l l l l X }
\hline
\multicolumn{6}{l}{\texttt{afqmc} method} \\
\hline
\multicolumn{6}{l}{parameters in \texttt{AFQMCInfo}} \\
   &   \bfseries name     & \bfseries datatype & \bfseries values & \bfseries default   & \bfseries description \\
   &   \texttt{NMO             } &  integer     & $\ge 0$ & no & number of molecular orbitals \\
   &   \texttt{NAEA            } &  integer     & $\ge 0$ & no & number of active electrons of spin up \\
   &   \texttt{NAEB            } &  integer     & $\ge 0$ & no & number of active electrons of spin down \\
\multicolumn{6}{l}{parameters in \texttt{Hamiltonian}}  \\
   &   \texttt{info            } &  argument   &               &      & name of \texttt{AFQMCInfo} block \\\\
   &   \texttt{filename        } &  string     &               & no   & name of file with the hamiltonian \\
   &   \texttt{filetype        } &  string     & hdf5          & yes  & native HDF5-based format of QMCPACK  \\ 
\multicolumn{6}{l}{parameters in \texttt{Wavefunction}}\\
   &   \texttt{info            } &  argument   &             &      & name of \texttt{AFQMCInfo} block \\
   &   \texttt{type            } &  argument & MSD      & no   & linear combination of (assumed non-orthogonal) Slater determinants \\
   &   \texttt{                } &           & PHMSD &    & CI-type multi-determinant wave function  \\
   &   \texttt{filetype        } &  string  & ascii       & no   & ASCII data file type \\
   &   \texttt{                } &          & hdf5        &      & HDF5 data file type \\
\multicolumn{6}{l}{parameters in \texttt{WalkerSet}} \\
   &   \texttt{walker$\_$type       } &  string    & collinear  & yes  & Request a collinear walker set. \\ 
   &   \texttt{       } &     & closed  & no  & Request a closed shell (doubly-occupied) walker set. \\ 
\multicolumn{6}{l}{parameters in \texttt{Propagator}} \\
   &   \texttt{type            } &  argument   & afqmc & afqmc & type of propagator \\
   &   \texttt{info            } &  argument   &       &       & name of \texttt{AFQMCInfo} block \\
   &   \texttt{hybrid   } &  string   & yes  & yes  & Use hybrid propagation algorithm. \\ 
   &   \texttt{   } &     & no  &  & Use local energy based propagation algorithm. \\ 
\multicolumn{6}{l}{parameters in \texttt{execute}} \\
   &   \texttt{wset            } &  argument    &         &      &  \\
   &   \texttt{ham             } & argument     &         &      &  \\
   &   \texttt{wfn             } & argument     &         &      &  \\
   &   \texttt{prop            } & argument     &         &      &  \\
   &   \texttt{info            } &  argument    &         &      & name of \texttt{AFQMCInfo} block \\
   &   \texttt{nWalkers        } &  integer     & $\ge 0$ & 5    & initial number of walkers per task group   \\
   &   \texttt{timestep        } &  real        & $> 0$   & 0.01 & time step in 1/a.u. \\
   &   \texttt{blocks          } &  integer     & $\ge 0$ & 100  & number of blocks            \\
   &   \texttt{step            } &  integer     & $> 0$   & 1    & number of steps within a block \\
   &   \texttt{substep         } &  integer     & $> 0$   & 1    & number of substeps within a step \\
   &   \texttt{ortho         } &  integer     & $> 0$   & 1    & number of steps between walker orthogonalization. \\ 
  \hline
\end{tabularx}
\end{center}
\caption{Input options for AFQMC in QMCPACK}
\label{table:afqmc_basic}
\end{table}

Below is a list of all input sections for AFQMC calculations, along with a detailed explanation of accepted parameters. Since the code is under active development, the list of parameters and their interpretation can change in the future.\\

\texttt{AFQMCInfo}: input block that defines basic information about the calculation. It is passed to all other input blocks to propagate the basic information:
\texttt{<AFQMCInfo name="info0">}
\begin{itemize}
\item \textbf{NMO}. Number of molecular orbitals, i.e., number of states in the single particle basis. 
\item \textbf{NAEA}. Number of Active Electrons-Alpha, i.e., number of spin-up electrons.
\item \textbf{NAEB}. Number of Active Electrons-Beta, i.e., number of spin-down electrons.
\end{itemize}

\texttt{Hamiltonian}: controls the object that reads, stores and manages the hamiltonian. 
  \texttt{<Hamiltonian name="ham0" type="SparseGeneral" info="info0">}
\begin{itemize}
\item \textbf{filename}. Name of file with the \texttt{Hamiltonian}. This is a required parameter.
\item \textbf{cutoff\_1bar}. Cutoff applied to integrals during reading. Any term in the hamiltonian smaller than this value is set to zero. (For filetype=``hdf5'', the cutoff is only applied to the 2-electron integrals). Default: 1e-8
\item \textbf{cutoff\_decomposition}. Cutoff used to stop the iterative cycle in the generation of the Cholesky decomposition of the 2-electron integrals. The generation of Cholesky vectors is stopped when the maximum error in the diagonal reaches this value. In case of an eigenvalue factorization, this becomes the cutoff applied to the eigenvalues. Only eigenvalues above this value are kept. Default: 1e-6
\item \textbf{nblocks}. This parameter controls the distribution of the 2-electron integrals among processors. In the default behavior (nblocks=1), all nodes contain the entire list of integrals. If nblocks $>$ 1, the of nodes in the calculation will be split in nblocks groups. Each node in a given group contains the same subset of integrals and subsequently operates on this subset during  any further operation that requires the hamiltonian. The maximum number of groups is NMO. Currently only works for filetype=``hdf5'' and the file must contain integrals.  Not yet implemented for input hamiltonians in the form of Cholesky vectors or for ASCII input. Coming soon!
    Default: No distribution
\item \textbf{printEig}. If ``yes'', prints additional information during the Cholesky decomposition.
    Default: no
\item \textbf{fix\_2eint}.  If this is set to ``yes'', orbital pairs that are found not to be positive definite are ignored in the generation of the Cholesky factorization. This is necessary if the 2-electron integrals are not positive definite due to round-off errors in their generation.
    Default: no \\
\end{itemize}

\texttt{Wavefunction}: controls the object that manages the trial wave-functions. This block expects a list of xml-blocks defining actual trial-wave functions for various roles. 
\texttt{<Wavefunction name="wfn0" type="MSD/PHMSD" info="info0">}
\begin{itemize}
\item \textbf{filename}. Name of file with wave-function information.
\item \textbf{cutoff}. cutoff applied to the terms in the calculation of the local energy. Only terms in the hamiltonian above this cutoff are included in the evaluation of the energy.
      Default: 1e-6
\item \textbf{nnodes}. Defines the parallelization of the local energy evaluation and the distribution of the \texttt{Hamiltonian} matrix (not to be confused with the list of 2-electron integrals managed by \texttt{Hamiltonian}. These are not the same.) If nnodes $>$ 1, the nodes in the simulation are split into groups of nnodes, each group works collectively in the evaluation of the local energy of their walkers. This helps distribute the effort involved in the evaluation of the local energy among the nodes in the group, but also distributes the memory associated with the wave-function among the nodes in the group.
      Default: No distribution
\item \textbf{ndet}. Number of determinants to read from file.
      Default: Read all determinants. 
\item \textbf{cutoff}. For sparse hamiltoniants, this defines the cutoff applied to the half-rotated 2-electron integrals. 
      Default: 0.0
\item \textbf{nbatch}. This turns on(>=1)/off(==0) batched calculation of density matrices and overlaps. -1 means all the walkers in the batch. 
      Default: 0 (CPU) / -1 (GPU) 
\item \textbf{nbatch\_qr}. This turns on(>=1)/off(==0) batched QR calculation. -1 means all the walkers in the batch.
      Default: 0 (CPU) / -1 (GPU) 
\end{itemize}

\texttt{WalkerSet}: Controls the object that handles the set of walkers.
\texttt{<WalkerSet name="wset0">}
\begin{itemize}
\item \textbf{walker\_type}. Type of walker set: closed or collinear. 
      Default: collinear
\item \textbf{pop\_control}. Population control algorithm. Options: ``simple'': Uses a simple branching scheme with a fluctuating population. Walkers with weight above max\_weight are split into multiple walkers of weight reset\_weight. Walkers with weight below min\_weight are killed with probability (weight/min\_weight); ``pair'': Fixed-population branching algorithm, based on QWalk's branching algorithm. Pairs of walkers with weight above/below max\_weight/min\_weight are combined into 2 walkers with weights equal to $(w_1+w_2)/2$. The probability of replicating walker w1 (larger weight) occurs with probability $w_1/(w_1+w_2)$, otherwise walker w2 (lower weight) is replicated; ``comb'': Fixed-population branching algorithm based on the Comb method. Will be available in the next release. Default: ``pair''
\item \textbf{min\_weight}. Weight at which walkers are possibly killed (with probability weight/min\_weight). Default: 0.05
\item \textbf{max\_weight}. Weight at which walkers are replicated. Default: 4.0
\item \textbf{reset\_weight}. Weight to which replicated walkers are reset to. Default: 1.0
\end{itemize}

\texttt{Propagator}: Controls the object that manages the propagators.
\texttt{<Propagator name="prop0" info="info0">}
\begin{itemize}
\item \textbf{cutoff}. Cutoff applied to Cholesky vectors. Elements of the Cholesky vectors below this value are set to zero. Only meaningful with sparse hamiltonians.
    Default: 1e-6
\item \textbf{substractMF}. If ``yes'', apply mean-field subtraction based on the ImpSamp trial wave-function. Must set to ``no'' to turn it off.
    Default: yes
\item \textbf{vbias\_bound}. Upper bound applied to the vias potential. Components of the vias potential above this value are truncated there. The bound is currently applied to $\sqrt{\tau} v_{bias}$, so a larger value must be used as either the time-step or the fluctuations increase (e.g. from running a larger system or using a poor trial wave-function).
    Default: 3.0
\item \textbf{apply\_constrain}. If ``yes'', apply the phaseless constrain to the walker propagation. Currently, setting this to ``no'' produces unknown behavior, since free propagation algorithm has not been tested.
    Default: yes
\item \textbf{hybrid}. If ``yes'', use hybrid propagation algorithm. This propagation scheme doesn't use the local energy during propagation, leading to significant speed ups when its evaluation  cost is high. The local energy of the ImpSamp trial wave-function is never evaluated. To obtain energy estimates in this case, you must define an Estimator xml-block with the \texttt{Wavefunction} block. The local energy of this trial wave-function is evaluated and printed. It is possible to use a previously defined trial wave-function in the Estimator block, just set its ``name'' argument to the name of a previously defined wave-function. In this case, the same object is used for both roles.
    Default: no
\item \textbf{nnodes}. Controls the parallel propagation algorithm. If nnodes $>$ 1, the nodes in the simulation are split into groups of nnodes nodes, each group working collectively to propagate their walkers.
    Default: 1 (Serial algorithm)
\item \textbf{nbatch}. This turns on(>=1)/off(==0) batched calculation of density matrices and overlaps. -1 means all the walkers in the batch.
      Default: 0 (CPU) / -1 (GPU)
\item \textbf{nbatch$\_$qr}. This turns on(>=1)/off(==0) batched QR calculation. -1 means all the walkers in the batch.
      Default: 0 (CPU) / -1 (GPU) 
\end{itemize}

\texttt{execute}: Defines an execution region. 
\texttt{<execute wset="wset0" ham="ham0" wfn="wfn0" prop="prop0" info="info0">}
\begin{itemize}
\item \textbf{nWalkers}. initial number of walkers per core group (see ncores). This sets the number of walkers for a given gorup of "ncores" on a node, the total number of walkers in the simulation depends on the total number of nodes and on the total number of cores on a node in the following way: $ \#_walkers_total = nWalkers * \#_nodes * \#_cores_total / ncores $. \\ 
    Default: 5
\item \textbf{timestep}. time step in 1/a.u. \\
    Default: 0.01
\item \textbf{blocks}. number of blocks. Slow operations occur once per block, e.g. write to file, slow observables, checkpoints, etc. \\
    Default: 100
\item \textbf{step}. number of steps within a block. Operations that occur at the step level include: load balance, orthogonalization, branching, etc. \\
    Default: 1
\item \textbf{substep}. number of substeps within a step. Only walker propagation occurs in a substep. \\
    Default: 1
\item \textbf{ortho}. number of steps between orthogonalization.
    Default: 1
\item \textbf{ncores}. Number of nodes in a task group. This number defines the number of cores on a node that share the parallel work associated with a distributed task. This number is used in the \texttt{Wavefunction} and \texttt{Propagator} task groups. The walker sets are shares by the ncores on a given node in the task group.
\item \textbf{checkpoint}. Number of blocks between checkpoint files are generated. If a value smaller than 1 is given, no file is generated. If \textbf{hdf\_write\_file} is not set, a default name is used. \textbf{Default: 0} 
%\item \textbf{samplePeriod}. Number of blocks between sample collection. \textbf{Default: 0}
\item \textbf{hdf\_write\_file}. If set (and checkpoint>0), a checkpoint file with this name will be written.
\item \textbf{hdf\_read\_file}. If set, the simulation will be restarted from the given file.\\
\end{itemize}

Within the \texttt{Estimators} xml block has an argument \textbf{name}: the type of estimator we want to measure. Currently available estimators include: ``basic'', ``energy'', ``mixed\_one\_rdm'', and ``back\_propagation''.  

The basic estimator has the following optional parameters:
\begin{itemize}
\item \textbf{timers}. print timing information. Default: true
\end{itemize}

The back\_propagation estimator has the following parameters:
\begin{itemize}
\item \textbf{ortho}. Number of back-propagation steps between orthogonalization. 
    Default: 10
\item \textbf{nsteps}. Maximum number of back-propagation steps. 
    Default: 10
\item \textbf{naverages}. Number of back propagation calculations to perform. The number of steps will be chosed equally distributed in the range {0,nsteps}. 
    Default: 1
\item \textbf{block\_size}. Number of blocks to use in the internal average of the back propagated estimator. This is used to block data and reduce the size of the output. 
    Default: 1 
\item \textbf{nskip}. Number of blocks to skip at the start of the calculation for equilibration purposes. 
    Default: 0
\end{itemize}

\section{File formats}
... Coming Soon ...

\section{Advice/Useful Information}

AFQMC calculations are computationally expensive and require some care in order to obtain reasonable performance.
Below is a growing list of useful advice for new users followed by a sample input for a large calculation.
\begin{itemize}
\item Generate Cholesky-decomposed integrals with external codes instead of the 2-electron integrals directly. The generation of the Cholesky factorization is faster and consumes less memory. 
\item Use the hybrid algorithm for walker propagation. Set steps/substeps to adequate values to reduce the number of energy evaluations. This is essential when using large multi-determinant expansions.
\item Adjust cutoffs in the wave-function and propagator bloxks until desired accuracy is reached. The cost of the calculation will depend on these cutoffs.
\item Adjust ncores/nWalkers to obtain better efficiency. Larger nWalkers will lead to more efficient linear algebra operations, but will increase the time per step. Larger ncores will reduce the time per step, but will reduce efficiency due to inefficiencies in the parallel implementation. For large calculations, values between 6-12 for both quantities should be reasonable, depending on architecture. 
\end{itemize}

\begin{lstlisting}[style=QMCPXML,caption=Example of sections of an AFQMC input file for a large calculation.]
...

  <Hamiltonian name="ham0" type="SparseGeneral" info="info0">
    <parameter name="filename">fcidump.h5</parameter>
    <parameter name="cutoff_1bar">1e-6</parameter>
    <parameter name="cutoff_decomposition">1e-5</parameter>
  </Hamiltonian>

  <Wavefunction name="wfn0" type="MSD" info="info0">
    <parameter name="filetype">ascii</parameter>
    <parameter name="filename">wfn.dat</parameter>
  </Wavefunction>

  <WalkerSet name="wset0">
    <parameter name="walker_type">closed</parameter>
  </WalkerSet>

  <Propagator name="prop0" info="info0">
    <parameter name="hybrid">yes</parameter>
  </Propagator>

  <execute wset="wset0" ham="ham0" wfn="wfn0" prop="prop0" info="info0">
    <parameter name="ncores">8</parameter>
    <parameter name="timestep">0.01</parameter>
    <parameter name="blocks">10000</parameter>
    <parameter name="steps">10</parameter>
    <parameter name="substeps">5</parameter>
    <parameter name="nWalkers">8</parameter>
    <parameter name="ortho">5</parameter>
  </execute>
\end{lstlisting}

\input{integrals_for_afqmc}



\chapter{Examples}
\label{chap:examples}

\textbf{WARNING: THESE EXAMPLES ARE NOT CONVERGED! YOU MUST CONVERGE PARAMETERS (SIMULATION CELL SIZE, JASTROW PARAMETER NUMBER/CUTOFF, TWIST NUMBER, DMC TIME STEP, DFT PLANE WAVE CUTOFF, DFT K-POINT MESH, ETC.) FOR REAL CALCUATIONS!}

The following examples should run in serial on a modern workstation in a few hours.

\section{Using QMCPACK directly}

In \ishell{examples/molecules}, there are the following examples.
Each directory also contains a \ishell{README} file with more details.

\begin{tabular}{l l}
Directory  & Description \\
\ishell{H2O} &  H2O molecule from GAMESS orbitals \\
\ishell{He} &  Helium atom with simple wavefunctions\\
\end{tabular}




\section{Using Nexus}

For more information about Nexus, see the User Guide in \ishell{nexus/documentation}.

For Python to find the Nexus library, the PYTHONPATH environment variable should be set to \ishell{<QMCPACK source>/nexus/library}.
For these examples to work properly, the executables for Quantum ESPRESSO and QMCPACK either
need to be on the path, or the paths in the script should be adjusted.

These examples can be found under the \ishell{nexus/examples/qmcpack} directory.
%\begin{itemize}
%\item \ishell{diamond} Bulk diamond with VMC
%\item \ishell{graphene} Graphene sheet with DMC
%\item \ishell{c20} C20 cage molecule
%\item \ishell{oxygen\_dimer} Binding curve for O$_2$ molecule
%\item \ishell{H2O} H$_2$O molecule with Quantum ESPRESSO orbitals
%\item \ishell{LiH} LiH crystal with Quantum ESPRESSO orbitals
%\end{itemize}

\begin{tabular}{l l}
Directory  & Description \\
\ishell{diamond} &  Bulk diamond with VMC \\
\ishell{graphene} & Graphene sheet with DMC \\
\ishell{c20} & C20 cage molecule \\
\ishell{oxygen\_dimer} & Binding curve for O$_2$ molecule \\
\ishell{H2O} & H$_2$O molecule with Quantum ESPRESSO orbitals \\
\ishell{LiH} & LiH crystal with Quantum ESPRESSO orbitals \\
\end{tabular}





%\subsection{Bulk diamond}
%The input files are in the directory \ishell{nexus/examples/qmcpack/diamond}.

%\subsection{Graphene Sheet}
%The input files are in the directory \ishell{nexus/examples/qmcpack/graphene}.

%\subsection{C20 cage molecule}
%The input files are in the directory \ishell{nexus/examples/qmcpack/c20}.

%\subsection{Binding curve for O$_2$ molecule}
%The input files are in the directory \ishell{nexus/examples/qmcpack/oxygen\_dimer}.

%\subsection{H$_2$O molecule with Quantum ESPRESSO orbitals}
%
%The input files are in the directory \ishell{nexus/examples/qmcpack/H2O}.

%The Nexus script is in \ishell{H2O.py} and \ishell{H2O.xyz} contains the atomic positions.
%To run the example, the BFD pseudopotentials are needed.  Create a \ishell{pseudopotentials} directory, and copy \ishell{O.BFD.upf}, \ishell{O.BFD.xml}, \ishell{H.BFD.upf}, and \ishell{H.BFD.xml} from \ishell{pseudopotentials/BFD} in the QMCPACK distribution.
%
%The \ishell{H2O.py} script generates the orbitals using Quantum ESPRESSO, runs QMCPACK to optimize the Jastrow and then performs DMC for H$_2$O in a box.
%
%
%
%\subsection{LiH crystal with Quantum ESPRESSO orbitals}
%The input files are in the directory \ishell{nexus/examples/qmcpack/LiH}.
%
%The \ishell{LiH.py} Nexus script uses CASINO-formatted Trail-Needs pseudopotentials (see Section~\ref{subsec:CASINO}) for Li and H in a subdirectory named \ishell{pseudopotentials} (both UPF and CASINO  formats, named \ishell{Li.TN-DF.upf}, \ishell{Li.pp.data}, \ishell{H.TN-DF.upf}, and \ishell{H.pp.data}, respectively), generates the orbitals using Quantum ESPRESSO, then runs QMCPACK to optimize the Jastrow and run DMC for LiH with periodic boundary conditions.


% labs: import each as a separate chapter for now

\chapter{Lab 1: MC Statistical Analysis}
\label{chap:lab_qmc_statistics}


\section{Topics covered in this lab} 

This lab focuses on the basics of analyzing data from MC
calculations.  In this lab, participants will use data from
VMC calculations of a simple 1-electron system with an analytically soluble
system (the ground state of the hydrogen atom) to understand how to interpret an
MC situation.  Most of these analyses will also carry over to DMC simulations.  Topics covered include:
\begin{itemize}
  \item{Averaging MC variables}
  \item{The statisical error bar of mean values}
  \item{The effects of autocorrelation and variance on the error bar}
  \item{The relationship between MC time step and autocorrelation}
  \item{The use of blocking to reduce autocorrelation}
  \item{The significance of the acceptance ratio}
  \item{The significance of the sample size}
  \item{How to determine whether an MC run was successful}
  \item{The relationship between wavefunction quality and variance}
  \item{Gauging the efficiency of MC runs}
  \item{The cost of scaling up to larger system sizes}
\end{itemize}


\hide{
\subsection{How to get the most out of this lab}
Be sure to practice using the various flags in the qmca tool to analyze the
data.  Although some features are not yet implemented, this will get you used
to seeing how the values in the data files produce the averages, which are the
ultimate result of the MC simulations.
}

\section{Lab directories and files}

\footnotesize
\begin{verbatim}
labs/lab1_qmc_statistics/
│
├── atom                              - H atom VMC calculation
│   ├── H.s000.scalar.dat                - H atom VMC data 
│   └── H.xml                            - H atom VMC input file
│
├── autocorrelation                   - varying autocorrelation
│   ├── H.dat                            - data for gnuplot
│   ├── H.plt                            - gnuplot for time step vs. E_L, tau_c
│   ├── H.s000.scalar.dat                - H atom VMC data: time step = 10 
│   ├── H.s001.scalar.dat                - H atom VMC data: time step =  5 
│   ├── H.s002.scalar.dat                - H atom VMC data: time step =  2 
│   ├── H.s003.scalar.dat                - H atom VMC data: time step =  1 
│   ├── H.s004.scalar.dat                - H atom VMC data: time step =  0.5
│   ├── H.s005.scalar.dat                - H atom VMC data: time step =  0.2
│   ├── H.s006.scalar.dat                - H atom VMC data: time step =  0.1
│   ├── H.s007.scalar.dat                - H atom VMC data: time step =  0.05 
│   ├── H.s008.scalar.dat                - H atom VMC data: time step =  0.02
│   ├── H.s009.scalar.dat                - H atom VMC data: time step =  0.01
│   ├── H.s010.scalar.dat                - H atom VMC data: time step =  0.005
│   ├── H.s011.scalar.dat                - H atom VMC data: time step =  0.002
│   ├── H.s012.scalar.dat                - H atom VMC data: time step =  0.001
│   ├── H.s013.scalar.dat                - H atom VMC data: time step =  0.0005
│   ├── H.s014.scalar.dat                - H atom VMC data: time step =  0.0002
│   ├── H.s015.scalar.dat                - H atom VMC data: time step =  0.0001
│   └── H.xml                            - H atom VMC input file
│
├── average                            - Python scripts for average/std. dev.
│   ├── average.py                         - average five E_L from H atom VMC
│   ├── stddev2.py                         - standard deviation using (E_L)^2
│   └── stddev.py                          - standard deviation around the mean
│
├── basis                              - varying basis set for orbitals
│   ├── H__exact.s000.scalar.dat           - H atom VMC data using STO basis
│   ├── H_STO-2G.s000.scalar.dat           - H atom VMC data using STO-2G basis
│   ├── H_STO-3G.s000.scalar.dat           - H atom VMC data using STO-3G basis
│   └── H_STO-6G.s000.scalar.dat           - H atom VMC data using STO-6G basis
│
├── blocking                           - varying block/step ratio
│   ├── H.dat                              - data for gnuplot
│   ├── H.plt                              - gnuplot for N_block vs. E, tau_c
│   ├── H.s000.scalar.dat                  - H atom VMC data 50000:1 blocks:steps
│   ├── H.s001.scalar.dat                  - "  "    "    "  25000:2 blocks:steps
│   ├── H.s002.scalar.dat                  - "  "    "    "  12500:4 blocks:steps
│   ├── H.s003.scalar.dat                  - "  "    "    "  6250: 8 blocks:steps
│   ├── H.s004.scalar.dat                  - "  "    "    "  3125:16 blocks:steps
│   ├── H.s005.scalar.dat                  - "  "    "    "  2500:20 blocks:steps
│   ├── H.s006.scalar.dat                  - "  "    "    "  1250:40 blocks:steps
│   ├── H.s007.scalar.dat                  - "  "    "    "  1000:50 blocks:steps
│   ├── H.s008.scalar.dat                  - "  "    "    "  500:100 blocks:steps
│   ├── H.s009.scalar.dat                  - "  "    "    "  250:200 blocks:steps
│   ├── H.s010.scalar.dat                  - "  "    "    "  125:400 blocks:steps
│   ├── H.s011.scalar.dat                  - "  "    "    "  100:500 blocks:steps
│   ├── H.s012.scalar.dat                  - "  "    "    "  50:1000 blocks:steps
│   ├── H.s013.scalar.dat                  - "  "    "    "  40:1250 blocks:steps
│   ├── H.s014.scalar.dat                  - "  "    "    "  20:2500 blocks:steps
│   ├── H.s015.scalar.dat                  - "  "    "    "  10:5000 blocks:steps
│   └── H.xml                             - H atom VMC input file
│
├── blocks                             -  varying total number of blocks
│   ├── H.dat                             - data for gnuplot
│   ├── H.plt                             - gnuplot for N_block vs. E
│   ├── H.s000.scalar.dat                 - H atom VMC data    500 blocks
│   ├── H.s001.scalar.dat                 - "  "    "    "    2000 blocks
│   ├── H.s002.scalar.dat                 - "  "    "    "    8000 blocks
│   ├── H.s003.scalar.dat                 - "  "    "    "   32000 blocks
│   ├── H.s004.scalar.dat                 - "  "    "    "  128000 blocks
│   └── H.xml                             - H atom VMC input file 
│
├── dimer                          - comparing no and simple Jastrow factor
│   ├── H2_STO___no_jastrow.s000.scalar.dat - H dimer VMC data without Jastrow
│   └── H2_STO_with_jastrow.s000.scalar.dat - H dimer VMC data with Jastrow
│
├──  docs                               - documentation
│   ├──  Lab_1_MC_Analysis.pdf             - this document
│   └──  Lab_1_Slides.pdf                  - slides presented in the lab
│
├── nodes                              - varying number of computing nodes
│   ├──  H.dat                             - data for gnuplot
│   ├──  H.plt                             - gnuplot for N_node vs. E
│   ├──  H.s000.scalar.dat                 - H atom VMC data with  32 nodes
│   ├──  H.s001.scalar.dat                 - H atom VMC data with 128 nodes
│   └──  H.s002.scalar.dat                 - H atom VMC data with 512 nodes
│
├── problematic                        - problematic VMC run
│   └──  H.s000.scalar.dat                 - H atom VMC data with a problem
│
└── size                                - scaling with number of particles
    ├──  01________H.s000.scalar.dat       - H atom VMC data
    ├──  02_______H2.s000.scalar.dat       - H dimer "   "
    ├──  06________C.s000.scalar.dat       - C atom  "   "
    ├──  10______CH4.s000.scalar.dat       - methane "   "
    ├──  12_______C2.s000.scalar.dat       - C dimer "   "
    ├──  16_____C2H4.s000.scalar.dat       - ethene  " 
    ├──  18___CH4CH4.s000.scalar.dat       - methane dimer VMC data
    ├──  32_C2H4C2H4.s000.scalar.dat       - ethene dimer   "   "
    ├──  nelectron_tcpu.dat                - data for gnuplot
    └──  Nelectron_tCPU.plt                - gnuplot for N_elec vs. t_CPU
\end{verbatim}
\normalsize

\section{Atomic units} 

QMCPACK operates in Ha atomic units to reduce the
number of factors in the Schr\"odinger equation.  Thus, the unit of length is
the bohr (5.291772 $\times 10^{-11}$ m = 0.529177 \AA); the unit of energy is
the Ha (4.359744 $\times 10^{-18}$ J = 27.211385 eV).  The energy of the
ground state of the hydrogen atom in these units is -0.5 Ha.


%\section{Monte Carlo data analysis:\newline average, error bars, variance}

\section{Reviewing statistics}
\label{sec:review}

We will practice taking the average (mean) and standard deviation of some MC data by hand to review the basic definitions.

Enter Python's command line by typing \texttt{python [Enter]}.
You will see a prompt ``\textgreater\textgreater\textgreater.''

The mean of a dataset is given by:
\begin{align}
  \overline{x} = \frac{1}{N}\sum_{i=1}^{N} x_i\:.
\end{align}

To calculate the average of five local energies from an MC calculation of the
ground state of an electron in the hydrogen atom, input (truncate at the
thousandths place if you cannot copy and paste; script versions are also
available in the \texttt{average} directory): 

\begin{lstlisting}[style=SHELL]
(
(-0.45298911858) + 
(-0.45481953564) + 
(-0.48066105923) + 
(-0.47316713469) + 
(-0.46204733302)
)/5.
\end{lstlisting} 

Then, press \texttt{[Enter]} to get:

\begin{shade}
>>> ((-0.45298911858) + (-0.45481953564) + (-0.48066105923) + 
(-0.47316713469) + (-0.4620473302))/5.  
-0.46473683566800006
\end{shade}

To understand the significance of the mean, we also need the standard deviation
around the mean of the data (also called the error bar), given by:

\begin{align}
  \sigma = \sqrt{\frac{1}{N(N-1)}\sum_{i=1}^{N} ({x_i} - \overline{x})^2}\:.
\end{align}

To calculate the standard deviation around the mean (-0.464736835668) of these
five data points, put in: 

\begin{lstlisting}[style=SHELL]
( (1./(5.*(5.-1.))) * ( 
(-0.45298911858-(-0.464736835668))**2 + \\
(-0.45481953564-(-0.464736835668))**2 + 
(-0.48066105923-(-0.464736835668))**2 + 
(-0.47316713469-(-0.464736835668))**2 + 
(-0.46204733302-(-0.464736835668))**2 ) 
)**0.5
\end{lstlisting} 

Then, press \texttt{[Enter]} to get:

\begin{shade}
>>> ( (1./(5.*(5.-1.))) * ( (-0.45298911858-(-0.464736835668))**2 +
(-0.45481953564-(-0.464736835668))**2 + (-0.48066105923-(-0.464736835668))**2 + 
(-0.47316713469-(-0.464736835668))**2 + (-0.46204733302-(-0.464736835668))**2 
) )**0.5
0.0053303187464332066
\end{shade}

Thus, we might report this data as having a value -0.465 +/- 0.005 Ha.
This calculation of the standard deviation assumes that the average for this
data is fixed, but we can continually add MC samples to the data, so it
is better to use an estimate of the error bar that does not rely on the overall
average.  Such an estimate is given by:

\begin{align}
  \tilde{\sigma} = \sqrt{\frac{1}{N-1}\sum_{i=1}^{N} \left[{(x^2)}_i - ({x_i})^2\right]}\:.
\end{align}

To calculate the standard deviation with this formula, input the following,
which includes the square of the local energy calculated with each
corresponding local energy:

\begin{lstlisting}[style=SHELL]
( (1./(5.-1.)) * ( 
(0.60984565298-(-0.45298911858)**2) + \\
(0.61641291630-(-0.45481953564)**2) + 
(1.35860151160-(-0.48066105923)**2) + \\
(0.78720769003-(-0.47316713469)**2) + 
(0.56393677687-(-0.46204733302)**2) ) 
)**0.5
\end{lstlisting}

and press \texttt{[Enter]} to get:

\begin{shade}
>>> ((1./(5.-1.))*((0.60984565298-(-0.45298911858)**2)+ 
(0.61641291630-(-0.45481953564)**2)+(1.35860151160-(-0.48066105923)**2)+ 
(0.78720769003-(-0.47316713469)**2)+(0.56393677687-(-0.46204733302)**2))
)**0.5
0.84491636672906634
\end{shade}

This much larger standard deviation, acknowledging that the mean of this small
data set is not the average in the limit of infinite sampling, more accurately
reports the value of the local energy as -0.5 +/- 0.8 Ha.

Type \texttt{quit()} and press \texttt{[Enter]} to exit the Python command line.

\section{Inspecting MC data}
\label{sec:inspect_data} 

QMCPACK outputs data from MC calculations into files ending in \texttt{scalar.dat}.
Several quantities are calculated and written for each block of MC
steps in successive columns to the right of the step index. 

Change directories to \texttt{atom}, and open the file ending in
\texttt{scalar.dat} with a text editor (e.g., \textbf{vi *.scalar.dat} or \textbf{emacs
*.scalar.dat}.  If possible, adjust the terminal so that lines do not wrap.
The data will begin as follows (broken into three groups to fit on this page):

\begin{shade}
#   index    LocalEnergy         LocalEnergy_sq      LocalPotential     ...
         0   -4.5298911858e-01    6.0984565298e-01   -1.1708693521e+00    
         1   -4.5481953564e-01    6.1641291630e-01   -1.1863425644e+00    
         2   -4.8066105923e-01    1.3586015116e+00   -1.1766446209e+00    
         3   -4.7316713469e-01    7.8720769003e-01   -1.1799481122e+00    
         4   -4.6204733302e-01    5.6393677687e-01   -1.1619244081e+00    
         5   -4.4313854290e-01    6.0831516179e-01   -1.2064503041e+00    
         6   -4.5064926960e-01    5.9891422196e-01   -1.1521370176e+00    
         7   -4.5687452611e-01    5.8139614676e-01   -1.1423627617e+00    
         8   -4.5018503739e-01    8.4147849706e-01   -1.1842075439e+00    
         9   -4.3862013841e-01    5.5477715836e-01   -1.2080979177e+00    
\end{shade}

The first line begins with a \#, indicating that this line does not contain MC
data but rather the labels of the columns.  After a blank line, the remaining
lines consist of the MC data.  The first column, labeled index, is an integer
indicating which block of MC data is on that line.  The second column contains
the quantity usually of greatest interest from the simulation: the local
energy.  Since this simulation did not use the exact ground state wavefunction, it does not produce -0.5 Ha as the local energy although the
value lies within about 10\%.  The value of the local energy fluctuates from
block to block, and the closer the trial wavefunction is to the ground state
the smaller these fluctuations will be.  The next column contains an important
ingredient in estimating the error in the MC average---the square of the local
energy---found by evaluating the square of the Hamiltonian.  

\begin{shade} 
...   Kinetic             Coulomb             BlockWeight        ... 
       7.1788023352e-01   -1.1708693521e+00    1.2800000000e+04   
       7.3152302871e-01   -1.1863425644e+00    1.2800000000e+04   
       6.9598356165e-01   -1.1766446209e+00    1.2800000000e+04   
       7.0678097751e-01   -1.1799481122e+00    1.2800000000e+04   
       6.9987707508e-01   -1.1619244081e+00    1.2800000000e+04   
       7.6331176120e-01   -1.2064503041e+00    1.2800000000e+04   
       7.0148774798e-01   -1.1521370176e+00    1.2800000000e+04   
       6.8548823555e-01   -1.1423627617e+00    1.2800000000e+04   
       7.3402250655e-01   -1.1842075439e+00    1.2800000000e+04   
       7.6947777925e-01   -1.2080979177e+00    1.2800000000e+04   
\end{shade}

The fourth column from the left consists of the values of the local potential
energy.  In this simulation, it is identical to the Coulomb potential
(contained in the sixth column) because the one electron in the simulation has
only the potential energy coming from its interaction with the nucleus.  In
many-electron simulations, the local potential energy contains contributions
from the electron-electron Coulomb interactions and the nuclear potential or
pseudopotential.  The fifth column contains the local kinetic energy value for
each MC block, obtained from the Laplacian of the wavefunction.  The sixth
column shows the local Coulomb interaction energy.  The seventh column displays
the weight each line of data has in the average (the weights are identical in
this simulation).   

\begin{shade} 
...    BlockCPU            AcceptRatio         
       6.0178991748e-03    9.8515625000e-01
       5.8323097461e-03    9.8562500000e-01
       5.8213412744e-03    9.8531250000e-01
       5.8330412549e-03    9.8828125000e-01
       5.8108362256e-03    9.8625000000e-01
       5.8254170264e-03    9.8625000000e-01
       5.8314813086e-03    9.8679687500e-01
       5.8258469971e-03    9.8726562500e-01
       5.8158433545e-03    9.8468750000e-01
       5.7959401123e-03    9.8539062500e-01
\end{shade}

The eighth column shows the CPU time (in seconds) to calculate the data in that
line.  The ninth column from the left contains the acceptance ratio (1 being
full acceptance) for MC steps in that line's data.  Other than the
block weight, all quantities vary from line to line.

Exit the text editor (\textbf{[Esc] :q! [Enter]} in vi, \textbf{[Ctrl]-x [Ctrl]-c} in
emacs).

\section{Averaging quantities in the MC data}
\label{sec:averaging} 

QMCPACK includes the qmca Python tool to average quantities in the \texttt{scalar.dat} file (and
also the \texttt{dmc.dat} file of DMC simulations).  Without any flags, qmca will output
the average of each column with a quantity in the \texttt{scalar.dat} file as follows. 

Execute qmca by \texttt{qmca *.scalar.dat}, which for this data outputs:

\begin{shade}

H  series 0 
LocalEnergy           =          -0.45446 +/-          0.00057
Variance              =             0.529 +/-            0.018 
Kinetic               =            0.7366 +/-           0.0020
LocalPotential        =           -1.1910 +/-           0.0016
Coulomb               =           -1.1910 +/-           0.0016 
LocalEnergy_sq        =             0.736 +/-            0.018
BlockWeight           =    12800.00000000 +/-       0.00000000
BlockCPU              =        0.00582002 +/-       0.00000067 
AcceptRatio           =          0.985508 +/-         0.000048
Efficiency            =        0.00000000 +/-       0.00000000 
\end{shade}

After one blank, qmca prints the title of the subsequent data, gleaned from the
data file name.  In this case, \texttt{H.s000.scalar.dat} became ``H  series 0.''
Everything before the first ``\texttt{.s}'' will be interpreted as the title, and the
number between ``\texttt{.s}'' and the next ``.'' will be interpreted as the series
number. 

The first column under the title is the name of each quantity qmca averaged.
The column to the right of the equal signs contains the average for the
quantity of that line, and the column to the right of the plus-slash-minus is
the statistical error bar on the quantity.  All quantities calculated from MC
simulations have and must be reported with a statistical error bar!

Two new quantities not present in the \texttt{scalar.dat} file are computed by qmca from
the data---variance and efficiency.  We will look at these later in this lab. 

To view only one value, \textbf{qmca} takes the \textbf{-q (quantity)} flag.
For example, the output of \texttt{qmca -q LocalEnergy *.scalar.dat} in this
directory produces a single line of output:

\begin{shade} 
H  series 0  LocalEnergy = -0.454460 +/- 0.000568 
\end{shade}

Type \texttt{qmca --help} to see the list of all quantities and their
abbreviations.

\section{Evaluating MC simulation quality}

There are several aspects of a MC simulation to consider in deciding how well
it went.  Besides the deviation of the average from an expected value (if there
is one), the stability of the simulation in its sampling, the autocorrelation
between MC steps, the value of the acceptance ratio (accepted steps over total
proposed steps), and the variance in the local energy all indicate the quality
of an MC simulation.  We will look at these one by one.

\subsection{Tracing MC quantities}

Visualizing the evolution of MC quantities over the course of the simulation by
a \textit{trace} offers a quick picture of whether the random walk had the expected
behavior.  qmca plots traces with the -t flag.

Type \texttt{qmca -q e -t H.s000.scalar.dat}, which produces a graph of the
trace of the local energy:

\FloatBarrier
\begin{figure}[ht!]
\begin{center}
\includegraphics[trim = 0mm 0mm 0mm 0mm, clip,width=0.75\columnwidth]{./figures/lab_qmc_statistics_tracing1.png}
\end{center}
\end{figure}
\FloatBarrier

%\includegraphics[scale=0.5]{E_L_H_STO-2G.png}

The solid black line connects the values of the local energy at each MC block
(labeled ``samples'').  The average value is marked with a horizontal, solid
red line.  One standard deviation above and below the average are marked with
horizontal, dashed red lines.  

The trace of this run is largely centered on the average with no
large-scale oscillations or major shifts, indicating a good-quality MC run. 

Try tracing the kinetic and potential energies, seeing that their behavior is
comparable with the total local energy.

Change to directory \texttt{problematic} and type \texttt{qmca -q e -t
H.s000.scalar.dat} to produce this graph:

\FloatBarrier
\begin{figure}[ht!]
\begin{center}
\includegraphics[trim = 0mm 0mm 0mm 0mm, clip,width=0.75\columnwidth]{./figures/lab_qmc_statistics_tracing2.png}
\end{center}
\end{figure}
\FloatBarrier

%\includegraphics[scale=0.5]{E_L_H_B-splines.png}

Here, the local energy samples cluster around the expected -0.5 Ha for the
first 150 samples or so and then begin to oscillate more wildly and increase
erratically toward 0, indicating a poor-quality MC run.

Again, trace the kinetic and potential energies in this run and see how their
behavior compares with the total local energy.

\subsection{Blocking away autocorrelation}

\textit{Autocorrelation} occurs when a given MC step biases subsequent MC
steps, leading to samples that are not statistically independent.  We must take
this autocorrelation into account to obtain accurate statistics.  qmca
outputs autocorrelation when given the {-}{-}sac flag.

Change to directory \texttt{autocorrelation} and type \texttt{qmca -q e
{-}{-}sac H.s000.scalar.dat}.  

\begin{shade} 
H  series 0  LocalEnergy = -0.454982 +/- 0.000430    1.0 
\end{shade}

The value after the error bar on the quantity is the autocorrelation (1.0 in
this case).

Proposing too small a step in configuration space, the MC \textit{time step},
can lead to autocorrelation since the new samples will be in the neighborhood
of previous samples.  Type \texttt{grep timestep H.xml} to see the varying time
step values in this QMCPACK input file (\texttt{H.xml}):

\begin{shade} 
<parameter name="timestep">10</parameter>
<parameter name="timestep">5</parameter> 
<parameter name="timestep">2</parameter> 
<parameter name="timestep">1</parameter>
<parameter name="timestep">0.5</parameter> 
<parameter name="timestep">0.2</parameter> 
<parameter name="timestep">0.1</parameter>
<parameter name="timestep">0.05</parameter> 
<parameter name="timestep">0.02</parameter> 
<parameter name="timestep">0.01</parameter>
<parameter name="timestep">0.005</parameter> 
<parameter name="timestep">0.002</parameter> 
<parameter name="timestep">0.001</parameter>
<parameter name="timestep">0.0005</parameter> 
<parameter name="timestep">0.0002</parameter> 
<parameter name="timestep">0.0001</parameter> 
\end{shade}

Generally, as the time step decreases, the autocorrelation will increase
(caveat: very large time steps will also have increasing autocorrelation). To
see this, type \texttt{qmca -q e {-}{-}sac *.scalar.dat} to see the energies
and autocorrelation times, then plot with gnuplot by inputting \texttt{gnuplot
H.plt}:

\FloatBarrier
\begin{figure}[ht!]
\begin{center}
\includegraphics[trim = 0mm 0mm 0mm 0mm, clip,width=0.75\columnwidth]{./figures/lab_qmc_statistics_blocking1.png}
\end{center}
\end{figure}
\FloatBarrier

%\includegraphics[scale=1.0]{timestep_vs_autocorrelation_energy_H_STO-2G.png}

The error bar also increases with the autocorrelation.  

Press \texttt{q [Enter]} to quit gnuplot.

To get around the bias of autocorrelation, we group the MC steps into blocks,
take the average of the data in the steps of each block, and then finally
average the averages in all the blocks.  QMCPACK outputs the block averages as
each line in the \texttt{scalar.dat} file.  (For DMC simulations, in addition to the
\texttt{scalar.dat}, QMCPACK outputs the quantities at each step to the \texttt{dmc.dat} file,
which permits reblocking the data differently from the specification in the
input file.) 

Change directories to \texttt{blocking}.  Here we look at the time step of the
last dataset in the \texttt{autocorrelation} directory.  Verify this by typing
\texttt{grep timestep H.xml} to see that all values are set to 0.001.  Now to
see how we will vary the blocking, type \texttt{grep -A1 blocks H.xml}.  The
parameter ``steps'' indicates the number of steps per block, and the parameter
``blocks'' gives the number of blocks.  For this comparison, the total number
of MC steps (equal to the product of ``steps'' and ``blocks'') is fixed at
50,000.  Now check the effect of blocking on autocorrelation---type \texttt{qmca
-q e {-}{-}sac *scalar.dat} to see the data and \texttt{gnuplot H.plt} to
visualize the data:

%\begin{shaded} 
%\begin{verbatim} 
%H  series 0  LocalEnergy = -0.454433 +/- 0.003970   189.2 
%H  series 1  LocalEnergy = -0.453352 +/- 0.004159   104.5 
%H  series 2  LocalEnergy = -0.449211 +/- 0.006544   114.1 
%H  series 3  LocalEnergy = -0.449491 +/- 0.014770   381.1 
%H  series 4  LocalEnergy = -0.446602 +/- 0.008809   78.2 
%H  series 5  LocalEnergy = -0.488471 +/- 0.006704   27.2 
%H  series 6  LocalEnergy = -0.427345 +/- 0.011377   50.0 
%H  series 7  LocalEnergy = -0.456044 +/- 0.014513   51.1 
%H  series 8  LocalEnergy = -0.453782 +/- 0.016594   24.1 
%H  series 9  LocalEnergy = -0.482306 +/- 0.028252   21.6 
%H  series 10  LocalEnergy = -0.405258 +/- 0.013696   22.4 
%H  series 11  LocalEnergy = -0.423111 +/- 0.003579    2.9 
%H  series 12  LocalEnergy = -0.474759 +/- 0.016879    9.6 
%H  series 13  LocalEnergy = -0.414045 +/- 0.003606    5.5 
%H  series 14  LocalEnergy = -0.432808 +/- 0.004773    3.3 
%H  series 15  LocalEnergy = -0.465723 +/- 0.004425    2.6 
%\end{verbatim}
%\end{shaded}

\FloatBarrier
\begin{figure}[ht!]
\begin{center}
\includegraphics[trim = 0mm 0mm 0mm 0mm, clip,width=0.75\columnwidth]{./figures/lab_qmc_statistics_blocking2.png}
\end{center}
\end{figure}
\FloatBarrier

%\includegraphics[scale=1.0]{steps_per_block_vs_autocorrelation_energy_H_STO-2G.png}

The greatest number of steps per block produces the smallest autocorrelation
time.  The larger number of blocks over which to average at small
step-per-block number masks the corresponding increase in error bar with
increasing autocorrelation.

Press \texttt{q [Enter]} to quit gnuplot.

\subsection{Balancing autocorrelation and acceptance ratio}

Adjusting the time step value also affects the ratio of accepted steps to
proposed steps.  Stepping nearby in configuration space implies that the
probability distribution is similar and thus more likely to result in an
accepted move.  Keeping the acceptance ratio high means the algorithm is
efficiently exploring configuration space and not sticking at particular
configurations.  Return to the \ishell{autocorrelation} directory.  Refresh your
memory on the time steps in this set of simulations by \texttt{grep timestep
H.xml}. Then, type \texttt{qmca -q ar *scalar.dat} to see the acceptance ratio
as it varies with decreasing time step:

\begin{shade} 
H  series 0  AcceptRatio = 0.047646 +/- 0.000206 
H  series 1  AcceptRatio = 0.125361 +/- 0.000308 
H  series 2  AcceptRatio = 0.328590 +/- 0.000340 
H  series 3  AcceptRatio = 0.535708 +/- 0.000313 
H  series 4  AcceptRatio = 0.732537 +/- 0.000234 
H  series 5  AcceptRatio = 0.903498 +/- 0.000156 
H  series 6  AcceptRatio = 0.961506 +/- 0.000083 
H  series 7  AcceptRatio = 0.985499 +/- 0.000051 
H  series 8  AcceptRatio = 0.996251 +/- 0.000025 
H  series 9  AcceptRatio = 0.998638 +/- 0.000014 
H  series 10  AcceptRatio = 0.999515 +/- 0.000009 
H  series 11  AcceptRatio = 0.999884 +/- 0.000004 
H  series 12  AcceptRatio = 0.999958 +/- 0.000003 
H  series 13  AcceptRatio = 0.999986 +/- 0.000002 
H  series 14  AcceptRatio = 0.999995 +/- 0.000001 
H  series 15  AcceptRatio = 0.999999 +/- 0.000000 
\end{shade}

By series 8 (time step = 0.02), the acceptance ratio is in excess of 99\%.  

Considering the increase in autocorrelation and subsequent increase in error
bar as time step decreases, it is important to choose a time step that trades
off appropriately between acceptance ratio and autocorrelation.  In this
example, a time step of 0.02 occupies a spot where the acceptance ratio is high
(99.6\%) and autocorrelation is not appreciably larger than the minimum value
(1.4 vs. 1.0).

\subsection{Considering variance}

Besides autocorrelation, the dominant contributor to the error bar is the
\textit{variance} in the local energy.  The variance measures the fluctuations
around the average local energy, and, as the fluctuations go to zero, the wavefunction reaches an exact eigenstate of the Hamiltonian.  qmca calculates this
from the local energy and local energy squared columns of the \texttt{scalar.dat}. 

Type \texttt{qmca -q v H.s009.scalar.dat} to calculate the variance on the run
with time step balancing autocorrelation and acceptance ratio:

\begin{shade}
H  series 9  Variance = 0.513570 +/- 0.010589  
\end{shade}

Just as the total energy does not tell us much by itself, neither does the
variance.  However, comparing the ratio of the variance with the energy indicates
how the magnitude of the fluctuations compares with the energy itself.   Type
\texttt{qmca -q ev H.s009.scalar.dat} to calculate the energy and variance on
the run side by side with the ratio:

\begin{shade}
                     LocalEnergy               Variance        ratio
H  series 0  -0.454460 +/- 0.000568   0.529496 +/- 0.018445   1.1651
\end{shade}

The very high ration of 1.1651 indicates the square of the fluctuations is on
average larger than the value itself.  In the next section, we will approach
ways to improve the variance that subsequent labs will build on.  

\section{Reducing statistical error bars}

\subsection{Increasing MC sampling}

Increasing the number of MC samples in a dataset reduces the error bar as the
inverse of the square root of the number of samples.  There are two ways to
increase the number of MC samples in a simulation: (1) running more samples in
parallel and (2) increasing the number of blocks (with fixed number of steps per
block, this increases the total number of MC steps).

To see the effect of running more samples in parallel, change to the
directory \ishell{nodes}.  The series here increases the number of nodes by
factors of four from 32 to 128 to 512.  Type \texttt{qmca -q ev *scalar.dat}
and note the change in the error bar on the local energy as the number of
nodes.  Visualize this with \textbf{gnuplot H.plt}:

\FloatBarrier
\begin{figure}[ht!]
\begin{center}
\includegraphics[trim = 0mm 0mm 0mm 0mm, clip,width=0.75\columnwidth]{./figures/lab_qmc_statistics_nodes.png}
\end{center}
\end{figure}
\FloatBarrier

%\includegraphics[scale=1.0]{nnode_vs_energy_H_STO-2G.png}

Increasing the number of blocks, unlike running in parallel, increases the
total CPU time of the simulation.  

Press \texttt{q [Enter]} to quit gnuplot.

To see the effect of increasing the block number, change to the directory
\ishell{blocks}. To see how we will vary the number of blocks, type
\texttt{grep -A1 blocks H.xml}.  The number of steps remains fixed, thus
increasing the total number of samples.   Visualize the tradeoff by inputting
\texttt{gnuplot H.plt}: 

\FloatBarrier
\begin{figure}[ht!]
\begin{center}
\includegraphics[trim = 0mm 0mm 0mm 0mm, clip,width=0.75\columnwidth]{./figures/lab_qmc_statistics_blocks.png}
\end{center}
\end{figure}
\FloatBarrier

%\includegraphics[scale=1.0]{nblock_vs_tcpu_energy_H_STO-2G.png}

Press \texttt{q [Enter]} to quit gnuplot.

\subsection{Improving the basis set}

In all of the previous examples, we are using the sum of two Gaussian functions
(STO-2G) to approximate what should be a simple decaying exponential for the wavefunction of the ground state of the hydrogen
atom.  The sum of multiple copies of a function varying each copy's width and
amplitude with coefficients is called a \textit{basis set}. As we add Gaussians
to the basis set, the approximation improves, the variance goes toward zero, and
the energy goes to -0.5 Ha.  In nearly every other case, the exact
function is unknown, and we add basis functions until the total energy does not
change within some threshold.

Change to the directory \ishell{basis} and look at the total energy and
variance as we change the wavefunction by typing \texttt{qmca -q ev H\_*}:

\begin{shade}
                            LocalEnergy               Variance        ratio 
H_STO-2G  series 0  -0.454460 +/- 0.000568   0.529496 +/- 0.018445   1.1651 
H_STO-3G  series 0  -0.465386 +/- 0.000502   0.410491 +/- 0.010051   0.8820 
H_STO-6G  series 0  -0.471332 +/- 0.000491   0.213919 +/- 0.012954   0.4539 
H__exact  series 0  -0.500000 +/- 0.000000   0.000000 +/- 0.000000   -0.0000 
\end{shade}

qmca also puts out the ratio of the variance to the local energy in a column to
the right of the variance error bar.  A typical high-quality value for this
ratio is lower than 0.1 or so---none of these few-Gaussian wavefunctions
satisfy that rule of thumb.

Use qmca to plot the trace of the local energy, kinetic energy, and potential
energy of H\_\_exact. The total energy is constantly -0.5 Ha even though
the kinetic and potential energies fluctuate from configuration to
configuration.

\subsection{Adding a Jastrow factor}

Another route to reducing the variance is the introduction of a Jastrow factor to 
account for electron-electron correlation (not the statistical autocorrelation
of MC steps but the physical avoidance that electrons have of one another).
To do this, we will switch to the hydrogen dimer with the exact ground state
wavefunction of the atom (STO basis)---this will not be exact for the dimer.
The ground state energy of the hydrogen dimer is -1.174 Ha.

Change directories to \ishell{dimer} and put in \texttt{qmca -q ev *scalar.dat}
to see the result of adding a simple, one-parameter Jastrow to the STO basis
for the hydrogen dimer at experimental bond length:

\begin{shade}
                               LocalEnergy               Variance           
H2_STO___no_jastrow  series 0  -0.876548 +/- 0.005313   0.473526 +/- 0.014910
H2_STO_with_jastrow  series 0  -0.912763 +/- 0.004470   0.279651 +/- 0.016405
\end{shade}

The energy reduces by 0.044 +/- 0.006 HA and the variance by 0.19 +/- 0.02.
This is still 20\% above the ground state energy, and subsequent labs will cover how
to improve on this with improved forms of the wavefunction that capture more
of the physics.

\section{Scaling to larger numbers of electrons}

\subsection{Calculating the efficiency}

The inverse of the product of CPU time and the variance measures the
\textit{efficiency} of an MC calculation.  Use qmca to calculate efficiency by
typing \texttt{qmca -q eff *scalar.dat} to see the efficiency of these two
H$_2$ calculations:

\begin{shade}
H2_STO___no_jastrow  series 0  Efficiency = 16698.725453 +/- 0.000000 
H2_STO_with_jastrow  series 0  Efficiency = 52912.365609 +/- 0.000000 
\end{shade}

The Jastrow factor increased the efficiency in these calculations by a factor
of three, largely through the reduction in variance (check the average block
CPU time to verify this claim).

\subsection{Scaling up}

To see how MC scales with increasing particle number, change directories to
\ishell{size}.  Here are the data from runs of increasing numbers of electrons
for H, H$_2$, C, CH$_4$, C$_2$, C$_2$H$_4$, (CH$_4$)$_2$, and (C$_2$H$_4$)$_2$
using the STO-6G basis set for the orbitals of the Slater determinant.  The file names begin with the number of electrons simulated for those data.

Use \texttt{qmca -q bc *scalar.dat} to see that the CPU time per block
increases with the number of electrons in the simulation; then plot the total CPU
time of the simulation by \textbf{gnuplot Nelectron\_tCPU.plt}:

\FloatBarrier
\begin{figure}[ht!]
\begin{center}
\includegraphics[trim = 0mm 0mm 0mm 0mm, clip,width=0.75\columnwidth]{./figures/lab_qmc_statistics_scaling.png}
\end{center}
\end{figure}
\FloatBarrier

%\includegraphics[scale=1.0]{nelectron_vs_tcpu_H_C_CH_STO-6G.png}

The green pluses represent the CPU time per block at each electron number.
The red line is a quadratic fit to those data.  For a fixed basis set size, we expect the time to scale quadratically up to 1,000s of electrons, at which point a cubic scaling term may become dominant.  Knowing the scaling allows you to roughly project the calculation time for a larger number of electrons.

Press \texttt{q [Enter]} to quit gnuplot.

This is not the whole story, however.  The variance of the energy also increases
with a fixed basis set as the number of particles increases at a faster rate
than the energy decreases.  To see this, type \texttt{qmca -q ev *scalar.dat}:

\begin{shade}
                            LocalEnergy               Variance           
01________H  series 0  -0.471352 +/- 0.000493      0.213020 +/- 0.012950 
02_______H2  series 0  -0.898875 +/- 0.000998      0.545717 +/- 0.009980 
06________C  series 0  -37.608586 +/- 0.020453   184.322000 +/- 45.481193
10______CH4  series 0  -38.821513 +/- 0.022740   169.797871 +/- 24.765674
12_______C2  series 0  -72.302390 +/- 0.037691   491.416711 +/- 106.090103
16_____C2H4  series 0  -75.488701 +/- 0.042919   404.218115 +/- 60.196642
18___CH4CH4  series 0  -58.459857 +/- 0.039309   498.579645 +/- 92.480126
32_C2H4C2H4  series 0  -91.567283 +/- 0.048392   632.114026 +/- 69.637760
\end{shade}

The increase in variance is not uniform, but the general trend is upward with a
fixed wavefunction form and basis set.  Subsequent labs will address how to
improve the wavefunction to keep the variance manageable.

\chapter{Lab 2: QMC basics}
\label{chap:lab_qmc_basics}



\section{Topics covered in this lab}
This lab focuses on the basics of performing quality QMC calculations.  As an example, participants test an oxygen pseudopotential within DMC by calculating atomic and dimer properties, a common step prior to production runs.  Topics covered include:
\begin{itemize}
  \item{Converting pseudopotentials into QMCPACK's FSATOM format}
  \item{Generating orbitals with QE}
  \item{Converting orbitals into QMCPACK's ESHDF format with pw2qmcpack}
  \item{Optimizing Jastrow factors with QMCPACK}
  \item{Removing DMC time step errors via extrapolation}
  \item{Automating QMC workflows with Nexus}
  \item{Testing pseudopotentials for accuracy}
  \hide{
  \item{(optional) Running QMCPACK for a general system of interest}
  }
\end{itemize}

\section{Lab outline}
\begin{enumerate}
  \item{Download and conversion of oxygen atom pseudopotential}
  \item{DMC time step study of the neutral oxygen atom}
  \begin{enumerate}
    \item{DFT orbital generation with QE}
    \item{Orbital conversion with \ishell{pw2qmcpack.x}}
    \item{Optimization of Jastrow correlation factor with QMCPACK}
    \item{DMC run with multiple time steps}
  \end{enumerate}
  \item{DMC time step study of the first ionization potential of oxygen}
  \begin{enumerate}
    \item{Repetition of a-d above for ionized oxygen atom}
  \end{enumerate}
  \item{Automated DMC calculations of the oxygen dimer binding curve}
\end{enumerate}


\section{Lab directories and files}
\footnotesize
\begin{verbatim}%
labs/lab2_qmc_basics/
│
├── oxygen_atom           - oxygen atom calculations 
│   ├── O.q0.dft.in          - Quantum ESPRESSO input for DFT run
│   ├── O.q0.p2q.in          - pw2qmcpack.x input for orbital conversion run
│   ├── O.q0.opt.in.xml      - QMCPACK input for Jastrow optimization run
│   ├── O.q0.dmc.in.xml      - QMCPACK input file for neutral O DMC
│   ├── ip_conv.py           - tool to fit oxygen IP vs timestep
│   └── reference            - directory w/ completed runs
│
├── oxygen_dimer          - oxygen dimer calculations
│   ├── dimer_fit.py         - tool to fit dimer binding curve
│   ├── O_dimer.py           - automation script for dimer calculations
│   ├── pseudopotentials     - directory for pseudopotentials
│   └── reference            - directory w/ completed runs
│
└── your_system           - performing calculations for an arbitrary system (yours)
    ├── example.py           - example nexus file for periodic diamond
    ├── pseudopotentials     - directory containing C pseudopotentials
    └── reference            - directory w/ completed runs
\end{verbatim}
\normalsize

\section{Obtaining and converting a pseudopotential for oxygen}
\label{sec:lqb_pseudo}
First enter the \ishell{oxygen_atom} directory:
\begin{shade}
cd labs/lab2_qmc_basics/oxygen_atom/
\end{shade}
\noindent
Throughout the rest of the lab, locations are specified with respect to \ishell{labs/lab2_qmc_basics} (e.g., \ishell{oxygen_atom}).

We use a potential from the Burkatzki-Filippi-Dolg pseudopotential database.  
Although the full database is available in QMCPACK distribution (\ishell{trunk/pseudopotentials/BFD/}), 
we use a BFD pseudopotential to illustrate the process of converting and testing an 
external potential for use with QMCPACK.   To obtain the pseudopotential, go to 
\href{http://www.burkatzki.com/pseudos/index.2.html}{http://www.burkatzki.com/pseudos/index.2.html}
and click on the ``Select Pseudopotential'' button.  Next click on oxygen in the 
periodic table.  Click on the empty circle next to ``V5Z'' (a large Gaussian 
basis set) and click on ``Next.''  Select the Gamess format and click on 
``Retrive Potential.''  Helpful information about the pseudopotential will be 
displayed.  The desired portion is at the bottom (the last 7 lines).  Copy 
this text into the editor of your choice (e.g., \ishell{emacs} or \ishell{vi}) 
and save it as \ishell{O.BFD.gamess} 
(be sure to include a new line at the end of the file).  To transform the 
pseudopotential into the FSATOM XML format used by QMCPACK, use the \ishell{ppconvert} 
tool:

\noindent
\ifws
\begin{shade}
ppconvert --gamess_pot O.BFD.gamess --s_ref "1s(2)2p(4)" \
 --p_ref "1s(2)2p(4)" --d_ref "1s(2)2p(4)" --xml O.BFD.xml
\end{shade}
\else
\begin{shade}
jobrun_vesta ppconvert --gamess_pot O.BFD.gamess --s_ref "1s(2)2p(4)" \
 --p_ref "1s(2)2p(4)" --d_ref "1s(2)2p(4)" --xml O.BFD.xml
\end{shade}
\fi

\noindent
Observe the notation used to describe the reference valence configuration for this helium-core PP: \ishell{1s(2)2p(4)}.  The \ishell{ppconvert} tool uses the following convention for the valence states: the first $s$ state is labeled \ishell{1s} (\ishell{1s}, \ishell{2s}, \ishell{3s}, \ldots), the first $p$ state is labeled \ishell{2p} (\ishell{2p}, \ishell{3p}, \ldots), and the first $d$ state is labeled \ishell{3d} (\ishell{3d}, \ishell{4d}, \ldots). Copy the resulting xml file into the \ishell{oxygen_atom} directory.

Note: The command to convert the PP into QE's UPF format is similar (both formats are required):

\noindent
\ifws
\begin{shade}
ppconvert --gamess_pot O.BFD.gamess --s_ref "1s(2)2p(4)" \
 --p_ref "1s(2)2p(4)" --d_ref "1s(2)2p(4)" --log_grid --upf O.BFD.upf
\end{shade}
\else
\begin{shade}
jobrun_vesta ppconvert --gamess_pot O.BFD.gamess --s_ref "1s(2)2p(4)" \
 --p_ref "1s(2)2p(4)" --d_ref "1s(2)2p(4)" --log_grid --upf O.BFD.upf
\end{shade}
\noindent
\fi

For reference, the text of \ishell{O.BFD.gamess} should be:
\begin{lstlisting}
O-QMC GEN 2 1
3
6.00000000 1 9.29793903
55.78763416 3 8.86492204
-38.81978498 2 8.62925665
1
38.41914135 2 8.71924452

\end{lstlisting}
\noindent
The full QMCPACK pseudopotential is also included in \ishell{oxygen_atom/reference/O.BFD.*}.


\section{DFT with QE to obtain the orbital part of the wavefunction}
\label{sec:lqb_dft}
With the pseudopotential in hand, the next step toward a QMC calculation is to obtain the Fermionic part of the wavefunction, in this case a single Slater determinant constructed from DFT-LDA orbitals for a neutral oxygen atom.  If you had trouble with the pseudopotential conversion step, preconverted pseudopotential files are located in the \ishell{oxygen_atom/reference} directory.  

QE input for the DFT-LDA ground state of the neutral oxygen atom can be found in \ishell{O.q0.dft.in} and also in listing~\ref{lst:O_q0_dft}.  Setting \ishell{wf_collect=.true.} instructs QE to write the orbitals to disk at the end of the run. Option \ishell{wf_collect=.true.} could be a potential problem in large simulations; therefore, we recommend avoiding it and using the converter pw2qmcpack in parallel (see details in Section~\ref{sec:pw2qmcpack}). Note that the plane-wave energy cutoff has been set to a reasonable value of 300 Ry here (\ishell{ecutwfc=300}).  This value depends on the pseudopotentials used, and, in general, should be selected by running DFT$\rightarrow$(orbital conversion)$\rightarrow$VMC with increasing energy cutoffs until the lowest VMC total energy and variance is reached.

\begin{lstlisting}[style=ESPRESSO, language=espresso, caption={QE input file for the neutral oxygen atom (\ishell{O.q0.dft.in})\label{lst:O_q0_dft}}.]
&CONTROL
   calculation       = 'scf'
   restart_mode      = 'from_scratch'
   prefix            = 'O.q0'
   outdir            = './'
   pseudo_dir        = './'
   disk_io           = 'low'
   wf_collect        = .true.
/

&SYSTEM
   celldm(1)         = 1.0
   ibrav             = 0
   nat               = 1
   ntyp              = 1
   nspin             = 2
   tot_charge        = 0
   tot_magnetization = 2
   input_dft         = 'lda'
   ecutwfc           = 300
   ecutrho           = 1200
   nosym             = .true.
   occupations       = 'smearing'
   smearing          = 'fermi-dirac'
   degauss           = 0.0001
/

&ELECTRONS
   diagonalization   = 'david'
   mixing_mode       = 'plain'
   mixing_beta       = 0.7
   conv_thr          = 1e-08
   electron_maxstep  = 1000
/


ATOMIC_SPECIES 
   O  15.999 O.BFD.upf

ATOMIC_POSITIONS alat
   O     9.44863067       9.44863161       9.44863255

K_POINTS automatic
   1 1 1  0 0 0 

CELL_PARAMETERS cubic
        18.89726133       0.00000000       0.00000000 
         0.00000000      18.89726133       0.00000000 
         0.00000000       0.00000000      18.89726133
\end{lstlisting}

Run QE by typing 
\ifws
\begin{shade}
mpirun -np 4 pw.x -input O.q0.dft.in >&O.q0.dft.out&
\end{shade}
\else
\begin{shade}
jobrun_vesta pw.x O.q0.dft.in
\end{shade}
\fi

The DFT run should take a few minutes to complete.  If desired, you can track the progress of the DFT run by typing ``\ifws\ishell{tail -f O.q0.dft.out}.\else\ishell{tail -f O.q0.dft.output}\fi'' Once finished, you should check the LDA total energy in \ifws\ishell{O.q0.dft.out}\: \else\ishell{O.q0.dft.output}\fi by typing  ``\ifws\ishell{grep '!  ' O.q0.dft.out}.\else\ishell{grep '!  ' O.q0.dft.output}\fi''  The result should be close to
\begin{shade}
!    total energy              =     -31.57553905 Ry
\end{shade} 
% both of the numbers below are for 200 Ry (too small as it turns out)
% 10 Angstrom cell
%!    total energy              =     -31.56729415 Ry
% 15 Angstrom cell
%!    total energy              =     -31.56730213 Ry



The orbitals have been written in a format native to QE in the \ishell{O.q0.save} directory.  We will convert them into the ESHDF format expected by QMCPACK by using the \ishell{pw2qmcpack.x} tool.  The input for \ishell{pw2qmcpack.x} can be found in the file \ishell{O.q0.p2q.in} and also in listing~\ref{lst:O_q0_p2q}. 

\begin{lstlisting}[caption={\ishell{pw2qmcpack.x} input file for orbital conversion (\ishell{O.q0.p2q.in})\label{lst:O_q0_p2q}}.]
&inputpp
  prefix     = 'O.q0'
  outdir     = './'
  write_psir = .false.
/
\end{lstlisting}

Perform the orbital conversion now by typing the following:
\ifws
\begin{shade}
mpirun -np 1 pw2qmcpack.x<O.q0.p2q.in>&O.q0.p2q.out&
\end{shade}
\else
\begin{shade}
jobrun_vesta pw2qmcpack.x O.q0.p2q.in
\end{shade}
\fi
\noindent
Upon completion of the run, a new file should be present containing the orbitals for QMCPACK: \ishell{O.q0.pwscf.h5}.  Template XML files for particle (\ishell{O.q0.ptcl.xml}) and wavefunction (\ishell{O.q0.wfs.xml}) inputs to QMCPACK should also be present.  


\section{Optimization with QMCPACK to obtain the correlated part of the wavefunction}\label{sec:optimization_walkthrough}
The wavefunction we have obtained to this point corresponds to a noninteracting Hamiltonian.  Once the Coulomb pair potential is switched on between particles, it is known analytically that the exact wavefunction has cusps whenever two particles meet spatially and, in general, the electrons become correlated.  This is represented in the wavefunction by introducing a Jastrow factor containing at least pair correlations:
\begin{align}
  &\Psi_{Slater-Jastrow}=e^{-J}\Psi_{Slater} \\
  &J = \sum_{\sigma\sigma'}\sum_{i<j}u^{\sigma\sigma'}_2(|r_i-r_j|) + \sum_\sigma\sum_{iI}u^{\sigma I}_1(|r_i-r_I|)\:.
\end{align}
Here $\sigma$ is a spin variable while $r_i$ and $r_I$ represent electron and ion coordinates, respectively.  The introduction of $J$ into the wavefunction is similar to F12 methods in quantum chemistry, though it has been present in essentially all QMC studies since the first applications the method (circa 1965).

How are the functions $u_2^{\sigma\sigma'}$ and $u_1^{\sigma}$ obtained?  Generally, they are approximated by analytical functions with several unknown parameters that are determined by minimizing the energy or variance directly within VMC.  This is effective because the energy and variance reach a global minimum only for the true ground state wavefunction ($\textrm{Energy}=E\equiv\expval{\Psi}{\hat{H}}{\Psi}$, $\textrm{Variance}=V\equiv\expval{\Psi}{(\hat{H}-E)^2}{\Psi}$).  For this exercise, we will focus on minimizing the variance.

% background on the wavefunction should be covered elsewhere in the manual
%   perhaps replace this with just the figure and a couple of brief comments 
\hide{
\subsubsection{Background on trial wavefunction and optimization}\label{sec:opt_background}
The trial wavefunction used to describe the neutral oxygen atom is of the 
standard Slater-Jastrow form:
\begin{align}  
  \Psi_T = e^{-(J_1+J_2)}D^\uparrow(\{\phi_u^\uparrow\}_{u=1}^{N^\uparrow})D^\downarrow(\{\phi_d^\downarrow\}_{d=1}^{N^\uparrow})
\end{align}
The orbitals forming the spin-restricted Slater determinants 
($D^\uparrow/D^\downarrow$) are obtained from DFT or Hartree-Fock (\emph{e.g.,} via QE) 
and are fixed.  The ground state of the (pseudo) oxygen atom is spin polarized 
with $N^{\uparrow}=4$ and $N^{\downarrow}=2$.  

The part of the wavefunction we will be optimizing is the Jastrow factor 
($e^{-(J_1+J_2)}$), which in this case includes 1- (electron-ion) and 2- 
(electron-electron) body correlation functions.  The Jastrow factor is symmetric 
under same-spin electron exchange and does not affect the DMC fixed node 
approximation.  Optimization of the Jastrow factor does, however, improve the 
efficiency of the DMC calculation and reduces additional approximations due to 
nonlocal pseudopotentials (locality approximation, T-moves).  Note that a 3-body 
term ($J_3$) is also available and is often necessary when using pseudopotentials 
for transition metal species or when high accuracy is desired for molecules.  


\begin{figure}
\begin{center}
\includegraphics[trim = 0mm 0mm 0mm 0mm, clip,width=0.75\columnwidth]{./figures/lab_qmc_basics_J1.png}
\end{center}
\caption{Optimized $U_1$ function for 1-body Jastrow factor of an oxygen atom.
\label{fig:u1_spline}
}
\end{figure}

The explicit form of the 1-body Jastrow factor we will be using is
\begin{align}\label{eq:J1}
  J_1 = \sum_{e=1}^{N^\uparrow+N^\downarrow}U_1^{\uparrow/\downarrow}(|r_e-r_O|)
\end{align}
where $r_e$ refers to the electron positions and $r_O$ is 
the position of the oxygen ion.  The $U_1^{\uparrow/\downarrow}$ term is a 
1D radial function represented with piecewise continuous cubic 
polynomials (B-splines).  The adjustable parameters to be optimized are the 
``knots'' of the B-splines, which are simply the values of the $U_1$ function at 
uniformly spaced grid points (See Figure~\ref{fig:u1_spline} for an example of a $U_1$ 
spline function with 8 knots).  

The 2-body Jastrow factor is spin resolved ($r^\uparrow/r^\downarrow$ are up/down electron positions):
\begin{align}\label{eq:J2}
  J_2 = \sum_{u<u'}U_2^{\uparrow\uparrow/\downarrow\downarrow}(|r_u^\uparrow-r_{u'}^\uparrow|) + \sum_{d<d'}U_2^{\uparrow\uparrow/\downarrow\downarrow}(|r_d^\downarrow-r_{d'}^\downarrow|) + \sum_{u,d} U_2^{\uparrow\downarrow}(|r_u^\uparrow-r_d^\downarrow|)
\end{align}
For an atom, Pad\'{e} functions are appropriate for $U_2^{\uparrow\uparrow/\downarrow\downarrow}$ and $U_2^{\uparrow\downarrow}$:
\begin{align}
  U_2(r) = \frac{Ar}{1+Br}
\end{align}
Only $B^{\uparrow\uparrow/\downarrow\downarrow}$ and $B^{\uparrow\downarrow}$ are adjustable since the $A$ parameters are fixed by the electron-electron cusp conditions.

Wavefunction optimization essentially relies on two inequalities regarding energy and variance:
\begin{align}
  E_T(P) &= \frac{\expvalh{\Psi_T(P)}{H}{\Psi_T(P)}}{\overlap{\Psi_T(P)}{\Psi_T(P)}} \ge E_0 \\
  V_T(P) &= \frac{\expval{\Psi_T(P)}{\hat{H}^2}{\Psi_T(P)}}{\overlap{\Psi_T(P)}{\Psi_T(P)}} - \left(\frac{\expval{\Psi_T(P)}{H}{\Psi_T(P)}}{\overlap{\Psi_T(P)}{\Psi_T(P)}}\right)^2 \ge 0   
\end{align}
Here $E_0$ is the ground state energy, $E_T(P)$ is the trial energy, $V_T(P)$ is the trial variance, and $P$ denotes the set of adjustable parameters in the trial wavefunction.  Equality is reached only for the true ground state wavefunction, and so the trial wavefunction can be improved by attempting to minimize a chosen cost function: 
\begin{align}
  C(P) = \alpha E_T(P) + (1-\alpha) V_T(P).
\end{align}  
Iterative varational MC methods have been developed to handle the nonlinear optimization problem $\min\limits_P C(P)$.  We will be using the linearized optimization method of Umrigar, \emph{et al.} (PRL \textbf{98} 110201 (2007)).  Let us try this now with QMCPACK.
}

First, we need to update the template particle and wavefunction information in \ishell{O.q0.ptcl.xml} and \ishell{O.q0.wfs.xml}.  We want to simulate the O atom in open boundary conditions (the default is periodic).  To do this, open \ishell{O.q0.ptcl.xml} with your favorite text editor (e.g., \ishell{emacs} or \ishell{vi}) and replace
\begin{lstlisting}[style=QMCPXML]
<parameter name="bconds">
   p p p
</parameter>
<parameter name="LR_dim_cutoff">
   15
</parameter>
\end{lstlisting}
with
\begin{lstlisting}[style=QMCPXML]
<parameter name="bconds">
   n n n 
</parameter>
\end{lstlisting}

Next we will select Jastrow factors appropriate for an atom.  In open boundary conditions, the B-spline Jastrow correlation functions should cut off to zero at some distance away from the atom.  Open \ishell{O.q0.wfs.xml} and add the following cutoffs (\ishell{rcut} in Bohr radii) to the correlation factors:
\begin{lstlisting}[style=QMCPXML]
...
<correlation speciesA="u" speciesB="u" size="8" rcut="10.0">
...
<correlation speciesA="u" speciesB="d" size="8" rcut="10.0">
...
<correlation elementType="O" size="8" rcut="5.0">
...
\end{lstlisting}
\noindent
These terms correspond to $u_2^{\uparrow\uparrow}/u_2^{\downarrow\downarrow}$, $u_2^{\uparrow\downarrow}$, and $u_1^{\uparrow O}/u_1^{\downarrow O}$, respectively.  In each case, the correlation function ($u_*$) is represented by piecewise continuous cubic B-splines.  Each correlation function has eight parameters, which are just the values of $u$ on a uniformly spaced grid up to \ishell{rcut}.  Initially the parameters (\ishell{coefficients}) are set to zero:
\begin{lstlisting}[style=QMCPXML]
<correlation speciesA="u" speciesB="u" size="8" rcut="10.0">
  <coefficients id="uu" type="Array">
     0.0 0.0 0.0 0.0 0.0 0.0 0.0 0.0
  </coefficients>
</correlation>
\end{lstlisting}

Finally, we need to assemble particle, wavefunction, and pseudopotential information into the main QMCPACK input file (\ishell{O.q0.opt.in.xml}) and specify inputs for the Jastrow optimization process.  Open \ishell{O.q0.opt.in.xml} and write in the location of the particle, wavefunction, and pseudopotential files (``\ishell{<!-- ... -->}'' are comments):
\begin{lstlisting}[style=QMCPXML]
...
<!-- include simulationcell and particle information from pw2qmcpqack -->
<include href="O.q0.ptcl.xml"/>
...
<!-- include wavefunction information from pw2qmcpqack -->
<include href="O.q0.wfs.xml"/>
...
<!-- O pseudopotential read from "O.BFD.xml" -->
<pseudo elementType="O" href="O.BFD.xml"/>
...
\end{lstlisting}
\noindent
The relevant portion of the input describing the linear optimization process is
\begin{lstlisting}[style=QMCPXML]
<loop max="MAX">  
  <qmc method="linear" move="pbyp" checkpoint="-1">
    <cost name="energy"              >  ECOST    </cost>
    <cost name="unreweightedvariance">  UVCOST   </cost>
    <cost name="reweightedvariance"  >  RVCOST   </cost>
    <parameter name="timestep"       >  TS       </parameter>
    <parameter name="samples"        >  SAMPLES  </parameter>
    <parameter name="warmupSteps"    >  50       </parameter>
    <parameter name="blocks"         >  200      </parameter>
    <parameter name="subSteps"       >  1        </parameter>
    <parameter name="nonlocalpp"     >  yes      </parameter>
    <parameter name="useBuffer"      >  yes      </parameter>
    ...
  </qmc>
</loop>
\end{lstlisting}
\noindent
An explanation of each input variable follows.  The remaining variables control specialized internal details of the linear optimization algorithm.  The meaning of these inputs is beyond the scope of this lab, and reasonable results are often obtained keeping these values fixed. 
\begin{description}
  \item[energy] Fraction of trial energy in the cost function.
  \item[unreweightedvariance] Fraction of unreweighted trial variance in the cost function.  Neglecting the weights can be more robust.
  \item[reweightedvariance] Fraction of trial variance (including the full weights) in the cost function.  
  \item[timestep] Time step of the VMC random walk, determines spatial distance moved by each electron during MC steps.  Should be chosen such that the acceptance ratio of MC moves is around 50\% (30--70\% is often acceptable).  Reasonable values are often between 0.2 and 0.6 $\textrm{Ha}^{-1}$.
  \item[samples] Total number of MC samples collected for optimization; determines statistical error bar of cost function.  It is often efficient to start with a modest number of samples (50k) and then increase as needed.  More samples may be required if the wavefunction contains a large number of variational parameters.  MUST be be a multiple of the number of threads/cores\labsw{}{(use multiples of 512 on Vesta)}.
  \item[warmupSteps]  Number of MC steps discarded as a warmup or equilibration period of the random walk.  If this is too small, it will bias the optimization procedure.
  \item[blocks]  Number of average energy values written to output files.  Should be greater than 200 for meaningful statistical analysis of output data ({e.g.,} via \ishell{qmca}).
  \item[subSteps] Number of MC steps in between energy evaluations.  Each energy evaluation is expensive, so taking a few steps to decorrelate between measurements can be more efficient.  Will be less efficient with many substeps.
  \item[nonlocalpp,useBuffer] If \ishell{nonlocalpp="no,"} then the nonlocal part of the pseudopotential is not included when computing the cost function.  If \ishell{useBuffer="yes,"} then temporary data is stored to speed up nonlocal pseudopotential evaluation at the expense of memory consumption.  
  \item[loop max] Number of times to repeat the optimization.  Using the resulting wavefunction from the previous optimization in the next one improves the results.  Typical choices range between 8 and 16.   
\end{description}
The cost function defines the quantity to be minimized during optimization. The three components of the cost function, energy, unreweighted variance, and reweighted variance should sum to one.  Dedicating 100\% of the cost function to unreweighted variance is often a good choice.  Another common choice is to try 90/10 or 80/20 mixtures of reweighted variance and energy.  Using 100\% energy minimization is desirable for reducing DMC pseudopotential localization errors, but the optimization process is less stable and should be attempted only after first performing several cycles of, for example, variance minimization (the entire \ishell{loop} section can be duplicated with a different cost function each time).

Replace \ishell{MAX}, \ishell{EVCOST}, \ishell{UVCOST}, \ishell{RVCOST}, \ishell{TS}, and \ishell{SAMPLES} in the \ishell{loop} with appropriate starting values in the \ishell{O.q0.opt.in.xml} input file.  Perform the optimization run by typing
\ifws
\begin{shade}
mpirun -np 4 qmcpack O.q0.opt.in.xml >&O.q0.opt.out&
\end{shade}
\else
\begin{shade}
jobrun_vesta qmcpack O.q0.opt.in.xml
\end{shade}
\fi
\noindent
The run should take only a few minutes for reasonable values of loop \ishell{max} and \ishell{samples}.  

The log file output will appear in \labsw{\ishell{O.q0.opt.out}}{\ishell{O.q0.opt.output}}.  The beginning of each linear optimization will be marked with text similar to
\begin{shade}
=========================================================
  Start QMCFixedSampleLinearOptimize
  File Root O.q0.opt.s011 append = no 
=========================================================
\end{shade}
\noindent
At the end of each optimization section the change in cost function, new values for the Jastrow parameters, and elapsed wall clock time are reported:
\begin{shade}
 OldCost: 7.0598901869e-01 NewCost: 7.0592576381e-01 Delta Cost:-6.3254886314e-05
...
  <optVariables href="O.q0.opt.s011.opt.xml">
uu_0 6.9392504232e-01 1 1  ON 0
uu_1 4.9690781460e-01 1 1  ON 1
uu_2 4.0934542375e-01 1 1  ON 2
uu_3 3.7875640157e-01 1 1  ON 3
uu_4 3.7308380014e-01 1 1  ON 4
uu_5 3.5419786809e-01 1 1  ON 5
uu_6 4.3139019377e-01 1 1  ON 6
uu_7 1.9344371667e-01 1 1  ON 7
ud_0 3.9219009713e-01 1 1  ON 8
ud_1 1.2352664647e-01 1 1  ON 9
ud_2 4.4048945133e-02 1 1  ON 10
ud_3 2.1415676741e-02 1 1  ON 11
ud_4 1.5201803731e-02 1 1  ON 12
ud_5 2.3708169445e-02 1 1  ON 13
ud_6 3.4279064930e-02 1 1  ON 14
ud_7 4.3334583596e-02 1 1  ON 15
eO_0 -7.8490123937e-01 1 1  ON 16
eO_1 -6.6726618338e-01 1 1  ON 17
eO_2 -4.8753453838e-01 1 1  ON 18
eO_3 -3.0913993774e-01 1 1  ON 19
eO_4 -1.7901872177e-01 1 1  ON 20
eO_5 -8.6199000697e-02 1 1  ON 21
eO_6 -4.0601160841e-02 1 1  ON 22
eO_7 -4.1358075061e-03 1 1  ON 23
  </optVariables>
...
  QMC Execution time = 2.8218972974e+01 secs
\end{shade}
\noindent
The cost function should decrease during each linear optimization (\ishell{Delta cost < 0}).  Try ``\labsw{\ishell{grep OldCost *opt.out.}}{\ishell{grep OldCost *opt.output}}''  You should see something like this:
\begin{shade}
 OldCost: 1.2655186572e+00 NewCost: 7.2443875597e-01 Delta Cost:-5.4107990118e-01
 OldCost: 7.2229830632e-01 NewCost: 6.9833678217e-01 Delta Cost:-2.3961524143e-02
 OldCost: 8.0649629434e-01 NewCost: 8.0551871147e-01 Delta Cost:-9.7758287036e-04
 OldCost: 6.6821241388e-01 NewCost: 6.6797703487e-01 Delta Cost:-2.3537901148e-04
 OldCost: 7.0106275099e-01 NewCost: 7.0078055426e-01 Delta Cost:-2.8219672877e-04
 OldCost: 6.9538522411e-01 NewCost: 6.9419186712e-01 Delta Cost:-1.1933569922e-03
 OldCost: 6.7709626744e-01 NewCost: 6.7501251165e-01 Delta Cost:-2.0837557922e-03
 OldCost: 6.6659923822e-01 NewCost: 6.6651737755e-01 Delta Cost:-8.1860671682e-05
 OldCost: 7.7828995609e-01 NewCost: 7.7735482525e-01 Delta Cost:-9.3513083900e-04
 OldCost: 7.2717974404e-01 NewCost: 7.2715201115e-01 Delta Cost:-2.7732880747e-05
 OldCost: 6.9400639873e-01 NewCost: 6.9257183689e-01 Delta Cost:-1.4345618444e-03
 OldCost: 7.0598901869e-01 NewCost: 7.0592576381e-01 Delta Cost:-6.3254886314e-05
\end{shade}

Blocked averages of energy data, including the kinetic energy and components of the potential energy, are written to \ishell{scalar.dat} files.  The first is named ``\ishell{O.q0.opt.s000.scalar.dat},'' with a series number of zero (\ishell{s000}).  In the end there will be \ishell{MAX} of them, one for each series. 

When the job has finished, use the \ishell{qmca} tool to assess the effectiveness of the optimization process.  To look at just the total energy and the variance, type ``\ishell{qmca -q ev O.q0.opt*scalar*}.''  This will print the energy, variance, and the variance/energy ratio in Hartree units:
\begin{shade}
                            LocalEnergy               Variance           ratio
O.q0.opt  series 0  -15.739585 +/- 0.007656   0.887412 +/- 0.010728   0.0564
O.q0.opt  series 1  -15.848347 +/- 0.004089   0.318490 +/- 0.006404   0.0201
O.q0.opt  series 2  -15.867494 +/- 0.004831   0.292309 +/- 0.007786   0.0184
O.q0.opt  series 3  -15.871508 +/- 0.003025   0.275364 +/- 0.006045   0.0173
O.q0.opt  series 4  -15.865512 +/- 0.002997   0.278056 +/- 0.006523   0.0175
O.q0.opt  series 5  -15.864967 +/- 0.002733   0.278065 +/- 0.004413   0.0175
O.q0.opt  series 6  -15.869644 +/- 0.002949   0.273497 +/- 0.006141   0.0172
O.q0.opt  series 7  -15.868397 +/- 0.003838   0.285451 +/- 0.007570   0.0180
...
\end{shade}
\noindent
Plots of the data can also be obtained with the ``\ishell{-p}'' option (``\ishell{qmca -p -q ev O.q0.opt*scalar*}'').

Identify which optimization series is the ``best'' according to your cost function.  It is likely that multiple series are similar in quality.  Note the \ishell{opt.xml} file corresponding to this series.  This file contains the final value of the optimized Jastrow parameters to be used in the DMC calculations of the next section of the lab.  

\vspace{1cm}
\begin{flushleft}
\textbf{\underline{Questions and Exercises}}
\end{flushleft}
\begin{enumerate}
  \item{What is the acceptance ratio of your optimization runs? (use ``\ishell{qmca -q ar O.q0.opt*scalar*}'')  Do you expect the MC sampling to be efficient?}
  \item{How do you know when the optimization process has converged?}
%  \item{Why is the mean and the error of the variance sometimes large?  Consider using \newline``\ishell{qmca -t -q ev O.q0.opt*scalar*}'' to investigate.}
  \item{(optional) Optimization is sometimes sensitive to initial guesses of the parameters.  If you have time, try varying the initial parameters, including the cutoff radius (\ishell{rcut}) of the Jastrow factors (remember to change \ishell{id} in the \ishell{<project/>} element).  Do you arrive at a similar set of final Jastrow parameters?  What is the lowest variance you are able to achieve?}
\end{enumerate}



\section{DMC timestep extrapolation I: neutral oxygen atom}
The DMC algorithm contains two biases in addition to the fixed node and pseudopotential approximations that are important to control: time step and population control bias.  In this section we focus on estimating and removing time step bias from DMC calculations.  The essential fact to remember is that the bias vanishes as the time step goes to zero, while the needed computer time increases inversely with the time step.   


% background on timestep error should be covered elsewhere in the manual
%   perhaps replace this with a brief formula of error (order tau^2) on total energy
\hide{

The following subsection briefly discusses the origin of time step and population control biases in DMC and how they can be minimized or extrapolated away.  As before, the second subsection contains the lab walkthrough with QMCPACK.  By the end of the section, we will have a solid DMC estimate of the ground state energy of oxygen.

\subsubsection{Background on time step and population control bias}\label{sec:opt_background}
DMC improves over the VMC algorithm by projecting toward the true many-body electronic ground state of the system.  The projection operator is the (importance sampled) imaginary time propagator, which is also known as the thermodynamic density matrix:
\begin{align}
  \hat{\rho} = e^{-t\hat{H}}
\end{align}
The direct action of the projection operator on a trial wavefunction in position space
\begin{align}
  \expval{R}{e^{-t\hat{H}}}{\Psi_T} = \int dR' \rho(R,R';t)\Psi_T(R')
\end{align}
cannot be calculated in a straightforward fashion since the analytic form of $\rho(R,R';t)=\expval{R}{\rho}{R'}$ is unknown.  To make the algorithm computationally tractable, the finite-time projection operator is expanded as a product of short-time projection operators
\begin{align}
  \expval{R}{e^{-t{H}}}{\Psi_T} &= \expval{R}{e^{-\tau\hat{H}}e^{-\tau\hat{H}}\cdots e^{-\tau\hat{H}}}{\Psi_T}\\
                                 &=\int dR_1dR_2\cdots dR_M \rho(R,R_1;\tau)\rho(R_1,R_2;\tau)\cdots\rho(R_{M-1},R_M;\tau)\Psi_T(R_M)
\end{align}
The advantage here is that reasonable approximations of the short-time propagators are known.  Common approximations have the form
\begin{align}
  \rho(R,R';\tau) = e^{D(R,R';\tau)}e^{B(R,R';\tau)} + \mathcal{O}(\tau^2)
\end{align} 
where $D(R,R';\tau)$ and $B(R,R';\tau)$ represent drift and branching terms, respectively.  DMC results are biased for any finite time step ($\tau$).  The bias can be eliminated by extrapolating to zero time step.  In practice this is done by performing a series of runs with decreasing time steps and then fitting the results.

The drift term can be sampled with standard MC methods, while the branching term is incorporated as a weight assigned to each random walker.  Instead of accumulating the weight, it is more efficient to ``branch'' each walker according to the weight, resulting in some walkers being deleted and others copied multiple times.  If left uncontrolled, the walker population $(P)$ may vanish or diverge.  A stable algorithm is obtained by adjusting the branching weight to preserve the overall number of walkers on average.  Population control also biases the results, but usually to a lesser extent than time step error (the bias is proportional to $1/P$).  A common rule of thumb is to use at least a couple thousand walkers.  This bias should be checked occasionally by performing runs with varying numbers of walkers.
}


In the same directory you used to perform wavefunction optimization (\ishell{oxygen_atom}) you will find a sample DMC input file for the neutral oxygen atom named \ishell{O.q0.dmc.in.xml}.  Open this file in a text editor and note the differences from the optimization case.  Wavefunction information is no longer included from \ishell{pw2qmcpack} but instead should come from the optimization run:
\begin{lstlisting}[style=QMCPXML]
<!-- OPT_XML is from optimization, e.g. O.q0.opt.s008.opt.xml -->
<include href="OPT_XML"/>
\end{lstlisting}
\noindent
Replace ``\ishell{OPT_XML}'' with the \ishell{opt.xml} file corresponding to the best Jastrow parameters you found in the last section (this is a file name similar to \ishell{O.q0.opt.s008.opt.xml}).  

The QMC calculation section at the bottom is also different.  The linear optimization blocks have been replaced with XML describing a VMC run followed by DMC.  Descriptions of the input keywords follow.

\begin{description}
  \item[timestep] Time step of the VMC/DMC random walk.  In VMC choose a time step corresponding to an acceptance ratio of about 50\%.  In DMC the acceptance ratio is often above 99\%.
  \item[warmupSteps]  Number of MC steps discarded as a warmup or equilibration period of the random walk.  
  \item[steps] Number of MC steps per block.  Physical quantities, such as the total energy, are averaged over walkers and steps.
  \item[blocks]  Number of blocks.  This is also the number of average energy values written to output files.  The number should be greater than 200 for meaningful statistical analysis of output data (e.g., via \ishell{qmca}).  The total number of MC steps each walker takes is \ishell{blocks}$\times$\ishell{steps}.
  \item[samples] VMC only. This is the number of walkers used in subsequent DMC runs.  Each DMC walker is initialized with electron positions sampled from the VMC random walk.
  \item[nonlocalmoves] DMC only.  If yes/no, use the locality approximation/T-moves for nonlocal pseudopotentials.  T-moves generally improve the stability of the algorithm and restore the variational principle for small systems (T-moves version 1).
\end{description}

The purpose of the VMC run is to provide initial electron positions for each DMC walker.  Setting $texttt{walkers}]=1$ in the VMC block ensures there will be only one VMC walker per execution thread.  There will be a total of \labsw{4}{512} VMC walkers in this case (see \ishell{O.q0.dmc.qsub.in}).  We want the electron positions used to initialize the DMC walkers to be decorrelated from one another.  A VMC walker will often decorrelate from its current position after propagating for a few Ha$^{-1}$ in imaginary time (in general, this is system dependent).  This leads to a rough rule of thumb for choosing \ishell{blocks} and \ishell{steps} for the VMC run (\labsw{$\texttt{VWALKERS}=4$}{$\texttt{VWALKERS}=512$} here):
\begin{align}
  \texttt{VBLOCKS}\times\texttt{VSTEPS} \ge \frac{\texttt{DWALKERS}}{\texttt{VWALKERS}} \frac{5~\textrm{Ha}^{-1}}{\texttt{VTIMESTEP}}
\end{align}
Fill in the VMC XML block with appropriate values for these parameters.  There should be more than one DMC walker per thread and enough walkers in total to avoid population control bias.  The general rule of thumb is to have more than $\sim 2,000$ walkers, although the dependence of the total energy on population size should be explicitly checked from time to time.

To study time step bias, we will perform a sequence of DMC runs over a range of time steps ($0.1$ Ha$^{-1}$ is too large, and time steps below $0.002$ Ha$^{-1}$ are probably too small).  A common approach is to select a fairly large time step to begin with and then decrease the time step by a factor of two in each subsequent DMC run.  The total amount of imaginary time the walker population propagates should be the same for each run.  A simple way to accomplish this is to choose input parameters in the following way
\begin{align}\label{eq:timestep_iter}
  \texttt{timestep}_{n}    &= \texttt{timestep}_{n-1}/2\nonumber\\
  \texttt{warmupSteps}_{n} &= \texttt{warmupSteps}_{n-1}\times 2\nonumber\\
  \texttt{blocks}_{n}      &= \texttt{blocks}_{n-1}\nonumber\\
  \texttt{steps}_{n}       &= \texttt{steps}_{n-1}\times 2
\end{align}
Each DMC run will require about twice as much computer time as the one preceding it.  Note that the number of blocks is kept fixed for uniform statistical analysis.  $\texttt{blocks}\times\texttt{steps}\times\texttt{timestep}\sim 60~\mathrm{Ha}^{-1}$ is sufficient for this system.

Choose an initial DMC time step and create a sequence of $N$ time steps according to~\ref{eq:timestep_iter}.  Make $N$ copies of the DMC XML block in the input file.
\begin{lstlisting}[style=QMCPXML]
   <qmc method="dmc" move="pbyp">
      <parameter name="warmupSteps"         >    DWARMUP         </parameter>
      <parameter name="blocks"              >    DBLOCKS         </parameter>
      <parameter name="steps"               >    DSTEPS          </parameter>
      <parameter name="timestep"            >    DTIMESTEP       </parameter>
      <parameter name="nonlocalmoves"       >    yes             </parameter>
   </qmc>
\end{lstlisting}
\noindent
Fill in \ishell{DWARMUP}, \ishell{DBLOCKS}, \ishell{DSTEPS}, and \ishell{DTIMESTEP} for each DMC run according to~\ref{eq:timestep_iter}.  Start the DMC time step extrapolation run by typing:  
\ifws
\begin{shade}
mpirun -np 4 qmcpack O.q0.dmc.in.xml >&O.q0.dmc.out&
\end{shade}
\else
\begin{shade}
jobrun_vesta qmcpack O.q0.dmc.in.xml
\end{shade}
\fi
\noindent
The run should take only a few minutes to complete.

QMCPACK will create files prefixed with \ishell{O.q0.dmc}.  The log file is \labsw{\ishell{O.q0.dmc.out}}{\ishell{O.q0.dmc.output}}.  As before, block-averaged data is written to \ishell{scalar.dat} files.  In addition, DMC runs produce \ishell{dmc.dat} files, which contain energy data averaged only over the walker population (one line per DMC step).  The \ishell{dmc.dat} files also provide a record of the walker population at each step.

Use the \ishell{PlotTstepConv.pl} to obtain a linear fit to the time step data (type ``\ishell{PlotTstepConv.pl O.q0.dmc.in.xml 40}'').  You should see a plot similar to Figure~\ref{fig:timestep_conv}.  The tail end of the text output displays the parameters for the linear fit.  The ``\ishell{a}'' parameter is the total energy extrapolated to zero time step in Hartree units. 

\begin{shade}
...
Final set of parameters            Asymptotic Standard Error
=======================            ==========================

a               = -15.8925         +/- 0.0007442    (0.004683%)
b               = -0.0457479       +/- 0.0422       (92.24%)
...
\end{shade}

\begin{figure}
\begin{center}
\ifdefined\HCode
\includegraphics[trim = 0mm 0mm 0mm 0mm, clip,width=0.75\columnwidth]{./figures/lab_qmc_basics_timestep_conv.dmn}
\else
\includegraphics[trim = 0mm 0mm 0mm 0mm, clip,width=0.75\columnwidth]{./figures/lab_qmc_basics_timestep_conv.pdf}
\fi
\end{center}
\caption{Linear fit to DMC timestep data from \ishell{PlotTstepConv.pl}.}
\label{fig:timestep_conv}
\end{figure}


\vspace{1cm}
\begin{flushleft}
\textbf{\underline{Questions and Exercises}}
\end{flushleft}
\begin{enumerate}
  \item{What is the $\tau\rightarrow 0$ extrapolated value for the total energy?}
  \item{What is the maximum time step you should use if you want to calculate the total energy to an accuracy of $0.05$ eV?  For convenience, $1~\textrm{Ha}=27.2113846~\textrm{eV}$.}
  \item{What is the acceptance ratio for this (bias $<0.05$ eV) run?  Does it follow the rule of thumb for sensible DMC (acceptance ratio $>99$\%) ?}
  \item{Check the fluctuations in the walker population (\ishell{qmca -t -q nw O.q0.dmc*dmc.dat --noac}).  Does the population seem to be stable?}
  \item{(Optional) Study population control bias for the oxygen atom.  Select a few population sizes. \labsw{}{(use multiples of 512 to fit cleanly on a single Vesta partition)}  Copy \ishell{O.q0.dmc.in.xml} to a new file and remove all but one DMC run (select a single time step).  Make one copy of the new file for each population, set ``\ishell{samples},'' and choose a unique \ishell{id} in \ishell{<project/>}.  \labsw{}{Run one job at a time to avoid crowding the lab allocation.}  Use \ishell{qmca} to study the dependence of the DMC total energy on the walker population.  How large is the bias compared with time step error?  What bias is incurred by following the ``rule of thumb'' of a couple thousand walkers?  Will population control bias generally be an issue for production runs on modern parallel machines?}
\end{enumerate}


\section{DMC time step extrapolation II: oxygen atom ionization potential}
In this section, we will repeat the calculations of the previous two sections (optimization, time step extrapolation) for the $+1$ charge state of the oxygen atom.  Comparing the resulting first ionization potential (IP) with experimental data will complete our first test of the BFD oxygen pseudopotential.  In actual practice, higher IPs could also be tested before performing production runs.

Obtaining the time step extrapolated DMC total energy for ionized oxygen should take much less (human) time than for the neutral case.  For convenience, the necessary steps are summarized as follows.
\begin{enumerate}
  \item{Obtain DFT orbitals with QE.}
  \begin{enumerate}
    \item{Copy the DFT input (\ishell{O.q0.dft.in}) to \ishell{O.q1.dft.in}}
    \item{Edit \ishell{O.q1.dft.in} to match the +1 charge state of the oxygen atom.}
    \begin{lstlisting}[style=espresso]
     ...
     prefix            = 'O.q1'
     ...
     tot_charge        = 1
     tot_magnetization = 3
     ...
    \end{lstlisting}
  \item{Perform the DFT run: \ifws\ishell{mpirun -np 4 pw.x -input O.q1.dft.in >&O.q1.dft.out&}
      \else\ishell{jobrun_vesta pw.x O.q1.dft.in}\fi}
  \end{enumerate}

  \item{Convert the orbitals to ESHDF format.}
  \begin{enumerate}
    \item{Copy the pw2qmcpack input (\ishell{O.q0.p2q.in}) to \ishell{O.q1.p2q.in}}
    \item{Edit \ishell{O.q1.p2q.in} to match the file prefix used in DFT.}
    \begin{verbatim}
     ...
     prefix = 'O.q1'
     ...
    \end{verbatim}
    \item{Perform the orbital conversion run: \ifws\ishell{mpirun -np 1 pw2qmcpack.x<O.q1.p2q.in>&O.q1.p2q.out&}\else\ishell{jobrun_vesta pw2qmcpack.x O.q1.p2q.in}\fi}
  \end{enumerate}

  \item{Optimize the Jastrow factor with QMCPACK.}
  \begin{enumerate}
    \item{Copy the optimization input (\ishell{O.q0.opt.in.xml}) to \ishell{O.q1.opt.in.xml}}
    \item{Edit \ishell{O.q1.opt.in.xml} to match the file prefix used in DFT.}
    \begin{lstlisting}[style=QMCPXML]
     ...
     <project id="O.q1.opt" series="0">
     ...
     <include href="O.q1.ptcl.xml"/>
     ...
     <include href="O.q1.wfs.xml"/>
     ...
    \end{lstlisting}
    \item{Edit the particle XML file (\ishell{O.q1.ptcl.xml}) to have open boundary conditions.}
    \begin{lstlisting}[style=QMCPXML]
      <parameter name="bconds">
        n n n 
      </parameter>
    \end{lstlisting}
    \item{Add cutoffs to the Jastrow factors in the wavefunction XML file (\ishell{O.q1.wfs.xml}).}
    \begin{lstlisting}[style=QMCPXML]
      ...
      <correlation speciesA="u" speciesB="u" size="8" rcut="10.0">
      ...
      <correlation speciesA="u" speciesB="d" size="8" rcut="10.0">
      ...
      <correlation elementType="O" size="8" rcut="5.0">
      ...
    \end{lstlisting}
    \item{Perform the Jastrow optimization run: \ifws\ishell{mpirun -np 4 qmcpack O.q1.opt.in.xml >&O.q1.opt.out&}\else\ishell{jobrun_vesta qmcpack O.q1.opt.in.xml}\fi}
    \item{Identify the optimal set of parameters with \ishell{qmca} (\ishell{[your opt.xml]}).}
  \end{enumerate}

  \item{DMC time step study with QMCPACK}
  \begin{enumerate}
    \item{Copy the DMC input (\ishell{O.q0.dmc.in.xml}) to \ishell{O.q1.dmc.in.xml}}
    \item{Edit \ishell{O.q1.dmc.in.xml} to use the DFT prefix and the optimal Jastrow.}
    \begin{lstlisting}[style=QMCPXML]
     ...
     <project id="O.q1.dmc" series="0">
     ...
     <include href="O.q1.ptcl.xml"/>
     ...
     <include href="[your opt.xml]"/>
     ...
    \end{lstlisting}
    \item{Perform the DMC run: \ifws\ishell{mpirun -np 4 qmcpack O.q1.dmc.in.xml >&O.q1.dmc.out&}\else\ishell{jobrun_vesta qmcpack O.q1.dmc.in.xml}\fi}
    \item{Obtain the DMC total energy extrapolated to zero time step with \ishell{PlotTstepConv.pl}.}
  \end{enumerate}
\end{enumerate}
The aforementioned process, which excludes additional steps for orbital generation and conversion, can become tedious to perform by hand in production settings where many calculations are often required.  For this reason, automation tools are introduced for calculations involving the oxygen dimer in Section~\ref{sec:dimer_automation} of the lab.  

\vspace{1cm}
\begin{flushleft}
\textbf{\underline{Questions and Exercises}}
\end{flushleft}
\begin{enumerate}
  \item{What is the $\tau\rightarrow 0$ extrapolated DMC value for the first ionization potential of oxygen?}
  \item{How does the extrapolated value compare with the experimental IP?  Go to\newline \href{http://physics.nist.gov/PhysRefData/ASD/ionEnergy.html}{http://physics.nist.gov/PhysRefData/ASD/ionEnergy.html} and enter ``\ishell{O I}'' in the box labeled ``\ishell{Spectra}'' and click on the ``\ishell{Retrieve Data}'' button.  
%For comparison the LDA value is $12.25$ eV.
}
  \item{What can we conclude about the accuracy of the pseudopotential?  What factors complicate this assessment?}
  \item{Explore the sensitivity of the IP to the choice of time step.  Type ``\ishell{./ip_conv.py}'' to view three time step extrapolation plots: two for the $q=0,$ one for total energies, and one for the IP.  Is the IP more, less, or similarly sensitive to time step than the total energy?}
  \item{What is the maximum time step you should use if you want to calculate the ionization potential to an accuracy of $0.05$ eV?  What factor of CPU time is saved by assessing time step convergence on the IP (a total energy difference) vs. a single total energy?}
  \item{Are the acceptance ratio and population fluctuations reasonable for the $q=1$ calculations?}
\end{enumerate}




\section{DMC workflow automation with Nexus}
Production QMC projects are often composed of many similar workflows.  The simplest of these is a single DMC calculation involving four different compute jobs:
\begin{enumerate}
  \item{Orbital generation via QE or GAMESS.}
  \item{Conversion of orbital data via \ishell{pw2qmcpack.x} or \ishell{convert4qmc}.}
  \item{Optimization of Jastrow factors via QMCPACK.}
  \item{DMC calculation via QMCPACK.}
\end{enumerate}
Simulation workflows quickly become more complex with increasing costs in terms of human time for the researcher.  Automation tools can decrease both human time and error if used well.

The set of automation tools we will be using is known as Nexus \cite{Krogel2016nexus}, which is distributed with QMCPACK.  Nexus is capable of generating input files, submitting and monitoring compute jobs, passing data between simulations (relaxed structures, orbital files, optimized Jastrow parameters, etc.), and data analysis.  The user interface to Nexus is through a set of functions defined in the Python programming language.  User scripts that execute simple workflows resemble input files and do not require programming experience.  More complex workflows require only basic programming constructs (e.g. for loops and if statements).  Nexus input files/scripts should be easier to navigate than QMCPACK input files and more efficient than submitting all the jobs by hand.

Nexus is driven by simple user-defined scripts that resemble keyword-driven input files.  An example Nexus input file that performs a single VMC calculation (with pregenerated orbitals) follows.  Take a moment to read it over and especially note the comments (prefixed with ``\ishell{\#}'') explaining most of the contents.  If the input syntax is unclear you may want to consult portions of Appendix~\ref{app:python_basics}, which gives a condensed summary of Python constructs.  An additional example and details about the inner workings of Nexus can be found in the reference publication \cite{Krogel2016nexus}. 

%For more information about the functionality and effective use of Nexus, consult \ishell{docs/Nexus.pdf}.  

%More information can be found in the user guide distributed with QMCPACK, although examples in this lab series and \ishell{Nexus.pdf} are more up to date (if \ishell{qmcpack} is the location of your QMCPACK distribution, the user guide can be found at \ishell{qmcpack/nexus/documentation/nexus_user_guide.pdf}).

\ifws
\begin{lstlisting}[style=Python]
#! /usr/bin/env python3

# import Nexus functions
from nexus import settings,job,get_machine,run_project 
from nexus import generate_physical_system
from nexus import generate_qmcpack,vmc

settings(                             # Nexus settings
    pseudo_dir    = './pseudopotentials', # location of PP files
    runs          = '',                   # root directory for simulations
    results       = '',                   # root directory for simulation results
    status_only   = 0,                    # show simulation status, then exit
    generate_only = 0,                    # generate input files, then exit
    sleep         = 3,                    # seconds between checks on sim. progress
    machine       = 'ws4',                # workstation with 4 cores
    ) 

qmcjob = job(                         # specify job parameters
    cores   = 4,                          # use 4 MPI tasks
    threads = 1,                          # 1 OpenMP thread per node
    app     = 'qmcpack'                   # use QMCPACK executable (assumed in PATH)
    )

qmc_calcs = [                         # list QMC calculation methods
    vmc(                                  #   VMC
        walkers     =   1,                #     1 walker
        warmupsteps =  50,                #    50 MC steps for warmup
        blocks      = 200,                #   200 blocks
        steps       =  10,                #    10 steps per block
        timestep    =  .4                 #   0.4 1/Ha timestep
        )]

dimer = generate_physical_system(     # make a dimer system
    type       = 'dimer',                 # system type is dimer
    dimer      = ('O','O'),               # dimer is two oxygen atoms
    separation = 1.2074,                  # separated by 1.2074 Angstrom
    Lbox       = 15.0,                    # simulation box is 15 Angstrom 
    units      = 'A',                     # Angstrom is dist. unit
    net_spin   = 2,                       # nup-ndown is 2
    O          = 6                        # pseudo-oxygen has 6 valence el.
    )

qmc = generate_qmcpack(                # make a qmcpack simulation 
    identifier   = 'example',             # prefix files with 'example'
    path         = 'scale_1.0',           # run in ./scale_1.0 directory
    system       = dimer,                 # run the dimer system
    job          = qmcjob,                # set job parameters
    input_type   = 'basic',               # basic qmcpack inputs given below    
    pseudos      = ['O.BFD.xml'],         # list of PP's to use
    orbitals_h5  = 'O2.pwscf.h5',         # file with orbitals from DFT
    bconds       = 'nnn',                 # open boundary conditions
    jastrows     = [],                    # no jastrow factors
    calculations = qmc_calcs              # QMC calculations to perform
    )
                       
run_project(qmc)                       # write input file and submit job
\end{lstlisting}

\else

\begin{lstlisting}[style=Python]
#! /usr/bin/env python3

# import Nexus functions
from nexus import settings,job,get_machine,run_project 
from nexus import generate_physical_system
from nexus import generate_qmcpack,vmc

settings(                             # Nexus settings
    pseudo_dir    = './pseudopotentials', # location of PP files
    runs          = '',                   # root directory for simulations
    results       = '',                   # root directory for simulation results
    status_only   = 0,                    # show simulation status, then exit
    generate_only = 0,                    # generate input files, then exit
    sleep         = 3,                    # seconds between checks on sim. progress
    machine       = 'vesta',              # name of local machine
    account       = 'QMCPACK-Training'    # charge account for cpu time
    ) 

vesta = get_machine('vesta')          # allow max of one job at a time (lab only)
vesta.queue_size = 1

qmcjob = job(                         # specify job parameters
    nodes   = 32,                         # use 32 Vesta nodes
    threads = 16,                         # 16 OpenMP threads per node (32 MPI tasks)
    hours   = 1,                          # wallclock limit of 1 hour
                                          # use QMCPACK executable
    app     = '/soft/applications/qmcpack/Binaries/qmcpack'
    )

qmc_calcs = [                         # list QMC calculation methods
    vmc(                                  #   VMC
        walkers     =   1,                #     1 walker
        warmupsteps =  50,                #    50 MC steps for warmup
        blocks      = 200,                #   200 blocks
        steps       =  10,                #    10 steps per block
        timestep    =  .4                 #   0.4 1/Ha timestep
        )]

dimer = generate_physical_system(     # make a dimer system
    type       = 'dimer',                 # system type is dimer
    dimer      = ('O','O'),               # dimer is two oxygen atoms
    separation = 1.2074,                  # separated by 1.2074 Angstrom
    Lbox       = 15.0,                    # simulation box is 15 Angstrom 
    units      = 'A',                     # Angstrom is dist. unit
    net_spin   = 2,                       # nup-ndown is 2
    O          = 6                        # pseudo-oxygen has 6 valence el.
    )

qmc = generate_qmcpack(                # make a qmcpack simulation 
    identifier   = 'example',             # prefix files with 'example'
    path         = 'scale_1.0',           # run in ./scale_1.0 directory
    system       = dimer,                 # run the dimer system
    job          = qmcjob,                # set job parameters
    input_type   = 'basic',               # basic qmcpack inputs given below    
    pseudos      = ['O.BFD.xml'],         # list of PP's to use
    orbitals_h5  = 'O2.pwscf.h5',         # file with orbitals from DFT
    bconds       = 'nnn',                 # open boundary conditions
    jastrows     = [],                    # no jastrow factors
    calculations = qmc_calcs              # QMC calculations to perform
    )
                       
run_project(qmc)                       # write input file and submit job
\end{lstlisting}
\fi





\section{Automated binding curve of the oxygen dimer}
\label{sec:dimer_automation}
In this section we will use Nexus to calculate the DMC total energy of the oxygen dimer over a series of bond lengths.  The equilibrium bond length and binding energy of the dimer will be determined by performing a polynomial fit to the data (Morse potential fits should be preferred in production tests).  Comparing these values with corresponding experimental data provides a second test of the BFD pseudopotential for oxygen.

Enter the \ishell{oxygen_dimer} directory.  Copy your BFD pseudopotential from the atom runs into \ishell{oxygen_dimer/pseudopotentials} (be sure to move both files: \ishell{.upf} and \ishell{.xml}).  Open \ishell{O_dimer.py} with a text editor.  The overall format is similar to the example file shown in the last section.  
%The header material, including Nexus imports, settings, and the job parameters for QMC are identical.  
The main difference is that a full workflow of runs (DFT orbital generation, orbital conversion, optimization and DMC) are being performed rather than a single VMC run.  


%Following the job parameters, inputs for the optimization method are given.  The keywords should all be familiar from the QMCPACK XML input files you used previously:
%\begin{lstlisting}
%linopt1 = linear(
%    energy               = 0.0,
%    unreweightedvariance = 1.0,
%    reweightedvariance   = 0.0,
%    timestep             = 0.4,
%    samples              = 10240, 
%    warmupsteps          = 50,
%    blocks               = 200,
%    substeps             = 1,
%    nonlocalpp           = True,
%    usebuffer            = True,
%    walkers              = 1,
%    minwalkers           = 0.5,
%    maxweight            = 1e9,
%    usedrift             = True,
%    minmethod            = 'quartic',
%    beta                 = 0.025,
%    exp0                 = -16,
%    bigchange            = 15.0,
%    alloweddifference    = 1e-4,
%    stepsize             = 0.2,
%    stabilizerscale      = 1.0,
%    nstabilizers         = 3
%    )
%\end{lstlisting}
%
%%\hide{
%\noindent
%Requesting multiple loop's with different numbers of samples is more compact than in XML:
%\begin{lstlisting}
%linopt1 = ...
%
%linopt2 = linopt1.copy()  
%linopt2.samples = 61440 # opt w/ 61440 samples
%
%opt_calcs = [loop(max=4,qmc=linopt1), # loops over opt's
%             loop(max=8,qmc=linopt2)]
%\end{lstlisting}
%%}
%
%\noindent
%The VMC/DMC method inputs should also look familiar:
%\begin{lstlisting}
%qmc_calcs = [
%    vmc(
%        walkers     =   1,
%        warmupsteps =  30,
%        blocks      =  20,
%        steps       =  10,
%        substeps    =   2,
%        timestep    =  .4,
%        samples     = 2048
%        ),
%    dmc(
%        warmupsteps   = 100, 
%        blocks        = 400,
%        steps         =  32,
%        timestep      = 0.01,
%        nonlocalmoves = True
%        )
%    ]
%\end{lstlisting}


As in the example in the last section, the oxygen dimer is generated with the \ishell{generate_physical_ 
system} function:
\begin{lstlisting}[style=Python]
dimer = generate_physical_system(
    type       = 'dimer',
    dimer      = ('O','O'),
    separation = 1.2074*scale,
    Lbox       = 10.0,
    units      = 'A',
    net_spin   = 2,
    O          = 6
    )
\end{lstlisting}
\noindent
Similar syntax can be used to generate crystal structures or to specify systems with arbitrary atomic configurations and simulation cells.  Notice that a ``\ishell{scale}'' variable has been introduced to stretch or compress the dimer.  

Next, objects representing a QE (PWSCF) run and subsequent orbital conversion step are constructed with respective \ishell{generate_*} functions:
\begin{lstlisting}[style=Python]
dft = generate_pwscf(
    identifier   = 'dft',
    ...
    input_dft    = 'lda',
    ...
    )
sims.append(dft)

# describe orbital conversion run                                                                    
p2q = generate_pw2qmcpack(
    identifier   = 'p2q',
    ...
    dependencies = (dft,'orbitals'),
    )
sims.append(p2q)
\end{lstlisting}
Note the \ishell{dependencies} keyword.  This keyword is used to construct workflows out of otherwise separate runs.  In this case, the dependency indicates that the orbital conversion run must wait for the DFT to finish before starting.

Objects representing QMCPACK simulations are then constructed with the \ishell{generate_qmcpack} function:
\begin{lstlisting}[style=Python]
opt = generate_qmcpack(
    identifier   = 'opt',
    ...
    jastrows     = [('J1','bspline',8,5.0), 
                    ('J2','bspline',8,10.0)],
    calculations = [
        loop(max=12,
             qmc=linear(
                energy               = 0.0,
                unreweightedvariance = 1.0,
                reweightedvariance   = 0.0,
                timestep             = 0.3,
                samples              = 61440,
                warmupsteps          = 50,
                blocks               = 200,
                substeps             = 1,
                nonlocalpp           = True,
                usebuffer            = True,
                walkers              = 1,
                minwalkers           = 0.5,
                maxweight            = 1e9,
                usedrift             = False,
                minmethod            = 'quartic',
                beta                 = 0.025,
                exp0                 = -16,
                bigchange            = 15.0,
                alloweddifference    = 1e-4,
                stepsize             = 0.2,
                stabilizerscale      = 1.0,
                nstabilizers         = 3,
                )
             )
        ],
    dependencies = (p2q,'orbitals'),
    )
sims.append(opt)

qmc = generate_qmcpack(
    identifier   = 'qmc',
    ...
    jastrows     = [],            
    calculations = [
        vmc(
            walkers     =   1,
            warmupsteps =  30,
            blocks      =  20,
            steps       =  10,
            substeps    =   2,
            timestep    =  .4,
            samples     = 2048
            ),
        dmc(
            warmupsteps   = 100,
            blocks        = 400,
            steps         =  32,
            timestep      = 0.01,
            nonlocalmoves = True,
            )
        ],
    dependencies = [(p2q,'orbitals'),(opt,'jastrow')],
    )
sims.append(qmc)
\end{lstlisting}
\noindent
Shared details such as the run directory, job, pseudopotentials, and orbital file have been omitted (\ishell{...}).  The ``\ishell{opt}'' run will optimize a 1-body B-spline Jastrow with 8 knots having a cutoff of 5.0 Bohr and a B-spline Jastrow (for up-up and up-down correlations) with 8 knots and cutoffs of 10.0 Bohr.  The Jastrow list for the DMC run is empty, and the previous use of \ishell{dependencies} indicates that the DMC run depends on the optimization run for the Jastrow factor.  Nexus will submit the ``\ishell{opt}'' run first, and upon completion it will scan the output, select the optimal set of parameters, pass the Jastrow information to the ``\ishell{qmc}'' run, and then submit the DMC job.  Independent job workflows are submitted in parallel when permitted. \labsw{}{(we have explicitly limited this for this lab by setting \ishell{queue_size=2} for Vesta)}  No input files are written or job submissions made until the ``\ishell{run_project}'' function is reached:
\begin{lstlisting}[style=Python]
run_project(sims)
\end{lstlisting}
\noindent
All of the simulation objects have been collected into a list (\ishell{sims}) for submission.

As written, \ishell{O_dimer.py} will perform calculations only at the equilibrium separation distance of 1.2074 {\AA} since the list of scaling factors (representing stretching or compressing the dimer)  contains only one value (\ishell{scales = [1.00]}).  Modify the file now to perform DMC calculations across a range of separation distances with each DMC run using the Jastrow factor optimized at the equilibrium separation distance.  Specifically, you will want to change the list of scaling factors to include both compression (\ishell{scale<1.0}) and stretch (\ishell{scale>1.0}): 
\begin{lstlisting}[style=Python]
scales = [1.00,0.90,0.95,1.05,1.10]
\end{lstlisting}
\noindent
Note that ``\ishell{1.00}'' is left in front because we are going to optimize the Jastrow factor first at the equilibrium separation and reuse this Jastrow factor for all other separation distances.  This procedure is used because it can reduce variations in localization errors (due to pseudopotentials in DMC) along the binding curve. 

Change the ``\ishell{status_only}'' parameter in the ``\ishell{settings}'' function to \ishell{1} and type ``\ishell{./O_dimer.py}'' at the command line.  This will print the status of all simulations:
\begin{shade}
Project starting 
  checking for file collisions 
  loading cascade images 
    cascade 0 checking in 
    cascade 10 checking in 
    cascade 4 checking in 
    cascade 13 checking in 
    cascade 7 checking in 
  checking cascade dependencies 
    all simulation dependencies satisfied 
  
  cascade status 
    setup, sent_files, submitted, finished, got_output, analyzed 
    000000  dft     ./scale_1.0 
    000000  p2q     ./scale_1.0 
    000000  opt     ./scale_1.0 
    000000  qmc     ./scale_1.0 
    000000  dft     ./scale_0.9 
    000000  p2q     ./scale_0.9 
    000000  qmc     ./scale_0.9 
    000000  dft     ./scale_0.95 
    000000  p2q     ./scale_0.95 
    000000  qmc     ./scale_0.95 
    000000  dft     ./scale_1.05 
    000000  p2q     ./scale_1.05 
    000000  qmc     ./scale_1.05 
    000000  dft     ./scale_1.1 
    000000  p2q     ./scale_1.1 
    000000  qmc     ./scale_1.1 
    setup, sent_files, submitted, finished, got_output, analyzed 
\end{shade}
\noindent
In this case, five simulation ``cascades'' (workflows) have been identified, each one starting and ending with ``\ishell{dft}'' and ``\ishell{qmc}'' runs, respectively.  The six status flags (\ishell{setup}, \ishell{sent_files}, \ishell{submitted}, \ishell{finished}, \ishell{got_output}, \ishell{analyzed}) each show \ishell{0}, indicating that no work has been done yet.  

Now change ``\ishell{status_only}'' back to \ishell{0}, set ``\ishell{generate_only}'' to \ishell{1}, and run \ishell{O_dimer.py} again.  This will perform a dry run of all simulations.  The dry run should finish in about 20 seconds:
\ifws
\begin{shade}
Project starting 
  checking for file collisions 
  loading cascade images 
    cascade 0 checking in 
    cascade 10 checking in 
    cascade 4 checking in 
    cascade 13 checking in 
    cascade 7 checking in 
  checking cascade dependencies 
    all simulation dependencies satisfied 
  
  starting runs:
  ~~~~~~~~~~~~~~~~~~~~~~~~~~~~~~ 
  poll 0  memory 91.03 MB 
    Entering ./scale_1.0 0 
      writing input files  0 dft 
    Entering ./scale_1.0 0 
      sending required files  0 dft 
      submitting job  0 dft 
  ...
  poll 1  memory 91.10 MB 
  ...
    Entering ./scale_1.0 0 
      Would have executed:  
        export OMP_NUM_THREADS=1
        mpirun -np 4 pw.x -input dft.in 

  poll 2  memory 91.10 MB 
    Entering ./scale_1.0 0 
      copying results  0 dft 
    Entering ./scale_1.0 0 
      analyzing  0 dft 
  ...
  poll 3  memory 91.10 MB 
    Entering ./scale_1.0 1 
      writing input files  1 p2q 
    Entering ./scale_1.0 1 
      sending required files  1 p2q 
      submitting job  1 p2q 
  ...
    Entering ./scale_1.0 1 
      Would have executed:  
        export OMP_NUM_THREADS=1
        mpirun -np 1 pw2qmcpack.x<p2q.in 

  poll 4  memory 91.10 MB 
    Entering ./scale_1.0 1 
      copying results  1 p2q 
    Entering ./scale_1.0 1 
      analyzing  1 p2q 
  ...
  poll 5  memory 91.10 MB 
    Entering ./scale_1.0 2 
      writing input files  2 opt 
    Entering ./scale_1.0 2 
      sending required files  2 opt 
      submitting job  2 opt 
  ...
    Entering ./scale_1.0 2 
      Would have executed:  
        export OMP_NUM_THREADS=1
        mpirun -np 4 qmcpack opt.in.xml 

  poll 6  memory 91.16 MB 
    Entering ./scale_1.0 2 
      copying results  2 opt 
    Entering ./scale_1.0 2 
      analyzing  2 opt 
  ...
  poll 7  memory 93.00 MB 
    Entering ./scale_1.0 3 
      writing input files  3 qmc 
    Entering ./scale_1.0 3 
      sending required files  3 qmc 
      submitting job  3 qmc 
  ...
    Entering ./scale_1.0 3 
      Would have executed:  
        export OMP_NUM_THREADS=1
        mpirun -np 4 qmcpack qmc.in.xml 
  ...
  poll 17  memory 93.06 MB 
Project finished
\end{shade}
\else
\begin{shade}
Project starting 
  checking for file collisions 
  loading cascade images 
    cascade 0 checking in 
    cascade 10 checking in 
    cascade 4 checking in 
    cascade 13 checking in 
    cascade 7 checking in 
  checking cascade dependencies 
    all simulation dependencies satisfied 
  
  starting runs:
  ~~~~~~~~~~~~~~~~~~~~~~~~~~~~~~ 
  poll 0  memory 91.03 MB 
    Entering ./scale_1.0 0 
      writing input files  0 dft 
    Entering ./scale_1.0 0 
      sending required files  0 dft 
      submitting job  0 dft 
  ...
  poll 1  memory 91.10 MB 
  ...
    Entering ./scale_1.0 1 
      Would have executed:  qsub --mode script --env BG_SHAREDMEMSIZE=32 dft.qsub.in 

  poll 2  memory 91.10 MB 
    Entering ./scale_1.0 0 
      copying results  0 dft 
    Entering ./scale_1.0 0 
      analyzing  0 dft 
  ...
  poll 3  memory 91.10 MB 
    Entering ./scale_1.0 1 
      writing input files  1 p2q 
    Entering ./scale_1.0 1 
      sending required files  1 p2q 
      submitting job  1 p2q 
  ...
    Entering ./scale_1.0 2 
      Would have executed:  qsub --mode script --env BG_SHAREDMEMSIZE=32 p2q.qsub.in 

  poll 4  memory 91.10 MB 
    Entering ./scale_1.0 1 
      copying results  1 p2q 
    Entering ./scale_1.0 1 
      analyzing  1 p2q
  ... 

  poll 5  memory 91.10 MB 
    Entering ./scale_1.0 2 
      writing input files  2 opt 
    Entering ./scale_1.0 2 
      sending required files  2 opt 
      submitting job  2 opt 
  ...
    Entering ./scale_1.0 3 
      Would have executed:  qsub --mode script --env BG_SHAREDMEMSIZE=32 opt.qsub.in 

  poll 6  memory 91.16 MB 
    Entering ./scale_1.0 2 
      copying results  2 opt 
    Entering ./scale_1.0 2 
      analyzing  2 opt 
  ...
  poll 7  memory 93.00 MB 
    Entering ./scale_1.0 3 
      writing input files  3 qmc 
    Entering ./scale_1.0 3 
      sending required files  3 qmc 
      submitting job  3 qmc 
  ...
    Entering ./scale_1.0 4 
      Would have executed:  qsub --mode script --env BG_SHAREDMEMSIZE=32 qmc.qsub.in 
  ...

  poll 17  memory 93.00 MB 
Project finished
\end{shade}
\fi

\noindent
Nexus polls the simulation status every 3 seconds and sleeps in between.  The ``\texttt{scale\_*}'' directories should now contain several files:
\ifws
\begin{shade}
scale_1.0
   dft.in
   O.BFD.upf
   O.BFD.xml
   opt.in.xml
   p2q.in
   pwscf_output
   qmc.in.xml
   sim_dft/
       analyzer.p
       input.p
       sim.p
   sim_opt/
       analyzer.p
       input.p
       sim.p
   sim_p2q/
       analyzer.p
       input.p
       sim.p
   sim_qmc/
       analyzer.p
       input.p
       sim.p
\end{shade}
\else
\begin{shade}
scale_1.0
   dft.in
   dft.qsub.in
   O.BFD.upf
   O.BFD.xml
   opt.in.xml
   opt.qsub.in
   p2q.in
   p2q.qsub.in
   pwscf_output
   qmc.in.xml
   qmc.qsub.in
   sim_dft/
       analyzer.p
       input.p
       sim.p
   sim_opt/
       analyzer.p
       input.p
       sim.p
   sim_p2q/
       analyzer.p
       input.p
       sim.p
   sim_qmc/
       analyzer.p
       input.p
       sim.p
\end{shade}
\fi
\noindent
Take a minute to inspect the generated input (\ishell{dft.in}, \ishell{p2q.in}, \ishell{opt.in.xml}, \ishell{qmc.in.xml}) \labsw{}{and submission (\ishell{dft.qsub.in}, \ishell{p2q.qsub.in}, \ishell{opt.qsub.in}, \ishell{qmc.qsub.in})} files.  The pseudopotential files (\ishell{O.BFD.upf} and \ishell{O.BFD.xml}) have been copied into each local directory. Four additional directories have been created: \ishell{sim_dft},  \ishell{sim_p2q}, \ishell{sim_opt} and \ishell{sim_qmc}.  The \ishell{sim.p} files in each directory contain the current status of each simulation.  If you run \ishell{O_dimer.py} again, it should not attempt to rerun any of the simulations:   
\begin{shade}
Project starting 
  checking for file collisions 
  loading cascade images 
    cascade 0 checking in 
    cascade 10 checking in 
    cascade 4 checking in 
    cascade 13 checking in 
    cascade 7 checking in 
  checking cascade dependencies 
    all simulation dependencies satisfied 
  
  starting runs:
  ~~~~~~~~~~~~~~~~~~~~~~~~~~~~~~ 
  poll 0  memory 64.25 MB 
Project finished
\end{shade}
\noindent
This way you can continue to add to the \ishell{O_dimer.py} file (e.g., adding more separation distances) without worrying about duplicate job submissions.

Now submit the jobs in the dimer workflow.  Reset the state of the simulations by removing the \ishell{sim.p} files (``\ishell{rm ./scale*/sim*/sim.p}''), set ``\ishell{generate_only}'' to \ishell{0}, and rerun \ishell{O_dimer.py}.  It should take about 20 minutes for all the jobs to complete.  You may wish to open another terminal to monitor the progress of the individual jobs while the current terminal runs \ishell{O_dimer.py} in the foreground.  You can begin the following first exercise once the optimization job completes.

\vspace{3cm}
\begin{flushleft}
\textbf{\underline{Questions and Exercises}}
\end{flushleft}
\begin{enumerate}
  \item{Evaluate the quality of the optimization at \ishell{scale=1.0} using the \ishell{qmca} tool.  Did the optimization succeed?  How does the variance compare with the neutral oxygen atom?  Is the wavefunction of similar quality to the atomic case?}

  \item{Evaluate the traces of the local energy and the DMC walker population for each separation distance with the \ishell{qmca} tool.  Are there any anomalies in the runs?  Is the acceptance ratio reasonable?  Is the wavefunction of similar quality across all separation distances?}

  \item{Use the \ishell{dimer_fit.py} tool located in \ishell{oxygen_dimer} to fit the oxygen dimer binding curve.   To get the binding energy of the dimer, we will need the DMC energy of the atom.  Before performing the fit, answer: What DMC time step should be used for the oxygen atom results?  The tool accepts three arguments (``\ishell{./dimer_fit.py P N E Eerr}''}), \ishell{P} is the prefix of the DMC input files (should be ``\ishell{qmc}'' at this point), \ishell{N} is the order of the fit (use 2 to start), \ishell{E} and \ishell{Eerr} are your DMC total energy and error bar, respectively, for the oxygen atom (in electron volts).  A plot of the dimer data will be displayed, and text output will show the DMC equilibrium bond length and binding energy as well as experimental values.  How accurately does your fit to the DMC data reproduce the experimental values?  What factors affect the accuracy of your results? 

  \item{Refit your data with a fourth-order polynomial.  How do your predictions change with a fourth-order fit?  Is a fourth-order fit appropriate for the available data?}
 
  \item{Add new ``\ishell{scale}'' values to the list in \ishell{O_dimer.py} that interpolate between the original set (e.g., expand to \ishell{[1.00,0.90,0.925,0.95,0.975,1.025,1.05,1.075,1.10]}).  Perform the DMC calculations and redo the fits.  How accurately does your fit to the DMC data reproduce the experimental values?  Should this pseudopotential be used in production calculations?}

  \item{(Optional) Perform optimization runs at the extreme separation distances corresponding to \ishell{scale=[0.90,1.10]}}.  Are the individually optimized wavefunctions of significantly better quality than the one imported from \ishell{scale=1.00}?  Why?  What form of Jastrow factor might give an even better improvement? 
\end{enumerate}




\section{(Optional) Running your system with QMCPACK}\label{sec:your_system}
This section covers a fairly simple route to get started on QMC calculations of an arbitrary system of interest using the Nexus workflow management system to set up input files and optionally perform the runs.  The example provided in this section uses QE (PWSCF) to generate the orbitals forming the Slater determinant part of the trial wavefunction.  PWSCF is a natural choice for solid-state systems, and it can be used for surface/slab and molecular systems as well, albeit at the price of describing additional vacuum space with plane waves.

To start out, you will need PPs for each element in your system in both the UPF (PWSCF) and FSATOM/XML (QMCPACK) formats.  A good place to start is the BFD pseudopotential database \newline (\href{http://www.burkatzki.com/pseudos/index.2.html}{http://www.burkatzki.com/pseudos/index.2.html}), which we have already used in our study of the oxygen atom.  The database does not contain PPs for the fourth and fifth row transition metals or any of the lanthanides or actinides.  If you need a PP that is not in the BFD database, you may need to generate and test one manually (e.g., with OPIUM, \href{http://opium.sourceforge.net/}{http://opium.sourceforge.net/}).  Otherwise, use \ishell{ppconvert} as outlined in Section~\ref{sec:lqb_pseudo} to obtain PPs in the formats used by PWSCF and QMCPACK.  Enter the \ishell{your_system} lab directory and place the converted PPs in \ishell{your_system/pseudopotentials}.

Before performing production calculations (more than just the initial setup in this section), be sure to converge the plane-wave energy cutoff in PWSCF as these PPs can be rather hard, sometimes requiring cutoffs in excess of 300 Ry.  Depending on the system under study, the amount of memory required to represent the orbitals (QMCPACK uses 3D B-splines) can become prohibitive, forcing you to search for softer PPs.

Beyond PPs, all that is required to get started are the atomic positions and the dimensions/shape of the simulation cell.  The Nexus file \ishell{example.py} illustrates how to set up PWSCF and QMCPACK input files by providing minimal information regarding the physical system (an 8-atom cubic cell of diamond in the example).  Most of the contents should be familiar from your experience with the automated calculations of the oxygen dimer binding curve in Section~\ref{sec:dimer_automation} (if you have skipped ahead you may want to skim that section for relevant information).  The most important change is the expanded description of the physical system:

\begin{lstlisting}[style=Python]
# details of your physical system (diamond conventional cell below)
my_project_name = 'diamond_vmc'   # directory to perform runs
my_dft_pps      = ['C.BFD.upf']   # pwscf pseudopotentials
my_qmc_pps      = ['C.BFD.xml']   # qmcpack pseudopotentials

#  generate your system
#    units      :  'A'/'B' for Angstrom/Bohr
#    axes       :  simulation cell axes in cartesian coordinates (a1,a2,a3)
#    elem       :  list of atoms in the system
#    pos        :  corresponding atomic positions in cartesian coordinates
#    kgrid      :  Monkhorst-Pack grid
#    kshift     :  Monkhorst-Pack shift (between 0 and 0.5)
#    net_charge :  system charge in units of e
#    net_spin   :  # of up spins - # of down spins
#    C = 4      :  (pseudo) carbon has 4 valence electrons
my_system = generate_physical_system(
    units      = 'A',
    axes       = [[ 3.57000000e+00, 0.00000000e+00, 0.00000000e+00],
                  [ 0.00000000e+00, 3.57000000e+00, 0.00000000e+00],
                  [ 0.00000000e+00, 0.00000000e+00, 3.57000000e+00]],
    elem       = ['C','C','C','C','C','C','C','C'],
    pos        = [[ 0.00000000e+00, 0.00000000e+00, 0.00000000e+00],
                  [ 8.92500000e-01, 8.92500000e-01, 8.92500000e-01],
                  [ 0.00000000e+00, 1.78500000e+00, 1.78500000e+00],
                  [ 8.92500000e-01, 2.67750000e+00, 2.67750000e+00],
                  [ 1.78500000e+00, 0.00000000e+00, 1.78500000e+00],
                  [ 2.67750000e+00, 8.92500000e-01, 2.67750000e+00],
                  [ 1.78500000e+00, 1.78500000e+00, 0.00000000e+00],
                  [ 2.67750000e+00, 2.67750000e+00, 8.92500000e-01]],
    kgrid      = (1,1,1),
    kshift     = (0,0,0),
    net_charge = 0,
    net_spin   = 0,
    C          = 4       # one line like this for each atomic species
    )

my_bconds       = 'ppp'  #  ppp/nnn for periodic/open BC's in QMC
                         #  if nnn, center atoms about (a1+a2+a3)/2
\end{lstlisting}

If you have a system you would like to try with QMC, make a copy of \ishell{example.py} and fill in the relevant information about the PPs, simulation cell axes, and atomic species/positions.  Otherwise, you can proceed with \ishell{example.py} as it is.

%The other new aspects are two additional compute jobs to generate the orbitals with PWSCF and convert them into the ESHDF format with \ishell{pw2qmcpack.x}:

%\begin{lstlisting}[style=Python]
%# scf run to generate orbitals
%scf = generate_pwscf(
%    identifier   = 'scf',
%    path         = my_project_name,
%    job          = Job(nodes=32,hours=2,app=pwscf),
%    input_type   = 'scf',
%    system       = my_system,
%    pseudos      = my_dft_pps,
%    input_dft    = 'lda', 
%    ecut         = 200,   # PW energy cutoff in Ry
%    conv_thr     = 1e-8, 
%    mixing_beta  = .7,
%    nosym        = True,
%    wf_collect   = True
%    )
%
%# conversion step to create h5 file with orbitals
%p2q = generate_pw2qmcpack(
%    identifier   = 'p2q',
%    path         = my_project_name,
%    job          = Job(cores=1,hours=2,app=pw2qmcpack),
%    write_psir   = False,
%    dependencies = (scf,'orbitals')
%    )
%\end{lstlisting}

Set ``\ishell{generate_only}'' to \ishell{1} and type ``\ishell{./example.py}'' or similar to generate the input files.  All files will be written to ``\ishell{./diamond_vmc}'' (``\ishell{./[my_project_name]}'' if you have changed ``\ishell{my_project_name}'' in the file).  The input files for PWSCF, pw2qmcpack, and QMCPACK are \ishell{scf.in}, \ishell{pw2qmcpack.in}, and \ishell{vmc.in.xml}, respectively.  Take some time to inspect the generated input files.  If you have questions about the file contents, or run into issues with the generation process, feel free to consult with a lab instructor.  

If desired, you can submit the runs directly with \ishell{example.py}.  To do this, first reset the Nexus simulation record by typing ``\ishell{rm ./diamond_vmc/sim*/sim.p}'' or similar and set ``\ishell{generate_only}'' back to \ishell{0}.  Next rerun \ishell{example.py}  (you may want to redirect the text output).  

Alternatively the runs can be submitted by hand:
\ifws
\begin{shade}
mpirun -np 4 pw.x<scf.in>&scf.out&

(wait until JOB DONE appears in scf.out)

mpirun -np 1 pw2qmcpack.x<p2q.in>&p2q.out&
\end{shade}
\else
\begin{shade}
qsub --mode script --env BG_SHAREDMEMSIZE=32 scf.qsub.in

(wait until JOB DONE appears in scf.output)

qsub --mode script --env BG_SHAREDMEMSIZE=32 p2q.qsub.in
\end{shade}
\fi
Once the conversion process has finished, the orbitals should be located in the file \newline \ishell{diamond_vmc/pwscf_output/pwscf.pwscf.h5}.  Open \ishell{diamond_vmc/vmc.in.xml} and replace ``\ishell{MISSING.h5}'' with ``\ishell{./pwscf_output/pwscf.pwscf.h5}''.  Next submit the VMC run:
\ifws
\begin{shade}
mpirun -np 4 qmcpack vmc.in.xml>&vmc.out&
\end{shade}
\else
\begin{shade}
qsub --mode script --env BG_SHAREDMEMSIZE=32 vmc.qsub.in
\end{shade}
\fi
Note: If your system is large, the preceding process may not complete within the time frame of this lab.  Working with a stripped down (but relevant) example is a good idea for exploratory runs.

Once the runs have finished, you may want to begin exploring Jastrow optimization and DMC for your system.  Example calculations are provided at the end of \ishell{example.py} in the commented out text.



%\section{(Optional) Revisiting the oxygen atom: 3-body Jastrow \& population control bias}

%\subsection{Optimization of 3-body Jastrow factors}

%\subsection{Investigation of DMC population control bias}




% cover basic python elsewhere in the manual?  refer to Nexus user guide or websites instead?
%\hide{

\section{Appendix A: Basic Python constructs\label{app:python_basics}}
Basic Python data types (\ishell{int}, \ishell{float}, \ishell{str}, \ishell{tuple}, \ishell{list}, \ishell{array}, \ishell{dict}, \ishell{obj}) and programming constructs (\ishell{if} statements, \ishell{for} loops, functions w/ keyword arguments) are briefly overviewed in the following.  All examples can be executed interactively in Python.  To do this, type ``\ishell{python}'' at the command line and paste any of the shaded text below at the ``\ishell{>>>}'' prompt.  For more information about effective use of Python, consult the detailed online documentation: \href{https://docs.python.org/2/}{https://docs.python.org/2/}.

\subsubsection{Intrinsic types: \ishell{int, float, str}}
\begin{shade}
#this is a comment
i=5                     # integer
f=3.6                   # float
s='quantum/monte/carlo' # string
n=None                  # represents "nothing"

f+=1.4                  # add-assign (-,*,/ also): 5.0
2**3                    # raise to a power: 8
str(i)                  # int to string: '5'
s+'/simulations'        # joining strings: 'quantum/monte/carlo/simulations'
'i={0}'.format(i)       # format string: 'i=5'

\end{shade}

 
\subsubsection{Container types: \ishell{tuple, list, array, dict, obj}}
\begin{shade}
from numpy import array  # get array from numpy module
from generic import obj  # get obj from Nexus' generic module 

t=('A',42,56,123.0)     # tuple

l=['B',3.14,196]        # list

a=array([1,2,3])        # array

d={'a':5,'b':6}         # dict

o=obj(a=5,b=6)          # obj

                        # printing
print t                 #  ('A', 42, 56, 123.0)
print l                 #  ['B', 3.1400000000000001, 196]
print a                 #  [1 2 3]
print d                 #  {'a': 5, 'b': 6}
print o                 #    a               = 5
                        #    b               = 6

len(t),len(l),len(a),len(d),len(o) #number of elements: (4, 3, 3, 2, 2)

t[0],l[0],a[0],d['a'],o.a  #element access: ('A', 'B', 1, 5, 5)

s = array([0,1,2,3,4])  # slices: works for tuple, list, array
s[:]                    #   array([0, 1, 2, 3, 4])
s[2:]                   #   array([2, 3, 4])
s[:2]                   #   array([0, 1])
s[1:4]                  #   array([1, 2, 3])
s[0:5:2]                #   array([0, 2, 4])

                        # list operations
l2 = list(l)            #   make independent copy
l.append(4)             #   add new element: ['B', 3.14, 196, 4]
l+[5,6,7]               #   addition: ['B', 3.14, 196, 4, 5, 6, 7]
3*[0,1]                 #   multiplication:  [0, 1, 0, 1, 0, 1]

b=array([5,6,7])        # array operations
a2 = a.copy()           #   make independent copy
a+b                     #   addition: array([ 6, 8, 10])
a+3                     #   addition: array([ 4, 5, 6])
a*b                     #   multiplication: array([ 5, 12, 21])
3*a                     #   multiplication: array([3, 6, 9])

                        # dict/obj operations
d2 = d.copy()           #   make independent copy
d['c'] = 7              #   add/assign element 
d.keys()                #   get element names: ['a', 'c', 'b']
d.values()              #   get element values: [5, 7, 6]

                        # obj-specific operations
o.c = 7                 #   add/assign element
o.set(c=7,d=8)          #   add/assign multiple elements

\end{shade}
An important feature of Python to be aware of is that assignment is most often by reference, that is, new values are not always created.  This point is illustrated with an \ishell{obj} instance in the following example, but it also holds for \ishell{list}, \ishell{array}, \ishell{dict}, and others.
\begin{shade}
>>> o = obj(a=5,b=6)
>>> 
>>> p=o
>>> 
>>> p.a=7
>>> 
>>> print o
  a               = 7
  b               = 6

>>> q=o.copy()
>>> 
>>> q.a=9
>>> 
>>> print o
  a               = 7
  b               = 6
\end{shade}
\noindent
Here \ishell{p} is just another name for \ishell{o}, while \ishell{q} is a fully independent copy of it.


\subsubsection{Conditional Statements: \ishell{if/elif/else}}
\begin{shade}
a = 5
if a is None:
    print 'a is None'
elif a==4:
    print 'a is 4'
elif a<=6 and a>2:
    print 'a is in the range (2,6]'
elif a<-1 or a>26:
    print 'a is not in the range [-1,26]'
elif a!=10: 
    print 'a is not 10'
else:
    print 'a is 10'
#end if

\end{shade}
The ``\ishell{\#end if}'' is not part of Python syntax, but you will see text like this throughout Nexus for clear encapsulation.

\subsubsection{Iteration: \ishell{for}}
\begin{shade}
from generic import obj

l = [1,2,3]              
m = [4,5,6]
s = 0
for i in range(len(l)):  # loop over list indices
    s += l[i] + m[i]
#end for

print s                  # s is 21

s = 0                    
for v in l:              # loop over list elements
    s += v
#end for

print s                  # s is 6

o = obj(a=5,b=6)
s = 0
for v in o:              # loop over obj elements
    s += v
#end for

print s                  # s is 11

d = {'a':5,'b':4}
for n,v in o.items():# loop over name/value pairs in obj
    d[n] += v
#end for

print d                  # d is {'a': 10, 'b': 10}

\end{shade}


\subsubsection{Functions: \ishell{def}, argument syntax}
\begin{shade}
def f(a,b,c=5):          # basic function, c has a default value
    print a,b,c
#end def f

f(1,b=2)                 # prints: 1 2 5


def f(*args,**kwargs):   # general function, returns nothing
    print args           #     args: tuple of positional arguments
    print kwargs         #   kwargs: dict of keyword arguments
#end def f

f('s',(1,2),a=3,b='t')   # 2 pos., 2 kw. args, prints:
                         #   ('s', (1, 2))
                         #   {'a': 3, 'b': 't'}

l = [0,1,2]
f(*l,a=6)                # pos. args from list, 1 kw. arg, prints:
                         #   (0, 1, 2)
                         #   {'a': 6}
o = obj(a=5,b=6)
f(*l,**o)                # pos./kw. args from list/obj, prints:
                         #   (0, 1, 2)
                         #   {'a': 5, 'b': 6}

f(                       # indented kw. args, prints
    blocks   = 200,      #   () 
    steps    = 10,       #   {'steps': 10, 'blocks': 200, 'timestep': 0.01}
    timestep = 0.01
    )

o = obj(                 # obj w/ indented kw. args
    blocks   = 100,
    steps    =  5,
    timestep = 0.02
    )

f(**o)                   # kw. args from obj, prints:
                         #   ()
                         #   {'timestep': 0.02, 'blocks': 100, 'steps': 5}
\end{shade}

%}

\chapter{Lab 3: Advanced molecular calculations}
\label{chap:lab_advanced_molecules}

\section{Topics covered in this lab}
This lab covers molecular QMC calculations with wavefunctions of increasing sophistication.  All of the trial wavefunctions are initially generated with the GAMESS code.  Topics covered include:
\begin{itemize}
  \item{Generating single-determinant trial wavefunctions with GAMESS (HF and DFT)}
  \item{Generating multideterminant trial wavefunctions with GAMESS (CISD, CASCI, and SOCI)}
  \item{Optimizing wavefunctions (Jastrow factors and CSF coefficients) with QMC}
  \item{DMC time step and walker population convergence studies}
  \item{Systematic progressions of Jastrow factors in VMC}
  \item{Systematic convergence of DMC energies with multideterminant wavefunctions}
  \item{Influence of orbitals basis choice on DMC energy}
  %\item{Influence of orbitals basis choice on rate of multi-determinant DMC convergence}
\end{itemize}

\section{Lab directories and files}
\footnotesize
\begin{verbatim}
labs/lab3_advanced_molecules/exercises
│    
├── ex1_first-run-hartree-fock    - basic work flow from Hatree-Fock to DMC
│   ├── gms                        - Hatree-Fock calculation using GAMESS
│   │   ├── h2o.hf.inp               - GAMESS input
│   │   ├── h2o.hf.dat               - GAMESS punch file containing orbitals
│   │   └── h2o.hf.out               - GAMESS output with orbitals and other info
│   ├── convert                    - Convert GAMESS wavefunction to QMCPACK format
│   │   ├── h2o.hf.out               - GAMESS output
│   │   ├── h2o.ptcl.xml             - converted particle positions
│   │   └── h2o.wfs.xml              - converted wave function
│   ├── opt                        - VMC optimization
│   │   └── optm.xml                 - QMCPACK VMC optimization input
│   ├── dmc_timestep               - Check DMC timestep bias
│   │   └── dmc_ts.xml               - QMCPACK DMC input
│   └── dmc_walkers                - Check DMC population control bias
│       └── dmc_wk.xml               - QMCPACK DMC input template
│   
├── ex2_slater-jastrow-wf-options - explore jastrow and orbital options
│   ├── jastrow                    - Jastrow options
│   │   ├── 12j                      - no 3-body Jastrow
│   │   ├── 1j                       - only 1-body Jastrow
│   │   └── 2j                       - only 2-body Jastrow
│   └── orbitals                   - Orbital options
│       ├── pbe                      - PBE orbitals
│       │   └── gms                    - DFT calculation using GAMESS
│       │      └── h2o.pbe.inp          - GAMESS DFT input
│       ├── pbe0                     - PBE0  orbitals
│       ├── blyp                     - BLYP  orbitals
│       └── b3lyp                    - B3LYP orbitals
│       
├── ex3_multi-slater-jastrow
│   ├── cisd                      - CISD wave function
│   │   ├── gms                     - CISD calculation using GAMESS
│   │   │   ├── h2o.cisd.inp           - GAMESS input
│   │   │   ├── h2o.cisd.dat           - GAMESS punch file containing orbitals
│   │   │   └── h2o.cisd.out           - GAMESS output with orbitals and other info
│   │   └── convert                 - Convert GAMESS wavefunction to QMCPACK format
│   │      └── h2o.hf.out             - GAMESS output
│   ├── casci                     - CASCI wave function
│   │   └── gms                     - CASCI calculation using GAMESS
│   └── soci                      - SOCI wave function
│       ├── gms                     - SOCI calculation using GAMESS
│       ├── thres0.01               - VMC optimization with few determinants
│       └── thres0.0075             - VMC optimization with more determinants
│   
└── pseudo
    ├── H.BFD.gamess             - BFD pseudopotential for H in GAMESS format
    ├── O.BFD.CCT.gamess         - BFD pseudopotential for O in GAMESS format
    ├── H.xml                    - BFD pseudopotential for H in QMCPACK format
    └── O.xml                    - BFD pseudopotential for H in QMCPACK format
\end{verbatim}
\normalsize

\section{Exercise \#1: Basics}

The purpose of this exercise is to show how to generate wavefunctions for QMCPACK
using GAMESS and to optimize the resulting wavefunctions using VMC. This will be
followed by a study of the time step and walker population dependence of DMC energies.
The exercise will be performed on a water molecule at the equilibrium geometry.


\subsection{Generation of a Hartree-Fock wavefunction with GAMESS}

From the top directory, go to ``\texttt{ex1\_first-run-hartree-fock/gms}.'' This directory contains an input
file for a HF calculation of a water molecule using BFD ECPs and the corresponding
cc-pVTZ basis set. The input file should be named: “h2o.hf.inp.” Study the input
file. See Section~\ref{sec:lab_adv_mol_gamess}, ``Appendix A: GAMESS input" for a more detailed description of the
GAMESS input syntax. However, there will be a better time to do this soon, so we recommend
continuing with the exercise at this point. After you are done, execute
GAMESS with this input and store the standard output in a file named ``h2o.hf.output.''
Finally, in the ``convert'' folder, use \texttt{convert4qmc} to generate the QMCPACK \texttt{particleset} and \texttt{wavefunction} files. It
is always useful to rename the files generated by \texttt{convert4qmc} to something meaningful
since by default they are called \texttt{sample.Gaussian-G2.xml} and \texttt{sample.Gaussian-G2.ptcl.xml}.
In a standard computer (without cross-compilation), these tasks can be accomplished by
the following commands.
\begin{lstlisting}[style=SHELL]
cd ${TRAINING TOP}/ex1_first-run-hartree-fock/gms
jobrun_vesta rungms h2o.hf 
cd ../convert
cp ../gms/h2o.hf.output
jobrun_vesta convert4qmc -gamessAscii h2o.hf.output -add3BodyJ
mv sample.Gaussian-G2.xml h2o.wfs.xml
mv sample.Gaussian-G2.ptcl.xml h2o.ptcl.xml
\end{lstlisting}
\noindent

%Due to the particular requirements of Vesta these executions can not be performed on
%the login nodes, they must be performed in the compute nodes. In order to accomplish
%this, we must make a submission script with the appropriate commands and submit it to
%the batch system. Two submission scripts have been provided, one for the GAMESS 
%execution called submit gamess.csh and one for the submission of convert4qmc, with a similar
%descriptive name. Study these input files. (In this and all other exercises, you will need
%to make all submission scripts executables with the command chmod u+x script.csh.)
%When you are ready, start by submitting the GAMESS execution to the batch system
%using ``./script gamess.csh'' (This script calls the GAMESS run script, which itself calls
%qsub. Do not attempt to submit this specific script with qsub). 
The HF energy of the
system is -16.9600590022 Ha. To search for the energy in the output file quickly, you can
use 
\begin{shade}
grep "TOTAL ENERGY =" h2o.hf.output
\end{shade}
%When the calculation completes,
%submit the execution of the converter using ``qsub submit convert.csh'' (all QMCPACK execution 
%scripts will be submitted with qsub). This is a good time to review section B, which
%contains a description on the use of the converter.
As the job runs on VESTA, it is a good time to review Section~\ref{sec:lab_adv_mol_convert4qmc}, ``Appendix B: convert4qmc," which contains a description on the use of the converter.


\subsection{Optimize the wavefunction}
When execution of the previous steps is completed, there should be two new
files called \texttt{h2o.wfs.xml} and \texttt{h2o.ptcl.xml}. Now we will use VMC to optimize the 
Jastrow parameters in the wavefunction.  From the top directory, go to
``\texttt{ex1\_first-run-hartree-fock/opt}.'' Copy the xml files generated in the previous step
to the current directory. This directory should already contain a basic QMCPACK input
file for an optimization calculation (\texttt{optm.xml}) %and a submission script (submit.csh). 
Open \texttt{optm.xml} with your favorite text editor and modify the name of the files that contain the
\texttt{wavefunction} and \texttt{particleset} XML blocks. These files are included with the commands:
\begin{lstlisting}[style=QMCPXML]
<include href=ptcl.xml/>
<include href=wfs.xml/>
\end{lstlisting}
(the particle set must be
defined before the wavefunction). The name of the particle set and wavefunction files should now be \texttt{h2o.ptcl.xml}
and \texttt{h2o.wfs.xml}, respectively. Study both files and submit when you are ready. Notice that the
location of the ECPs has been set for you; in your own calculations you have to make
sure you obtain the ECPs from the appropriate libraries and convert them to QMCPACK
format using ppconvert. While these calculations finish is a good time to study Section~\ref{sec:lab_adv_mol_opt_appendix}, ``Appendix C: Wavefunction optimization XML block," which contains a review
of the main parameters in the optimization XML block.  The
previous steps can be accomplished by the following commands:
\begin{shade}
cd ${TRAINING TOP}/ex1_first-run-hartree-fock/opt
cp ../convert/h2o.wfs.xml ./
cp ../convert/h2o.ptcl.xml ./
# edit optm.xml to include the correct ptcl.xml and wfs.xml
jobrun_vesta qmcpack optm.xml
\end{shade}

Use the analysis tool \texttt{qmca} to analyze the results of the calculation. Obtain the VMC
energy and variance for each step in the optimization and plot it using your favorite program.
Remember that \texttt{qmca} has built-in functions to plot the analyzed data.
\begin{shade}
qmca -q e *scalar.dat -p
\end{shade}

\begin{figure}
\begin{center}
\includegraphics[trim = 0mm 0mm 0mm 0mm, clip,width=0.75\columnwidth]{./figures/lab_advanced_molecules_opt_conv.png}
\end{center}
\caption{VMC energy as a function of optimization step.}
\label{fig:lam_opt_conv}
\end{figure}

The resulting energy as a function of the optimization step should look qualitatively similar to Figure \ref{fig:lam_opt_conv}.
The energy should decrease quickly as a function of the number of optimization steps. After 6--8 steps, the energy should be converged to $\sim$2--3 mHa. To improve convergence,
we would need to increase the number of samples used during optimization (You can
check this for yourself later.). With optimized wavefunctions, we are in a position
to perform VMC and DMC calculations. The modified wavefunction files after each step
are written in a file named \texttt{ID.sNNN.opt.xml}, where ID is the identifier of the calculation
defined in the input file (this is defined in the project XML block with parameter “id”) and
NNN is a series number that increases with every executable xml block in the input file.


\subsection{Time-step study}
Now we will study the dependence of the DMC energy with time step. From the top directory, 
go to “\texttt{ex1\_first-run-hartree-fock/dmc\_timestep}.” This folder contains a basic XML input
file (\texttt{dmc\_ts.xml}) that performs a short VMC calculation and three DMC calculations
with varying time steps (0.1, 0.05, 0.01). Link the \texttt{particleset} and the last \texttt{optimization}
file from the previous folder (the file called \texttt{jopt-h2o.sNNN.opt.xml} with the largest value of
NNN). Rename the optimized \texttt{wavefunction} file to any suitable name if you wish (for example,
\texttt{h2o.opt.xml}) and change the name of the \texttt{particleset} and \texttt{wavefunction} files in the
input file. An optimized wavefunction can be found in the reference files (same location)
in case it is needed. %Using the submission script of the previous exercise as a base, create a
%submission script for this step and submit the run. Set the number of nodes to 32 (2 places
%must be changed), the number of threads to 16 and leave the number of tasks at 1.

The main steps needed to perform this exercise are:
\begin{lstlisting}[style=SHELL]
cd \$\{TRAINING TOP\}/ex1_first-run-hartree-fock/dmc_timestep
cp ../opt/h2o.ptcl.xml ./
cp ../opt/jopt-h2o.s007.opt.xml h2o.opt.wfs.xml
# edit dmc_ts.xml to include the correct ptcl.xml and wfs.xml
jobrun_vesta qmcpack dmc_ts.xml
\end{lstlisting}
While these runs complete, go to Section\ref{sec:lab_adv_mol_vmcdmc_appendix}, ``Appendix D: VMC and DMC XML block," and review the basic VMC and DMC input
blocks. Notice that in the current DMC blocks the time step is decreased as the number of blocks is increased. Why is this?

When the simulations are finished, use \texttt{qmca} to analyze the output files and plot the
DMC energy as a function of time step. Results should be qualitatively similar to those
presented in Figure \ref{fig:lam_dmc_timestep}; in this case we present more time steps with well converged results to
better illustrate the time step dependence. In realistic calculations, the time step must be
chosen small enough so that the resulting error is below the desired accuracy. Alternatively,
various calculations can be performed and the results extrapolated to the zero time-step
limit.


\begin{figure}
\begin{center}
\includegraphics[trim = 0mm 0mm 0mm 0mm, clip,width=0.75\columnwidth]{./figures/lab_advanced_molecules_dmc_timestep.png}
\end{center}
\caption{DMC energy as a function of time step.}
\label{fig:lam_dmc_timestep}
\end{figure}


\subsection{Walker population study}
Now we will study the dependence of the DMC energy with the number of walkers in the
simulation. Remember that, in principle, the DMC distribution is reached in the limit of
an infinite number of walkers. In practice, the energy and most properties converge to high
accuracy with $\sim$100--1,000 walkers. The actual number of walkers needed in a calculation
will depend on the accuracy of the VMC wavefunction and on the complexity and size of
the system. Also notice that using too many walkers is not a problem; at worse it will be
inefficient since it will cost more computer time than necessary. In fact, this is the strategy
used when running QMC calculations on large parallel computers since we can reduce the
statistical error bars efficiently by running with large walker populations distributed across
all processors.

From the top directory, go to ``\texttt{ex1\_first-run-hartree-fock/dmc\_walkers}.'' Copy the
optimized \texttt{wavefunction} and \texttt{particleset} files used in the previous calculations to the current
folder; these are the files generated during step 2 of this exercise. An optimized \texttt{wavefunction} file can be found in the reference files (same location) in case it is needed. The directory
contains a sample DMC input file and submission script. Create three  directories named NWx,
with x values of 120,240,480, and copy the input file to each one. Go
to ``NW120,'' and, in the input file, change the name of the \texttt{wavefunction} and \texttt{particleset}
files (in this case they will be located one directory above, so use ``\texttt{../dmc\_timestep/h2.opt.xml},'' for
example); change the PP directory so that it points to one directory above; change ``targetWalkers'' to 120; and change the number of steps to 100, the time step
to 0.01, and the number of blocks to 400. Notice that ``targetWalkers'' is one way to set the desired (average) number of walkers in a DMC calculation. One can alternatively set ``samples'' in the \ixml{<qmc method="vmc"} block to carry over de-correlated VMC configurations as DMC walkers. For your own simulations, we generally recommend setting $\sim$2*(\#threads)
walkers per node (slightly smaller than this value).

The main steps needed to perform this exercise are
\begin{shade}
cd ${TRAINING TOP}/ex1_first-run-hartree-fock/dmc_walkers
cp ../opt/h2o.ptcl.xml ./
cp ../opt/jopt-h2o.s007.opt.xml h2o.opt.wfs.xml
# edit dmc_wk.xml to include the correct ptcl.xml and wfs.xml and 
#  use the correct pseudopotential directory
mkdir NW120
cp dmc_wk.xml NW120
# edit dmc_wk.xml to use the desired number of walkers, 
#  and collect the desired amount of statistics
jobrun_vesta qmcpack dmc_wk.xml
# repeat for NW240, NW480
\end{shade}

\begin{figure}
\begin{center}
\includegraphics[trim = 0mm 0mm 0mm 0mm, clip,width=0.75\columnwidth]{./figures/lab_advanced_molecules_dmc_popcont.png}
\end{center}
\caption{DMC energy as a function of the average number of walkers.}
\label{fig:lam_dmc_popcont}
\end{figure}

Repeat the same procedure in the other folders by setting (targetWalkers=240,
steps=100, timestep=0.01, blocks=200) in NW240 and (targetWalkers=480, 
steps=100, timestep=0.01, blocks=100) in NW480. When
the simulations complete, use \texttt{qmca} to analyze and plot the energy as a function of the
number of walkers in the calculation. As always, Figure \ref{fig:lam_dmc_popcont} 
shows representative results of the
energy dependence on the number of walkers for a single water molecule. As shown,
less than 240 walkers are needed to obtain an accuracy of 0.1 mHa.


\section{Exercise \#2 Slater-Jastrow wavefunction options}
From this point on in the tutorial we assume familiarity with the basic parameters in the
optimization, VMC, and DMC XML input blocks of QMCPACK. In addition, we assume
familiarity with the submission system. As a result, the folder structure will not contain
any prepared input or submission files, so you will need to generate them using 
input files from exercise 1. In the case of QMCPACK sample 
files, you will find \texttt{optm.xml}, \texttt{vmc dmc.xml}, and \texttt{submit.csh files}. Some of
the options in these files can be left unaltered, but many of them will need to be tailored to
the particular calculation.

In this exercise we will study the dependence of the DMC energy on the choices made
in the wavefunction ansatz. In particular, we will study the influence/dependence of the
VMC energy with the various terms in the Jastrow. We will also study the influence of
the VMC and DMC energies on the SPOs used to form the Slater determinant 
in single-determinant wavefunctions. For this we will use wavefunctions generated
with various exchange-correlation functionals in DFT. Finally, we will optimize a simple
multideterminant wavefunction and study the dependence of the energy on the number of
configurations used in the expansion. All of these exercises will be performed on the water 
molecule at equilibrium.


\subsection{Influence of Jastrow on VMC energy with HF wavefunction}
In this section we will study the dependence of the VMC energy on the various Jastrow
terms (e.g., 1-body, 2-body and 3-body. From the top directory, go to ``\texttt{ex2\_slater-jastrow-wf-options/jastrow.''} 
We will compare the single-determinant VMC energy using a 2-body 
Jastrow term, both 1- and 2-body terms, and finally 1-, 2- and 3-body
terms. Since we are interested in the influence of the Jastrow, we will use the HF orbitals
calculated in exercise \#1. Make three folders named 2j, 12j, and 123j. For both 2j and
12j, %(we have already optimized a wave-function for the 1-2-3J case, so the steps will be
%slightly different in this case)
 copy the input file \texttt{optm.xml} %and the sample submission file
from ``\texttt{ex1\_first-run-hartree-fock/opt.}'' This input file performs both wavefunction optimization 
and a VMC calculation. Remember to correct relative paths to the PP directory. Copy the un-optimized HF \texttt{wavefunction} and \texttt{particleset} files
from ``\texttt{ex1\_first-run-hartree-fock/convert}''; if you followed the instructions in exercise \#1 these should be
named \texttt{h2o.wfs.xml} and \texttt{h2o.ptcl.xml}. Otherwise, you can obtained them from the
REFERENCE files. Modify the \texttt{h2o.wfs.xml} file to remove the appropriate Jastrow
blocks. For example, for a 2-body Jastrow (only), you need to eliminate the Jastrow
blocks named \ixml{<jastrow name="J1"} and \ixml{<jastrow name="J3."} In the case of 12j, remove
only \ixml{<jastrow name="J3."} Recommended settings for the optimization run are nodes=32,
threads=16, blocks=250, samples=128000, time-step=0.5, 8 optimization loops. Recommended settings in the
VMC section are walkers=16, blocks=1000, steps=1, substeps=100. Notice that
samples should always be set to blocks*threads per node*nodes = 32*16*250=128000. Repeat 
the process in both 2j and 12j cases. For the 123j case, the wavefunction has
already been optimized in the previous exercise. Copy the optimized HF wavefunction and
the particleset from ``\texttt{ex1\_first-run-hartree-fock/opt.}'' Copy the input file from any of the previous runs and remove the optimization block from the
input, just leave the VMC step. In all three cases, modify the submission script and submit the run.

\begin{figure}
\begin{center}
\includegraphics[trim = 0mm 0mm 0mm 0mm, clip,width=0.75\columnwidth]{./figures/lab_advanced_molecules_vmc_jastrow.png}
\end{center}
\caption{VMC energy as a function of Jastrow type.}
\label{fig:lam_vmc_jastrow}
\end{figure}

Because these simulations will take several minutes to complete, this is an excellent opportunity
to go to Section~\ref{sec:lab_adv_mol_wf_appendix}, ``Appendix E: Wavefunction XML block," and review the wavefunction XML block used by QMCPACK. When the
simulations are completed, use \texttt{qmca} to analyze the output files. Using your favorite plotting
program (e.g., gnu plot), plot the energy and variance as a function of the Jastrow form.
Figure \ref{fig:lam_vmc_jastrow} shows a typical result for this calculation. As can be seen, the VMC energy and
variance depends strongly on the form of the Jastrow. Since the DMC error bar is directly
related to the variance of the VMC energy, improving the Jastrow will always lead to a
reduction in the DMC effort. In addition, systematic approximations (time step, number of
walkers, etc.) are also reduced with improved wavefunctions.


\subsection{Generation of wavefunctions from DFT using GAMESS}
In this section we will use GAMESS to generate wavefunctions for QMCPACK from
DFT calculations. From the top folder, go to ``\texttt{ex2\_slater-jastrow-wf-options/orbitals}.'' To demonstrate
the variation in DMC energies with the choice of DFT orbitals, we will choose the following
set of exchange-correlation functionals (PBE, PBE0, BLYP, B3LYP). For each functional,
make a directory using your preferred naming convention (e.g., the name of the functional).
Go into each folder and copy a GAMESS input file from %for a ROHF calculation from
``\texttt{ex1\_first-run-hartree-fock/gms}.'' %, a file named rohf.inp should exist.
 Rename the file with your preferred naming convention; we suggest using \texttt{h2o.[dft].inp}, where [dft] is the name of
the functional used in the calculation. At this point, this input file should be identical to the
one used to generate the HF wavefunction in exercise \#1. To perform a DFT
calculation we only need to add ``DFTTYP'' to the \igamess{\$CONTRL ... \$END} section and set
it to the desired functional type, for example, ``DFTTYP=PBE'' for a PBE functional. This
variable must be set to (PBE, PBE0, BLYP, B3LYP) to obtain the appropriate functional in
GAMESS. For a complete list of implemented functionals, see the GAMESS input manual.


\subsection{Optimization and DMC calculations with DFT wavefunctions}
In this section we will optimize the wavefunction generated in the previous step and
perform DMC calculations. From the top directory, go to “\texttt{ex2\_slater-jastrow-wf-options/orbitals}.”
The steps required to achieve this are identical to those used to optimize the wavefunction
with HF orbitals. Make individual folders for each calculation and obtain the necessary files
to perform optimization, for example, VMC and DMC calculations from ``for \texttt{ex1\_first-run-hartree-fock/opt}'' and ``\texttt{ex1\_first-run-hartree-fock/dmc\_ts}.''
%A file named optm vmc dmc.xml should exist that contains all three execution blocks. 
For each functional, make the appropriate modifications to the input files and copy the \texttt{particleset} and \texttt{wavefunction} files from the appropriate directory in “\texttt{ex2\_slater-jastrow-wf-options/orbitals/[dft]}.” We
recommend the following settings: nodes=32, threads=16, (in optimization) blocks=250,
samples=128000, timestep=0.5, 8 optimization loops, (in VMC) walkers=16, blocks=100,
steps=1, substeps=100, (in DMC) blocks 400, targetWalkers=960, and timestep=0.01. Submit
the runs and analyze the results using \texttt{qmca}.

How do the energies compare against each other? How do they compare against DMC
energies with HF orbitals?
%Orbital Sets and Configurations in 
\section{Exercise \#3: Multideterminant wavefunctions}
In this exercise we will study the dependence of the DMC energy on the set of orbitals
and the type of configurations included in a multideterminant wavefunction. 

\subsection{Generation of a CISD wavefunctions using GAMESS}
In this section we will use GAMESS to generate a multideterminant wavefunction with
configuration interaction with single and double excitations (CISD). In CISD, the Schrodinger equation is solved exactly on a basis of determinants 
including the HF determinant and all its single and double excitations. 

Go to ``\texttt{ex3\_multi-slater-jastrow/cisd/gms}'' and you will see input and output files named \texttt{h2o.cisd.inp} and \texttt{h2o.cisd.out}. Because of technical problems with GAMESS in the BGQ architecture of VESTA, we are unable to use CISD properly in GAMESS. Consequently, the output of the calculation is already provided in the directory. 

%You'll see several input and output files named h2o.XXX.inp
%and h2o.XXX.out, where XXX is one of the following multi-determinant methods: CISD,
%CASSCF, CASCI, SOCI. 

There will be time in the next step to study the GAMESS input
files and the description in Section~\ref{sec:lab_adv_mol_gamess}, ``Appendix A: GAMESS input." %In the next exercise we will use the CISD output, in
%the next exercise we will use the remaining files. 
Since the output is already provided, the
only action needed is to use the converter to generate the appropriate QMCPACK files.  %Copy a submission script from GAMESS/Generic Files and execute the converter for all the output 
%files in the directory (with the exception of CASSCF, which is used to generate orbitals).
%but it doesn’t contain appropriate CI coefficients). 
\begin{shade}
jobrun_vesta convert4qmc h2o.cisd.out -ci h2o.cisd.out \
-readInitialGuess 57 -threshold 0.0075
\end{shade}

We used the PRTMO=.T. flag in the GUESS section to include orbitals in the output file. You should read these orbitals from the output (-readInitialGuess 40).
The highest occupied orbital in any determinant should be 34, so reading 40 orbitals is a safe choice. In this case, it is important to rename the XML files with meaningful names, for example, \texttt{h2o.cisd.wfs.xml}. A threshold of 0.0075 is sufficient for the calculations in the training.


\subsection{Optimization of a multideterminant wavefunction}

In this section we will optimize the wavefunction generated in the previous step. There
is no difference in the optimization steps if a single determinant and a multideterminant wavefunction.
QMCPACK will recognize the presence of a multideterminant wavefunction and will automatically 
optimize the linear coefficients by default. Go to ``\texttt{ex3\_multi-slater-jastrow/cisd}'' and make a folder called 
\texttt{thres0.01}. Copy the \texttt{particleset} and \texttt{wavefunction} files created in the previous step to the current 
directory. With your favorite text editor, open the \texttt{wavefunction} file \texttt{h2o.wfs.xml}. Look for 
the multideterminant XML block and change the ``cutoff'' parameter in detlist to 0.01. Then follow 
the same steps used in Section 9.4.3, ``Optimization and DMC calculations with DFT wavefunctions''
to optimize the wavefunction. Similar to this case, design a QMCPACK input file that performs
wavefunction optimization followed by VMC and DMC calculations. Submit the calculation.

This is a good time to review the GAMESS input file description in Section~\ref{sec:lab_adv_mol_gamess}, ``Appendix A. GAMESS input."
When the run is completed, go to the previous directory and make a new folder named
\texttt{thres0.0075}. Repeat the previous steps to optimize the wavefunction with a cutoff of 0.01, but use a cutoff of 0.0075 this time. This will increase the number of determinants used in the calculation. Notice that the ``cutoff'' parameter in the XML should be less than the ``-threshold 0.0075'' flag passed to the converted, which is further bounded by the PRTTOL flag in the GAMESS input.

After the wavefunction is generated, we are ready to optimize. Instead of starting from an un-optimized wavefunction, we can start from the optimized wavefunction from thres0.01 to speed up convergence. You will need to modify the file and change the cutoff in detlist to 0.0075 with a text editor. Repeat the optimization steps and submit the calculation.

\begin{figure}
\begin{center}
\includegraphics[trim = 0mm 0mm 0mm 0mm, clip,width=0.75\columnwidth]{./figures/lab_advanced_molecules_dmc_ci_cisd.png}
\end{center}
\caption{DMC energy as a function of the sum of the square of CI coefficients from CISD.}
\label{fig:lam_dmc_ci_cisd}
\end{figure}

When you are done, use \texttt{qmca} to analyze the results. Compare the energies at these two
coefficient cutoffs with the energies obtained with DFT orbitals. Because of the time limitations of this tutorial, it is not practical to optimize the wavefunctions with a smaller cutoff since this would require more samples and longer runs due to the larger number of optimizable parameters. Figure \ref{fig:lam_dmc_ci_cisd} shows the results of such exercise: the DMC energy as a function of the cutoff in the wavefunction. As can be seen, a large improvement in the energy is obtained as the number of configurations is increased.


%Since the un-optimized wave-functions were generated in subsection “Generation of a CISD wavefunctions 
%using GAMESS” of exercise \#2, we can skip this section and go straight to the
%wave-function optimization. 

\subsection{CISD, CASCI, and SOCI}

Go to “\texttt{ex3\_multi-slater-jastrow}” and inspect the folders for the remaining wavefunction types: CASCI and SOCI. Follow the steps in the previous exercise and obtain the optimized wavefunctions for these determinant choices. Notice that the SOCI GAMESS output is not included because it is large. Already converted XML inputs can be found in ``\texttt{ex3\_multi-slater-jastrow/soci/thres*}.'' %The exercise has already been performed with a CISD wave-function in exercise \#2.

A CASCI wavefunction is produced from a CI calculation that includes all the determinants 
in a complete active space (CAS) calculation, in this case using the orbitals from a previous CASSCF
calculation. In this case we used a CAS(8,8) active space that includes all determinants
generated by distributing 8 electrons in the lowest 8 orbitals. A SOCI calculation is similar
to the CAS-CI calculation, but in addition to the determinants in the CAS it also includes
all single and double excitations from all of them, leading to a much larger determinant
set. Since you now have considerable experience optimizing wavefunctions and calculating
DMC energies, we will leave it to you to complete the remaining tasks on your own.
If you need help, refer to previous exercises in the tutorial. Perform optimizations for both
wavefunctions using cutoffs in the CI expansion of 0.01 an 0.0075. If you have time, try to optimize the wavefunctions with a cutoff of 0.005. Analyze the results and plot
the energy as a function of cutoff for all three cases: CISD, CAS-CI, and SOCI.

Figure  \ref{fig:lam_dmc_ci_cisd} shows the result of similar calculations using more samples and smaller cutoffs.
The results should be similar to those produced in the tutorial. For reference, the exact
energy of the water molecule with ECPs is approximately -17.276 Ha. From the results of the
tutorial, how does the selection of determinants relate to the expected DMC energy?
What about the choice in the set of orbitals?


\newpage
\section{Appendix A: GAMESS input}\label{sec:lab_adv_mol_gamess}
In this section we provide a brief description of the GAMESS input needed to produce
trial wavefunction for QMC calculations with QMCPACK. We assume basic familiarity
with GAMESS input structure, particularly regarding the input of atomic coordinates and
the definition of Gaussian basis sets. This section focuses on generation of the output
files needed by the converter tool, \texttt{convert4qmc}. For a description of the converter, see Section~\ref{sec:lab_adv_mol_convert4qmc}, ``Appendix B: convert4qmc."

Only a subset of the methods available in GAMESS can be used to generate wavefunctions 
for QMCPACK, and we restrict our description to these.
For a complete description of all the options and methods available
in GAMESS, please refer to the official documentation at ``http://www.msg.ameslab.gov/gamess/documentation.html.”

Currently, \texttt{convert4qmc} can process output for the following methods in GAMESS (in
SCFTYP): RHF, ROHF, and MCSCF. Both HF and DFT calculations (any DFT
type) can be used in combination with RHF and ROHF calculations. For MCSCF and CI
calculations, ALDET, ORMAS, and GUGA drivers can be used (details follow).


\subsection{HF input}
The following input will perform a restricted HF calculation on a closed-shell singlet 
(multiplicity=1). This will generate RHF orbitals for any molecular system defined in 
\ishell{\$DATA ... \$END}.

\begin{lstlisting}[style=GAMESS]
$CONTRL SCFTYP=RHF RUNTYP=ENERGY MULT=1
ISPHER=1 EXETYP=RUN COORD=UNIQUE MAXIT=200 $END
$SYSTEM MEMORY=150000000 $END
$GUESS GUESS=HUCKEL $END
$SCF DIRSCF=.TRUE. $END
$DATA
...
Atomic Coordinates and basis set
...
$END
\end{lstlisting}

Main options:
\begin{enumerate}
  \item{SCFTYP: Type of SCF method, options: RHF, ROHF, MCSCF, UHF and NONE.}
  \item{RUNTYP: Type of run. For QMCPACK wavefunction generation this should always be ENERGY.}
  \item{MULT: Multiplicity of the molecule.}
  \item{ISPHER: Use spherical harmonics (1) or Cartesian basis functions (-1).}
  \item{COORD: Input structure for the atomic coordinates in \$DATA.}
\end{enumerate}


\subsection{DFT calculations}
The main difference between the input for a RHF/ROHF calculation and a DFT calculation 
is the definition of the DFTTYP parameter. If this is set in the \$CONTROL
section, a DFT calculation will be performed with the appropriate functional. Notice that
although the default values are usually adequate, DFT calculations have many options involving
the integration grids and accuracy settings. Make sure you study the input manual to be
aware of these. Refer to the input manual for a list of the implemented exchange-correlation
functionals.


\subsection{MCSCF}
MCSCF calculations are performed by setting SCFTYP=MCSCF in the CONTROL
section. If this option is set, an MCSCF section must be added to the input file with the
options for the calculation. An example section for the water molecule used in the tutorial
follows.

\begin{lstlisting}[style=GAMESS]
$MCSCF CISTEP=GUGA MAXIT=1000 FULLNR=.TRUE. ACURCY=1.0D-5 $END
\end{lstlisting}

The most important parameter is CISTEP, which defines the CI package used. The only
options compatible with QMCPACK are: ALDET, GUGA, and ORMAS. Depending on the
package used, additional input sections are needed.


\subsection{CI}
Configuration interaction (full CI, truncated CI, CAS-CI, etc) calculations are performed
by setting \igamess{SCFTYP=NONE} and \igamess{CITYP=GUGA,ALDET,ORMAS}. Each one of these packages 
requires further input sections, which are typically slightly different from the input sections
needed for MCSCF runs.


\subsection{GUGA: Unitary group CI package}
The GUGA package is the only alternative if one wants CSFs with GAMESS. We subsequently provide a very brief description of the input sections needed to perform MCSCF, CASCI,
truncated CI, and SOCI with this package. For a complete description of these methods and
all the options available, please refer to the GAMESS input manual.

\subsubsection{GUGA-MCSCF}
The following input section performs a CASCI calculation with a CAS that includes 8
electrons in 8 orbitals (4 DOC and 4 VAL), for example, CAS(8,8). NMCC is the number of frozen
orbitals (doubly occupied orbitals in all determinants), NDOC is the number of double
occupied orbitals in the reference determinant, NVAL is the number of singly occupied
orbitals in the reference (for spin polarized cases), and NVAL is the number of orbitals in
the active space. Since FORS is set to .TRUE., all configurations in the active space will
be included. ISTSYM defines the symmetry of the desired state.

\begin{lstlisting}[style=GAMESS]
$MCSCF CISTEP=GUGA MAXIT=1000 FULLNR=.TRUE. ACURCY=1.0D-5 $END
$DRT GROUP=C2v NMCC=0 NDOC=4 NALP=0 NVAL=4 ISTSYM=1 MXNINT= 500000 FORS=.TRUE. $END
\end{lstlisting}

\subsubsection{GUGA-CASCI}
The following input section performs a CASCI calculation with a CAS that includes 8
electrons in 8 orbitals (4 DOC and 4 VAL), for example, CAS(8,8). NFZC is the number of frozen
orbitals (doubly occupied orbitals in all determinants). All other parameters are identical
to those in the MCSCF input section.

\begin{lstlisting}[style=GAMESS]
$CIDRT GROUP=C2v NFZC=0 NDOC=4 NALP=0 NVAL=4 NPRT=2 ISTSYM=1 FORS=.TRUE. MXNINT= 500000 $END
$GUGDIA PRTTOL=0.001 CVGTOL=1.0E-5 ITERMX=1000 $END
\end{lstlisting}

\subsubsection{GUGA-truncated CI}
The following input sections will lead to a truncated CI calculation. In this particular case
it will perform a CISD calculation since IEXCIT is set to 2. Other values in IEXCIT will lead
to different CI truncations; for example, IEXCIT=4 will lead to CISDTQ. Notice that only
the lowest 30 orbitals will be included in the generation of the excited determinants in this
case. For a full CISD calculation, NVAL should be set to the total number of virtual orbitals.

\begin{lstlisting}[style=GAMESS]
$CIDRT GROUP=C2v NFZC=0 NDOC=4 NALP=0 NVAL=30 NPRT=2 ISTSYM=1 IEXCIT=2 MXNINT= 500000 $END
$GUGDIA PRTTOL=0.001 CVGTOL=1.0E-5 ITERMX=1000 $END
\end{lstlisting}

\subsubsection{GUGA-SOCI}
The following input section performs a SOCI calculation with a CAS that includes 8
electrons in 8 orbitals (4 DOC and 4 VAL), for example, CAS(8,8). Since SOCI is set to .TRUE.,
all single and double determinants from all determinants in the CAS(8,8) will be included.

\begin{lstlisting}[style=GAMESS]
$CIDRT GROUP=C2v NFZC=0 NDOC=4 NALP=0 NVAL=4 NPRT=2 ISTSYM=1 SOCI=.TRUE. NEXT=30 MXNINT= 500000 $END
$GUGDIA PRTTOL=0.001 CVGTOL=1.0E-5 ITERMX=1000 $END
\end{lstlisting}


\subsection{ECP}
To use ECPs in GAMESS, you must define a \{\ishell{\$ECP ... \$END}\} 
block. There must be a definition of a potential for every atom in the system, including
symmetry equivalent ones. In addition, they must appear in the particular order expected
by GAMESS. The following example shows an ECP input block for a single water molecule using
BFD ECPs. To turn on the use of ECPs, the option “ECP=READ” must be added to the
CONTROL input block.

\begin{lstlisting}[style=GAMESS]
$ECP
O-QMC GEN 2 1
3
6.00000000 1 9.29793903
55.78763416 3 8.86492204
-38.81978498 2 8.62925665
1
38.41914135 2 8.71924452
H-QMC GEN 0 0
3
1.000000000000 1 25.000000000000
25.000000000000 3 10.821821902641
-8.228005709676 2 9.368618758833
H-QMC
$END
\end{lstlisting}


\newpage
\section{Appendix B: convert4qmc}\label{sec:lab_adv_mol_convert4qmc}
To generate the particleset and wavefunction XML blocks required by QMCPACK in
calculations with molecular systems, the converter \texttt{convert4qmc} must be used. The converter
will read the standard output from the appropriate quantum chemistry calculation and will
generate all the necessary input for QMCPACK. In the following, we describe the main options of the
converter for GAMESS output. In general, there are three ways to use the converter depending
on the type of calculation performed. The minimum syntax for each option is shown subsequently.
For a description of the XML files produced by the converter, see Section~\ref{sec:lab_adv_mol_wf_appendix}, ``Appendix E: Wavefunction XML block."

\begin{enumerate}
  \item{For all single-determinant calculations (HF and DFT with any DFTTYP):}
  \begin{shade}
convert4qmc -gamessAscii single det.out
  \end{shade}
  \begin{itemize}
    \item{single det.out is the standard output generated by GAMESS.}
  \end{itemize}
  \item{\textit{(This option is not recommended. Use the following option to avoid mistakes.)} For 
    multideterminant calculations where the orbitals and configurations are read from different
    files (e.g., when using orbitals from a MCSCF run and configurations from a
    subsequent CI run):}
  \begin{shade}
convert4qmc -gamessAscii orbitals multidet.out -ci cicoeff multidet.out
  \end{shade}
  \begin{itemize}
    \item{orbitals\_multidet.out is the standard output from the calculation that generates the
       orbitals. cicoeff multidet.out is the standard output from the calculation that calculates 
       the CI expansion.}
  \end{itemize}
  \item{For multideterminant calculations where the orbitals and configurations are read from
    the same file, using PRTMO=.T. in the GUESS input block:}
  \begin{shade}
convert4qmc -gamessAscii multi det.out -ci multi det.out -readInitialGuess Norb
  \end{shade}
  \begin{itemize}
    \item{multi\_det.out is the standard output from the calculation that calculates the CI expansion.}
  \end{itemize}
\end{enumerate}

Options:
\begin{itemize}
\item{\textbf{-gamessAscii file.out}: Standard output of GAMESS calculation. With the exception 
of determinant configurations and coefficients in multideterminant calculations,
everything else is read from this file including atom coordinates, basis sets, SPOs, ECPs, number of electrons, multiplicity, etc.}

\item{\textbf{-ci file.out}: In multideterminant calculations, determinant configurations and 
coefficients are read from this file. Notice that SPOs are NOT read
from this file. Recognized CI packages are ALDET, GUGA, and ORMAS. Output
produced with the GUGA package MUST have the option “NPRT=2” in the CIDRT
or DRT input blocks.}

\item{\textbf{-threshold cutoff}: Cutoff in multideterminant expansion. Only configurations with
coefficients above this value are printed.}

\item{\textbf{-zeroCI}: Sets to zero the CI coefficients of all determinants, with the exception of the
first one.}

\item{\textbf{-readInitialGuess Norb}: Reads Norb initial orbitals (“INITIAL GUESS ORBITALS”) 
from GAMESS output. These are orbitals generated by the GUESS input
block and printed with the option “PRTMO=.T.”. Notice that this is useful only in
combination with the option “GUESS=MOREAD” and in cases where the orbitals
are not modified in the GAMESS calculation, e.g. CI runs. This is the recommended
option in all CI calculations.}

\item{\textbf{-NaturalOrbitals Norb}: Read Norb “NATURAL ORBITALS” from GAMESS
output. The natural orbitals must exists in the output, otherwise the code aborts.}

\item{\textbf{-add3BodyJ}: Adds 3-body Jastrow terms (e-e-I) between electron pairs (both
same spin and opposite spin terms) and all ion species in the system. The radial
function is initialized to zero, and the default cutoff is 10.0 bohr. The converter will
add a 1- and 2-body Jastrow to the wavefunction block by default.}
\end{itemize}

\subsection{Useful notes}
\begin{itemize}
  \item{The type of SPOs read by the converter depends on the type of
calculation and on the options used. By default, when neither -readInitialGuess nor
-NaturalOrbitals are used, the following orbitals are read in each case (notice that
-readInitialGuess or -NaturalOrbitals are mutually exclusive):}
  \begin{itemize}
    \item{RHF and ROHF: “EIGENVECTORS”}
    \item{MCSCF: “MCSCF OPTIMIZED ORBITALS”}
    \item{GUGA, ALDET, ORMAS: Cannot read orbitals without -readInitialGuess or -NaturalOrbitals options.}
  \end{itemize}
  \item{The SPOs and printed CI coefficients in MCSCF calculations are
not consistent in GAMESS. The printed CI coefficients correspond to the next-to-last
iteration; they are not recalculated with the final orbitals. So to get appropriate 
CI coefficients from MCSCF calculations, a subsequent CI (no SCF) calculation
is needed to produce consistent orbitals. In principle, it is possible to read the orbitals 
from the MCSCF output and the CI coefficients and configurations from the
output of the following CI calculations. This could lead to problems in principle since
GAMESS will rotate initial orbitals by default to obtain an initial guess consistent 
with the symmetry of the molecule. This last step is done by default and can
change the orbitals reported in the MCSCF calculation before the CI is performed.
To avoid this problem, we highly recommend using the preceding option \#3 to
read all the information from the output of the CI calculation; this requires the use
of “PRTMO=.T.” in the GUESS input block. Since the orbitals are printed after any
symmetry rotation, the resulting output will always be consistent.}
\end{itemize}


\newpage
\section{Appendix C: Wavefunction optimization XML block}\label{sec:lab_adv_mol_opt_appendix}

\begin{lstlisting}[style=QMCPXML,caption=``Sample XML optimization block.",label=lst:lam_xml_opt]
  <loop max="10">
    <qmc method="linear" move="pbyp" checkpoint="-1" gpu="no">
    <parameter name="blocks">     10  </parameter>
      <parameter name="warmupsteps"> 25 </parameter>
      <parameter name="steps"> 1 </parameter>
      <parameter name="substeps"> 20 </parameter>
      <parameter name="timestep"> 0.5 </parameter>
      <parameter name="samples"> 10240  </parameter>
      <cost name="energy">                   0.95 </cost>
      <cost name="unreweightedvariance">     0.0 </cost>
      <cost name="reweightedvariance">       0.05 </cost>
      <parameter name="useDrift">  yes </parameter>
      <parameter name="bigchange">10.0</parameter>
      <estimator name="LocalEnergy" hdf5="no"/>
      <parameter name="usebuffer"> yes </parameter>
      <parameter name="nonlocalpp"> yes </parameter>
      <parameter name="MinMethod">quartic</parameter>
      <parameter name="exp0">-6</parameter>
      <parameter name="alloweddifference"> 1.0e-5 </parameter>
      <parameter name="stepsize">  0.15 </parameter>
      <parameter name="nstabilizers"> 1 </parameter>
    </qmc>
  </loop>
\end{lstlisting}

Options:
\begin{itemize}
  \item{bigchange: (default 50.0) Largest parameter change allowed}
  \item{usebuffer: (default no) Save useful information during VMC}
  \item{nonlocalpp: (default no) Include nonlocal energy on 1-D min}
  \item{MinMethod: (default quartic) Method to calculate magnitude of parameter change
quartic: fit quartic polynomial to four values of the cost function obtained using reweighting 
along chosen direction linemin: direct line minimization using reweighting rescale:
no 1-D minimization. Uses Umrigars suggestions.}
  \item{stepsize: (default 0.25) Step size in either quartic or linemin methods.}
  \item{alloweddifference: (default 1e-4) Allowed increase in energy}
  \item{exp0: (default -16.0) Initial value for stabilizer (shift to diagonal of H). Actual value
of stabilizer is 10 exp0}
  \item{nstabilizers: (default 3) Number of stabilizers to try}
  \item{stabilizaterScale: (default 2.0) Increase in value of exp0 between iterations.}
  \item{max its: (default 1) Number of inner loops with same sample}
  \item{minwalkers: (default 0.3) Minimum value allowed for the ratio of effective samples
to actual number of walkers in a reweighting step. The optimization will stop if the
effective number of walkers in any reweighting calculation drops below this value. Last
set of acceptable parameters are kept.}
  \item{maxWeight: (defaul 1e6) Maximum weight allowed in reweighting. Any weight above
this value will be reset to this value.}
\end{itemize}

Recommendations:
\begin{itemize}
  \item{Set samples to equal to (\#threads)*blocks.}
  \item{Set steps to 1. Use substeps to control correlation between samples.}
  \item{For cases where equilibration is slow, increase both substeps and warmupsteps.}
  \item{For hard cases (e.g., simultaneous optimization of long MSD and 3-Body J), set exp0
to 0 and do a single inner iteration (max its=1) per sample of configurations.}
\end{itemize}


\newpage
\section{Appendix D: VMC and DMC XML block}\label{sec:lab_adv_mol_vmcdmc_appendix}

\begin{lstlisting}[style=QMCPXML,caption=``Sample XML blocks for VMC and DMC calculations.",label=lst:lam_xml_vmc_dmc]
  <qmc method="vmc" move="pbyp" checkpoint="-1">
    <parameter name="useDrift">yes</parameter>
    <parameter name="warmupsteps">100</parameter>
    <parameter name="blocks">100</parameter>
    <parameter name="steps">1</parameter>
    <parameter name="substeps">20</parameter>
    <parameter name="walkers">30</parameter>
    <parameter name="timestep">0.3</parameter>
    <estimator name="LocalEnergy" hdf5="no"/>
  </qmc>
  <qmc method="dmc" move="pbyp" checkpoint="-1">
    <parameter name="nonlocalmoves">yes</parameter>
    <parameter name="targetWalkers">1920</parameter>
    <parameter name="blocks">100</parameter>
    <parameter name="steps">100</parameter>
    <parameter name="timestep">0.1</parameter>
    <estimator name="LocalEnergy" hdf5="no"/>
  </qmc>
\end{lstlisting}

General Options:
\begin{itemize}
\item{\textbf{move}: (default ``walker”) Type of electron move. Options: ``pbyp” and ``walker.”}
\item{\textbf{checkpoint}: (default ``-1”) (If > 0) Generate checkpoint files with given frequency.
The calculations can be restarted/continued with the produced checkpoint files.}
\item{\textbf{useDrift}: (default ``yes”) Defines the sampling mode. useDrift = ``yes” will
use Langevin acceleration to sample the VMC and DMC distributions, while
useDrift=``no” will use random displacements in a box.}
\item{\textbf{warmupSteps}: (default 0) Number of steps warmup steps at the beginning of the
calculation. No output is produced for these steps.}
\item{\textbf{blocks}: (default 1) Number of blocks (outer loop).}
\item{\textbf{steps}: (default 1) Number of steps per blocks (middle loop).}
\item{\textbf{sub steps}: (default 1) Number of substeps per step (inner loop). During substeps,
the local energy is not evaluated in VMC calculations, which leads to faster execution.
In VMC calculations, set substeps to the average autocorrelation time of the desired
quantity.}
\item{\textbf{time step}: (default 0.1) Electronic time step in bohr.}
\item{\textbf{samples}: (default 0) Number of walker configurations saved during the current 
calculation.}
\item{\textbf{walkers}: (default \#threads) In VMC, sets the number of walkers per node. The total
number of walkers in the calculation will be equal to walkers*(\# nodes).}
\end{itemize}

Options unique to DMC:
\begin{itemize}
\item{\textbf{targetWalkers}: (default \#walkers from previous calculation, e.g., VMC). Sets the
target number of walkers. The actual population of walkers will fluctuate around this
value. The walkers will be distributed across all the nodes in the calculation. On a
given node, the walkers are split across all the threads in the system.}
\item{\textbf{nonlocalmoves}: (default ``no”) Set to ``yes” to turns on the use of Casula’s T-moves.}
\end{itemize}


\newpage
\section{Appendix E: Wavefunction XML block}\label{sec:lab_adv_mol_wf_appendix}

\begin{lstlisting}[style=QMCPXML,caption=``Basic framework for a single-determinant determinantset XML block.",label=lst:lam_xml_determinantset]
<wavefunction name="psi0" target="e">
  <determinantset type="MolecularOrbital" name="LCAOBSet"
   source="ion0" transform="yes">
    <basisset name="LCAOBSet">
      <atomicBasisSet name="Gaussian-G2" angular="cartesian" type="Gaussian" elementType="O" normalized="no">
      ...
      </atomicBasisSet>
    </basisset>
    <slaterdeterminant>
      <determinant id="updet" size="4">
        <occupation mode="ground"/>
        <coefficient size="57" id="updetC">
        ...
        </coefficient>
      </determinant>
      <determinant id="downdet" size="4">
        <occupation mode="ground"/>
        <coefficient size="57" id="downdetC">
        ...
        </coefficient>
      </determinant>
    </slaterdeterminant>

  </determinantset>

  <jastrow name="J2" type="Two-Body" function="Bspline" print="yes">
  ...
  </jastrow>

</wavefunction>
\end{lstlisting}

In this section we describe the basic format of a QMCPACK wavefunction XML block.
Everything listed in this section is generated by the appropriate converter tools. Little to
no modification is needed when performing standard QMC calculations. As a result, this
section is meant mainly for illustration purposes. Only experts should attempt to modify
these files (with very few exceptions like the cutoff of CI coefficients and the cutoff in Jastrow
functions) since changes can lead to unexpected results.

A QMCPACK wavefunction XML block is a combination of a determinantset, which
contains the antisymmetric part of the wavefunction and one or more Jastrow blocks.
The syntax of the antisymmetric block depends on whether the wavefunction is a single
determinant or a multideterminant expansion. Listing \ref{lst:lam_xml_determinantset} 
shows the general structure of the
single-determinant case. The determinantset block is composed of a basisset block, which
defines the atomic orbital basis set, and a slaterdeterminant block, which defines the SPOs and occupation numbers of the Slater determinant.
% Listing \ref{lst:lam_xml_basisset} 
%shows a section
%of a basisset block for a gold atom. The structure of this block is rigid and should not
%be modified.
 Listing \ref{lst:lam_xml_slaterdeterminant} shows a (piece of a) sample of a 
slaterdeterminant block. The
slaterdeterminant block consists of two determinant blocks, one for each electron spin. The
parameter ``size” in the determinant block refers to the number of SPOs
present while the ``size” parameter in the coefficient block refers to the number of atomic
basis functions per SPO.

\begin{lstlisting}[style=QMCPXML,caption=``Sample XML block for the single Slater determinant case.",label=lst:lam_xml_slaterdeterminant]
      <slaterdeterminant>
        <determinant id="updet" size="5">
          <occupation mode="ground"/>
          <coefficient size="134" id="updetC">
  9.55471000000000e-01 -3.87000000000000e-04  6.51140000000000e-02  2.17700000000000e-03
  1.43900000000000e-03  4.00000000000000e-06 -4.58000000000000e-04 -5.20000000000000e-05
 -2.40000000000000e-05  6.00000000000000e-06 -0.00000000000000e+00 -0.00000000000000e+00
 -0.00000000000000e+00 -0.00000000000000e+00 -0.00000000000000e+00 -0.00000000000000e+00
 -0.00000000000000e+00 -0.00000000000000e+00 -0.00000000000000e+00 -0.00000000000000e+00
 -0.00000000000000e+00 -0.00000000000000e+00 -0.00000000000000e+00 -0.00000000000000e+00
 -0.00000000000000e+00 -0.00000000000000e+00 -0.00000000000000e+00 -0.00000000000000e+00
 -0.00000000000000e+00 -0.00000000000000e+00 -0.00000000000000e+00 -0.00000000000000e+00
 -0.00000000000000e+00 -0.00000000000000e+00 -0.00000000000000e+00 -0.00000000000000e+00
 -0.00000000000000e+00 -5.26000000000000e-04  2.63000000000000e-04  2.63000000000000e-04
 -0.00000000000000e+00 -0.00000000000000e+00 -0.00000000000000e+00 -1.27000000000000e-04
  6.30000000000000e-05  6.30000000000000e-05 -0.00000000000000e+00 -0.00000000000000e+00
 -0.00000000000000e+00 -3.20000000000000e-05  1.60000000000000e-05  1.60000000000000e-05
 -0.00000000000000e+00 -0.00000000000000e+00 -0.00000000000000e+00  7.00000000000000e-06
\end{lstlisting}
Listing \ref{lam_xml_multideterminant} shows the general structure of the multideterminant case. 
Similar to the
single-determinant case, the determinantset must contain a basisset block. This definition is
identical to the one described previously. In this case, the definition of the SPOs
must be done independently from the definition of the determinant configurations; the latter
is done in the sposet block, while the former is done on the multideterminant block. Notice
that two sposet sets must be defined, one for each electron spin. The name of each sposet set
is required in the definition of the multideterminant block. The determinants are defined in
terms of occupation numbers based on these orbitals.

\begin{lstlisting}[style=QMCPXML,caption=``Basic framework for a multideterminant determinantset XML block.",label=lam_xml_multideterminant]
  <wavefunction id="psi0" target="e">
    <determinantset name="LCAOBSet" type="MolecularOrbital" transform="yes" source="ion0">
      <basisset name="LCAOBSet">
        <atomicBasisSet name="Gaussian-G2" angular="cartesian" type="Gaussian" elementType="O" normalized="no">
        ...
        </atomicBasisSet>
        ...
      </basisset>
      <sposet basisset="LCAOBSet" name="spo-up" size="8">
        <occupation mode="ground"/>
        <coefficient size="40" id="updetC">
        ...
</coefficient>
      </sposet>
      <sposet basisset="LCAOBSet" name="spo-dn" size="8">
        <occupation mode="ground"/>
        <coefficient size="40" id="downdetC">
        ...
      </coefficient>
      </sposet>
      <multideterminant optimize="yes" spo_up="spo-up" spo_dn="spo-dn">
        <detlist size="97" type="CSF" nca="0" ncb="0" nea="4" neb="4" nstates="8" cutoff="0.001">
          <csf id="CSFcoeff_0" exctLvl="0" coeff="0.984378" qchem_coeff="0.984378" occ="22220000">
            <det id="csf_0-0" coeff="1" alpha="11110000" beta="11110000"/>
          </csf>
          ...
        </detlist>
      </multideterminant>
    </determinantset>
    <jastrow name="J2" type="Two-Body" function="Bspline" print="yes">
    ...
    </jastrow>
  </wavefunction>
\end{lstlisting}

There are various options in the multideterminant block that users should be aware of.
\begin{itemize}
  \item{cutoff: (IMPORTANT! ) Only configurations with (absolute value) “qchem coeff”
larger than this value will be read by QMCPACK.}
  \item{optimize: Turn on/off the optimization of linear CI coefficients.}
  \item{coeff: (in csf ) Current coefficient of given configuration. Gets updated during 
wavefunction optimization.}
  \item{qchem coeff: (in csf ) Original coefficient of given configuration from GAMESS 
calculation. This is used when applying a cutoff to the configurations read from the file.
The cutoff is applied on this parameter and not on the optimized coefficient.}
  \item{nca and nab: Number of core orbitals for up/down electrons. A core orbital is an
orbital that is doubly occupied in all determinant configurations, not to be confused
with core electrons. These are not explicitly listed on the definition of configurations.}
  \item{nea and neb: Number of up/down active electrons (those being explicitly correlated).}
  \item{nstates: Number of correlated orbitals}.
  \item{size (in detlist ): Contains the number of configurations in the list.}
\end{itemize}
The remaining part of the determinantset block is the definition of Jastrow factor. Any
number of these can be defined. Figure \ref{fig:lam_xml_jastrow} shows a sample Jastrow 
block including 1-, 2- and 3-body terms. This is the standard block produced by 
\texttt{convert4qmc} with the option -add3BodyJ (this particular example is for a water molecule). 
Optimization of individual radial functions can be turned on/off using the “optimize” 
parameter. It can be added to any coefficients block, even though it is currently not 
present in the J1 and J2 blocks.

\begin{lstlisting}[style=QMCPXML,caption=``Sample Jastrow XML block.",label=fig:lam_xml_jastrow]
<jastrow name="J2" type="Two-Body" function="Bspline" print="yes">
      <correlation rcut="10" size="10" speciesA="u" speciesB="u">
        <coefficients id="uu" type="Array">0.0 0.0 0.0 0.0 0.0 0.0 0.0 0.0 0.0 0.0</coefficients>
      </correlation>
      <correlation rcut="10" size="10" speciesA="u" speciesB="d">
        <coefficients id="ud" type="Array">0.0 0.0 0.0 0.0 0.0 0.0 0.0 0.0 0.0 0.0</coefficients>
      </correlation>
    </jastrow>
    <jastrow name="J1" type="One-Body" function="Bspline" source="ion0" print="yes">
      <correlation rcut="10" size="10" cusp="0" elementType="O">
        <coefficients id="eO" type="Array">0.0 0.0 0.0 0.0 0.0 0.0 0.0 0.0 0.0 0.0</coefficients>
      </correlation>
      <correlation rcut="10" size="10" cusp="0" elementType="H">
        <coefficients id="eH" type="Array">0.0 0.0 0.0 0.0 0.0 0.0 0.0 0.0 0.0 0.0</coefficients>
      </correlation>
    </jastrow>
    <jastrow name="J3" type="eeI" function="polynomial" source="ion0" print="yes">
      <correlation ispecies="O" especies="u" isize="3" esize="3" rcut="10">
        <coefficients id="uuO" type="Array" optimize="yes">
        </coefficients>
      </correlation>
      <correlation ispecies="O" especies1="u" especies2="d" isize="3" esize="3" rcut="10">
        <coefficients id="udO" type="Array" optimize="yes">
        </coefficients>
      </correlation>
      <correlation ispecies="H" especies="u" isize="3" esize="3" rcut="10">
        <coefficients id="uuH" type="Array" optimize="yes">
        </coefficients>
      </correlation>
      <correlation ispecies="H" especies1="u" especies2="d" isize="3" esize="3" rcut="10">
        <coefficients id="udH" type="Array" optimize="yes">
        </coefficients>
      </correlation>
      </jastrow>
\end{lstlisting}

This training assumes basic familiarity with the UNIX operating system. In particular,
we use simple scripts written in “csh.” In addition, we assume you have obtained
all the necessary files and executables and that the training files are located
at \$\{TRAINING TOP\}.

The goal of this training is not only to familiarize you with the execution and
options in QMCPACK but also to introduce you to important concepts in QMC calculations and many-body electronic structure calculations.



%%% Local Variables:
%%% mode: latex
%%% TeX-master: "just_labs"
%%% End:

\chapter{Lab 4: Condensed matter calculations} % changed title to match schedule, can revert if desired
%\chapter{Lab 4: Using PWSCF and QMCPACK to perform total energy calculations of
%condensed systems}


\hide{
\begin{flushleft}
\textbf{Lab author: Luke Shulenburger}\footnote{Sandia National Laboratories is a multiprogram
laboratory managed and operated by Sandia Corporation, a wholly owned
subsidiary of Lockheed Martin Corporation, for the U.S. Department of Energy's
National Nuclear Security Administration under Contract No.
DE-AC04-94AL85000.}

\textbf{Creation date: July 17, 2014}
\end{flushleft}
}


\section{Topics covered in this lab}
\begin{itemize}
  \item{Tiling DFT primitive cells into QMC supercells}
  \item{Reducing finite-size errors via extrapolation}
  \item{Reducing finite-size errors via averaging over twisted boundary conditions}
  \item{Using the B-spline mesh factor to reduce memory requirements}
  \item{Using a coarsely resolved vacuum buffer region to reduce memory requirements}
  \item{Calculating the DMC total energies of representative 2D and 3D extended systems}
\end{itemize}



\section{Lab directories and files}

\footnotesize
\begin{verbatim}
labs/lab4_condensed_matter/
├── Be-2at-setup.py           - DFT only for prim to conv cell
├── Be-2at-qmc.py             - QMC only for prim to conv cell
├── Be-16at-qmc.py            - DFT and QMC for prim to 16 atom cell
├── graphene-setup.py         - DFT and OPT for graphene
├── graphene-loop-mesh.py     - VMC scan over orbital bspline mesh factors
├── graphene-final.py         - DMC for final meshfactor
└── pseudopotentials          - pseudopotential directory
    ├── Be.ncpp                 - Be PP for Quantum ESPRESSO
    ├── Be.xml                  - Be PP for QMCPACK
    ├── C.BFD.upf               - C  PP for Quantum ESPRESSO
    └── C.BFD.xml               - C  PP for QMCPACK
\end{verbatim}
\normalsize

The goal of this lab is to introduce you to the somewhat specialized problems involved in performing DMC calculations on condensed matter as opposed to the atoms and molecules that were the focus of the preceding labs.   Calculations will be performed on two different systems.  Firstly, we will perform a series of calculations on BCC beryllium, focusing on the necessary methodology to limit finite-size effects.  Secondly, we will perform calculations on graphene as an example of a system where QMCPACK’s capability to handle cases with mixed periodic and open boundary conditions is useful.  This example will also focus on strategies to limit memory usage for such systems.
All of the calculations performed in this lab will use the Nexus workflow management system, which vastly simplifies the process by automating the steps of generating trial wavefunctions and performing DMC calculations.

\newcommand{\vp}{\mathbf{a}^\text{p}}
\newcommand{\vs}{\mathbf{a}^\text{s}} 
\newcommand{\Smat}{\mathbf{S}}
\section{Preliminaries}
For any DMC calculation, we must start with a trial wavefunction. As is typical for our calculations of condensed matter, we will produce this wavefunction using DFT.  Specifically, we will use QE to generate a Slater determinant of SPOs.  This is done as a three-step process.  First, we calculate the converged charge density by performing a DFT calculation with a fine grid of k-points to fully sample the Brillouin zone.  Next, a non-self- consistent calculation is performed at the specific k-points needed for the supercell and twists needed in the DMC calculation (more on this later).  Finally, a wavefunction is converted from the binary representation used by QE to the portable hdf5 representation used by QMCPACK.

The choice of k-points necessary to generate the wavefunctions depends on both the supercell chosen for the DMC calculation and by the supercell twist vectors needed.  Recall that the wavefunction in a plane-wave DFT calculation is written using Bloch's theorem as:
\begin{equation}
\Psi(\vec{r}) = e^{i\vec{k}\cdot\vec{r}}u(\vec{r})\:,
\end{equation}
where $\vec{k}$ is confined to the first Brillouin zone of the cell chosen and $u(\vec{r})$ is periodic in this simulation cell.  A plane-wave DFT calculation stores the periodic part of the wavefunction as a linear combination of plane waves for each SPO at all k-points selected.  The symmetry of the system allows us to generate an arbitrary supercell of the primitive cell as follows:  Consider the set of primitive lattice vectors, $ \{ \mathbf{a}^p_1, \mathbf{a}^p_2,
\mathbf{a}^p_3\} $.  We may write these vectors in a matrix, $\mathbf{L}_p$, the
rows of which are the primitive lattice vectors.  Consider a nonsingular
matrix of integers, $\Smat$.  A corresponding set of supercell lattice
vectors, $\{\mathbf{a}^s_1, \mathbf{a}^s_2, \mathbf{a}^s_3\}$, can be constructed by the matrix
product 
\begin{equation}
\mathbf{a}^s_i = S_{ij} \mathbf{a}^p_j]\:.
\end{equation}
If the primitive cell contains $N_p$ atoms, the supercell will then
contain $N_s = |\det(\Smat)| N_p$ atoms.

Now, the wavefunciton at any point in this new supercell can be related to the wavefunction in the primitive cell by  finding the linear combination of primitive lattice vectors that maps this point back to the primitive cell:
\begin{equation}
\vec{r}' = \vec{r} + x \mathbf{a}^p_1 + y \mathbf{a}^p_2 + z\mathbf{a}^p_3 = \vec{r} + \vec{T}\:,
\end{equation}
where $x, y, z$ are integers.   Now the wavefunction in the supercell at point $\vec{r}'$ can be written in terms of the wavefunction in the primitive cell at $\vec{r}'$ as:
\begin{equation}
\Psi(\vec{r}) = \Psi(\vec{r}') e^{i \vec{T} \cdot \vec{k}}\:,
\end{equation}
where $\vec{k}$ is confined to the first Brillouin zone of the primitive cell.  We have also chosen the supercell twist vector, which places a constraint on the form of the wavefunction in the supercell.  The combination of these two constraints allows us to identify family of N k-points in the primitive cell that satisfy the constraints.  Thus, for a given supercell tiling matrix and twist angle, we can write the wavefunction everywhere in the supercell by knowing the wavefunction a N k-points in the primitive cell.  This means that the memory necessary to store the wavefunction in a supercell is only linear in the size of the supercell rather than the quadratic cost if symmetry were neglected.

\section{Total energy of BCC beryllium}

When performing calculations of periodic solids with QMC, it is essential to work with a reasonable size supercell rather than the primitive cells that are common in mean field calculations.  Specifically, all of the finite-size correction schemes discussed in the morning require that the exchange-correlation hole be considerably smaller than the periodic simulation cell.  Additionally, finite-size effects are lessened as the distance between the electrons in the cell and their periodic images increases, so it is advantageous to generate supercells that are as spherical as possible to maximize this distance.  However, a competing consideration is that when calculating total energies we often want to extrapolate the energy per particle to the thermodynamic limit by means of the following formula in three dimensions:
\begin{equation}
E_{\inf} = C + E_{N}/N\:.
\end{equation}
This formula derived assuming the shape of the supercells is consistent (more specifically that the periodic distances scale uniformly with system size), meaning we will need to do a uniform tiling, that is, $2\times2\times2$, $3\times3\times3$, etc.  As a $3\times3\times3$ tiling is 27 times larger than the supercell and the practical limit of DMC is on the order of 200 atoms (depending on Z), sometimes it is advantageous to choose a less spherical supercell with fewer atoms rather than a more spherical one that is too expensive to tile.

In the case of a BCC crystal, it is possible to tile the one atom primitive cell to a cubic supercell only by doubling the number of electrons.  This is the best possible combination of a small number of atoms that can be tiled and a regular box that maximizes the distance between periodic images.  We will need to determine the tiling matrix S that generates this cubic supercell by solving the following equation for the coefficients of the S matrix:
\begin{equation}
 \left[\begin{array}{rrr}
  1 & 0 & 0 \\
  0 & 1 & 0 \\
  0 & 0 & 1 
  \end{array}\right] =  \left[\begin{array}{rrr}
  s_{11} & s_{12} & s_{13} \\
  s_{21} & s_{22} & s_{23} \\
  s_{31} & s_{32} & s_{33} 
  \end{array}\right] \cdot 
\left[\begin{array}{rrr}
  0.5 &  0.5 & -0.5 \\
 -0.5 &  0.5 &  0.5 \\
  0.5 & -0.5 &  0.5
\end{array}\right]\:. 
\end{equation}

We will now use Nexus to generate the trial wavefunction for this BCC beryllium.

Fortunately, the Nexus will handle determination of the proper k-vectors given the tiling matrix.  All that is needed is to place the tiling matrix in the \texttt{Be-2at-setup.py} file.   Now the definition of the physical system is

\begin{lstlisting}[style=Python]
    bcc_Be = generate_physical_system(
        lattice    = 'cubic',
        cell       = 'primitive',
        centering  = 'I',
        atoms      = 'Be',
        constants  = 3.490,
        units      = 'A',
        net_charge = 0,
        net_spin   = 0,
        Be         = 2,
        tiling     = [[a,b,c],[d,e,f],[g,h,i]],
        kgrid      = kgrid,
        kshift     = (.5,.5,.5)
        )
\end{lstlisting}
where the tiling line should be replaced with the preceding row major tiling matrix.  This script file will now perform a converged DFT calculation to generate the charge density in a directory called \texttt{bcc-beryllium/scf} and perform a non-self-consistend DFT calculation to generate SPOs in the directory \texttt{bcc-beryllium/nscf}.  Fortunately, Nexus will calculate the required k-points needed to tile the wavefunction to the supercell, so all that is necessary is the granularity of the supercell twists and whether this grid is shifted from the origin.  Once this is finished, it performs the conversion from pwscf's binary format to the hdf5 format used by QMCPACK.  Finally, it will optimize the coefficients of 1-body and 2-body Jastrow factors in the supercell defined by the tiling matrix.

Run these calculations by executing the script \texttt{Be-2at-setup.py}.  You will notice the small calculations required to generate the wavefunction of beryllium in a one-atom cell are rather inefficient to run on a high-performance computer such as vesta in terms of the time spent doing calculations versus time waiting on the scheduler and booting compute nodes.  One of the benefits of the portable HDF format that is used by QMCPACK is that you can generate data like wavefunctions on a local workstation or other convenient resource and use high-performance clusters for the more expensive QMC calculations.

In this case, the wavefunction is generated in the directory \texttt{bcc-beryllium/nscf-2at\_222/pwscf\_ output} in a file called \texttt{pwscf.pwscf.h5}.  For debugging purposes, it can be useful to verify that the contents of this file are what you expect.  For instance, you can use the tool \texttt{h5ls} to check the geometry of the cell where the DFT calculations were performed or the number of k-points or electrons in the calculation.  This is done with the command h5ls -d pwscf.pwscf.h5/supercell or h5ls -d pwscf.pwscf.h5/electrons.

In the course of running \texttt{Be-2at-setup.py}, you will get an error when attempting to perform the VMC and wavefunction optimization calculations.  This is because the wavefunction has generated supercell twists of the form (+/- 1/4, +/- 1/4, +/- 1/4).  In the case that the supercell twist contains only 0 or 1/2, it is possible to operate entirely with real arithmetic.  The executable that has been indicated in \texttt{Be-2at-setup.py} was compiled for this case.  Note that where possible, the memory use is a factor of two less than the general case and the calculations are somewhat faster.  However, it is often necessary to perform calculations away from these special twist angles to reduce finite-size effects.  To fix this, delete the directory \texttt{bcc-beryllium/opt-2at}, change the line near the top of \texttt{Be-2at-setup.py} from 
\begin{lstlisting}[style=Python]
qmcpack    = '/soft/applications/qmcpack/Binaries/qmcpack'
\end{lstlisting}
to
\begin{lstlisting}[style=Python]
qmcpack    = '/soft/applications/qmcpack/Binaries/qmcpack_comp'
\end{lstlisting}
and rerun the script.

When the optimization calculation has finished, check that everything has proceeded correctly by looking at the output in the \texttt{opt-2at} directory.  Firstly, you can grep the output file for Delta to see if the cost function has indeed been decreasing during the optimization.  You should find something like this:
\begin{shade}
 OldCost: 4.8789147e-02 NewCost: 4.0695360e-02 Delta Cost:-8.0937871e-03
 OldCost: 3.8507795e-02 NewCost: 3.8338486e-02 Delta Cost:-1.6930674e-04
 OldCost: 4.1079105e-02 NewCost: 4.0898345e-02 Delta Cost:-1.8076319e-04
 OldCost: 4.2681333e-02 NewCost: 4.2356598e-02 Delta Cost:-3.2473514e-04
 OldCost: 3.9168577e-02 NewCost: 3.8552883e-02 Delta Cost:-6.1569350e-04
 OldCost: 4.2176276e-02 NewCost: 4.2083371e-02 Delta Cost:-9.2903058e-05
 OldCost: 4.3977361e-02 NewCost: 4.2865751e-02 Delta Cost:-1.11161830-03
 OldCost: 4.1420944e-02 NewCost: 4.0779569e-02 Delta Cost:-6.4137501e-04
\end{shade}
which shows that the starting wavefunction was fairly good and that most of the optimization occurred in the first step.  Confirm this by using \texttt{qmca} to look at how the energy and variance changed over the course of the calculation with the command: \texttt{qmca -q ev -e 10 *.scalar.dat} executed in the \texttt{opt-2at directory}.  You should get output like the following:
\begin{shade}
                 LocalEnergy               Variance             ratio
opt  series 0  -2.159139 +/- 0.001897   0.047343 +/- 0.000758   0.0219 
opt  series 1  -2.163752 +/- 0.001305   0.039389 +/- 0.000666   0.0182 
opt  series 2  -2.160913 +/- 0.001347   0.040879 +/- 0.000682   0.0189 
opt  series 3  -2.162043 +/- 0.001223   0.041183 +/- 0.001250   0.0190 
opt  series 4  -2.162441 +/- 0.000865   0.039597 +/- 0.000342   0.0183 
opt  series 5  -2.161287 +/- 0.000732   0.039954 +/- 0.000498   0.0185 
opt  series 6  -2.163458 +/- 0.000973   0.044431 +/- 0.003583   0.0205 
opt  series 7  -2.163495 +/- 0.001027   0.040783 +/- 0.000413   0.0189 
\end{shade}

Now that the optimization has completed successfully, we can perform DMC calculations.  The first goal of the calculations will be to try to eliminate the 1-body finite-size effects by twist averaging.  The script \texttt{Be-2at-qmc.py} has the necessary input.  Note that on line 42 two twist grids are specified, (2,2,2) and (3,3,3).  Change the tiling matrix in this input file as in \texttt{Be-2at-qmc.py} and start the calculations.  Note that this workflow takes advantage of QMCPACK's capability to group jobs.  If you look in the directory \texttt{dmc-2at\_222} at the job submission script (\texttt{dmc.qsub.in}), you will note that rather than operating on an XML input file, \texttt{qmcapp} is targeting a text file called \texttt{dmc.in}.  This file is a simple text file that contains the names of the eight XML input files needed for this job, one for each twist.  When operated in this mode, QMCPACK will use MPI groups to run multiple copies of itself within the same MPI context.  This is often useful both in terms of organizing calculations and for taking advantage of the large job sizes that computer centers often encourage.

The DMC calculations in this case are designed to complete in a few minutes.  When they have finished running, first look at the \texttt{scalar.dat} files corresponding to the DMC calculations at the various twists in \texttt{dmc-2at\_222}.  Using a command such as \texttt{qmca -q ev -e 32 *.s001.scalar.dat} (with a suitably chosen number of blocks for the equilibration), you will see that the DMC energy in each calculation is nearly identical within the statistical uncertainty of the calculations.  In the case of a large supercell, this is often indicative of a situation where the Brillouin zone is so small that the 1-body finite-size effects are nearly converged without any twist averaging.  In this case, however, this is because of the symmetry of the system.  For this cubic supercell, all of the twist angles chosen in this shifted $2\times2\times2$ grid are equivalent by symmetry.  In the case where substantial resources are required to equilibrate the DMC calculations, it can be beneficial to avoid repeating such twists and instead simply weight them properly.  In this case, however, where the equilibration is inexpensive, there is no benefit to adding such complexity as the calculations can simply be averaged together and the result is equivalent to performing a single longer calculation.

Using the command \texttt{qmc -a -q ev -e 16 *.s001.scalar.dat}, average the DMC energies in \texttt{dmc-2at\_222 and dmc-2at\_333} to see whether the 1-body finite-size effects are converged with a $3\times3\times3$ grid of twists.  When using beryllium as a metal, the convergence is quite poor (0.025 Ha/Be or 0.7 eV/Be).  If this were a production calculation it would be necessary to perform calculations on much larger grids of supercell twists to eliminate the 1-body finite-size effects.

In this case there are several other calculations that would warrant a high priority.  Script \texttt{Be-16at-qmc.py} has been provided in which you can input the appropriate tiling matrix for a 16-atom cell and perform calculations to estimate the 2-body finite-size effects, which will also be quite large in the 2-atom calculations.  This script will take approximately 30 minutes to run to completion, so depending on your interest,  you can either run it or work to modify the scripts to address the other technical issues that would be necessary for a production calculation such as calculating the population bias or the time step error in the DMC calculations.  

Another useful exercise would be to attempt to validate this PP by calculating the ionization potential and electron affinity of the isolated atom and compare it with the experimental values:  IP = 9.3227 eV , EA = 2.4 eV.

\section{Handling a 2D system: graphene}
In this section we examine a calculation of an isolated sheet of graphene. Because graphene is a 2D system, we will take advantage of QMCPACK's capability to mix periodic and open boundary conditions to eliminate and spurious interaction of the sheet with its images in the z direction.  Run the script \texttt{graphene-setup.py}, which will generate the wavefunction and optimize one and two body jastrow factors.  In the script; notice line 160: bconds = 'ppn' in the generate\_qmcpack function, which specifies this mix of open and periodic boundary conditions.  Consequently, the atoms will need to be kept away from this open boundary in the z direction as the electronic wavefunction will not be defined outside of the simulation box in this direction.  For this reason, all of the atom positions at the beginning of the file have z coordinates 7.5.  At this point, run the script \texttt{graphene-setup.py}.

Aside from the change in boundary conditions, the main thing that distinguishes this kind of calculation from the previous beryllium example is the large amount of vacuum in the cell.  Although this is a very small calculation designed to run quickly in the tutorial, in general a more converged calculation would quickly become memory limited on an architecture like BG/Q.  When the initial wavefunction optimization has completed to your satisfaction, run the script \texttt{graphene-loop-mesh.py}.  This examines within VMC an approach to reducing the memory required to store the wavefunction.  In \texttt{graphene-loop-mesh.py}, the spacing between the B-spline points is varied uniformly.  The mesh spacing is a prefactor to the linear spacing between the spline points, so the memory use goes as the cube of the meshfactor.  When you run the calculations, examine the \texttt{.s000.scalar.dat} files with \texttt{qmca} to determine the lowest possible mesh spacing that preserves both the VMC energy and the variance.  

Finally, edit the file \texttt{graphene-final.py}, which will perform two DMC calculations.  In the first, (qmc1) replace the following lines:
\begin{lstlisting}[style=Python]
    meshfactor   = xxx,
    precision    = '---',
\end{lstlisting}
with the values you have determined will perform the calculation with as small as possible wavefunction.  Note that we can also use single precision arithmetic to store the wavefunction by specifying precision=`single.'  When you run the script, compare the output of the two DMC calculations in terms of energy and variance.  Also, see if you can calculate the fraction of memory that you were able to save by using a meshfactor other than 1 and single precision arithmetic.

\section{Conclusion}
Upon completion of this lab, you should be able to use Nexus to perform DMC calculations on periodic solids when provided with a PP.  You should also be able to reduce the size of the wavefunction in a solid-state calculation in cases where memory is a limiting factor.

\hide{
\section{Acknowledgment}
 This tutorial was created with support from Sandia National Laboratories.

 Sandia National Laboratories is a multiprogram laboratory managed and operated by
 Sandia Corporation, a wholly owned subsidiary of Lockheed Martin Corporation, for
 the U.S. Department of Energy's National Nuclear Security Administration under
 Contract No. DE-AC04-94AL85000.
}

\chapter{Lab 5: Excited State Calculations}
\label{chap:excited}

\hide{
	\begin{flushleft}
		\textbf{Lab author: Kayahan Saritas}\footnote{Oak Ridge National Laboratory}
		
		\textbf{Creation date: November 29, 2018}
	\end{flushleft}
}

\section{Topics covered in this Lab}
\begin{itemize}
	\item{Tiling DFT primitive cells into optimal QMC supercells}
	\item{Fundamentals of  between neutral and charged calculations}
	\item{Calculating quasiparticle excitation energies of condensed matter systems}
	\item{Calculating optical excitation energies of condensed matter systems}
\end{itemize}

\section{Lab directories and files}

\begin{verbatim}
labs/lab5_excited_properties/
├── band.py           - Band structure calculation for Carbon Diamond
├── optical.py        - VMC optical gap calculation using the tiling matrix from band.py
├── quasiparticle.py  - VMC quasiparticle gap calculation using the tiling matrix from band.py
└── pseudopotentials      - pseudopotential directory
    ├── C.BFD.upf         - C PP for Quantum ESPRESSO
    └── C.BFD.xml         - C PP for QMCPACK
\end{verbatim}

The goal of this lab is to perform neutral and charged excitation calculations in condensed matter systems using QMCPACK. 
Throughout this lab, a working knowledge of \textit{Lab4 Condensed Matter Calculations} is assumed. 
First, we will introduce the concepts of neutral and charged excitations. 
We will briefly discuss these in relation to the specific experimental studies that must be used to benchmark DMC results. 
Secondly, we will perform charged (quasiparticle) and neutral (optical) excitations calculations on C-diamond.

\section{Basics and excited state experiments}
Although VMC and DMC methods are better suited for studying ground state properties of materials, they can still provide useful information regarding the excited states. 
Unlike the applications of band structure theory such as DFT and GW, it is more challenging to obtain the complete excitation spectra using DMC. 
However, it is relatively straightforward to calculate the band gap minimum of a condensed matter system using DMC. 

We will briefly discuss the two main ways of obtaining the band gap minimum through experiments: photoemission and absorption studies.  
The energy required to remove an electron from a neutral system is called the ionization potential (IP), which is available from direct photoemission experiments. 
In contrast, the emission energy of a negatively charged system (or the energy required to convert a negatively charged system to a neutral system) known as electron affinity (EA) and it is available from inverse photoemission experiments. 
Outline of these experiments are shown in Fig.~\ref{fig:lab_ex_exp}. 

Following the explanation in the previous paragraph and Fig.~\ref{fig:lab_ex_exp}, the \textit{quasiparticle} band gap of a material can be defined as:
\begin{equation}
	E_g=EA-IP=(E_{N+1}^{CBM}-E_{N}^{K'})-(E_{N}^{K'}-E_{N-1}^{VBM})=E_{N+1}^{CBM}+E_{N-1}^{VBM}-2*E_{N}^{K'}\label{eq:qp}
\end{equation}
where $N$ is the number of electrons in the neutral system and $E_{N}$ is the ground state energy of the neutral system. 
CBM and VBM stand for the conduction band minimum and valence band maximum, respectively. K' can formally be arbitrary at the infinite limit.
However, in practical calculations, a supertwist which accommodates both CBM and VBM can be more efficient in terms of computational time and systematic finite size error cancellation. 
In the literature, the quasiparticle gap is also called the electronic gap. 
The term electronic comes from the fact that in both photoemission experiments, it is assumed that the perturbed electron is non-interacting with the sample. 

\begin{figure}
	\centering
	\includegraphics[width=0.5\textwidth]{./figures/lab_excited_experiments}
	\caption{Direct and inverse photoemission experiments involve charged excitations, whereas optical absorption experiments involves excitation that are just enough to be excited to the conduction band. From ref. \cite{Onida2002a}}
	\label{fig:lab_ex_exp}
\end{figure}

Additionally, one can also perform absorption experiments where electrons are perturbed at relatively lower energies, just enough to be excited into the conduction band. 
In absorption experiments,  electrons are perturbed at lower energies. 
Therefore, they are not completely free and the system is still considered neutral. 
Since a \textit{quasihole} and \textit{quasielectron} are formed simultaneously, it creates a bound state, unlike the free electron in the quasiparticle gap as described above. 
This process is also known as \textit{optical} excitation, which is schematically shown in Fig.~\ref{fig:lab_ex_exp}, under "Absorption". 
The optical gap can be formulated as follows:
\begin{equation}
E_g^{K_1 {\rightarrow} K_2}=E^{K_1 {\rightarrow} K_2}- E_{0}\label{eq:optical}
\end{equation}
where $E^{K_1 {\rightarrow} K_2}$ is the energy of the system when a valence electron at wavevector $K_1$ is promoted to the conduction band at wavevector $K_2$. 
Therefore, the $E_g^{K_1 {\rightarrow} K_2}$ is called the optical gap for promoting an electron at $K_1$ to $K_2$.
If both CBM and VBM are on the same k-vector then the material is called direct band gap, since it can directly emit photons without any external perturbation (phonons). 
However, if CBM and VBM share different k-vectors, then the photon emitting electron has to transfer some of its momenta to the crystal lattice and then decay to the ground state. 
As this process involves an intermediate step, this property is called the indirect band gap. 
Difference between the optical and electronic band gaps are called the exciton binding energy. 
Exciton binding energy is very important for optoelectronic applications such as lasers. 
Since the recombination usually occurs between free holes and free electrons, a bound electron and hole state means that the spectrum of emission energies will be narrower. 
In the examples that follow, we will investigate the optical excitations of C-diamond.

\begin{figure}
	\hfill
	\includegraphics[width=0.41\textwidth]{./figures/lab_excited_xcrysden1}
	\includegraphics[width=0.48\textwidth]{./figures/lab_excited_xcrysden2}
	\hfill
	\caption{Visualizing the Brillouin Zone using XCRYSDEN.}
	\label{fig:lab_ex_xcrysden}
\end{figure}


\section{Preparation for the excited state calculations}\label{sec:lab_ex_prep}

In this section, we will study the preparation steps to perform excited state calculations with quantum Monte Carlo. 
Here, the most basic steps are listed in the implementation order:
\begin{enumerate}
	\item Identify the high symmetry k-points of the standardized primitive cell 
	\item Perform DFT band structure calculation along high symmetry paths
	\item Find a supertwist which includes all the k-points of interest
	\item Identify the indexing of k-points in the supertwist to be used in QMCPACK
\end{enumerate}

\subsection{Identifying high-symmetry k-points}\label{sec:lab_ex_highk}
Primitive cell is the most basic, non-unique repeat unit of a crystal in the real space. 
However, the translations of the repeat unit, the Bravais lattice is unique for each crystal, and can be represented using discrete translation operations, $R_n$:
\begin{equation}
{\bf R_n} = n_1{\bf a_1} + n_2{\bf a_2} + n_3{\bf a_3}
\end{equation}
$a_n$ are the real space lattice vectors in three dimensions. Thanks to the periodicity of the Bravais lattice, a crystal can also be represented using periodic functions in the reciprocal space:
\begin{equation}
f({\bf R_n + r})= \sum_{m}f_me^{iG_m({\bf R_n+r})}\label{eqn:lab_ex_rec_real}
\end{equation}
where $G_m$ are called as the reciprocal lattice vectors. Equation~\ref{eqn:lab_ex_rec_real} also satisfies the equality $G_m\cdot{R_n}=2{\pi}N$. High-symmetry structures can be represented using a subspace of the BZ, which is called as the irreducible Brillouin Zone (iBZ). If we choose series of  paths of high-symmetry k-points which encapsulates the iBZ, we can determine the band gap and electronic structure of the material. For more discussion, please refer to any solid state physics textbook. 

There are multiple practical ways to find the high-symmetry k-point path. 
For example, one can use pymatgen, \cite{Ong2013} XCRYSDEN \cite{Kokalj1999} or SeeK-path \cite{Hinuma2017}. 
Figure~\ref{fig:lab_ex_xcrysden} shows the procedure for visualizing the Brillouin Zone using XCRYSDEN after the structure file is loaded. 
However, the primitive cell is not unique, and the actual shape of the BZ can depend on the structure used. 
In our example, we use the python libraries of SeeK-path, using a wrapper written in Nexus. 

SeeK-path includes routines to standardize primitive cells, which will be useful for our work.

SeeK-path can be installed easily using \ishell{pip}:
\begin{shade}
>pip install --user seekpath
\end{shade}
 
In the \ishell{band.py} script, identification of high symmetry k-points and band structure calculations are done within the workflow. 
In the script, where the \ishell{dia} PhysicalSystem object is used as the input structure, \ishell{dia2_structure} is the standardized primitive cell and \ishell{dia2_kpath} is the respective k-path around the iBZ. 
\ishell{dia2_kpath} has a dictionary of the k-path in various coordinate systems, please make sure you are using the right one. 

\begin{lstlisting}[style=Python]
from structure import get_primitive_cell, get_kpath
dia2_structure   = get_primitive_cell(structure=dia.structure)['structure']
dia2_kpath       = get_kpath(structure=dia2_structure)
\end{lstlisting}

\begin{figure}
	\centering
	\includegraphics[width=0.5\textwidth]{figures/lab_excited_band_si}
	\caption{Band structure calculation of C-diamond performed at DFT-LDA level. Conduction band minimum (CBM) are shown with red points, and the valence band maximum(VBM)  are shown with the green points both at $\Gamma$.  DFT-LDA calculations suggest that the material has an indirect band gap from $\Gamma\rightarrow{\Delta}$. However, $\Gamma\rightarrow{\Gamma}$ transition can also be investigated for more complete check. }
	\label{fig:lab_ex_bands}
\end{figure}

\subsection{DFT band structure calculation along high symmetry paths}
After the high-symmetry kpoints are identified, one can perform band structure calculations in DFT. 
For an insulating structure, DFT can provide VBM and CBM wavevectors which would be of interest to the DMC calculations. 
However, if available, CBM and VBM from DFT would need to be compared to the experiments.  
Basically,  \ishell{band.py} will:
\begin{enumerate}
	\item Perform an SCF calculation in QE using a high density reciprocal grid.
	\item Identifies the high-symmetry k-points on the iBZ and provides a k-path.
	\item Perform a 'band' calculation in QE explicitly writing all the k-points on the path. (Make sure to add extra unoccupied bands)
	\item Plot the band structure curves and the location of VBM/CBM if available.
\end{enumerate}
In Fig.~\ref{fig:lab_ex_bands}, C-diamond is shown to have an indirect band gap between the red and green dots (CBM and VBM respectively). 
VBM is located at $\Gamma$. CBM is not located on a high symmetry k-point in this case. 
Therefore, we can use the symbol $\Delta$ to denote the CBM wavevector in the rest of this document. 
In \ishell{band.py} script, once the band structure calculation is finished, you can use the following lines to get the exact location of VBM and CBM using:
\begin{lstlisting}[style=Python]
p = band.load_analyzer_image()
print "VBM:\n{0}".format(p.bands.vbm)
print "CBM:\n{0}".format(p.bands.cbm)
\end{lstlisting}
Output must be the following:
\begin{lstlisting}[style=Python]
VBM:
  band_number     = 3
  energy          = 13.2874
  index           = 0
  kpoint_2pi_alat = [0. 0. 0.]
  kpoint_rel      = [0. 0. 0.]
  pol             = up

CBM:
  band_number     = 4
  energy          = 17.1545
  index           = 51
  kpoint_2pi_alat = [0.        0.1095605 0.       ]
  kpoint_rel      = [0.3695652 0.        0.3695652]
  pol             = up
\end{lstlisting}
\subsection{Finding a supertwist which includes all the k-points of interest}
Using the VBM and CBM wavevectors defined in the previous section, we now construct the supertwist which will hopefully contain both VBM and CBM. In Fig.~\ref{fig:lab_ex_twists}, we provide a simple example using 2D rectangular lattice. 
Let us assume that we are interested in the indirect transition, $\Gamma \rightarrow X_1$. 
In Fig.~\ref{fig:lab_ex_twists}a, the first BZ of the primitive cell is shown as the square centered on $\Gamma$, which is drawn using dashed lines. Due to the periodicity of the lattice, this primitive cell BZ repeats itself with spacings equal to the reciprocal lattice vectors: (2$\pi$/a, 0) and (0, 2$\pi$/a) (or (1,0) and (0,1) in crystal coordinates). 
We are interested in the  first BZ, where $X_1$ is at (0,0.5). 
In Fig.~\ref{fig:lab_ex_twists}b, the first BZ of the 2x2 supercell is the smaller square, drawn using solid lines. 
In Fig.~\ref{fig:lab_ex_twists}c, the BZ of the 2x2 supercell also repeats in the space, similar to Fig.~\ref{fig:lab_ex_twists}a. 
Therefore, in the 2x2 supercell, $X_1$, $X_2$ and $R$ are only the periodic images of $\Gamma$.  2x2 supercell calculation can be performed in reciprocal space using [2,2] tiling matrix. 
Therefore, individual kpoints (twists) of the primitive cell are combined in the supercell calculation, which are then called as supertwists. 
In more complex primitive cell (hence BZ), more general criteria would be constructing a set of supercell reciprocal lattice vectors which contains the $\Gamma \rightarrow X_1$ (e.g. $G_1$ in Fig.~\ref{fig:lab_ex_twists}) vector within their convex hull. 
Under this constraint, Wigner-Seitz radius of the simulation cell can be maximized to in an effort to reduce finite size errors. 

\begin{figure}
	\includegraphics[width=\textwidth]{figures/lab_excited_twists}
	\caption{a) First Brillouin Zone (BZ) of the primitive cell centered on $\Gamma$. Dashed lines indicate zone boundaries. b) First BZ of the 2x2 supercell inside the first BZ of the primitive cell. First BZ boundaries of the supercell are shown using solid lines. c) Periodic translations of the first BZ of the supercell showing that $\Gamma$ and $X_1$ are periodic images of each other given the supercell BZ. }
	\label{fig:lab_ex_twists}
\end{figure}

For the case of the indirect band gap in Diamond, one may need to deal with using several approximations to generate a supertwist which corresponds to a reasonable simulation cell. 
$\Delta$ in Diamond band gap is at \ishell{[0.3695653, 0., 0.3695653]}. 
In your calculations, the $\Delta$ wavevector and the eigenvalues you find can be slightly different in value. 
Closest simple fraction to this number with the smallest denominator is 1/3. If we use $\Delta'=[1/3, 0., 1/3]$, we could use 3x1x3 supercell as the simple choice and include both $\Delta'$ and $\Gamma$ in the same supertwist exactly. 
Near  $\Delta$, the LDA band curvature is very low and using  $\Delta'$ can indeed be a good approximation. 
We can compare the eigenvalues using their index numbers:
\begin{lstlisting}[mathescape=true,style=Python]
>>> print p.bands.up[51] ## CBM, $\Delta$ ##
  eigs            = [-3.2076  4.9221  7.5433  7.5433 17.1545 19.7598 28.3242 28.3242]
  index           = 51
  kpoint_2pi_alat = [0.        0.1095605 0.       ]
  kpoint_rel      = [0.3695652 0.        0.3695652]
  occs            = [1. 1. 1. 1. 0. 0. 0. 0.]
  pol             = up
>>> print p.bands.up[46] ## $\Delta'$ ##
  eigs            = [-4.0953  6.1376  7.9247  7.9247 17.1972 20.6393 27.3653 27.3653]
  index           = 46
  kpoint_2pi_alat = [0.        0.0988193 0.       ]
  kpoint_rel      = [0.3333333 0.        0.3333333]
  occs            = [1. 1. 1. 1. 0. 0. 0. 0.]
  pol             = up
\end{lstlisting}
This shows that the eigenvalues of the first unoccupied bands in $\Delta$ and $\Delta'$ are 17.1545 and 17.1972 eV respectively, meaning that according to LDA, a correction of nearly -40 meV is obtained. 
After electronic transitions between $\Gamma$ and $\Delta'$ are studied using DMC, one can apply the LDA correction to extrapolate the results to $\Gamma$ and $\Delta$ transitions.

\subsection{Identifying the indexing of k-points of interest in the supertwist}
At this stage, we must have performed \textit{scf} calculation using a converged k-point grid and then an \textit{nscf} calculation using the supertwist kpoints given above. 
We will be using the orbitals from neutral DFT calculations, therefore we need to explicitly define the band and twist indexes of the excitations in QMCPACK (e.g. in order to define electron promotion).
In C-diamond, we can give an example by finding the band and twist indexes of $\Gamma$ and $\Delta'$. 
For this end, one can run a mock VMC calculation and read \ishell{einspline.tile_300010003} \ishell{.spin_0.tw_0.g0.bandinfo.dat} file. Einspline file prints out the eigenstates information from DFT calculations. 
Therefore, we can obtain the band and the state index from this file, which can later be used to define the electron promotion. 
Below, you can see an explanation of how the band and twist indexes are defined using a portion of the \ishell{einspline.tile_300010003.spin_0.tw_0.g0.bandinfo.dat} file. 
Spin\_0 in the file name suggests that we are reading the spin up eigenstates. Band, state, twistindex and bandindex numbers all start from zero. We know that we have 72 electrons in the simulation cell, where 36 of them are spin-up polarized. 
Since state number starts from 0, state number 35 must be occupied while state 36 should be unoccupied. 
States 35 and 36 have the same reciprocal crystal coordinates (K1,K2,K3) as $\Gamma$ and $\Delta'$, respectively. 
Therefore, one should promote an electron from state number 35 to 36 to study the indirect band gap here.
\begin{lstlisting}[style=SHELL]
#  Band State TwistIndex BandIndex Energy Kx Ky Kz K1 K2 K3 KmK
33 33 0  1     0.488302  0.0000  0.0000  0.0000 -0.0000 -0.0000 -0.0000      1
34 34 0  2     0.488302  0.0000  0.0000  0.0000 -0.0000 -0.0000 -0.0000      1
35 35 0  3     0.488302  0.0000  0.0000  0.0000 -0.0000 -0.0000 -0.0000      1
36 36 4  4     0.631985  0.0000 -0.6209  0.0000 -0.3333 -0.0000 -0.3333      1
37 37 8  4     0.631985  0.0000 -1.2418  0.0000 -0.6667 -0.0000 -0.6667      1
38 38 0  4     0.691907  0.0000  0.0000  0.0000 -0.0000 -0.0000 -0.0000      1
\end{lstlisting}
However, one should always check whether this is really what we want. 
It can be seen  that band \# 33, 34 and 35 are degenerate (energy eigenvalues are listed in the 5th column), but also they have the same reciprocal coordinates in (K1,K2,K3). 
This is actually expected as one can see from Fig.~\ref{fig:lab_ex_bands}, in the band diagram the band structure is threefold degenerate at $\Gamma$.  
Here, we can choose the state with the largest band index: (0,3). 
Following the (twistindex, bandindex) notation, we can say that $\Gamma$ to $\Delta'$ transition can be defined as from (0,3) to (4,4). 

Alternatively, one can also read the band and twist indexes using PwscfAnalyzer and determine the band/twist indexes on the go:
\begin{lstlisting}[style=Python]
p = nscf.load_analyzer_image()
print 'band information'
print p.bands.up
print 'twist 0 k-point:',p.bands.up[0].kpoint_rel
print 'twist 4 k-point:',p.bands.up[4].kpoint_rel
print 'twist 0 band 3 eigenvalue:',p.bands.up[0].eigs[3]
print 'twist 4 band 4 eigenvalue:',p.bands.up[4].eigs[4]
\end{lstlisting}
Giving output:
\begin{lstlisting}[style=Python]
  0
    eigs            = [-8.0883 13.2874 13.2874 13.2874 18.8277 18.8277 18.8277 25.9151]
    index           = 0
    kpoint_2pi_alat = [0. 0. 0.]
    kpoint_rel      = [0. 0. 0.]
    occs            = [1. 1. 1. 1. 0. 0. 0. 0.]
    pol             = up
  1
    eigs            = [-5.0893  3.8761 10.9518 10.9518 21.5031 21.5031 21.5361 28.2574]
    index           = 1
    kpoint_2pi_alat = [-0.0494096  0.0494096  0.0494096]
    kpoint_rel      = [0.3333333 0.        0.       ]
    occs            = [1. 1. 1. 1. 0. 0. 0. 0.]
    pol             = up
  2
    eigs            = [-5.0893  3.8761 10.9518 10.9518 21.5031 21.5031 21.5361 28.2574]
    index           = 2
    kpoint_2pi_alat = [-0.0988193  0.0988193  0.0988193]
    kpoint_rel      = [0.6666667 0.        0.       ]
    occs            = [1. 1. 1. 1. 0. 0. 0. 0.]
    pol             = up
  3
    eigs            = [-5.0893  3.8761 10.9518 10.9518 21.5031 21.5031 21.5361 28.2574]
    index           = 3
    kpoint_2pi_alat = [ 0.0494096  0.0494096 -0.0494096]
    kpoint_rel      = [0.        0.        0.3333333]
    occs            = [1. 1. 1. 1. 0. 0. 0. 0.]
    pol             = up
  4
    eigs            = [-4.0954  6.1375  7.9247  7.9247 17.1972 20.6393 27.3652 27.3652]
    index           = 4
    kpoint_2pi_alat = [0.        0.0988193 0.       ]
    kpoint_rel      = [0.3333333 0.        0.3333333]
    occs            = [1. 1. 1. 1. 0. 0. 0. 0.]
    pol             = up
  5
    eigs            = [-0.6681  2.3791  3.7836  8.5596 19.3423 26.2181 26.6666 28.0506]
    index           = 5
    kpoint_2pi_alat = [-0.0494096  0.1482289  0.0494096]
    kpoint_rel      = [0.6666667 0.        0.3333333]
    occs            = [1. 1. 1. 1. 0. 0. 0. 0.]
    pol             = up
  6
    eigs            = [-5.0893  3.8761 10.9518 10.9518 21.5031 21.5031 21.5361 28.2574]
    index           = 6
    kpoint_2pi_alat = [ 0.0988193  0.0988193 -0.0988193]
    kpoint_rel      = [0.        0.        0.6666667]
    occs            = [1. 1. 1. 1. 0. 0. 0. 0.]
    pol             = up
  7
    eigs            = [-0.6681  2.3791  3.7836  8.5596 19.3423 26.2181 26.6666 28.0506]
    index           = 7
    kpoint_2pi_alat = [ 0.0494096  0.1482289 -0.0494096]
    kpoint_rel      = [0.3333333 0.        0.6666667]
    occs            = [1. 1. 1. 1. 0. 0. 0. 0.]
    pol             = up
  8
    eigs            = [-4.0954  6.1375  7.9247  7.9247 17.1972 20.6393 27.3652 27.3652]
    index           = 8
    kpoint_2pi_alat = [0.        0.1976385 0.       ]
    kpoint_rel      = [0.6666667 0.        0.6666667]
    occs            = [1. 1. 1. 1. 0. 0. 0. 0.]
    pol             = up

twist 0 k-point: [0. 0. 0.]
twist 4 k-point: [0.3333333 0.        0.3333333]
twist 0 band 3 eigenvalue: 13.2874
twist 4 band 4 eigenvalue: 17.1972
\end{lstlisting}

\section{Quasiparticle (electronic) gap calculations}\label{sec:lab_ex_qp}
In quasiparticle calculations, it is essential to work with reasonably large sized supercells in order to avoid spurious "1/N effects". 
Since quasiparticle calculations involve charged cells, large simulation cells ensure that the extra charge is diluted over the simulation cell. Coulombic interactions are conditionally convergent for neutral periodic systems, but they are divergent for the charged systems. 
A typical workflow for a quasiparticle calculation includes:
\begin{enumerate}
	\item SCF calculation in a neutral charged cell with QE using a high-density reciprocal grid.
	\item Choose a tiling matrix which will at least approximately include VBM and CBM k-points. 
	\item 'nscf'/'p2q' calculations using the tiling matrix 
	\item VMC/DMC calculations for the neutral, positively and negatively charged cells in QMCPACK
	\item Check the convergence of the quasiparticle gap with respect to the simulation cell size
\end{enumerate}
\begin{lstlisting}[style=QMCPXML]
<particleset name="e" random="yes">
  <group name="u" size="36" mass="1.0"> ##Change size to 35
    <parameter name="charge"              >    -1                    </parameter>
    <parameter name="mass"                >    1.0                   </parameter>
  </group>
...
...
<determinantset>
  <slaterdeterminant>
    <determinant id="updet" group="u" sposet="spo_u" size="36"> ##Change size to 35
      <occupation mode="ground" spindataset="0"/>	
    </determinant>
    <determinant id="downdet" group="d" sposet="spo_d" size="36">
      <occupation mode="ground" spindataset="1"/>	
    </determinant>
  </slaterdeterminant>
</determinantset>
\end{lstlisting}
Going back to equation~\ref{eq:qp}, one can see that it is essential to include VBM and CBM wavevectors in the same twist for quasiparticle calculations as well. 
Therefore, the added electron will sit at CBM while the subtracted electron will be removed from VBM. 
However, for the charged cell calculations, one may need to make changes in the input files for the fourth step.  Alternatively, in \ishell{quasiparticle.py} file the changes in the qmc input are shown for negatively charged system:
\begin{lstlisting}[style=Python]
qmc.input.simulation.qmcsystem.particlesets.e.groups.u.size +=1
(qmc.input.simulation.qmcsystem.wavefunction.determinantset
 .slaterdeterminant.determinants.updet.size += 1)
\end{lstlisting}
Here, the number of up electrons are increased by one (negatively charged system), and QMCPACK is instructed to read more one orbital in the up channel from the .h5 file. 

QE uses symmetry in order to reduce the number of k-points required for the calculation. 
Therefore, all symmetry tags in QE (\ishell{nosym}, \ishell{noinv} and \ishell{nosym_evc}) must be set to false. 
An easy way to check whether this is the case is to see that all KmK values \ishell{einspline} files are equal to 1. 
Above, the input for the neutral cell is given, while the changes are denoted as comments for the positively charged cell. 
Notice that, we have used \ishell{det_format      = "old"} in the \ishell{vmc_+/-e.py} files.
\section{Optical gap calculations}
Routines for the optical gap calculations are very similar to the quasiparticle gap calculations. 
The first three items in the quasiparticle band gap calculations can be reused for the optical gap calculations. 
However, at the VMC/DMC level, one should explicitly state the electronic transitions that are performed. 
Therefore, compared to the quasiparticle calculations, only the item number 4 is different for optical gap calculations. 
Here, the modified input file is given for the $\Gamma\rightarrow\Delta'$ transition, which can be compared to the ground state input file in the previous section. 
\begin{lstlisting}[style=QMCPXML]
<determinantset>
  <slaterdeterminant>
    <determinant id="updet" group="u" sposet="spo_u" size="36">
      <occupation mode="excited" spindataset="0" format="band" pairs="1" >
        0 3 4 4
      </occupation>
    </determinant>
    <determinant id="downdet" group="d" sposet="spo_d" size="36">
      <occupation mode="ground" spindataset="1"/>	
    </determinant>
  </slaterdeterminant>
</determinantset>
\end{lstlisting}
We have used the (twistindex, bandindex) notation in the annihilaion/creation order for the up spin electrons.
After resubmitting the batch job, in the output, you should be able to see the following lines in the \ishell{vmc.out} file:
\begin{lstlisting}[style=SHELL]
Sorting the bands now:
  Occupying bands based on (ti,bi) data.
removing orbital 35
adding orbital 36
We will read 36 distinct orbitals.
There are 0 core states and 36 valence states.
\end{lstlisting}
And the \ishell{einspline.tile_300010003.spin_0.tw_0.g0.bandinfo.dat} file must be changed in the following way: 
\begin{lstlisting}[style=SHELL]
#  Band State TwistIndex BandIndex Energy Kx Ky Kz K1 K2 K3 KmK
33 33 0	1 0.499956	0.0000  0.0000 0.0000  0.0000 0.0000  0.0000 1
34 34 0	2 0.500126	0.0000  0.0000 0.0000  0.0000 0.0000  0.0000 1
35 35 4	4 0.637231	0.0000 -0.6209 0.0000 -0.3333 0.0000 -0.3333 1
36 36 0	3 0.502916	0.0000  0.0000 0.0000  0.0000 0.0000  0.0000 1
37 37 8	4 0.637231	0.0000 -1.2418 0.0000 -0.6667 0.0000 -0.6667 1
38 38 0	4 0.699993	0.0000  0.0000 0.0000  0.0000 0.0000  0.0000 1
\end{lstlisting}
Alternatively, one can define the excitations within Nexus as shown in \ishell{optical.py} file:
\begin{lstlisting}[style=Python]
qmc = generate_qmcpack(
    ...
    excitation = ['up', '0 3 4 4'], # (ti, bi) notation
    #excitation = ['up', '-35 + 36'], # Orbital (state) index notation
    ...
    )
\end{lstlisting}



\chapter{Additional Tools}
\label{chap:additional_tools}
QMCPACK provides a set of lightweight executables that address certain
common problems in QMC workflow and analysis.  These range from conversion utilities between 
different file formats and QMCPACK (e.g., \ishell{ppconvert} and \ishell{convert4qmc}),  
(qmc-extract-eshdf-kvectors) to postprocessing utilities (\ishell{trace-density} and \ishell{qmcfinitesize}) to many others.  In this chapter, we cover the use cases, syntax, and features of all additional tools provided with QMCPACK.  

\section{Initialization Tools}
  \subsection{qmc-get-supercell}

\section{Postprocessing}
  \subsection{qmca}
    \ishell{qmca} is a versatile tool to analyze and plot the raw data from QMCPACK \ishell{*.scalar.dat} files.
    It is a Python executable and part of the Nexus suite of tools.  It can be found in 
    \ishell{qmcpack/nexus/executables}. For details, see Section~\ref{sec:qmca}.
  \subsection{qmc-fit}
    \ishell{qmc-fit} is a curve fitting tool used to obtain statistical error bars on fitted parameters.
    It is useful for DMC time step extrapolation.  For details, see Section~\ref{sec:qmcfit}.
  \subsection{qdens}
    \ishell{qdens} is a command line tool to produce density files from QMCPACK's \ishell{stat.h5} output files.  For details, see Section~\ref{sec:qdens}.
  \subsection{qmcfinitesize}
    \ishell{qmcfinitesize} is a utility to compute many-body finite-size corrections to the energy.  It
    is a C++ executable that is built alongside the QMCPACK executable.  It can be found in 
    \ishell{build/bin}.

\section{Converters} 
\subsection{convert4qmc}
\label{sec:convert4qmc}
\ishell{Convert4qmc} allows conversion of orbitals and wavefunctions from
quantum chemistry output files to \qmcpack XML and HDF5 input files.
It is a small C++ executable that is built alongside the \qmcpack
executable and can be found in \texttt{build/bin}.\\

To date, \texttt{convert4qmc} supports the following codes:
GAMESS\cite{schmidt93}, PySCF\cite{Sun2018}, QP\cite{QP}
and GAMESS-FMO\cite{Fedorov2004,schmidt93}

\subsubsection{General use}
General use of \texttt{convert4qmc} can be prompted by running with no options:

\begin{lstlisting}[style=SHELL]
>convert4qmc

Defaults : -gridtype log -first 1e-6 -last 100 -size 1001 -ci required -threshold 0.01 -TargetState 0 -prefix sample

 convert [-gaussian|-casino|-gamesxml|-gamess|-gamessFMO|-QP|-pyscf|-orbitals] 
 filename                                                          
[-nojastrow -hdf5 -prefix title -addCusp -production -NbImages NimageX NimageY NimageZ]
[-psi_tag psi0 -ion_tag ion0 -gridtype log|log0|linear -first ri -last rf]
[-size npts -ci file.out -threshold cimin -TargetState state_number
-NaturalOrbitals NumToRead -optDetCoeffs]                                        
Defaults : -gridtype log -first 1e-6 -last 100 -size 1001 -ci required 
-threshold 0.01 -TargetState 0 -prefix sample                                
When the input format is missing, the  extension of filename is used to determine
the format                                                      
 *.Fchk -> gaussian; *.out -> gamess; *.data -> casino; *.xml -> gamesxml
\end{lstlisting}

As an example, to convert a GAMESS calculation using a single determinant, the following use is sufficient:\\
\begin{lstlisting}[style=SHELL]
convert4qmc -gamess MyGamessOutput.out
\end{lstlisting}

By default, the converter will generate multiple files:\\
\begin{table}[h]
\begin{center}
\begin{tabularx}{\textwidth}{l l l l X }
\hline
\multicolumn{5}{l}{\texttt{convert4qmc} output} \\
\hline
%\multicolumn{2}{l}{Outputfiles}  & \multicolumn{3}{l}{}\\
   &   \bfseries output     & \bfseries file type & \bfseries default   & \bfseries description \\
   &   \texttt{*.qmc.in-wfs.xml             } &  XML  & default& Main input file for QMCPACK\\
   &   \texttt{*.qmc.in-wfnoj.xml             } &  XML  & default& Main input file for QMCPACK\\
   &   \texttt{*.structure.xml             } &  XML   &default   & File containing the structure of the system\\
   &   \texttt{*.wfj.xml             } &  XML  & default & Wavefunction file with 1-, 2-, and 3-body Jastrows\\
   &   \texttt{*.wfnoj.xml             } &  XML   & default & Wavefunction file with no Jastrows \\
   &   \texttt{*.orbs.h5             } &  HDF5   & with -hdf5   & HDF5 file containing all wavefunction data\\
    \hline
    \end{tabularx}
\end{center}
\end{table}

If no \ishell{-prefix} option is specified, the prefix is taken from
the input file name. For instance, if the GAMESS output file is
\texttt{Mysim}.out, the files generated by \texttt{convert4qmc} will use the
prefix \texttt{Mysim} and output files will be
\ishell{Mysim.qmc.in-wfs.xml}, \ishell{Mysim.structure.xml}, and so on.

\begin{itemize}
 \item Files \texttt{.in-wfs.xml} and \texttt{.in-wfnoj.xml} \\ These
   are the input files for \qmcpack.  The geometry and the
   wavefunction are stored in external files \ishell{*.structure.xml}
   and \ishell{*.wfj.xml} (referenced from \ishell{*.in-wfs.xml}) or
   \ishell{*.qmc.wfnoj.xml} (referenced from
   \ishell{*.qmc.in-wfnoj.xml}). The Hamiltonian section is included,
   and the presence or lack of presence of an ECP is detected during the
   conversion. If use of an ECP is detected, a default ECP name is
   added (e.g., \ishell{H.qmcpp.xml}), and it is the responsibility of
   the user to modify the ECP name to match the one used to generate
   the wavefunction.\\
\begin{lstlisting}[style=QMCPXML]
  <?xml version="1.0"?>
<simulation>
  <!--
 
Example QMCPACK input file produced by convert4qmc
 
It is recommend to start with only the initial VMC block and adjust
parameters based on the measured energies, variance, and statistics.

-->
  <!--Name and Series number of the project.-->
  <project id="gms" series="0"/>
  <!--Link to the location of the Atomic Coordinates and the location of 
      the Wavefunction.-->
  <include href="gms.structure.xml"/>
  <include href="gms.wfnoj.xml"/>
  <!--Hamiltonian of the system. Default ECP filenames are assumed.-->
  <hamiltonian name="h0" type="generic" target="e">
    <pairpot name="ElecElec" type="coulomb" source="e" target="e" 
                                                   physical="true"/>
    <pairpot name="IonIon" type="coulomb" source="ion0" target="ion0"/>
    <pairpot name="PseudoPot" type="pseudo" source="ion0" wavefunction="psi0" 
                                                           format="xml">
      <pseudo elementType="H" href="H.qmcpp.xml"/>
      <pseudo elementType="Li" href="Li.qmcpp.xml"/>
    </pairpot>
  </hamiltonian>

 \end{lstlisting}

 The \ishell{qmc.in-wfnoj.xml} file will have one VMC block with a
 minimum number of blocks to reproduce the HF/DFT energy used to
 generate the trial wavefunction.
 
 \begin{lstlisting}[style=QMCPXML]
  <qmc method="vmc" move="pbyp" checkpoint="-1">
    <estimator name="LocalEnergy" hdf5="no"/>
    <parameter name="warmupSteps">100</parameter>
    <parameter name="blocks">20</parameter>
    <parameter name="steps">50</parameter>
    <parameter name="substeps">8</parameter>
    <parameter name="timestep">0.5</parameter>
    <parameter name="usedrift">no</parameter>
  </qmc>
</simulation>
 \end{lstlisting}

If the \ishell{qmc.in-wfj.xml} file is used, Jastrow optimization
blocks followed by a VMC and DMC block are included. These blocks
contain default values to allow the user to test the accuracy of a
system; however, they need to be updated and optimized for each
system. The initial values might only be suitable for a small molecule.

\begin{lstlisting}[style=QMCPXML]
  <loop max="4">
    <qmc method="linear" move="pbyp" checkpoint="-1">
      <estimator name="LocalEnergy" hdf5="no"/>
      <parameter name="warmupSteps">100</parameter>
      <parameter name="blocks">20</parameter>
      <parameter name="timestep">0.5</parameter>
      <parameter name="walkers">1</parameter>
      <parameter name="samples">16000</parameter>
      <parameter name="substeps">4</parameter>
      <parameter name="usedrift">no</parameter>
      <parameter name="MinMethod">OneShiftOnly</parameter>
      <parameter name="minwalkers">0.0001</parameter>
    </qmc>
  </loop>
  <!--

Example follow-up VMC optimization using more samples for greater accuracy:

-->
  <loop max="10">
    <qmc method="linear" move="pbyp" checkpoint="-1">
      <estimator name="LocalEnergy" hdf5="no"/>
      <parameter name="warmupSteps">100</parameter>
      <parameter name="blocks">20</parameter>
      <parameter name="timestep">0.5</parameter>
      <parameter name="walkers">1</parameter>
      <parameter name="samples">64000</parameter>
      <parameter name="substeps">4</parameter>
      <parameter name="usedrift">no</parameter>
      <parameter name="MinMethod">OneShiftOnly</parameter>
      <parameter name="minwalkers">0.3</parameter>
    </qmc>
  </loop>
  <!--

Production VMC and DMC:

Examine the results of the optimization before running these blocks.
For example, choose the best optimized jastrow from all obtained, put in the 
wavefunction file, and do not reoptimize.

-->
  <qmc method="vmc" move="pbyp" checkpoint="-1">
    <estimator name="LocalEnergy" hdf5="no"/>
    <parameter name="warmupSteps">100</parameter>
    <parameter name="blocks">200</parameter>
    <parameter name="steps">50</parameter>
    <parameter name="substeps">8</parameter>
    <parameter name="timestep">0.5</parameter>
    <parameter name="usedrift">no</parameter>
    <!--Sample count should match targetwalker count for 
      DMC. Will be obtained from all nodes.-->
    <parameter name="samples">16000</parameter>
  </qmc>
  <qmc method="dmc" move="pbyp" checkpoint="20">
    <estimator name="LocalEnergy" hdf5="no"/>
    <parameter name="targetwalkers">16000</parameter>
    <parameter name="reconfiguration">no</parameter>
    <parameter name="warmupSteps">100</parameter>
    <parameter name="timestep">0.005</parameter>
    <parameter name="steps">100</parameter>
    <parameter name="blocks">100</parameter>
    <parameter name="nonlocalmoves">yes</parameter>
  </qmc>
</simulation>

\end{lstlisting}

 \item File \texttt{.structure.xml} \\
 This file will be referenced from the main QMCPACK input. It contains the geometry of the system, position of the atoms, number of atoms, atomic types and charges, and number of electrons.
 
 \item Files \texttt{.wfj.xml} and \texttt{.wfnoj.xml}\\
 These files contain the basis set detail, orbital coefficients, and the 1-, 2-, and 3-body Jastrow (in the case of \texttt{.wfj.xml}). If the wavefunction is multideterminant, the expansion will be at the end of the file. We recommend using the option \texttt{-hdf5} when large molecules are studied to store the data more compactly in an HDF5 file.
 
 \item File \texttt{.orbs.h5} \\
 This file is generated only if the option \texttt{-hdf5} is added as follows:
 \begin{shade}
  convert4qmc -gamess MyGamessOutput.out -hdf5
 \end{shade}
In this case,  the \texttt{.wfj.xml} or \texttt{.wfnoj.xml} files will point to this HDF file.  Information about the basis set, orbital coefficients, and the multideterminant expansion is put in this file and removed from the wavefunction files, making them smaller. 

\end{itemize}



\begin{table}[h]
\begin{center}
\begin{tabularx}{\textwidth}{l l l l }
\hline
\multicolumn{4}{l}{\ishell{convert4qmc} input type} \\
\hline
   &   \bfseries Option name     &\bfseries description\\
   &   \texttt{-orbitals    } &  Generic HDF5 input file. Mainly automatically generated from QP and PySCF.  & Actively maintained\\
   &   \texttt{-pyscf       } &  PySCF code & Actively maintained\\
   &   \texttt{-QP          } &  QP code & Actively maintained\\
   &   \texttt{-gamess      } &  Gamess code & Maintained\\
   &   \texttt{-gamesFMO    } &  Gamess FMO & Maintained\\
   &   \texttt{-gaussian    } &  Gaussian code & Obsolete/untested \\
   &   \texttt{-casino      } &  Casino code & Obsolete/untested \\
   &   \texttt{-gamesxml    } &  Gamess xml format code  & Obsolete/untested\\
    \hline

    \end{tabularx}
\end{center}
\end{table}

\subsubsection{Command line options}

 \begin{table}[h]
 \begin{center}
 \begin{tabularx}{\textwidth}{l l l l l }
 \hline
 \multicolumn{5}{l}{\texttt{convert4qmc} command line options} \\
 \hline
    &   \bfseries Option name      & \bfseries Value & \bfseries default   & \bfseries description \\
    &   \texttt{-nojastrow    } &  -      &   - & Force no Jastrow. \texttt{qmc.in.wfj} will not be generated  \\
    &   \texttt{-hdf5         } &  -      &   - & Force the wf to be in HDF5 format   \\
    &   \texttt{-prefix       } & string  &   - & All created files will have the name of the string   \\
    &   \texttt{-multidet     } & string  &   - & HDF5 file containing a multideterminant expansion   \\
    &   \texttt{-addCusp      } &  -      &   - & Force to add orbital cusp correction (ONLY for all-electron)  \\
    &   \texttt{-production   } &  -      &   - & Generates specific blocks in the input     \\
    &   \texttt{-psi\_tag     } & string  & psi0& Name of the electrons particles inside \qmcpack   \\
    &   \texttt{-ion\_tag     } & string  & ion0& Name of the ion particles inside \qmcpack      \\
    \hline
     \end{tabularx}
 \end{center}
 \end{table}
\begin{itemize}
\item \texttt{-multidet}\\
This option is to be used when a multideterminant expansion (mainly a CI expansion) is present in an HDF5 file. The trial wavefunction file will not display the full list of multideterminants and will add a path to the HDF5 file as follows (full example for the C2 molecule in qmcpack/tests/molecules/C2\_pp).\\
  
\begin{lstlisting}[style=QMCPXML]
<?xml version="1.0"?>
<qmcsystem>
  <wavefunction name="psi0" target="e">
    <determinantset type="MolecularOrbital" name="LCAOBSet" source="ion0" transform="yes" href="C2.h5">
      <sposet basisset="LCAOBSet" name="spo-up" size="58">
        <occupation mode="ground"/>
        <coefficient size="58" spindataset="0"/>
      </sposet>
      <sposet basisset="LCAOBSet" name="spo-dn" size="58">
        <occupation mode="ground"/>
        <coefficient size="58" spindataset="0"/>
      </sposet>
      <multideterminant optimize="no" spo_up="spo-up" spo_dn="spo-dn">
        <detlist size="202" type="DETS" nca="0" ncb="0" nea="4" neb="4" nstates="58" cutoff="1e-20" href="C2.h5"/>
      </multideterminant>
    </determinantset>
  </wavefunction>
</qmcsystem>
\end{lstlisting}



To generate such trial wavefunction, the converter has to be invoked as follows:

\begin{shade}
> convert4qmc -orbitals C2.h5 -multidet C2.h5 
\end{shade}


\item \texttt{-nojastrow}\\
This option generates only an input file, \ishell{*.qmc.in.wfnoj.xml}, containing no Jastrow optimization blocks and references a wavefunction file, \ishell{*.wfnoj.xml}, containing no Jastrow section.

\item \texttt{-hdf5}\\
This option generates the \ishell{*.orbs.h5} HDF5 file containing the basis set and the orbital coefficients. If the wavefunction contains a multideterminant expansion from QP, it will also be stored in this file. This option minimizes the size of the \ishell{*.wfj.xml} file, which points to the HDF file, as in the following example: 

\begin{lstlisting}[style=QMCPXML]
 <?xml version="1.0"?>
<qmcsystem>
  <wavefunction name="psi0" target="e">
    <determinantset type="MolecularOrbital" name="LCAOBSet" source="ion0"
       transform="yes" href="test.orbs.h5">
      <slaterdeterminant>
        <determinant id="updet" size="39">
          <occupation mode="ground"/>
          <coefficient size="411" spindataset="0"/>
        </determinant>
        <determinant id="downdet" size="35">
          <occupation mode="ground"/>
          <coefficient size="411" spindataset="0"/>
        </determinant>
      </slaterdeterminant>
    </determinantset>
  </wavefunction>
</qmcsystem>
\end{lstlisting}

Jastrow functions will be included if the option ``-nojastrow'' was
not specified. Note that when initially optimization a wavefunction, we recommend
temporarily removing/disabling the 3-body Jastrow.

\item \textbf{-prefix}\\
Sets the prefix for all output generated by \texttt{convert4qmc}. \\
If not specified, \texttt{convert4qmc} will use the defaults for the following:\\
\begin{itemize}
 \item \textbf{Gamess}\\
If the Gamess output file  is named ``\textbf{Name}.out'' or ``\textbf{Name}.output,'' all files generated by \texttt{convert4qmc} will carry \textbf{Name} as a prefix (i.e., \textbf{Name}.qmc.in.xml).\\ 
\item \textbf{PySCF}\\
If the PySCF output file  is named ``\textbf{Name}.H5,'' all files generated by \texttt{convert4qmc} will carry \textbf{Name} as a prefix (i.e., \textbf{Name}.qmc.in.xml).\\ 
\item \textbf{QP}\\
If the QP output file  is named ``\textbf{Name}.dump,'' all files generated by \texttt{convert4qmc} will carry \textbf{Name} as a prefix (i.e., \textbf{Name}.qmc.in.xml).\\ 
\item \textbf{Generic HDF5 input}\\
If a generic HDF5 file (either from PySCF or QP in the HDF5 format) is named ``\textbf{Name}.H5,'' all files generated by \texttt{convert4qmc} will carry \textbf{Name} as a prefix (i.e., \textbf{Name}.qmc.in.xml).\\ 

\end{itemize}


\item \textbf{-addCusp} \\ This option is very important for
  all-electron (AE) calculations. In this case, orbitals have to be
  corrected for the electron-nuclear cusp. The cusp correction scheme
  follows the algorithm described by Ma et al. \cite{Ma2005}
  When this option is present, the wavefunction file has a new set of
  tags:

\begin{lstlisting}[style=QMCPXML]
 qmcsystem>
  <wavefunction name="psi0" target="e">
    <determinantset type="MolecularOrbital" name="LCAOBSet" source="ion0"
      transform="yes" cuspCorrection="yes">
      <basisset name="LCAOBSet">
\end{lstlisting}

The tag ``cuspCorrection'' in the \ishell{wfj.xml} (or \ishell{wfnoj.xml}) wavefunction file will force correction of the orbitals at the beginning of the \qmcpack run. \\
In the ``orbitals`` section of the wavefunction file, a new tag ``cuspInfo'' will be added for orbitals spin-up and orbitals spin-down:

\begin{lstlisting}[style=QMCPXML]
   <slaterdeterminant>
        <determinant id="updet" size="2"
            cuspInfo="../CuspCorrection/updet.cuspInfo.xml">
          <occupation mode="ground"/>
          <coefficient size="135" id="updetC">
          
  <determinant id="downdet" size="2"
           cuspInfo="../CuspCorrection/downdet.cuspInfo.xml">
          <occupation mode="ground"/>
          <coefficient size="135" id="downdetC">
\end{lstlisting}

These tags will point to the files \ishell{updet.cuspInfo.xml} and
\ishell{downdet.cuspInfo.xml}. By default, the converter assumes that
the files are located in the relative path
\texttt{../CuspCorrection/}. If the directory
\ishell{../CuspCorrection} does not exist, or if the files are not
present in that directory, \qmcpack will run the cusp correction
algorithm to generate both files.  If the files exist, then /qmcpack
will apply the corrections to the orbitals. \\

\textbf{Important notes:}\\
- The cusp correction implementations has been parallelized and performance improved.  However, since the correction needs
to be applied for every ion and then for every orbital on that ion, this operation can be costly (slow) for large
systems. We recommend saving and reusing the computed cusp correction files \ishell{updet.cuspInfo.xml} and
\ishell{downdet.cuspInfo.xml}, and transferring them between computer systems where relevant.
\\

\item \textbf{-psi\_tag}\\
\qmcpack builds the wavefunction as a named object. In the vast majority of cases, one wavefunction is simulated at a time, but there may be situations where we want to distinguish different parts of a wavefunction, or even use multiple wavefunctions. This option can change the name for these cases. 

\begin{lstlisting}[style=QMCPXML]
   <wavefunction name="psi0" target="e">
\end{lstlisting}

\item \textbf{-ion\_tag} \\
Although similar to \textbf{-psi\_tag}, this is used for the type of ions. \\
\begin{shade}
  <particleset name="ion0" size="2">
\end{shade}


\item \textbf{-production}\\

Without this option, input files with standard optimization, VMC, and
DMC blocks are generated. When the ``-production'' option is
specified, an input file containing complex options that may be
more suitable for large runs at HPC centers is generated. This option
is for users who are already familiar with QMC and \qmcpack. We encourage feedback
on the standard and production sample inputs.


\end{itemize}

The following options are specific to using MCSCF multideterminants from Gamess. 

 \begin{table}[h]
 \begin{center}
 \begin{tabularx}{\textwidth}{l l l l l }
 \hline
 \multicolumn{5}{l}{\texttt{convert4qmc} MCSCF arguments} \\
 \hline
    &   \bfseries option      & \bfseries Value & \bfseries default   & \bfseries description \\
    &   \texttt{-ci    } & String     &   none & Name of the file containing the CI expansion  \\
    &   \texttt{-threshold         } &  double    &  1e-20 & Cutoff of the weight of the determinants  \\
    &   \texttt{-TargetState      } & int  &  none & ?  \\
    &   \texttt{-NaturalOrbitals      } &  int      &  none   & ?  \\
    &   \texttt{-optDetCoeffs      } &  -      &   no & Enables the optimization of CI coefficients \\
    \hline
     \end{tabularx}
 \end{center}
 \end{table}
\begin{itemize}
\item keyword \textbf{-ci}\\
Path/name of the file containing the CI expansion in a Gamess Format.
\item keyword \textbf{-threshold}\\
The CI expansion contains coefficients (weights) for each determinant. This option sets the maximum coefficient to include in the QMC run. By default it is set to 1e-20 (meaning all determinants in an expansion are taken into account). At the same time, if the threshold is set to a different value, for example $1e-5$, any determinant with a weight $abs(weight) < 1e-5$ will be discarded and the determinant will not be considered. 
\item keyword \textbf{-TargetState}\\
?
\item keyword \textbf{-NaturalOrbitals}\\
?
\item keyword \textbf{-optDetCoeffs}\\
This flag enables optimization of the CI expansion coefficients. By default, optimization of the coefficients is disabled during wavefunction optimization runs. 
\end{itemize}

Examples and more thorough descriptions of these options can be found in the lab section of this manual: Chapter-\ref{chap:lab_advanced_molecules}\\

\subsubsection{Grid options}
                                          
% 
These parameters control how the basis set is projected on a grid. The default parameters are chosen to be very efficient. Unless you have a very good reason, we do not recommend modifying them. 

\begin{table}[h]
 \begin{center}
 \begin{tabularx}{\textwidth}{l l l l l }
 \hline
 \multicolumn{5}{l}{\texttt{convert4qmc} Grid Keywords} \\
 \hline
 \multicolumn{2}{l}{Tags}  & \multicolumn{3}{l}{}\\
    &   \bfseries keyword      & \bfseries Value & \bfseries default   & \bfseries description \\
    &   \texttt{-gridtype    } &  log|log0|linear      &  log & Grid type  \\
    &   \texttt{-first         } & double  &  1e-6 & First point of the grid   \\
    &   \texttt{-last       } & double  & 100 & Last point of the grid \\
    &   \texttt{-size      } &  int    &  1001& Number of point in the grid   \\
     \hline
     \end{tabularx}
 \end{center}
 \end{table}
\begin{itemize}
\item \textbf{-gridtype}\\
Grid type can be logarithmic, logarithmic base 10, or linear \\
\item \textbf{-first}\\
First value of the grid\\
\item \textbf{-last}\\
Last value of the grid\\
\item \textbf{-size}\\
Number of points in the grid between ``first'' and ``last.'' \\
\end{itemize}


\subsubsection{Supported codes}

\begin{itemize}
\item \textbf{PySCF}\\

PySCF\cite{Sun2018} is an all-purpose quantum chemistry code that can
run calculations from simple Hartree-Fock to DFT, MCSCF, and CCSD, and
for both isolated systems and periodic boundary conditions. PySCF can
be downloaded from \url{https://github.com/sunqm/pyscf}. Many examples
and tutorials can be found on the PySCF website, and all types of
single determinants calculations are compatible with \qmcpack, thanks
to active support from the authors of PySCF. A few additional steps
are necessary to generate an output readable by \texttt{convert4qmc}.


This example shows how to run a Hartree-Fock calculation for the $LiH$
dimer molecule from PySCF and convert the wavefunction for \qmcpack.\\

\begin{itemize}
\item \textbf{Python path}\\
  \begin{sloppypar}
    PySCF is a Python-based code. A Python module named \textbf{PyscfToQmcpack} containing the function \textbf{savetoqmcpack} is provided by \qmcpack and is located at\linebreak
    \ishell{qmcpack/src/QMCTools/PyscfToQmcpack.py}.
To be accessible to the PySCF script, this path must be added to the PYTHONPATH environment variable.
For the bash shell, this can be done as follows:\\
\end{sloppypar}
\begin{shade}
 export PYTHONPATH=/PATH_TO_QMCPACK/qmcpack/src/QMCTools:\$PYTHONPATH
\end{shade}


 \item \textbf{PySCF Input File}\\
 
Copy and paste the following code in a file named LiH.py.

\begin{lstlisting}[style=Python]
#!/usr/bin/env python
from pyscf import gto, scf, df
import numpy

cell = gto.M(
   atom ='''
Li 0.0 0.0 0.0
H  0.0 0.0 3.0139239778''',
   basis ='cc-pv5z',
   unit="bohr",
   spin=0,
   verbose = 5,
   cart=False,
)
mf = scf.ROHF(cell)
mf.kernel()

###SPECIFIC TO QMCPACK###
title='LiH'
from PyscfToQmcpack import savetoqmcpack

savetoqmcpack(cell,mf,title)
\end{lstlisting}

The arguments to the function \textbf{savetoqmcpack} are:\\
\begin{itemize}
 \item \textbf{cell}\\
 This is the object returned from gto.M, containing the type of atoms, geometry, basisset, spin, etc. \\
 \item \textbf{mf}\\
This is an object representing the PySCF level of theory, in this example, ROHF. This object contains the orbital coefficients of the calculations. \\
 \item \textbf{title}\\
 The name of the output file generated by PySCF. By default, the name of the generated file will be ``default'' if nothing is specified.\\
 \end{itemize}

By adding the three lines below the ``SPECIFIC TO QMCPACK'' comment  in the input file, the script will dump all the necessary data for \qmcpack into an HDF5 file using the value of ``title'' as an output name. PySCF is run as follows:\\
\begin{shade}
 >python LiH.py
\end{shade}


The generated HDF5 can be read by \texttt{convert4qmc} to generate the appropriate \qmcpack input files.\\

 \item \textbf{Generating input files}\\
 
 As described in the previous section, generating input files for PySCF is as follows:\\
 \begin{shade}
  > convert4qmc -pyscf LiH.h5 
 \end{shade}

The HDF5 file produced by ``savetoqmcpack'' contains the wavefunction in a form directly readable by \qmcpack.
The wavefunction files from \texttt{convert4qmc} reference this HDF file as if the ``-hdf5" option were specified
(converting from PySCF implies the ``-hdf5'' option is always present).

\end{itemize}

 An implementation of periodic boundary conditions with Gaussian orbitals from PySCF is under development. 

\item \textbf{Quantum Package}\\
QP\cite{QP} is a quantum chemistry code developed by the LCPQ laboratory in Toulouse, France. It can be downloaded from \url{https://github.com/LCPQ/quantum_package}, and the tutorial within is quite extensive. The tutorial section of QP can guide you on how to install and run the code.\\

After a QP calculation, the data needed for \texttt{convert4qmc} can be generated through\\
\begin{shade}
 qp_run save_for_qmcpack Myrun.ezfio &> Myrun.dump
\end{shade}

\texttt{convert4qmc} can read this format and generate \qmcpack input files in XML and HDF5 format.  For example:

\begin{shade}
 convert4qmc -QP Myrun.dump
\end{shade}


The main reason to use QP is to access the CIPSI algorithm to generate a multideterminant wavefunction.
CIPSI is the preferred choice for generating a selected CI trial wavefunction for \qmcpack. 
An example on how to use QP for Hartree-Fock and selected CI can be found in Section-\ref{sec:cipsi}  of this manual.
The converter code is actively maintained and codeveloped by both \qmcpack and QP developers.\\

We recommend using a trial wavefunction stored in HDF5 format to reduce the reading time when a multideterminant expansion is too large (more than 1K determinants). This can be done with two paths:\\

using the \textit{-hdf5} option in the converter as follows:\\

 \item \textbf{Using -hdf5 tag}\\

\begin{shade}
 convert4qmc -QP Myrun.dump -hdf5
\end{shade}

This will read the multideterminant expansion in the \texttt{Myrun.dump} file and store it in \texttt{Myrun.dump.orbs.h5}. Note that this method will be deprecated as QP automatically generates a compatible HDF5 file usable by \qmcpack directly. \\

 \item \textbf{Using h5 file }\\

QP version 2.0 (released in 2019) directly generates an HDF5 file that completely mimics the \qmcpack readable format. This file can be generated after a CIPSI, Hartree-Fock, or range-separated DFT in QP as follows: \\

\begin{shade}
 qp_run save_for_qmcpack Myrun.ezfio > Myrun.dump
\end{shade}

In addition to \texttt{Myrun.dump}, an HDF5 file always named \texttt{QMC.h5} is created containing all relevant information to start a QMC run. Input files can be generated as follows:\\

\begin{shade}
 convert4qmc -orbitals QMC.h5 -multidet QMC.h5
\end{shade}

Note that the \texttt{QMC.h5} combined with the tags \texttt{-orbitals} and \texttt{-multidet} allows the user to choose orbitals from a different code such as PYSCF and the multideterminant section from QP. These two codes are fully compatible, and this route is also the only possible route for multideterminants for solids. 

\item \textbf{GAMESS}\\
\qmcpack can use the output of GAMESS\cite{schmidt93} for any type of single determinant calculation (HF or DFT) or multideterminant (MCSCF) calculation. A description with an example can be found in the Advanced Molecular Calculations Lab (Section~\ref{chap:lab_advanced_molecules}).
\end{itemize}

    
  \subsection{pw2qmcpack.x}
\label{sec:pw2qmcpack}
\ishell{pw2qmcpack.x} is an executable that converts PWSCF wavefunctions to QMCPACK readable 
HDF5 format.  This utility is built alongside the QE postprocessing utilities.
This utility is written in Fortran90 and is distributed as a patch of the QE 
source code.  The patch, as well as automated QE download and patch scripts, can be found in 
\ishell{qmcpack/external_codes/quantum_espresso}.

pw2qmcpack can be used in serial in small systems and should be used in parallel with large systems for best performance. The K\_POINT gamma optimization is not supported.

\begin{lstlisting}[style=ESPRESSO,caption={Sample \ishell{pw2qmcpack.x} input file \ishell{p2q.in}}]
&inputpp
  prefix     = 'bulk_silicon'
  outdir     = './'
  write_psir = .false.
/
\end{lstlisting}

This example will cause \ishell{pw2qmcpack.x} to convert wavefunctions saved from PWSCF with the prefix ``bulk\_silicon.'' Perform the conversion via, for example:

\begin{shade}
mpirun -np 1 pw2qmcpack.x < p2q.in>& p2q.out
\end{shade}

Because of the large plane-wave energy cutoffs in the pw.x calculation required by accurate PPs and the large system sizes of interest, one limitation of QE can be easily reached:
that \ishell{wf_collect=.true.} results in problems of writing and loading correct plane-wave coefficients on disks by pw.x because of the 32 bit integer limits. Thus, \ishell{pw2qmcpack.x} fails to convert the orbitals for QMCPACK. Since the release of QE v5.3.0, the converter has been fully parallelized to overcome this limitation completely.

By setting \ishell{wf_collect=.false.} (by default \ishell{.false.} in v6.1 and before and \ishell{.true.} since v6.2), pw.x does not collect the whole wavefunction into individual files for each k-point but instead writes one smaller file for each processor.
By running \ishell{pw2qmcpack.x} in the same parallel setup (MPI tasks and k-pools) as the last scf/nscf calculation with pw.x,
the orbitals distributed among processors will first be aggregated by the converter into individual temporal HDF5 files for each k-pool and then merged into the final file.
In large calculations, users should benefit from a significant reduction of time in writing the wavefunction by pw.x thanks to avoiding the wavefunction collection.

pw2qmcpack has been included in the test suite of QMCPACK (see instructions about how to activate the tests in Section~\ref{sec:buildqe}).
There are tests labeled ``no-collect'' running the pw.x with the setting \ishell{wf_collect=.false.}
The input files are stored at \ishell{examples/solids/dft-inputs-polarized-no-collect}.
The scf, nscf, and pw2qmcpack runs are performed on 16, 12, and 12 MPI tasks with 16, 2, and 2 k-pools respectively.
\subsection{convertpw4qmc}
Convertpw4qmc is an executable that reads xml from a plane wave based DFT code and produces a QMCPACK readable 
HDF5 format wavefunction.  For the moment, the only supported code is QBox, although others may follow.

In order to save the wavefunction from QBox so that convertpw4qmc can work on it, one needs to add a line to the
QBox input like
\begin{shade}
save -text -serial basename.sample
\end{shade}
after the end of a converged dft calculation.  This will write an ascii wavefunction file and will avoid
QBox's optimized parallel IO (which is not currently supported).

After the wavefunction file is written (basename.sample in this case) one can use convertpw4qmc as follows:
\begin{shade}
convertpw4qmc basename.sample -o qmcpackWavefunction.h5
\end{shade}

This reads the Qbox wavefunction and performs the Fourier transform before saving to a QMCPACK eshdf format wavefunction.  Currently multiple k-points are supported, but due to difficulties with the qbox wavefunction file format, the single particle orbitals do not have their proper energies associated with them.  This means that when tiling from a primitive cell to a supercell, the lowest n single particle orbitals from all necessary k-points will be used.  This can be problematic in the case of a metal and this feature should be used with EXTREME caution.


\subsection{ppconvert}
\label{sec:ppconvert}
\texttt{ppconvert} is a utility to convert PPs between different commonly used formats.
It is a stand-alone C++ executable that is not built by default but that is accessible via adding
\ishell{-DBUILD_PPCONVERT=1} to CMake and then typing \ishell{make ppconvert}.
Currently it converts CASINO, FHI, UPF (generated by OPIUM), BFD, and GAMESS formats to several other formats
including XML (QMCPACK) and UPF (QE). See all the formats via \ishell{ppconvert -h}.
For output formats requiring Kleinman-Bylander projectors, the atom will be solved with DFT
if the projectors are not provided in the input formats.
This requires providing reference states and sometimes needs extra tuning for heavy elements.
To avoid ghost states, the local channel can be changed via the \ishell{--local_channel} option. Ghost state considerations are similar to those of DFT calculations but could be worse if ghost states were not considered during the original PP construction.
To make the self-consistent calculation converge, the density mixing parameter may need to be reduced
via the \ishell{--density_mix} option.
Note that the reference state should include only the valence electrons.
One reference state should be included for each channel in the PP.
For example, for a sodium atom with a neon core, the reference state would be ``1s(1).''
\ishell{--s_ref} needs to include a 1s state, \ishell{--p_ref} needs to include a 2p state,
\ishell{--d_ref} needs to include a 3d state, etc. If not specified, a corresponding state with zero occupation is added.
If the reference state is chosen as the neon core, setting empty reference states ``'' is technically correct.
In practice, reasonable reference states should be picked with care.
For PP with semi-core electrons in the valence, the reference state can be long.
For example, Ti PP has 12 valence electrons. When using the neutral atom state,
\ishell{--s_ref}, \ishell{--p_ref}, and \ishell{--d_ref} are all set as ``1s(2)2p(6)2s(2)3d(2).''
When using an ionized state, the three reference states are all set as ``1s(2)2p(6)2s(2)'' or ``1s(2)2p(6)2s(2)3d(0).''
Unfortunately, if the generated UPF file is used in QE, the calculation may be incorrect because of the presence of ``ghost'' states. Potentially these can be removed by adjusting the local channel (e.g., by setting \ishell{--local_channel 1}, which chooses the p channel as the local channel instead of d.
For this reason, validation of UPF PPs is always required from the third row and is strongly encouraged in general. For example, check that the expected ionization potential and electron affinities are obtained for the atom and that dimer properties are consistent with those obtained by a quantum chemistry code or a plane-wave code that does not use the Kleinman-Bylander projectors.
\input{pseudopotentials}
\section{Interfaces}
\label{sec:interf}
QMCPack includes a generic interface class to manage the generation of single particle orbitals (SPOs), along with the
implementation of two derived classes for reading orbitals from and HDF5 file and generating the orbitals with a Quantum
Espresso computation performed during the QMCPack run. In this section the behaviour of these classes is explained. 
Other derived classes can be implemented e.g. to call other DFT codes from QMCPack, using these ones as reference.

\subsection{The ESInterfaceBase class}
The \ixml{ESInterfaceBase} class contains the definitions of types and virtual methods that are to be inherited by specific 
interface derived classes, in particular the creator/distructor, methods to inizialize, finalize and update the 
single particle orbitals and method to set/get from the interface relevant quantities (e.g. the number and position 
of the atoms, the number and weights of the twists and the atomic orbital themselves).

\subsection{The ESHDF5Interface class}
The \ixml{ESHDF5Interface} class is derived from \ixml{ESInterfaceBase}; this class uses uses the generic interface 
infrastructure to read and process information stored in a suitable HDF5 files, like e.g. the ones generated using 
Quantum Espresso and pw2qmcpack.
In a way using this class is an alternative of the default way of reading input using the keyword \ixml{bspline} in the \ixml{determinantset} xml block.
In Listing \ref{list:h5example} we show an example of the use of this interface in the \ixml{determinantset} block. The element \ixml{href}
refers to the name of the HDF5 input file. If not specified the file name will default to \ixml{pwscf.h5}.
\begin{lstlisting}[style=QMCPXML,caption=Example of \ixml{determinantset} block using the HDF5 interface \label{list:h5example}]
 <wavefunction name="psi0" target="e">
  ...
   <determinantset type="interfaceh5" href="O.pwscf.h5" sort="1" tilematrix="1 0 0 0 1 0 0 0 1" twistnum="0" source="ion0" version="0.10">
  ...
 </wavefunction>
\end{lstlisting}

\subsection{The ESPWSCFInterface class}
When using the \ixml{ESPWSCFInterface} derived class we generate the SPOs during a QMCPack simulation, calling the \ixml{pw.x} program from
within the simulation. In order to do so we need to properly patch and compile Quantum Espresso 5.3. In order to do so we have to execute the 
\ixml{QMCQEPack_download_and_patch_qe.sh} script in the \ixml{external_codes/quantum_espresso} directory. After that we need to run the 
\ixml{configure} script in the resulting directori \ixml{q-e-qe-5.3} and finally build the code using \ixml{make pw}. Note that when building the code 
it may be required to use the internal Quantum Espresso version of the FFTW libraries. In order to do so if is sufficient to change in the \ixml{DFLAGS} field 
of the \ixml{make.sys} file \ixml{-D__FFTW3} with \ixml{-D__FFTW}. The resulting \ixml{libpwinterface.so} library will
 be included in QMCPack if QMCPack is built using the options \ixml{QMC_COMPLEX} and \ixml{QE_INTERFACE}, e.g. by using

\ixml{cmake -DQMC_COMPLEX=1 -DQE_INTERFACE=1 <qmcpack directory>}.

In Listing \ref{list:pwexample} we show an example of the use of this interface in the \ixml{determinantset} block. The element \ixml{href}
refers to the name of the Quantum Espresso input file. If not specified the file name will default to \ixml{pwscf.in}. When run using this option QMCPack will call Quantum Espresso during the orbital generation, performing a DFT computation, splining and storing the SPOs and then it will go on with the regular QMC simulations specified in the input file. The Espresso standard output will be part of the QMCPack standard output.

\begin{lstlisting}[style=QMCPXML,caption=Example of \ixml{determinantset} block using the PWSCF interface \label{list:pwexample}]
 <wavefunction name="psi0" target="e">
  ...
      <determinantset type="qmcqepack" href="QMCQEPack.scf.in" sort="1" tilematrix="1 0 0 0 1 0 0 0 1" twistnum="0" source="ion" version="0.10">
  ...
 </wavefunction>
\end{lstlisting}


\chapter{External Tools}
\label{chap:external_tools}
This chapter provides some information on using QMCPACK with external tools.

\section{LLVM Sanitizer Libraries}\label{tool:LLVM-Sanitizer-Libraries}

Using CMake, set one of these flags for using the clang sanitizer libraries with or without lldb.

\begin{shade}
-DLLVM_SANITIZE_ADDRESS    link with the %*\href{https://clang.llvm.org/docs/AddressSanitizer.html}{Clang address sanitizer library}*
-DLLVM_SANITIZE_MEMORY     link with the %*\href{https://clang.llvm.org/docs/MemorySanitizer.html}{Clang memory sanitizer library}*
\end{shade}

These set the basic flags required to build with either of these sanitizer libraries. They require a build of clang with dynamic libraries somehow visible (i.e., through \ishell{LD_FLAGS=-L/your/path/to/llvm/lib}). You must link through clang, which is generally the default when building with it. Depending on your system and linker, this may be incompatible with the ``Release'' build, so set \ishell{-DCMAKE_BUILD_TYPE=Debug}. They have been tested with the default spack install of llvm@7.0.0 and been manually built with llvm 7.0.1. See the preceding  links for additional information on use, run time, and build options of the sanitizers.

In general, the address sanitizer libraries will catch most pointer-based errors. ASAN can also catch memory links but requires that additional options be set. MSAN will catch more subtle memory management errors but is difficult to use without a full set of MSAN-instrumented libraries.

\section{Intel VTune}

Intel's VTune profiler has an API that allows program control over collection (pause/resume) and can add information to the profile data (e.g., delineating tasks).

\subsection{VTune API}

If the variable \ishell{USE\_VTUNE\_API} is set, QMCPACK will check that the
include file (\ishell{ittnotify.h}) and the library (\ishell{libittnotify.a}) can
be found.
To provide CMake with the VTune paths, add the include path to \ishell{CMAKE\_CXX\_FLAGS} and the library path to \ishell{CMAKE\_LIBRARY\_PATH}.

An example of options to be passed to CMake:
\begin{shade}
 -DCMAKE_CXX_FLAGS=-I/opt/intel/vtune_amplifier_xe/include \
 -DCMAKE_LIBRARY_PATH=/opt/intel/vtune_amplifier_xe/lib64
\end{shade}

\section{NVIDIA Tools Extensions}

NVIDIA's Tools Extensions (NVTX) API enables programmers to annotate their source code when used with the NVIDIA profilers.

\subsection{NVTX API}

If the variable \ishell{USE_NVTX_API} is set, QMCPACK will add the library (\ishell{libnvToolsExt.so}) to the QMCPACK target. To add NVTX annotations
to a function, it is necessary to include the \ishell{nvToolsExt.h} header file and then make the appropriate calls into the NVTX API. For more information
about the NVTX API, see \url{https://docs.nvidia.com/cuda/profiler-users-guide/index.html#nvtx}. Any additional calls to the NVTX API should be guarded by
the \ishell{USE\_NVTX\_API} compiler define.

\subsection{Timers as Tasks}
To aid in connecting the timers in the code to the profile data, the start/stop of
timers will be recorded as a task if \ishell{USE_VTUNE_TASKS} is set.

In addition to compiling with \ishell{USE_VTUNE_TASKS}, an option needs to be set at run time to collect the task API data.
In the graphical user interface (GUI), select the checkbox labeled ``Analyze user tasks" when setting up the analysis type.
For the command line, set the \ishell{enable-user-tasks} knob to \ishell{true}. For example,
\begin{shade}
amplxe-cl -collect hotspots -knob enable-user-tasks=true ...
\end{shade}

Collection with the timers set at ``fine" can generate too much task data in the profile.
Collection with the timers at ``medium" collects a more reasonable amount of task data.

\section{Scitools Understand}

Scitools Understand (\url{https://scitools.com/}) is a tool for static
code analysis. The easiest configuration route is to use the JSON output
from CMake, which the Understand project importer can read directly:
\begin{enumerate}
\item Configure QMCPACK by running CMake with
  \ishell{CMAKE_EXPORT_COMPILE_COMMANDS=ON}, for example:
  \begin{lstlisting}[style=SHELL]
  cmake -DCMAKE_C_COMPILER=clang -DCMAKE_CXX_COMPILER=clang++
  -DQMC_MPI=0 -DCMAKE_EXPORT_COMPILE_COMMANDS=ON ../qmcpack/
  \end{lstlisting}
\item Run Understand and create a new C++ project. At the import files
  and settings dialog, import the \ishell{compile_commands.json} created by
  CMake in the build directory.  
\end{enumerate}


\chapter{Contributing to the Manual}
\label{chap:contrib}

This section briefly describes how to contribute to the manual.  This is primarily ``by developers, for developers''.   This section should iterate until a consistent view on style/contents is reached.

\textbf{\underline{Desirable:}}
\begin{itemize}
\item{Use the table templates below when describing XML input.}
\item{Instead of \ilatex{\\texttt} or \ilatex{\\verb} use
    \begin{itemize}
      \item{\ilatex{\\ishell} for shell text}
      \item{\ilatex{\\ixml} for xml text}
      \item{\ilatex{\\icode} for C++ text}
      \end{itemize}
     \bf{Except} within tabularx or math environments}
    \item{Instead of \ilatex{\\begin\{verbatim\}} environments use the appropriate \ilatex{\\begin\{lstlisting\}[style=<see qmcpack_listings.sty>]}}
\item{\ilatex{\\begin\{shade\}} can be used in place of \ilatex{\\begin\{lstlisting\}[style=SHELL]}}
\item{Unicode rules}
\begin{itemize}
\item Do not use characters for which well established latex idiom exists, especially dashes, quotes, and apostrophes.
\item Use math mode markup instead of unicode characters for equations.
\item Be cautious of WYSIWYG word processors, cutting and pasting can pickup characters promoted to unicode by the program.
\item Take a look at your text multibyte expanded i.e. open in (emacs and `esc-x toggle-enable-multibyte-characters`), see any unicode you didn't intend?
\end{itemize}
\item{Place unformatted text targeted at developers working on the latex in comments.  Include generously.}
\item{Encapsulate formatted text aimed at developers (like this entire chapter), in \ilatex{\\dev\{\}}.  Text encapsulated in this way will be removed from the user version of the manual by editing the definition of \ilatex{\\dev\{\}} in \ishell{qmcpack_manual.tex}.  Existing but deprecated or partially functioning features fall in this category.}
\item Newly added entries to a Bib file should be complete as possible. Use a tool such as JabRef or Zotero that can automate creation of these entries from just a DOI.
\end{itemize}

\textbf{\underline{Forbidden:}}
\begin{itemize}
\item Including images instead of using lstlisting sections for text.
\item Packages the LaTeX community considers \href{https://latex.org/forum/viewtopic.php?f=37&t=6637}{deprecated}.
\item Do not use packages, features, or fonts not included in texlive 2017 unless you insure they degrade reasonably for 2017.
\item Don't add packages unless they are bringing great value and are supported by tex4ht (unless you are willing to add the support).
\item Tex files and Bib files are UTF8 encoded, do not save them in other encodings. Some may report being ASCII encoded since they contain no unicode characters.
\end{itemize}


\textbf{\underline{Missing sections (these are opinions, not decided priorities):}}
\begin{itemize}
  \item{Description of XML input in general.  Discuss XML format, use of attributes and \texttt{<parameter/>}'s in general, case sensitivity (input is generally case sensitive), and behavior of \qmcpack when unrecognized XML elements are encountered (they are generally ignored without notification).}
  \item{Overview of the input file in general, broad structure, and at least one full example that works in isolation.}
\end{itemize}


\textbf{\underline{Information currently missing for a complete reference specification:}}
\begin{itemize}
  \item{Noting how many instances of each child element are allowed.  Examples: \texttt{simulation}--1 only, \texttt{method}--1 or more, \texttt{jastrow}--0 or more}.
\end{itemize}


Below are template tables for describing XML elements in reference fashion.  A number of examples can be found in \textit{e.g.} Chapter~\ref{chap:hamiltobs}.  Preliminary style is (please weigh in with opinions): typewriter text (\ilatex{\\texttt\{\}}) for XML element, attribute, and parameter names, normal text for literal information in datatype/values/default columns, bold (\ilatex{\\textbf\{\}}) text if an attribute/parameter must take on a particular value (values column), italics (\ilatex{\\textit\{\}}) for descriptive (non-literal) information in the values column (e.g. \textit{anything}, \textit{non-zero}, etc.), required/optional attributes/parameters noted by \texttt{some\_attr$^r$}/\texttt{some\_attr$^o$} superscripts.  Valid datatypes are text, integer, real, boolean, and arrays of each.  Fixed lengh arrays can be noted, \textit{e.g.} by ``real array(3)''.


Template for a generic XML element:
\FloatBarrier
\begin{table}[h]
\begin{center}
\begin{tabularx}{\textwidth}{l l l l l X }
\hline
\multicolumn{6}{l}{\texttt{generic} element} \\
\hline
\multicolumn{2}{l}{parent elements:} & \multicolumn{4}{l}{\texttt{parent1 parent2}}\\
\multicolumn{2}{l}{child  elements:} & \multicolumn{4}{l}{\texttt{child1 child2 child3 ...}}\\
\multicolumn{2}{l}{attributes}  & \multicolumn{4}{l}{}\\
   &   \bfseries name     & \bfseries datatype & \bfseries values & \bfseries default   & \bfseries description \\
   &   \texttt{attr1}$^r$ &  text              &                  &                     &                       \\
   &   \texttt{attr2}$^r$ &  integer           &                  &                     &                       \\
   &   \texttt{attr3}$^o$ &  real              &                  &                     &                       \\
   &   \texttt{attr4}$^o$ &  boolean           &                  &                     &                       \\
   &   \texttt{attr5}$^o$ &  text array        &                  &                     &                       \\
   &   \texttt{attr6}$^o$ &  integer array     &                  &                     &                       \\
   &   \texttt{attr7}$^o$ &  real array        &                  &                     &                       \\
   &   \texttt{attr8}$^o$ &  boolean array     &                  &                     &                       \\
\multicolumn{2}{l}{parameters}  & \multicolumn{4}{l}{}\\
   &   \bfseries name     & \bfseries datatype & \bfseries values & \bfseries default   & \bfseries description \\
   &   \texttt{param1}$^r$&  text              &                  &                     &                       \\
   &   \texttt{param2}$^r$&  integer           &                  &                     &                       \\
   &   \texttt{param3}$^o$&  real              &                  &                     &                       \\
   &   \texttt{param4}$^o$&  boolean           &                  &                     &                       \\
   &   \texttt{param5}$^o$&  text array        &                  &                     &                       \\
   &   \texttt{param6}$^o$&  integer array     &                  &                     &                       \\
   &   \texttt{param7}$^o$&  real array        &                  &                     &                       \\
   &   \texttt{param8}$^o$&  boolean array     &                  &                     &                       \\
\multicolumn{2}{l}{body text}  & \multicolumn{4}{l}{}\\
   &                           & \multicolumn{4}{l}{Long form description of body text format}                   \\
  \hline
\end{tabularx}
\end{center}
\end{table}
\FloatBarrier



``Factory'' elements are XML elements that share a tag, but whose contents change based on the value an attribute (or sometimes multiple attributes take).  The attribute(s) that determine the allowed contents is referred to below as the ``type selector''  (\textit{e.g.} for \ixml{<estimator/>} elements, the type selector is usually the \ixml{type} attribute).  These types of elements are frequently encountered as they correspond (sometimes loosely, sometimes literally) to polymorphic classes in \qmcpack that are built in ``factories''.  This name is true to the underlying code, but may be obscure to the general user (is there a better name to retain the general meaning?).   

The template below should be provided each time a new ``factory'' type is encountered (like \ixml{<estimator/>}).  The table lists all types of possible elements (see ``type options'' below) and any attributes that are common to all possible related elements.  Specific ``derived'' elements are then described one at a time with the template above, noting the type selector in addition to the XML tag (\textit{e.g.} ``\ixml{estimator type=density} element'').

Template for shared information about ``factory'' elements.
\FloatBarrier
\begin{table}[h]
\begin{center}
\begin{tabularx}{\textwidth}{l l l l l X }
\hline
\multicolumn{6}{l}{\texttt{generic} factory element} \\
\hline
\multicolumn{2}{l}{parent elements:} & \multicolumn{4}{l}{\texttt{parent1 parent2}}\\
\multicolumn{2}{l}{child  elements:} & \multicolumn{4}{l}{\texttt{child1 child2 child3 ...}}\\
\multicolumn{2}{l}{type   selector:} & \multicolumn{4}{l}{\texttt{some} attribute}\\
\multicolumn{2}{l}{type   options :} & \multicolumn{4}{l}{Selection1}\\
\multicolumn{2}{l}{                } & \multicolumn{4}{l}{Selection2}\\
\multicolumn{2}{l}{                } & \multicolumn{4}{l}{Selection3}\\
\multicolumn{2}{l}{                } & \multicolumn{4}{l}{...}\\
\multicolumn{2}{l}{shared attributes:} & \multicolumn{4}{l}{}\\
   &   \bfseries name     & \bfseries datatype & \bfseries values & \bfseries default   & \bfseries description \\
   &   \texttt{attr1}     &  text              &                  &                     &                       \\
   &   \texttt{attr2}     &  integer           &                  &                     &                       \\
   &   ...                &                    &                  &                     &                       \\
  \hline
\end{tabularx}
\end{center}
\end{table}
\FloatBarrier

\chapter{Unit Testing}
\label{chap:unit_testing}

Unit testing is a standard software engineering practice to aid in ensuring a quality product. A good suite of unit tests provides confidence in refactoring and changing code, provides some documentation on how classes and functions are used, and can drive a more decoupled design.

If unit tests do not already exist for a section of code, you are encouraged to add them when modifying that section of code.  New code additions should also include unit tests.
When possible, fixes for specific bugs should also include a unit test that would have caught the bug.

\section {Unit testing framework} The Catch framework is used for unit testing.
See the project site for a tutorial and documentation: \url{https://github.com/philsquared/Catch}

Catch consists solely of header files. It is distributed as a single include file about 400KB in size.  In QMCPACK, it is stored in \ishell{external\_codes/catch}.

\section{Unit test organization}

\begin{sloppypar}
The source for the unit tests is located in the \ishell{tests} directory under each directory in \ishell{src} (e.g. \ishell{src/QMCWavefunctions/tests}).
All of the tests in each \ishell{tests} directory get compiled into an executable.
After building the project, the individual unit test executables can be found in \ishell{build/tests/bin}.
For example, the tests in \ishell{src/QMCWavefunctions/tests} are compiled into \ishell{build/tests/bin/test\_wavefunction}.
\end{sloppypar}

All the unit test executables are collected under CTest with the \ishell{unit} label.
When checking the whole code, it's useful to run through CMake (\ishell{cmake -L unit}).
When working on an individual directory, it's useful to run the individual executable.

Some of the tests reference input files. The unit test CMake setup places those input files in particular locations under the \ishell{tests} directory (e.g. \ishell{tests/xml\_test}).  The individual test needs to be run from that directory to find the expected input files.

Command line options are available on the unit test executables.  Some of the more useful ones are
\begin{description}
\item[\ishell{-h}]  List command line options.
\item [\ishell{--list-tests}] List all the tests in the executable.
\end{description}

A test name can be given on the command line to execute just that test.  This is useful when iterating
on a particular test, or when running in the debugger.   Test names often contain spaces, so most command line environments require enclosing the test name in single or double quotes.



\section{Example}

The first example is one test from \ishell{src/Numerics/tests/test\_grid\_functor.cpp}

\begin{minipage}{\linewidth}
\begin{lstlisting}[language=C++,caption={Unit test example using Catch},label=CatchExample,basicstyle=\ttfamily]
TEST_CASE("double_1d_grid_functor", "[numerics]")
{
  LinearGrid<double> grid;
  OneDimGridFunctor<double> f(&grid);

  grid.set(0.0, 1.0, 3);

  REQUIRE(grid.size() == 3);
  REQUIRE(grid.rmin() == 0.0);
  REQUIRE(grid.rmax() == 1.0);
  REQUIRE(grid.dh() == Approx(0.5));
  REQUIRE(grid.dr(1) == Approx(0.5));
}
\end{lstlisting}
\end{minipage}

The test function declaration is
\ishell{TEST\_CASE("double\_1d\_grid\_functor","[numerics]")}.
The first argument is the test name, and it must be unique in the test suite.
The second argument is an optional list of tags.  Each tag is a name surrounded by brackets (\ishell{"[tag1][tag2]"}).  It can also be the empty string.

The \ishell{REQUIRE} macro accepts expressions with C++ comparison operators and records an error if the value of the expression is false.

Floating point numbers may have small differences due to roundoff, etc.   The \ishell{Approx} class adds some tolerance to the comparison.  Place it on either side of the comparison (e.g. \ishell{Approx(a) == 0.3} or \ishell{a = Approx(0.3)}).   To adjust the tolerance, use the \ishell{epsilon} and \ishell{scale} methods to \ishell{Approx} (\ishell{REQUIRE(Approx(a).epsilon(0.001) = 0.3);}.

\subsection{Expected output}

When running the test executables individually, the output of a run with no failures should look like
\begin{shade}
===============================================================================
All tests passed (26 assertions in 4 test cases)
\end{shade}

A test with failures will look like

\begin{minipage}{\linewidth}
\begin{shade}
~~~~~~~~~~~~~~~~~~~~~~~~~~~~~~~~~~~~~~~~~~~~~~~~~~~~~~~~~~~~~~~~~~~~~~~~~~~~~~~
test_numerics is a Catch v1.4.0 host application.
Run with -? for options

-------------------------------------------------------------------------------
double_1d_grid_functor
-------------------------------------------------------------------------------
/home/user/qmcpack/src/Numerics/tests/test_grid_functor.cpp:29
...............................................................................

/home/user/qmcpack/src/Numerics/tests/test_grid_functor.cpp:39: FAILED:
  REQUIRE( grid.dh() == Approx(0.6) )
with expansion:
  0.5 == Approx( 0.6 )

===============================================================================
test cases:  4 |  3 passed | 1 failed
assertions: 25 | 24 passed | 1 failed
\end{shade}
\end{minipage}


\section{Adding tests}
There are three scenarios covered here: adding a new test in an existing file, adding a new test file, or adding a new \ishell{test} directory.

\subsection{Adding a test to existing file}
Copy an existing test, or from the example shown here.  Be sure to change the test name.

\subsection{Adding a test file}
When adding a new test file,
create a file in the test directory, or copy from an existing file.  Add the file name to the \ishell{ADD\_EXECUTABLE} in the \ishell{CMakeLists.txt} file in that directory.

One (and only one) file must define the \ishell{main} function for the test executable by defining \ishell{CATCH\_CONFIG\_MAIN} before including the Catch header.  If more than one file defines this value, there will be linking errors about multiply defined values.

Some of the tests need to shut down MPI properly to avoid extraneous error messages. Those tests include \ishell{Message/catch\_mpi\_main.hpp} instead of defining \ishell{CATCH\_CONFIG\_MAIN}.


\subsection{Adding a test directory}
Copy the CMakeLists.txt file from an existing \ishell{tests} directory.
Change the \ishell{SRC\_DIR} name and the  files in the \ishell{ADD\_EXECUTABLES} line.  The libraries to link in \ishell{TARGET\_LINK\_LIBRARIES} may need to be updated.

Add the new test directory to \ishell{src/CMakeLists.txt} in the \ishell{BUILD\_UNIT\_TESTS} section near the end.


\section{Testing with random numbers}
Many algorithms and parts of the code depend on random numbers, which makes validating the results difficult.
One solution is to verify that certain properties hold for any random number.
This approach is valuable at some levels of testing, but is unsatisfying at the unit test level.

The \ishell{Utilities} directory contains a 'fake' random number generator that can be used for deterministic tests of these parts of the code.
Currently it outputs a single, fixed value every time it is called, but it could be expanded to produce more varied, but still deterministic, sequences.
See \ishell{src/QMCDrivers/test\_vmc.cpp} for an example of using the fake random number generator.


\chapter{QMCPACK Design and Feature Documentation}
\label{chap:design_features}

This section contains information on the overall design of QMCPACK.  Also included in this section are detailed explanations/derivations of major features and algorithms present in the code.


\section{QMCPACK Design}
TBD.



\newpage
\section{Feature: Optimized Long-Ranged Breakup (Ewald)}

% Written by Ken Esler as part of the Common codebase used in wfconvert
% Originally titled ``Ewald Breakup for Long-Range Potentials in PIMC''
% PIMC-specific portions have been commented out

Consider a group of particles interacting with long-ranged central
potentials, $v^{\alpha \beta}(|r^{\alpha}_i - r^{\beta}_j|)$, where the Greek superscripts
represent the particle species (eg. $\alpha=\text{electron}$,
$\beta=\text{proton}$), and Roman subscripts refer to particle number
within a species.  We can then write the total interaction energy for
the system as,
\newcommand{\vr}{\mathbf{r}}
\newcommand{\vR}{\mathbf{R}}
\newcommand{\vk}{\mathbf{k}}
\newcommand{\vq}{\mathbf{q}}
\begin{equation}
V = \sum_\alpha \left\{\sum_{i<j} v^{\alpha\alpha}(|\vr^\alpha_i - \vr^\alpha_j|) +
\sum_{\beta<\alpha} 
\sum_{i,j} v^{\alpha \beta}(|\vr^{\alpha}_i - \vr^{\beta}_j|) \right\}
\label{eq:Vperiodic}
\end{equation}
\newcommand{\va}{\mathbf{a}}
\newcommand{\vb}{\mathbf{b}}
\newcommand{\vL}{\mathbf{L}}

\subsection{The Long-Range Problem}
Consider such a system in periodic boundary conditions in a cell
defined by primitive lattice vectors $\va_1$, $\va_2$, and $\va_3$.
Let $\vL \equiv n_1 \va_1 + n_2 \va_2 + n_3\va_3$ be a direct lattice
vector.  Then the interaction energy per cell for the periodic system
is given by
\begin{equation}
\begin{split}
V = & \sum_\vL \sum_\alpha \left\{ 
\overbrace{\sum_{i<j} v^{\alpha\alpha}(|\vr^\alpha_i - \vr^\alpha_j + \vL|)}^{\text{homologous}} +
\overbrace{\sum_{\beta<\alpha} 
\sum_{i,j} v^{\alpha \beta}(|\vr^{\alpha}_i - \vr^{\beta}_j+\vL|)}^{\text{heterologous}}
\right\}  \\
& + \underbrace{\sum_{\vL \neq \mathbf{0}} \sum_\alpha N^\alpha v^{\alpha \alpha} (|\vL|)}_\text{Madelung}
\end{split}
\label{eq:direct},
\end{equation}
where $N^\alpha$ is the number particles of species $\alpha$.
If the potentials $v^{\alpha\beta}(r)$ are indeed long-range, the
summation over direct lattice vectors will not converge in this naive
form.  A solution to the problem was posited by Ewald.  We break the
central potentials into two pieces -- a short range and a long range
part define by
\begin{equation}
v^{\alpha \beta}(r) = v_s^{\alpha\beta}(r) + v_l^{\alpha \beta}(r).
\end{equation}
We will perform the summation over images for the short-range part in
real space, while performing the sum for the long-range part in
reciprocal space.  For simplicity, we choose $v^{\alpha \beta}_s(r)$
so that it is identically zero at the half the box length.  This
eliminates the need to sum over images in real space.


\subsection{Reciprocal-Space Sums}
\subsubsection{Heterologous terms}
We begin with Eq.~\ref{eq:direct}, starting with the heterologous terms,
i.e. the terms involving particles of different species.  The
short-range terms are trivial, so we neglect them here.
\begin{equation}
\text{heterologous} = \frac{1}{2} \sum_{\alpha \neq \beta} \sum_{i,j} \sum_\vL
v^{\alpha\beta}_l(\vr_i^\alpha - \vr_j^\beta + \vL)
\end{equation}
We insert the resolution of unity in real space twice,
\begin{eqnarray}
\text{heterologous} & = & \frac{1}{2}\sum_{\alpha \neq \beta} \int_\text{cell} d\vr \, d\vr' \, \sum_{i,j}
\delta(\vr_i^\alpha - \vr) \delta(\vr_j^\beta-\vr') \sum_\vL
v^{\alpha\beta}_l(|\vr - \vr' + \vL|) \\
& = & \frac{1}{2\Omega^2}\sum_{\alpha \neq \beta} \int_\text{cell} d\vr \, d\vr' \, \sum_{\vk, \vk', i, j} e^{i\vk\cdot(\vr_i^\alpha
  - \vr)} e^{i\vk'\cdot(\vr_j^\beta - \vr')} \sum_\vL
v^{\alpha\beta}_l(|\vr - \vr' + \vL|) \nonumber \\
& = & \frac{1}{2\Omega^2} \sum_{\alpha \neq \beta} \int_\text{cell} d\vr \, d\vr'\,
\sum_{\vk, \vk', \vk'', i, j} e^{i\vk\cdot(\vr_i^\alpha - \vr)}
e^{i\vk'\cdot(\vr_j^\beta-\vr')} e^{i\vk''\cdot(\vr -\vr')}
v^{\alpha\beta}_{\vk''}, \nonumber.
\end{eqnarray}
Here, the $\vk$ summations are over reciprocal lattice vectors given
by $\vk = m_1 \vb_1 + m_2\vb_2 + m_3\vb_3$, where
\begin{eqnarray}
\vb_1 & = & 2\pi \frac{\va_2 \times \va_3}{\va_1 \cdot (\va_2 \times
  \va_3)} \nonumber \\
\vb_2 & = & 2\pi \frac{\va_3 \times \va_1}{\va_1 \cdot (\va_2 \times
  \va_3)} \\
\vb_3 & = & 2\pi \frac{\va_1 \times \va_2}{\va_1 \cdot (\va_2 \times
  \va_3)} \nonumber.
\end{eqnarray}
We note that $\vk \cdot \vL = 2\pi(n_1 m_1 + n_2 m_2 + n_3 m_3)$. 

\begin{eqnarray}
v_{k''}^{\alpha \beta} & = & 
\frac{1}{\Omega} \int_{\text{cell}} d\vr'' \sum_\vL
e^{-i\vk''\cdot(|\vr''+\vL|)} v^{\alpha\beta}(|\vr''+\vL|), \\
& = & \frac{1}{\Omega} \int_\text{all space} d\tilde{\vr} \, 
    e^{-i\vk'' \cdot \tilde{\vr}} v^{\alpha\beta}(\tilde{r}), \label{eq:vk}
\end{eqnarray}
where $\Omega$ is the volume of the cell. Here we have used the fact
that summing over all cells of the integral over the cell is
equivalent to integrating over all space.
\begin{equation}
\text{hetero} = \frac{1}{2\Omega^2} \sum_{\alpha \neq \beta}
\int_\text{cell} d\vr \, d\vr' \, \sum_{\vk, \vk', \vk'', i, j}
e^{i(\vk \cdot \vr_i^\alpha + \vk' \cdot\vr_j^\beta)} e^{i(\vk''-\vk)\cdot \vr}
e^{-i(\vk'' + \vk')\cdot \vr'} v^{\alpha \beta}_{\vk''}.
\end{equation}
We have
\begin{equation}
\frac{1}{\Omega} \int d\vr \  e^{i(\vk -\vk')\cdot \vr} =
\delta_{\vk,\vk'},
\end{equation}
Then, performing the integrations we have
\begin{eqnarray}
\text{hetero} = \frac{1}{2} \sum_{\alpha \neq \beta}
\sum_{\vk, \vk', \vk'', i, j}
e^{i(\vk \cdot \vr_i^\alpha + \vk' \cdot\vr_j^\beta)} \delta_{\vk,\vk''}
\delta_{-\vk', \vk''} v^{\alpha \beta}_{\vk''}.
\end{eqnarray}
We now separate the summations, yielding
\begin{equation}
\text{hetero} = \frac{1}{2} \sum_{\alpha \neq \beta} \sum_{\vk, \vk'}
\underbrace{\left[\sum_i e^{i\vk  \cdot \vr_i^\alpha} \rule{0cm}{0.705cm}
    \right]}_{\rho_\vk^\alpha}
\underbrace{\left[\sum_j e^{i\vk' \cdot \vr_j^\beta} \right]}_{\rho_{\vk'}^\beta}
 \delta_{\vk,\vk''} \delta_{-\vk', \vk''} v^{\alpha
  \beta}_{\vk''}.
\end{equation}
Summing over $\vk$ and $\vk'$, we have
\begin{equation}
\text{hetero} = \frac{1}{2} \sum_{\alpha \neq \beta} \sum_{\vk''}
\rho_{\vk''}^\alpha \, \rho_{-\vk''}^\beta v_{k''}^{\alpha \beta}.
\end{equation}
We can simplify the calculation a bit further by rearranging the
sums over species,
\begin{eqnarray}
\text{hetero} & = & \frac{1}{2} \sum_{\alpha > \beta} \sum_{\vk}
\left(\rho^\alpha_\vk \rho^\beta_{-\vk} + \rho^\alpha_{-\vk}
\rho^\beta_\vk\right) v_{k}^{\alpha\beta} \\
& = & \sum_{\alpha > \beta} \sum_\vk \mathcal{R}e\left(\rho_\vk^\alpha
\rho_{-\vk}^\beta\right)v_k^{\alpha\beta} .
\end{eqnarray}


\subsubsection{Homologous Terms}
We now consider the terms involving particles of the same species
interacting with each other.  The algebra is very similar to that
above, with the slight difficulty of avoiding the self-interaction term.
\begin{eqnarray}
\text{homologous} & = & \sum_\alpha \sum_L \sum_{i<j} v_l^{\alpha
  \alpha}(|\vr_i^\alpha - \vr_j^\alpha + \vL|) \\
 & = & \frac{1}{2} \sum_\alpha \sum_L \sum_{i\neq j} v_l^{\alpha
  \alpha}(|\vr_i^\alpha - \vr_j^\alpha + \vL|) 
\end{eqnarray}
\begin{eqnarray}
\text{homologous} & = & \frac{1}{2} \sum_\alpha \sum_L 
\left[
-N^\alpha v_l^{\alpha \alpha}(|\vL|)  + \sum_{i,j} v^{\alpha \alpha}_l(|\vr_i^\alpha - \vr_j^\alpha + \vL|)
  \right] \\
& = & \frac{1}{2} \sum_\alpha \sum_\vk \left(|\rho_k^\alpha|^2 - N
\right) v_k^{\alpha \alpha}
\end{eqnarray}

\subsubsection{Madelung Terms}
Let us now consider the Madelung term for a single particle of species
$\alpha$.  This term corresponds to the interaction of a particle with
all of its periodic images.  
\begin{eqnarray}
v_M^{\alpha} & = & \frac{1}{2} \sum_{\vL \neq \mathbf{0}} v^{\alpha
  \alpha}(|\vL|) \\
& = & \frac{1}{2} \left[ -v_l^{\alpha \alpha}(0) + \sum_\vL v^{\alpha
  \alpha}(|\vL|) \right] \\
& = & \frac{1}{2} \left[ -v_l^{\alpha \alpha}(0) + \sum_\vk v^{\alpha
  \alpha}_\vk \right]  
\end{eqnarray}

\subsubsection{$\vk=\mathbf{0}$ terms}
Thus far, we have neglected what happens at the special point $\vk =
\mathbf{0}$.  For many long-range potentials, such as the coulomb
potential, $v_k^{\alpha \alpha}$ diverges for $k=0$.  However, we
recognize that for a charge-neutral system, the divergent part of the
terms cancel each other.  If all the potential in the system were
precisely coulomb, the $\vk=\mathbf{0}$ terms would cancel precisely,
yielding zero.  For systems involving pseudopotentials, however, it
may be the case the resulting term is finite, but nonzero.  Consider
the terms from $\vk=\mathbf{0}$,
\begin{eqnarray}
V_{k=0} & = & \sum_{\alpha>\beta} N^\alpha N^\beta v^{\alpha \beta}_{k=0}
+ \frac{1}{2} \sum_\alpha \left(N^{\alpha}\right)^2 v^{\alpha\alpha}_{k=0} \\
& = & \frac{1}{2} \sum_{\alpha,\beta} N^\alpha N^\beta v^{\alpha
  \beta}_{k=0}.
\label{eq:kzero}
\end{eqnarray}
Next, we must compute $v^{\alpha \beta}_{k=0}$.  
\begin{equation}
v^{\alpha \beta}_{k=0} = \frac{4 \pi}{\Omega} \int_0^\infty dr\ r^2
v_l^{\alpha \beta}(r)
\end{equation}
We recognize that this integral will not converge because of the
large-$r$ behavior.  However, we recognize that when we do the sum in
Eq.~\ref{eq:kzero}, the large-$r$ parts of the integrals will cancel
precisely.  Therefore, we define
\begin{equation}
\tilde{v}^{\alpha \beta}_{k=0} = \frac{4 \pi}{\Omega} 
\int_0^{r_\text{end}} dr\ r^2 v_l^{\alpha \beta}(r),
\end{equation}
where $r_{\text{end}}$ is some cutoff value after which the potential
tails precisely cancel.

\subsubsection{Neutralizing Background Terms}
For systems with a net charge, such as the one-component plasma
(jellium), we add a uniform background charge which makes the system
neutral.  When we do this, we must add a term which comes from the
interaction of the particle with the neutral background.  It is a
constant term, independent of the particle positions.  In general, we
have a compensating background for each species, which largely cancels
out for neutral systems.
\begin{equation}
V_\text{background} = -\frac{1}{2} \sum_\alpha \left(N^\alpha\right)^2 
v^{\alpha \alpha}_{s\mathbf{0}}
-\sum_{\alpha > \beta} N_\alpha N_\beta
v^{\alpha\beta}_{s\mathbf{0}},
\end{equation}
where $v^{\alpha \beta}_{s\mathbf{0}}$ is given by
\begin{eqnarray}
v^{\alpha \beta}_{s\mathbf{0}} & = & \frac{1}{\Omega} \int_0^{r_c} d^3 r\ 
v^{\alpha \beta}_s(r) \\
& = & \frac{4 \pi}{\Omega} \int_0^{r_c} r^2 v_s(r) \ dr \nonumber
\end{eqnarray}


\subsection{Combining Terms}
Here, we sum all of the terms we computed in the sections above,
\begin{eqnarray}
V & = & \sum_{\alpha > \beta} \left[\sum_{i,j} v_s(|\vr_i^\alpha
  -\vr_j^\beta|) + \sum_\vk \mathcal{R}e\left(\rho_\vk^\alpha
  \rho_{-\vk}^\beta\right)v^{\alpha\beta}_k  -N^\alpha N^\beta
  v^{\alpha \beta}_{s\mathbf{0}}  \right] \nonumber \\
& + & \sum_\alpha \left[ N^\alpha v_M^\alpha + \sum_{i>j} v_s(|\vr_i^\alpha -
  \vr_j^\alpha|) + \frac{1}{2} \sum_\vk \left( |\rho_\vk^\alpha|^2 -
  N\right) v^{\alpha\alpha}_\vk -\frac{1}{2}\left(N_\alpha\right)^2 v_{s\mathbf{0}}^{\alpha\alpha}\right] \nonumber \\
& = & \sum_{\alpha > \beta} \left[\sum_{i,j} v_s(|\vr_i^\alpha
  -\vr_j^\beta|) + \sum_\vk \mathcal{R}e\left(\rho_\vk^\alpha
  \rho_{-\vk}^\beta\right) v^{\alpha \beta}_k   -N^\alpha N^\beta
  v^{\alpha \beta}_{s\mathbf{0}}  +\tilde{V}_{k=0} \right] \\
& + & \sum_\alpha \left[ -\frac{N^\alpha v_l^{\alpha \alpha}(0)}{2}  + \sum_{i>j} v_s(|\vr_i^\alpha -
  \vr_j^\alpha|) + \frac{1}{2} \sum_\vk |\rho_\vk^\alpha|^2 v^{\alpha\alpha}_\vk - \frac{1}{2}\left(N_\alpha\right)^2
  v_{s\mathbf{0}}^{\alpha\alpha} +\tilde{V}_{k=0}\right]  \nonumber
\end{eqnarray}

\subsection {Computing the Reciprocal Potential}
Now we return to Eq.~\ref{eq:vk}.  Without loss of generality, we define
for convenience $\vk = k\hat{\mathbf{z}}$.
\begin{equation}
v^{\alpha \beta}_k = \frac{2\pi}{\Omega} \int_0^\infty dr \int_{-1}^1
  d\cos(\theta) \ r^2 e^{-i k r \cos(\theta)} v_l^{\alpha \beta}(r)
\end{equation}
We do the angular integral first.  By inversion symmetry, the
imaginary part of the integral vanishes, yielding
\begin{equation}
v^{\alpha \beta}_k = \frac{4\pi}{\Omega k}\int _0^\infty dr\ r \sin(kr)
v^{\alpha \beta}_l(r).
\label{eq:vkint}
\end{equation}

\subsection{The Coulomb Potential}
For the case of the Coulomb potential, the above integral is not
formally convergent if we do the integral naively. We may remedy the
situation by including a convergence factor, $e^{-k_0 r}$.  For a
potential of the form $v^{\text{coul}}(r) = q_1 q_2/r$, this yields
\begin{eqnarray}
v^{\text{screened coul}}_k & = & \frac{4\pi q_1 q_2}{\Omega k} \int_0^\infty dr\ \sin(kr)
e^{-k_0r} \\ 
& = & \frac{4\pi q_1 q_2}{\Omega (k^2 + k_0^2)}
\end{eqnarray}
Allowing the convergence factor to tend to zero, we have
\begin{equation}
v_k^\text{coul} = \frac{4 \pi q_1 q_2}{\Omega k^2}
\end{equation}

For more generalized potentials with a coulomb tail, we cannot
evaluate Eq.~\ref{eq:vkint} numerically but must handle the coulomb part
analytically.  In this case, we have
\begin{equation}
v_k^{\alpha \beta} = \frac{4\pi}{\Omega} 
\left\{ \frac{q_1 q_2}{k^2} + \int_0^\infty dr \ r \sin(kr) \left[ v_l^{\alpha \beta}(r) -
  \frac{q_1 q_2}{r} \right] \right\}
\end{equation}

\subsection{Efficient calculation methods}
\subsubsection{Fast computation of $\rho_\vk$}
We wish to quickly calculate the quantity
\begin{equation}
\rho_\vk^\alpha \equiv \sum_i e^{i\vk \cdot r_i^\alpha}
\end{equation}
First, we write 
\begin{eqnarray}
\vk & = & m_1 \vb_1 + m_2 \vb_2 + m_3 \vb_3 \\
\vk \cdot \vr_i^\alpha & = &  m_1 \vb_1 \cdot \vr_i^\alpha + 
m_2 \vb_2 \cdot \vr_i^\alpha + m_3 \vb_3 \cdot \vr_i^\alpha \\
e^{i\vk \cdot r_i^\alpha} & = & 
{\underbrace{\left[e^{i \vb_1 \cdot\vr_i^\alpha}\right]}_{C^{i\alpha}_1}}^{m_1}
{\underbrace{\left[e^{i \vb_2 \cdot\vr_i^\alpha}\right]}_{C^{i\alpha}_2}}^{m_2}
{\underbrace{\left[e^{i \vb_3 \cdot\vr_i^\alpha}\right]}_{C^{i\alpha}_3}}^{m_3}
\end{eqnarray}
Now, we note that
\begin{equation}
[C^{i\alpha}_1]^{m_1} = C^{i\alpha}_1 [C^{i\alpha}]^{(m_1-1)}.
\end{equation}
This allows us to recursively build up an array of the $C^{i\alpha}$s,
and then compute $\rho_\vk$ for all $\vk$-vectors by looping over all
k-vectors, requiring only two complex multiplies per particle per
$\vk$.
\begin{algorithm}
\caption{Algorithm to quickly calculate $\rho_\vk^\alpha$.}
\begin{algorithmic}
\STATE Create list of $\vk$-vectors and corresponding $(m_1, m_2,
m_3)$ indices.
\FORALL{$\alpha \in $ species}
  \STATE Zero out $\rho_\vk^\alpha$
  \FORALL{$i \in $ particles}
    \FOR{$j \in [1\cdots3]$}
      \STATE Compute $C^{i \alpha}_j \equiv e^{i \vb_j \cdot
        \vr^{\alpha}_i}$
       \FOR{$m \in [-m_{\text{max}}\dots m_{\text{max}}]$}
         \STATE Compute $[C^{i \alpha}_j]^m$ and store in array
       \ENDFOR
    \ENDFOR
     \FORALL{$(m_1, m_2, m_3) \in $ index list}
       \STATE Compute $e^{i \vk \cdot r^\alpha_i} =
         [C^{i\alpha}_1]^{m_1} [C^{i\alpha}_2]^{m_2}
         [C^{i\alpha}_3]^{m_3}$ from array
    \ENDFOR
  \ENDFOR
\ENDFOR
\end{algorithmic}
\end{algorithm}

\subsection{Gaussian Charge Screening Breakup}
This original approach to the short and long-ranged breakup adds an
opposite screening charge of gaussian shape around each point charge.
It then removes the charge in the long-ranged part of the potential.
In this potential,
\begin{equation}
v_{\text{long}}(r) = \frac{q_1 q_2}{r} \text{erf}(\alpha r),
\end{equation}
where $\alpha$ is an adjustable parameter used to control how
short-ranged the potential should be.  If the box size is $L$, a
typical value for $\alpha$ might be $7/(Lq_1 q_2)$. We should note
that this form for the long-ranged potential should also work for any
general potential with a coulomb tail, e.g. pseudo-Hamiltonian
potentials.  For this form of the long-ranged potential, we have in $k$-space
\begin{equation}
v_k = \frac{4\pi q_1 q_2 \exp\left[\frac{-k^2}{4\alpha^2}\right]}{\Omega k^2}.
\end{equation}

\subsection{Optimized Breakup Method}
In this section, we undertake the task of choosing a
long-range/short-range partitioning of the potential which is optimal
in that it minimizes the error for given real and $k$-space cutoffs
$r_c$ and $k_c$.  Here, we modify slightly the method introduced
Natoli and Ceperley\cite{Natoli1995}. We choose $r_c =
\frac{1}{2}\min\{L_i\}$, so that we require the nearest image in
real space summation.  $k_c$ is then chosen so as to satisfy our
accuracy requirements.

Here we modify our notation slightly to accommodate details not
required above.  We restrict our discussion to the interaction of two
particles species (which may be the same), and drop our species
indices.  Thus we are looking for short and long-range potentials
defined by,
\renewcommand{\vs}{v^s}
\newcommand{\vl}{v^\ell}
\begin{equation}
v(r) = \vs(r) + \vl(r)
\end{equation}
Define $\vs_k$ and $\vl_k$ to be the respective Fourier transforms of
the above.  The goal is to choose $v_s(r)$ such that its value and
first two derivatives vanish at $r_c$, while making $\vl(r)$ as smooth as
possible so that $k$-space components, $\vl_k$, are very small for
$k>k_c$.  Here, we describe how to do this is an optimal way.

Define the periodic potential, $V_p$, as 
\begin{equation}
V_p(\vr) = \sum_l v(|\vr + \mathbf{l}|),
\end{equation}
where $\vr$ is the displacement between the two particles and
$\mathbf{l}$ is a lattice vector.  Let us then define our
approximation to this potential, $V_a$, as
\begin{equation}
V_a(\vr) = \vs(r) + \sum_{|\vk| < k_c} \vl_k e^{i\mathbf \vk \cdot \vr}
\end{equation}
Now, we seek to minimize the RMS error over the cell,
\begin{equation}
\chi^2 = \frac{1}{\Omega}\int_\Omega d^3 \mathbf{r} \ 
\left| V_p(\vr) - V_a(\vr)\right|^2 
\end{equation}
We may write
\begin{equation}
V_p(\vr) = \sum_{\vk} v_k e^{i \vk \cdot \vr},
\end{equation}
where 
\begin{equation}
v_k = \frac{1}{\Omega} \int d^3\vr \ e^{-i\vk\cdot\vr}v(r).
\end{equation}
We now need a basis in which to represent the broken up potential.  We
may choose to represent either $\vs(r)$ or $\vl(r)$ in a real-space
basis.  Natoli and Ceperley chose the prior in their paper.  We choose
the latter for a number of reasons.  First, singular potentials are
difficult to represent in a linear basis unless the singularity is
explicitly included.  This requires a separate basis for each type of
singularity.  The short-range potential may have an arbitrary number
of features for $r<r_c$ and still be a valid potential.  By
construction, however, we desire that $\vl(r)$ be smooth in real-space
so that its Fourier transform falls off quickly with increasing $k$.
We therefore expect that, in general, $\vl(r)$ should be
well-represented by fewer basis functions than $\vs(r)$.  Therefore,
we define,
\begin{equation}
\vl(r) \equiv
\begin{cases}
 \sum_{n=0}^{J-1} t_n h_n(r) & \text{for } r \le r_c \\
 v(r) & \text{for } r > r_c.
\end{cases}
\end{equation}
where the $h_n(r)$ are a set of $J$ basis functions.  We require that
the two cases agree on the value and first two derivatives at $r_c$.
We may then define
\begin{equation}
c_{nk} \equiv \frac{1}{\Omega} \int_0^{r_c} d^3 \vr \ e^{-i\vk\cdot\vr} h_n(r).
\end{equation}
Similarly, we define
\begin{equation}
x_k \equiv -\frac{1}{\Omega} \int_{r_c}^\infty d^3\vr \ e^{-i\vk\cdot\vr} v(r)
\end{equation}
Therefore,
\begin{equation}
\vl_k = -x_k + \sum_{n=0}^{J-1} t_n c_{nk} 
\end{equation}
Because $\vs(r)$ goes identically to zero at the box edge, inside the
cell we may write
\begin{equation}
\vs(\vr) = \sum_\vk \vs_k e^{i\vk \cdot \vr}
\end{equation}
We then write
\begin{equation}
\chi^2 = \frac{1}{\Omega} \int_\Omega d^3 \vr \ 
\left| \sum_\vk e^{i\vk \cdot \vr} \left(v_k - \vs_k \right)
-\sum_{|\vk| \le k_c} \vl_k \right|^2
\end{equation}
We see that if we define
\begin{equation}
\vs(r) \equiv v(r) - \vl(r)
\end{equation}
Then
\begin{equation}
\vl_k + \vs_k = v_k,
\end{equation}
which then cancels out all terms for $|\vk| < k_c$.  Then we have
\begin{eqnarray}
\chi^2 & = & \frac{1}{\Omega} \int_\Omega d^3 \vr \ 
\left|\sum_{|\vk|>k_c} e^{i\vk\cdot\vr} 
\left(v_k -\vs_k \right)\right|^2 \\
& = & \frac{1}{\Omega} \int_\Omega d^3 \vr \ 
\left|\sum_{|\vk|>k_c} e^{i\vk\cdot\vr} \vl_k \right|^2 \\ 
& = & 
\frac{1}{\Omega} \int_\Omega d^3 \vr
\left|\sum_{|\vk|>k_c} e^{i\vk\cdot\vr}\left( -x_k + \sum_{n=0}^{J-1} t_n
c_{nk}\right) \right|^2
\end{eqnarray}
We expand the summation,
\newcommand{\ns}{\negthickspace}
\begin{equation}
\chi^2 = \frac{1}{\Omega} \int_\Omega d^3 \vr \ns \ns \ns
\sum_{\{|\vk|,|\vk'|\}>k_c} \ns\ns\ns\ns\ns
 e^{i(\vk-\vk')\cdot \vr}
\left(x_k -\sum_{n=0}^{J-1} t_n c_{nk} \right)
\left(x_k -\sum_{m=0}^{J-1} t_{m} c_{mk'} \right)
\end{equation}
We take the derivative w.r.t. $t_{m}$,
\begin{equation}
\frac{\partial (\chi^2)}{\partial t_{m}} =
\frac{2}{\Omega}\int_\Omega d^3 \vr \ns \ns \ns
\sum_{\{|\vk|,|\vk'|\}>k_c} \ns\ns\ns\ns\ns
 e^{i(\vk-\vk')\cdot \vr}
\left(x_k -\sum_{n=0}^{J-1} t_n c_{nk} \right) c_{mk'}
\end{equation}
We integrate w.r.t. $\vr$, yielding a Kronecker $\delta$.
\begin{equation}
\frac{\partial (\chi^2)}{\partial t_{m}} =
2 \ns\ns\ns\ns\ns\ns\ns 
\sum_{\ \ \ \ \{|\vk|,|\vk'|\}>k_c} \ns\ns\ns\ns\ns\ns\ns \delta_{\vk, \vk'} 
\left(x_k -\sum_{n=0}^{J-1} t_n c_{nk} \right) c_{mk'}
\end{equation}
Summing over $\vk'$ and equating the derivative to zero, we find the
minimum of our error function is given by
\begin{equation}
\sum_{n=0}^{J-1} \sum_{|\vk|>k_c} c_{mk}c_{nk} t_n = 
\sum_{|\vk|>k_c} x_k c_{mk},
\end{equation}
which is equivalent in form to equation (19) in \cite{Natoli1995}, where
we have $x_k$, instead of $V_k$.  Thus, we see that we may optimize
the short-range or long-range potential in simply by choosing to use
$V_k$ or $x_k$ in the above equation.  We now define
\begin{eqnarray}
A_{mn} & \equiv & \sum_{|\vk|>k_c} c_{mk} c_{nk} \\
b_{m} & \equiv & \sum_{|\vk|>k_c} x_k c_{mk}
\end{eqnarray}
Thus, it becomes clear that our minimization equations can be cast in
the canonical linear form,
\newcommand{\bA}{\mathbf{A}}
\newcommand{\bU}{\mathbf{U}}
\newcommand{\bV}{\mathbf{V}}
\newcommand{\bb}{\mathbf{b}}
\newcommand{\bS}{\mathbf{S}}
\begin{equation}
\bA\mathbf{t} = \mathbf{b}.
\end{equation}

\subsubsection{Solution by SVD}
In practice, we note that the matrix $\bA$ frequently becomes singular
in practice.  For this reason, we use the singular value decomposition
to solve for $t_n$.  This factorization decomposes $A$ as
\begin{equation}
\bA = \bU \bS \bV^T,
\end{equation}
where $\bU^T\bU = \bV^T\bV = 1$ and $\bS$ is diagonal.  In this form, we have
\begin{equation}
\mathbf{t} = \sum_{i=0}^{J-1} \left( \frac{\bU_{(i)} \cdot
  \bb}{\bS_{ii}} \right) \bV_{(i)},
\end{equation}
where the parenthesized subscripts refer to columns.  The advantage of
this form is that if $\bS_{ii}$ is zero or very near zero, the
contribution of the $i^{\text{th}}$ of $\bV$, may be neglected, since
it represents a numerical instability and has little physical
meaning.  It represents the fact that the system cannot distinguish
between two linear combinations of the basis functions.  Using the SVD
in this manner is guaranteed to be stable.  This decomposition is
available in LAPACK in the DGESVD subroutine.

\subsubsection{Constraining Values}
Often, we wish to constrain the value of $t_n$ to have a fixed value
to enforce a boundary condition, for example.  To do this, we define
\begin{equation}
\bb' \equiv \vb - t_n \bA_{(n)}.
\end{equation}
We then define $\bA^*$ as $\bA$ with the $n^{\text{th}}$ row and column
removed, and $\bb^*$ as $\vb'$ with the $n^{\text{th}}$ element removed.  Then
we solve the reduced equation $\bA^* \mathbf{t}^* = \bb^*$, and
finally insert $t_n$ back into the appropriate place in $\mathbf{t}^*$
to recover the complete, constrained vector $\mathbf{t}$.  This may be
trivially generalized to an arbitrary number of constraints.
\label{sec:contraints}

\subsubsection{The LPQHI basis}
The above discussion was general and independent of the basis used to
represent $\vl(r)$.  In this section, we introduce a convenient basis
of localized interpolant functions, similar to those used for
splines, which have a number of properties which are convenient for
our purposes.  

First, we divide the region from 0 to $r_c$ into $M-1$ subregions,
bounded above and below by points we term {\em knots}, defined by $r_j
\equiv j\Delta$, where $\Delta \equiv r_c/(M-1)$.  We then define
compact basis elements, $h_{j\alpha}$ which span the region
$[r_{j-1},r_{j+1}]$, except for $j=0$ and $j=M$.  For $j=0$, only the
region $[r_0,r_1]$, while for $j=M$, only $[r_{M-1}, r_M]$.  Thus the
index $j$ identify the knot the element is centered on, while $\alpha$
is an integer from 0 to 2 indicating one of three function shapes.
The dual index can be mapped to the single index above by the
relation, $n = 3j + \alpha$.  The basis functions are then defined as
\begin{equation}
h_{j\alpha}(r) = 
\begin{cases}
\ \ \ \, \Delta^\alpha \, \, \sum_{n=0}^5 S_{\alpha n} 
\left( \frac{r-r_j}{\Delta}\right)^n,    & r_j < r \le r_{j+1} \\
(-\Delta)^\alpha \sum_{n=0}^5 S_{\alpha n} 
\left( \frac{r_j-r}{\Delta}\right)^n,    & r_{j-1} < r \le r_j \\
\quad\quad\quad\quad\quad 0, & \text{otherwise},
\end{cases}
\end{equation}
where the matrix $S_{\alpha n}$ is given by
\begin{equation}
S = 
\left[\begin{matrix}
1 & 0 & 0 & -10 & 15 & -6 \\
0 & 1 & 0 & -6  &  8 & -3 \\
0 & 0 & \frac{1}{2} & -\frac{3}{2} & \frac{3}{2} & -\frac{1}{2}
\end{matrix}\right].
\end{equation}
\begin{figure}
\begin{center}
  \ifdefined\HCode
  \includegraphics[width=3.5in]{./figures/LPQHI.dmn}
  \else
  \includegraphics[width=3.5in]{./figures/LPQHI.pdf}
  \fi
\caption{Basis functions $h_{j0}$, $h_{j1}$, and $h_{j2}$ are shown.
We note at the left and right extremes, the values and first two
derivatives of the functions are zero, while at the center, $h_{j0}$
has a value of 1, $h_{j1}$ has a first derivative of 1, and $h_{j2}$
has a second derivative of 1. }
\label{fig:LPQHI} 
\end{center}
\end{figure}
Figure~\ref{fig:LPQHI} shows plots of these function shapes.

The basis functions have the property that at the left and right
extremes, i.e. $r_{j-1}$ and $r_{j+1}$, their values and first two
derivatives are zero.  At the center, $r_j$, we have the properties,
\begin{eqnarray}
h_{j0}(r_j)=1, & h'_{j0}(r_j)=0, & h''_{j0}(r_j)= 0 \\
h_{j1}(r_j)=0, & h'_{j1}(r_j)=1, & h''_{j1}(r_j)= 0 \\
h_{j2}(r_j)=0, & h'_{j2}(r_j)=0, & h''_{j2}(r_j)= 1 
\end{eqnarray}
These properties allow the control of the value and first two derivatives
of the represented function at any knot value simply by setting the
coefficients of the basis functions centered around that knot.  Used
in combination with the method described in
section~\ref{sec:contraints} above, boundary conditions can easily be
enforced.  In our case, we wish require that
\begin{equation}
h_{M0} = v(r_c), \ \ h_{M1} = v'(r_c), \ \ \text{and} \ \  h_{M2} = v''(r_c).
\end{equation}
This ensures that $\vs$ and its first two derivatives vanish at $r_c$.

\subsubsection*{Fourier coefficients}
We wish now to calculate the Fourier transforms of the basis
functions, defined as
\begin{equation}
c_{j\alpha k} \equiv \frac{1}{\Omega} \int_0^{r_c} d^3 \vr 
e^{-i \vk \cdot \vr} h_{j\alpha}(r)
\end{equation}
We then may write,
\begin{equation}
c_{j\alpha k} = 
\begin{cases}
\Delta^\alpha \sum_{n=0}^5 S_{\alpha n} D^+_{0 k n}, & j = 0 \\
\Delta^\alpha \sum_{n=0}^5 S_{\alpha n} (-1)^{\alpha+n} D^-_{M k n}, &
j = M \\
\Delta^\alpha \sum_{n=0}^5 S_{\alpha n} 
\left[ D^+_{j k n} + (-1)^{\alpha+n}D^-_{j k n} \right] & \text{otherwise},
\end{cases}
\end{equation}
where
\begin{equation}
D^{\pm}_{jkn} \equiv \frac{1}{\Omega} \int_{r_j}^{r_{j\pm1}} d^3\!\vr \ 
e^{-i\vk \cdot \vr} \left( \frac{r-r_j}{\Delta}\right)^n.
\end{equation}
We then further make the definition that
\renewcommand{\Im}{\text{Im}}
\begin{equation}
D^{\pm}_{jkn} = \pm \frac{4\pi}{k \Omega} 
\left[ \Delta \Im \left(E^{\pm}_{jk(n+1)}\right) + 
r_j \Im \left(E^{\pm}_{jkn}\right)\right]
\end{equation}
It can then be shown that 
\begin{equation}
E^{\pm}_{jkn} =
\begin{cases}
-\frac{i}{k} e^{ikr_j} \left( e^{\pm i k \Delta} - 1 \right) &
\text{if } n=0, \\
-\frac{i}{k} 
\left[ \left(\pm1\right)^n e^{i k (r_j \pm \Delta)} - \frac{n}{\Delta}
E^\pm_{jk(n-1)}  \right] & \text{otherwise}.
\end{cases}
\end{equation}
Note that these equations correct typographical errors present in \cite{Natoli1995}.
\subsubsection{Enumerating $k$-points}
We note that the summations over $k$ which have been ubiquitous in
this paper requires enumeration of the $k$-vectors.  In particular, we
should sum over all $|\vk| > k_c$.  In practice, we must limit our
summation to some finite cutoff value $k_c < |\vk| < k_{\text{max}}$,
where $k_{\text{max}}$ should be of order $3000/L$, where $L$ is the
minimum box dimension.  Enumerating these vectors in a naive fashion
even for this finite cutoff would prove quite prohibitive, as it
requires $\sim 10^9$ vectors.

Our first optimization come in realizing that all quantities in this
calculation require only $|\vk|$, and not $\vk$ itself.  Thus, we may
take advantage of the great degeneracy of $|\vk|$.  We create a list
of $(k,N)$ pairs, where $N$ is the number of vectors with magnitude $k$.
We make nested loops over
$n_1$, $n_2$, and $n_3$, yielding $\vk = n_1 \vb_1 + n_2 \vb_2 + n_3
\vb_3$. If $|\vk|$ is in the required range, we check to see if there
is already an entry with that magnitude on our list, incrementing the
corresponding $N$ if there is, or creating a new entry if not.  Doing
so typically saves a factor of $\sim 200$ in storage and computation.

This reduction is still not sufficient for large $k_max$, since it
requires that we still look over $10^9$ entries.  To further reduce
cost, we may pick an intermediate cutoff, $k_{\text{cont}}$, above which
we will approximate the degeneracy assuming a continuum of
$k$-points.  We stop our exact enumeration at $k_{\text{cont}}$, and
then add $\sim 1000$ points, $k_i$, uniformly spaced between $k_{\text{cont}}$
and $k_{\text{max}}$. We then approximate the degeneracy by
\begin{equation}
N_i = \frac{4 \pi}{3} \frac{\left( k_b^3 -k_a^3\right)}{(2\pi)^3/\Omega},
\end{equation}
where $k_b = (k_i + k_{i+1})/2$ and $k_a = (k_i + k_{i-1})$.  In doing
so, we typically reduce our total number of k-points to sum over $\sim
2500$ from the $10^9$ we had to start.

\subsubsection{Calculating $x_k$'s}
\subsubsection*{The coulomb potential}
For $v(r) = \frac{1}{r}$, $x_k$ is given by
\begin{equation}
x_k^{\text{coulomb}} = -\frac{4 \pi}{\Omega k^2} \cos(k r_c)
\end{equation}

\subsection*{The $1/r^2$ potential}
For $v(r) = \frac{1}{r^2}$, $x_k$ is given by
\begin{equation}
x_k^{1/r^2} = \frac{4 \pi}{\omega k} 
\left[ \text{Si}(k r_c) -\frac{\pi}{2}\right],
\end{equation}
where the {\em sin integral}, $\text{Si}(z)$, is given by
\begin{equation}
\text{Si}(z) \equiv \int_0^z \frac{\sin \ t}{t} dt.
\end{equation}

\subsection*{The $1/r^3$ potential}
For $v(r) = \frac{1}{r^3}$, $x_k$ is given by
\begin{equation}
x_k^{1/r^3} = \frac{4\pi}{\Omega k} 
\left[k\text{Ci}(k r_c) - \frac{\sin(k r_c)}{r_c} \right],
\end{equation}
where the {\em cosine integral}, $\text{Ci}(z)$, is given by
\begin{equation}
\text{Ci}(z) \equiv -\int_z^\infty \frac{\cos t}{t} dt.
\end{equation}

\subsection*{The $1/r^4$ potential}
For $v(r) = \frac{1}{r^4}$, $x_k$ is given by
\begin{equation}
x_k^{1/r^4} = -\frac{4 \pi}{\Omega k} 
\left\{
\frac{k \cos(k r_c)}{2 r_c} + \frac{\sin(k r_c)}{2r_c^2} + \frac{k^2}{2} \left[ \text{Si}(k r_c) - \frac{\pi}{2}\right]\right\}
\end{equation}


%\section{Adapting to PIMC}
%\subsection{Pair actions}
%Let us begin by summarizing what we have done so far.  We began with the many-body Hamiltonian given by 
%\begin{equation}
%\mathcal{H} = \sum_i -\lambda_i \nabla_i^2 + V,
%\end{equation}
%where $V$ is the periodic potential given by \ref{eq:Vperiodic}, and $\lambda \equiv \frac{\hbar^2}{2m_i}$. 
%
%We approximately solved the action of this Hamiltonian by considering
%the particles pairwise, and solving for the density matrix for the
%density matrix of each pair exactly using the matrix squaring method.
%This yields the the {\em pair action}, defined by
%\begin{equation}
%\rho^{\alpha \beta}(\vr, \vr';\tau) \equiv \rho_0(\vr, \vr';\tau)
%e^{-u^{\alpha \beta}(\vr, \vr';\tau)},
%\end{equation}
%where $\rho_0$ is the {\em free particle} density matrix for species
%$\alpha$ interacting with species $\beta$.  $\rho^{\alpha \beta}$ is
%the density matrix for the pair Hamiltonian
%\begin{equation}
%H^{\alpha\beta} = -\lambda^{\alpha\beta} \nabla^2 + v^{\alpha\beta}(|\vr|),
%\end{equation}
%where $\vr \equiv \vr_i - \vr_j$ and particles $i$ and $j$ are members
%of species $\alpha$ and $\beta$, respectively, and
%$\lambda^{\alpha\beta}$ is given by
%\begin{equation}
%\lambda^{\alpha \beta} = \frac{\hbar^2}{2m_{\alpha}} +
%\frac{\hbar^2}{2m_\beta}.
%\end{equation}
%If the potential $v^{\alpha \beta}(r)$ is long range, then the action,
%$u^{\alpha \beta}(\vr, \vr';\beta)$, will also be long range.  We
%note, however, that the action is not a simple function of the scalar
%$r$, as the potential is.  Experience shows, however, that at large
%distances, the action is well-approximated by
%\begin{eqnarray}
%u^{\alpha\beta}(\vr, \vr';\tau) & \approx & 
%\frac{1}{2} \left[ u^{\alpha\beta}(\vr,\vr;\tau) +
%  u^{\alpha\beta}(\vr',\vr';\tau)\right] \\
%& = & \frac{1}{2} \left[ u^{\alpha\beta}_\text{diag}(r,\tau)+
%u^{\alpha\beta}_\text{diag}(r',\tau)\right]
%\end{eqnarray}
%This is known as the {\em diagonal approximation}.  Thus, as long as
%this approximation is valid at half the minimum box dimension, we may
%break up the diagonal action as we did the potential.  This
%effectively neglects the off-diagonal parts of the action for
%particles more than a half-box length apart, but experience has shown
%that these contributions are usually quite small.  The same
%analysis follows for the $\tau$-derivative the the action, which is
%required to compute the total energy.  Note that PIMC simulation
%requires the pair action at several values of $\tau$, so that in
%practice, we need to do several optimized breakups for each
%$u_\text{diag}^{\alpha\beta}$ and $\dot{u}_\text{diag}^{\alpha\beta}$ and a single breakup for
%each $v^{\alpha\beta}$.
%
%\subsection{Beyond the pair approximation: RPA improvements}
%Consider the limit of a dense gas of charged particles.  We know from
%solid state theory that collective density fluctuations, known as
%plasmons, contribute significantly to the energy spectrum of such a system.
%An approximation to the density matrix determined by considering only
%pairs of particles will neglect these contributions at finite $\tau$.
%As $\tau$ approaches zero, Trotter still guarantees we will approach
%the right limit.
%
%Nonetheless, it is possible to significantly reduce the finite-$\tau$
%timestep error by utilizing a different approximation for the long
%range part of the action.  We begin by defining our effective,
%long-range potential.  As noted above, we may perform an optimized
%breakup on the diagonal action, $u_\text{diag}^{\alpha\beta}(r)$.
%\begin{equation}
%u_\text{diag}^{\alpha\beta}(r) = \hat{u}^{\alpha\beta}_\text{diag}(r) +
%\bar{u}^{\alpha\beta}_\text{diag}(r),
%\end{equation}
%where the $\hat{u}$ and $\bar{u}$ refer to the short and long range
%diagonal actions, respectively, borrowing the notation for short and
%long vowels.
%We subtract the long range part form the total pair action in a
%quasi-primitive approximation by defining
%\begin{equation}
%\bar{u}^{\alpha\beta}_\text{diag}(r) \equiv \tau \bar{v}^{\alpha \beta}(r).
%\end{equation}
%Let $\bar{v}^{\alpha \beta}_k$ represent the Fourier transform the the
%effective potential, $\bar{v}^{\alpha\beta}(r)$.  Finally, let its
%short-range counterpart be defined by 
%\begin{equation}
%\hat{v}^{\alpha \beta}_k \equiv v^{\alpha\beta}_k - \bar{v}^{\alpha\beta}_k
%\end{equation}
%
%Now, we wish to reintroduce a new long range action, which we will
%calculate in $k$-space within the {\em Random Phase Approximation
%  (RPA)}.   We begin with the Bloch equation,
%\begin{equation}
%\dot{\rho} = -\mathcal{H} \rho,
%\end{equation}
%where the dot refers to differentiation w.r.t. $\tau$.  The
%Hamiltonian is given by
%\begin{equation}
%\mathcal{H} = \left[\sum_\alpha \sum_{i\in \alpha} -\lambda_\alpha
%\nabla_i^2\right] + \hat{V} + \bar{V},
%\end{equation}
%where $\hat{V}$ and $\bar{V}$ are the total short and long range
%periodic potentials, respectively.
%Let us now make the partitioning that
%\begin{equation}
%\rho(\vR, \vR';\tau) = \rho_0(\vR, \vR';\tau) e^{-\hat{U}(\vR,
%  \vR';\tau)} e^{-\bar{U}(\vR, \vR';\tau)},
%\end{equation}
%We assume that $\rho_s \equiv \rho_0 e^{-\hat{U}}$ satisfies the Bloch
%equation for the short-range Hamiltonian,
%\begin{equation}
%\mathcal{H}_s = \left[\sum_\alpha \sum_{i\in \alpha} -\lambda_\alpha
%\nabla_i^2\right] + \hat{V}.
%\end{equation}
%In fact, this is only strictly true in the limit that $\tau=0$, but
%this relation will suffice for our present analysis.
%
%
%Recall that $\nabla^2(ab) = a\nabla^2 b + b\nabla^2a +2(\nabla a)
%\cdot (\nabla b)$.  Thus, we have for our Bloch equation,
%\begin{eqnarray}
%-\left [\dot{\rho_s} -\rho_s\dot{\bar{U}}\right] e^{-\bar{U}} & = &
%\sum_{\alpha,\  i\in\alpha} -\lambda_\alpha
%\left[\rho_s \nabla^2_i e^{-\bar{U}} + e^{-\bar{U}} \nabla^2_i \rho_s
%  + 2(\nabla_i \rho_s)\cdot (\nabla_i e^{-\bar{U}}) 
%\right] \nonumber \\ & & + (\hat{V} + \bar{V}) \rho_s e^{-\bar{U}}.
%\end{eqnarray} 
%Subtracting the Bloch equation for the short range part,
%we are left with
%\begin{equation}
%\left[\dot{\bar{U}}-\bar{V}\right] \rho_s e^{-\bar{U}}  = 
%\sum_{\alpha,\  i\in\alpha} -\lambda_\alpha
%\left[\rho_s \nabla^2_i e^{-\bar{U}}
%  + 2(\nabla_i \rho_s)\cdot (\nabla_i e^{-\bar{U}}) 
%\right].
%\end{equation} 
%Recall that
%\begin{eqnarray}
%\nabla e^{-\bar{U}} & = & -\nabla\bar{U}e^{-\bar{U}} \\
%\nabla \rho_0 & = & 0 \ \ \ \ \ \ \ \ \ \ \ \ \ \ \ \ \ \ \ \ \ \ \ \
%\ \ \ \ \ \ \ \ \ 
%\text{ (for $\vR = \vR'$)} \\
%\nabla \rho_s & = & -\rho_s \nabla \hat{U} \\
%\nabla^2 e^{-\bar{U}} & = & 
%\left[(\nabla \bar{U})^2 - \nabla^2 \bar{U}\right] e^{-\bar{U}} 
%\end{eqnarray} 
%We now attempt to solve the Bloch equation under the restriction that
%$\vR = \vR'$, i.e. along the diagonal of the density matrix.  Hence
%let us define
%\begin{eqnarray} 
%\bar{U}(\vR, \vR';\tau) & \equiv &
%  \frac{1}{2}\left[\bar{\mathcal{U}}(\vR;\tau) +
%  \bar{\mathcal{U}}(\vR';\tau) \right] \\ 
% \hat{U}(\vR,\vR';\tau) & \equiv & 
%  \frac{1}{2} \left[\hat{\mathcal{U}}(\vR;\tau) +
%  \hat{\mathcal{U}}(\vR';\tau) \right] 
%\end{eqnarray}
%as the long and short range diagonal actions written as
%functions of only one spatial argument.  Then we have, along the
%diagonal,
%\begin{eqnarray}
%\nabla U   & = & \frac{1}{2} \nabla\mathcal{U} \\
%\nabla^2 U & = & \frac{1}{2} \nabla^2\mathcal{U}.
%\end{eqnarray}
%Substituting back into out Bloch equation,
%\begin{equation}
%\dot{\bar{\mathcal{U}}} = \sum_{\alpha, \ i\in \alpha} -\lambda_\alpha
%\left\{ \frac{1}{4} (\nabla_i \bar{\mathcal{U}})^2 - 
%\frac{1}{2}\nabla_i^2 \bar{\mathcal{U}} + \frac{1}{2} (\nabla_i\hat{\mathcal{U}})
%  \cdot (\nabla_i \bar{\mathcal{U}}) \right\} +\bar{V}
%\end{equation}
%
%We recall that the long range potential, $\bar{V}$, may be written as
%\begin{equation}
%\bar{V} = \sum_\vk \sum_{\alpha} \left[ 
%\frac{1}{2} \left| \rho^\alpha_\vk\right|^2 \bar{v}^{\alpha
%  \alpha}_k + 
%\sum_{\beta < \alpha} \mathcal{R}e \left( \rho^{\alpha}_\vk
%  \rho^\beta_{-\vk} \bar{v}^{\alpha\beta}_k \right)
%\right] 
%\end{equation}
%When we wrote this expression above, we did so to optimize the speed
%of computation.  For the following analysis, we will find it more
%convenient to write
%\begin{equation}
%\bar{V} = \frac{1}{2} \sum_\vk \sum_{\alpha, \beta} \rho_\vk^\alpha
%\rho_{-\vk}^\beta v^{\alpha \beta}_k.
%\end{equation}
%The sum is guaranteed to be real since for every $\vk$, we have a
%corresponding $-\vk$ in the sum.  Hence we need not be concerned by
%taking the real part. We may similarly write $\bar{\mathcal{U}}$ and $\hat{\mathcal{U}}$ in
%terms of $\bar{u}_k^{\alpha\beta}$ and $\hat{u}_k^{\alpha\beta}$.
%
%
%We now proceed to calculate gradients and laplacians.
%Recall that 
%\begin{equation}
%\rho_\vk^\alpha = \sum_{i\in\alpha} e^{i\vk \cdot \vr_i}
%\end{equation}
%\begin{eqnarray}
%\nabla_i \mathcal{U} & = & \frac{1}{2}\sum_\vk \left[ i\vk e^{i\vk \cdot \vr_i} \sum_\alpha
%\rho_{-\vk}^\alpha u^{\alpha \beta}_k + \text{c.c.} \right] \\
%& = & \frac{1}{2} \sum_\vk 2\mathcal{R}e \left[i\vk e^{i\vk \cdot \vr_i} \sum_\alpha
%\rho_{-\vk}^\alpha u^{\alpha \beta}_k\right] \\ 
%& = & \mathcal{R}e \left[ \sum_\vk i\vk e^{i\vk \cdot \vr_i} \sum_\alpha
%\rho_{-\vk}^\alpha u^{\alpha \beta}_k \right] \\
%& = & \sum_\vk i\vk e^{i\vk \cdot \vr_i} \sum_\alpha
%\rho_{-\vk}^\alpha u^{\alpha \beta}_k.
%\end{eqnarray}
%In the last line, we have again recognized that for every $\vk$
%there is a corresponding $-\vk$, so that the sum is purely real.
%
%Next, we compute the Laplacian w.r.t. the $i^{\text{th}}$ particle.
%\begin{eqnarray}
%\nabla^2_i \mathcal{U} & = & \nabla_i \cdot \nabla_i \mathcal{U} \\
%& = & \nabla_i \cdot \sum_\vk i\vk e^{i\vk \cdot \vr_i} \sum_\alpha
%\rho_{-\vk}^\alpha u^{\alpha \sigma_i}_k \\
%& = & \sum_\vk i\vk \cdot \nabla_i \left[ e^{i\vk \cdot \vr_i}
%  \sum_\alpha \rho_{-\vk}^\alpha u^{\alpha\sigma_i}_k \right] \\
%& = & \sum_{\vk} k^2 \left[ u^{\sigma_i \sigma_i}_k - e^{i\vk\cdot\vr_i}\sum_\alpha \rho_{-\vk}
%u_k^{\alpha \sigma_i}\right],
%\end{eqnarray}
%where $\sigma_i$ is the species of the $i^{\text{th}}$ particle.  Now,
%let us sum over all particles,
%\begin{eqnarray}
%\sum_i \lambda_i \nabla^2_i \mathcal{U} & = & \sum_\vk k^2 \left[\sum_\beta N_\beta u_k^{\beta
%  \beta} - \rho_{\vk}^\beta \sum_\alpha \rho_{-\vk} u_k^{\alpha \beta}
%  \right] \\
%& = & \sum_{\vk} k^2 \sum_{\alpha, \beta}
%  \lambda_\beta \left[N^{\alpha}\delta_{\alpha,\beta} -
%  \rho_{-\vk}^{\alpha}\rho_\vk^\beta \right]u_k^{\alpha \beta} 
%%\\
%%& = & \sum_{\vk} k^2 \sum_{\alpha, \beta}
%%  \left(\frac{\lambda_\alpha +\lambda_\beta}{2} \right) \left[N^{\alpha}\delta_{\alpha,\beta} -
%%  \rho_{-\vk}^{\alpha}\rho_\vk^\beta \right]u_k^{\alpha \beta}.
%\end{eqnarray}
%%In the last step, we have added half the sum with $\alpha$ and $\beta$
%%swapped so as to symmetrize the summation.
%Now, let us consider the cross term,
%\begin{eqnarray}
%(\nabla_i \hat{\mathcal{U}}) \cdot ( \nabla_i \bar{\mathcal{U}} ) 
%& = & \left[\sum_\vk i\vk e^{i\vk\cdot \vr_i} \sum_\alpha
%  \rho_{-\vk}^\alpha \hat{u}^{\sigma_i \alpha}_k \right] \cdot
%\left[\sum_\vq i\vq e^{i\vq\cdot \vr_i} \sum_\beta
%  \rho_{-\vk}^\beta \bar{u}^{\sigma_i \beta}_k \right] \nonumber \\
%& = & -\sum_{\vk,\vq} \vk \cdot \vq e^{i(\vk + \vq)\cdot \vr_i}
%\sum_{\alpha, \beta} \rho_{-\vk}^\alpha \rho_{-\vq}^\beta 
%\hat{u}^{\alpha \sigma_i}_k \bar{u}^{\beta \sigma_i}_k
%\end{eqnarray}
%Again, summing over all particles,
%\begin{equation}
%\sum_i (\nabla_i \hat{\mathcal{U}}) \cdot ( \nabla_i \bar{\mathcal{U}} ) =
%-\sum_{\vk, \vq} \vk \cdot \vq \sum_{\alpha, \beta, \gamma}
%\rho_{\vk + \vq}^\gamma \rho_{-\vk}^{\alpha} \rho_{-\vq}^\beta
%\hat{u}^{\alpha \gamma}_k \bar{u}^{\beta \gamma}_k
%\end{equation}
%Similarly,
%\begin{equation}
%\sum_i (\nabla_i \bar{\mathcal{U}})^2 = -\sum_{\vk, \vq} \vk \cdot \vq 
%\sum_{\alpha, \beta, \gamma} \rho^\gamma_{\vk+\vq} \rho^\alpha_{-\vk}
%\rho^\gamma_{-\vq} \bar{u}^{\alpha \gamma}_k \bar{u}^{\beta \gamma}_k
%\end{equation}
%
%The {\em Random Phase Approximation} (RPA) amounts to the assumption
%that $\rho^\gamma_{\vk + \vq} \approx N_\gamma \delta_{\vk + \vq}$. 
%Then we have,
%\begin{eqnarray}
%\sum_i (\nabla_i \hat{\mathcal{U}}) \cdot ( \nabla_i \bar{\mathcal{U}}
%) & \overset{\text{RPA}}{=} &
%\sum_\vk k^2 \sum_{\alpha, \beta, \gamma} N_\gamma \rho^\alpha_{-\vk}
%\rho^\beta_\vk \hat{u}^{\alpha \gamma}_k \bar{u}^{\beta \gamma}_k \\
%\sum_i (\nabla_i \bar{\mathcal{U}})^2 & \overset{\text{RPA}}{=} &
%\sum_\vk k^2 \sum_{\alpha, \beta, \gamma} N_\gamma
%\rho^{\alpha}_{-\vk} \rho^\beta_\vk \bar{u}_k^{\alpha \gamma}
%\bar{u}_k^{\beta \gamma}
%\end{eqnarray}
%We now return to the Bloch equation
%\begin{equation}
%\begin{split}
%\sum_\vk \sum_{\alpha,\beta} & \left\{ 
%\frac{1}{2} \rho_\vk^\alpha
%\rho_{-\vk}^\beta \left(\dot{\bar{u}}_k - \bar{v}_k^{\alpha \beta} \right)
%+\frac{1}{2} \lambda_\alpha k^2 \bar{u}_k^{\alpha \beta}
%\left(\rho_{-\vk}^\alpha 
%  \rho_\vk^\beta - N_\beta \delta_{\alpha,\beta} \right) 
%\right. \\
%& \left. -\sum_\gamma k^2 N_\gamma \lambda_\gamma \rho_{-\vk}^\alpha
%  \rho_\vk^\beta \left[ 
%\frac{1}{4} \hat{u}^{\alpha \gamma}_k \bar{u}^{\beta\gamma}_k +
%\frac{1}{2} \bar{u}^{\alpha \gamma}_k \bar{u}^{\beta \gamma}
%\right]\right\} = 0
%\end{split}
%\end{equation}
%Next, we symmetrize this equation w.r.t $\alpha$ and $\beta$.
%\begin{equation}
%\begin{split}
%\sum_{\vk, \alpha, \beta} & \left\{ \left( \rho^\alpha_{\vk} \rho^\beta_{-\vk} 
%+ \rho^\alpha_{-\vk} \rho^\beta_{\vk} \right) \left[
%\dot{\bar{u}}_k^{\alpha \beta} - \bar{v}_k^{\alpha \beta}
% +k^2 \left(\frac{\lambda_\alpha+\lambda_\beta}{2}\right) 
%\bar{u}_k^{\alpha \beta} \rule{0cm}{0.6cm} \right.\right. \\
%& \ \ \left.\left.+\sum_\gamma \frac{k^2}{2} N^\gamma \left(
%\bar{u}_k^{\alpha \gamma} \bar{u}_k^{\beta \gamma} +
%\hat{u}_k^{\alpha \gamma} \bar{u}_k^{\beta \gamma} +
%\bar{u}_k^{\alpha \gamma} \hat{u}_k^{\beta \gamma}  \right)
%\right]
%- k^2 N^\alpha \delta_{\alpha \beta}
%\right\} = 0
%\end{split}
%\end{equation}
%We require that this expression hold independent of the positions of
%the particles, i.e. independent of the values of $\rho^\alpha_{\vk}$ and
%$\rho^\beta_{\vk}$.  Thus, the equations separate for each value of
%$\vk$, $\alpha$, and $\beta$.  For $\vk \neq 0$,
%\begin{equation}
%\dot{\bar{u}}_k^{\alpha \beta} = \bar{v}_k^{\alpha \beta} 
%- k^2 \left(\frac{\lambda_\alpha+\lambda_\beta}{2}\right)
%\bar{u}_k^{\alpha\beta} -\frac{k^2}{2} \sum_\gamma N_\gamma
%\left(
%\bar{u}_k^{\alpha \gamma} \bar{u}_k^{\beta \gamma} +
%\hat{u}_k^{\alpha \gamma} \bar{u}_k^{\beta \gamma} +
%\bar{u}_k^{\alpha \gamma} \hat{u}_k^{\beta \gamma}
%\right)
%\end{equation}
%Next, we need an equation for the time propagation of
%$\hat{u}_k^{\alpha \beta}$.  Above, we assumed that $\hat{U}$ was the
%solution to the short-range problem.  Our Bloch equation for
%$\hat{mathcal{U}}$ is then given by
%\begin{equation}
%\dot{\hat{\mathcal{U}}} = \sum_i -\lambda_i
%\left\{ \frac{1}{4} (\nabla_i \hat{\mathcal{U}} 
%-\frac{1}{2} \nabla_i^2 \hat{\mathcal{U}} \right\} + \hat{V}
%\end{equation}
%Following the RPA procedure above, we arrive at the following
%equations for $\hat{u}_k^{\alpha\beta}$.
%\begin{equation}
%\dot{\hat{u}}^{\alpha \beta}_k = \hat{v}^{\alpha \beta}_k
%-k^2 \left( \frac{\lambda_\alpha + \lambda_\beta}{2} \right)
%\hat{u}^{\alpha \beta}_k - \frac{k^2}{2} \sum_\gamma N_\gamma
%\hat{u}^{\alpha \gamma}_k \hat{u}^{\beta \gamma}_k.
%\end{equation}
%Hence, for each value of $k$, we have a coupled set of differential
%equations we must solve.  We note that while the equations for
%$\bar{u}$ couple to $\hat{u}$, those for $\hat{u}$ do not couple to
%$\bar{u}$.
%%% \begin{equation}
%%% \begin{split}
%%% \sum_\vk \sum_{\alpha,\beta} 
%%%  & \left\{
%%% \rho_{-\vk}^\alpha \rho_\vk^\beta 
%%% \left[
%%% \dot{\bar{u}}^{\alpha \beta}_k + k^2 \sum_\gamma \lambda_\gamma
%%% N_\gamma 
%%% \left(\frac{\bar{u}^{\alpha \gamma}_k \bar{u}^{\beta \gamma}_k}{4} -
%%% \frac{\hat{u}^{\alpha \gamma}_k \bar{u}^{\beta \gamma}_k}{2}
%%% \right) 
%%% \frac{k^2}{2} \lambda_\beta \bar{u}_k^{\alpha \beta} +
%%% \bar{v}_k^{\alpha \beta}
%%% \right] \right.\\
%%%  & \left. \rule{0pt}{0.6cm}
%%% + \frac{k^2}{2}\lambda_\beta N_\beta \delta_{\alpha,\beta}
%%% \bar{u}_k^{\alpha \beta}
%%% \right\} = 0
%%% \end{split}
%%% \end{equation}
%
%
%%% While this
%%% is correct in the limit that the timestep, $\tau$, goes to zero, it
%%% may incur a substantial error for finite $\tau$.  In this section, we
%%% describe a method to reduce the timestep error of the long range part
%%% of the action by using the Bloch equation combined with the Random
%%% Phase Approximation (RPA).  
%
%%% The Bloch equation may be written,
%%% \begin{equation}
%%% \dot{\rho} = -\mathcal{H} \rho,
%%% \end{equation}
%%% where the dot indicates differentiation with respect to $\tau$.  Now,
%%% we define
%%% \begin{equation}
%%% \rho = \rho_0 e^{-U_s}e^{-U_l}.
%%% \end{equation}
%%% \begin{equation}
%%% \mathcal{H} = \left[ -\lambda \sum_i \nabla_i^2 \right] + V_s + V_l
%%% \end{equation}
%%% The Bloch equation gives us 




\newpage
\section{Feature: Optimized Long-Ranged Breakup (Ewald) 2}

% Written by Simone Chiesa for the FITPN code/tool (Ceperley)
% Originally titled ``Notes on fitnp''

\newcommand{\rv}{\mathbf{r}}
\newcommand{\kv}{\mathbf{k}}
\newcommand{\Rv}{\mathbf{R}}
\newcommand{\Lv}{\mathbf{L}}
\newcommand{\Rc}{\mathcal{R}}
\newcommand{\tV}{\widetilde{V}}
\newcommand{\tW}{\widetilde{W}}
\newcommand{\tc}{\widetilde{c}}
\newcommand{\tY}{\widetilde{Y}}
\newcommand{\Nk}{N_{\text{knot}}}
\newcommand{\wk}{w_{\text{knot}}}

Given a lattice of vectors $\Lv$, its associated reciprocal
lattice of vectors $\kv$ and a function $\psi(\rv)$ periodic
on the lattice we define its Fourier transform $\widetilde{\psi}(\kv)$ as
\begin{equation}
\widetilde{\psi}(\kv)=\frac{1}{\Omega}\int_\Omega d\rv \psi(\rv) e^{-i\kv\rv}
\end{equation}
where we indicated both the cell domain and the cell volume by $\Omega$. 
$\psi(\rv)$ can then be expressed as
\begin{equation}
\psi(\rv)=\sum_{\kv} \widetilde{\psi}(\kv)e^{i\kv\rv}
\end{equation}
The potential generated by charges sitting on the lattice positions
at a particular point $\rv$ inside the cell is given by
\begin{equation}
V(\rv)=\sum_{\Lv}v(|\rv+\Lv|)
\end{equation}
and its Fourier transform can be explicitly written as a function of $V$ or $v$
\begin{equation}
\widetilde{V}(\kv)=\frac{1}{\Omega}\int_\Omega d\rv V(\rv) e^{-i\kv\rv}=
\frac{1}{\Omega}\int_{\mathbb{R}^3} d\rv v(\rv) e^{-i\kv\rv}
\end{equation}
where $\mathbb{R}^3$ denotes the whole 3-dimensional space.
We now want to find the best (``best'' to be defined later) approximate 
potential of the form
\begin{equation}
V_a(\rv)=\sum_{k\le k_c} \widetilde{Y}(k) e^{i\kv\rv} + W(r)
\end{equation}
where $W(r)$ has been chosen to go to $0$ smoothly when $r=r_c$, being
$r_c$ lower or equal to the Wigner-Seitz radius of the cell. Note also
the cutoff $k_c$ on the momentum summation.

The best form of $\widetilde{Y}(k)$ and $W(r)$ is given by minimizing
\begin{equation}
  \chi^2=\frac{1}{\Omega}\int d\rv \left(V(\rv)-W(\rv)-
  \sum_{k\le k_c}\widetilde{Y}(k)e^{i\kv\rv}\right)^2
  \label{chi2r}
\end{equation}
or the reciprocal space equivalent
\begin{equation}
  \chi^2=\sum_{k\le k_c}(\tV(k)-\tW(k)-\tY(k))^2+\sum_{k>k_c}(\tV(k)-\tW(k))^2
  \label{chi2k}
\end{equation}
Eq.\ref{chi2k} follows from Eq.\ref{chi2r} and the unitarity
(norm conservation) of the Fourier transform.

This last condition is minimized by
\begin{equation}
\tY(k)=\tV(k)-\tW(k)\qquad \min_{\tW(k)}\sum_{k>k_c}(\tV(k)-\tW(k))^2
\label{mincond}
\end{equation}
We now use a set of basis function $c_i(r)$ vanishing smoothly at $r_c$
to expand $W(r)$ i.e.
\begin{equation}
W(r)=\sum_i t_i c_i(r)\qquad\text{or}\qquad \tW(k)=\sum_i t_i \tc_i(k)
\end{equation}
Inserting the reciprocal space expansion of $\tW$ in the second condition of
Eq.\ref{mincond} and minimizing with respect to $t_i$ leads immediately
to the linear system $\mathbf{A}\mathbf{t}=\mathbf{b}$ where
\begin{center}
\vskip 3mm
\begin{eqnarray}
A_{ij}=\sum_{k>k_c}\tc_i(k)\tc_j(k)\qquad b_j=\sum_{k>k_c} V(k) \tc_j(k)
\label{matrix_elements}
\end{eqnarray}

\end{center}
\vskip 3mm

\subsection{Basis functions}
The basis functions are splines. We define a uniform grid 
with $\Nk$ uniformly spaced knots at position $r_i=i\frac{r_c}{\Nk}$ 
where $i\in[0,\Nk-1]$. On each knot we center $m+1$ piecewise polynomials
$c_{i\alpha}(r)$ with $\alpha\in[0,m]$, defined as
\vskip 3mm
\begin{center}
\begin{eqnarray}
c_{i\alpha}(r)=\begin{cases}
\Delta^\alpha \sum_{n=0}^\mathcal{N} S_{\alpha n}(\frac{r-r_i}{\Delta})^n & r_i<r\le r_{i+1} \\
\Delta^{-\alpha} \sum_{n=0}^\mathcal{N} S_{\alpha n}(\frac{r_i-r}{\Delta})^n & r_{i-1}<r\le r_i \\
0 & |r-r_i| > \Delta
\end{cases}
\label{basisdef}
\end{eqnarray}

\end{center}
\vskip 3mm
These functions and their derivatives are, by construction, continuous and odd (even)
(with respect to $r-r_i\rightarrow r_i-r$) when $\alpha$ is odd (even).
We further ask them to satisfy
\begin{eqnarray}
\left.\frac{d^\beta}{dr^\beta} c_{i\alpha}(r)\right|_{r=r_i}=
\delta_{\alpha\beta} \quad \beta\in[0,m]\\
\left.\frac{d^{\beta}}{dr^{\beta}} c_{i\alpha}(r)\right|_{r=r_{i+1}}=0\quad \beta\in[0,m]
\label{constr}
\end{eqnarray}
(The parity of the functions guarantees that the second constraint is satisfied
at $r_{i-1}$ as well). These constraints have a simple interpretation: the basis functions
and their first $m$ derivatives are $0$ on the boundary of the subinterval where they
are defined; the only function to have a non zero $\beta$-th derivative in $r_i$ is $c_{i\beta}$.
These $2(m+1)$ constraints therefore impose $\mathcal{N}=2m+1$. 
Inserting the definitions of Eq.~\ref{basisdef} in the constraints of Eq.~\ref{constr}
leads to the set of $2(m+1)$ linear equation that fixes the value of $S_{\alpha n}$: 
\begin{eqnarray}
\Delta^{\alpha-\beta} S_{\alpha\beta} \beta!=\delta_{\alpha\beta}
\label{Smatrix1}\\
\Delta^{\alpha-\beta}\sum_{n=\beta}^{2m+1} S_{\alpha n} \frac{n!}{(n-\beta)!}=0
\end{eqnarray}
One can further simplify inserting the first of these equations into the second and write
the linear system as
\begin{equation}
\sum_{n=m+1}^{2m+1} S_{\alpha n} \frac{n!}{(n-\beta)!}=\begin{cases}
-\frac{1}{(\alpha-\beta)!}& \alpha\ge \beta \\
0 & \alpha < \beta
\end{cases}
\label{Smatrix2}
\end{equation}

\subsection{Fourier components of the basis functions in 3D}
\subsubsection*{$k\ne 0$, non coulomb case.}
We now need to evaluate the Fourier transform $\tc_{i\alpha}(k)$. Let us start
by writing the definition
\begin{equation}
\tc_{i\alpha}(k)=\frac{1}{\omega}\int_\Omega d\rv  e^{-i\kv\rv} c_{i\alpha}(r)
\end{equation}
Because $c_{i\alpha}$ is different from zero only inside the spherical crown
defined by $r_{i-1}<r<r_i$ one can conveniently compute the integral in spherical
coordinates as
\vskip 3mm
\begin{center}
\begin{eqnarray}
\tc_{i\alpha}(k)=\Delta^\alpha\sum_{n=0}^\mathcal{N} S_{\alpha n} \left[
D_{in}^+(k) +\wk(-1)^{\alpha+n}D_{in}^-(k)\right]
\label{fourier_transform}
\end{eqnarray}
\end{center}
\vskip 3mm
where we used the definition $\wk=1-\delta_{i0}$ and
\begin{equation}
D_{in}^\pm(k)=\pm\frac{4\pi}{k\Omega}\Im\left[\int_{r_i}^{r_i\pm\Delta}
dr\left(\frac{r-r_i}{\Delta}\right)^n r e^{ikr}\right]
\label{D+-}
\end{equation}
obtained by integrating the angular part of the Fourier transform.
Using the identity
\begin{equation}
\left(\frac{r-r_i}{\Delta}\right)^n r=\Delta\left(\frac{r-r_i}{\Delta}\right)^{n+1}+\left(\frac{r-r_i}{\Delta}\right)^n r_i
\end{equation}
and the definition
\begin{equation}
E_{in}^\pm(k)=\int_{r_i}^{r_i\pm\Delta}
dr\left(\frac{r-r_i}{\Delta}\right)^n e^{ikr}
\end{equation}
we rewrite Eq.\ref{D+-} as
\begin{center}
\vskip 3mm
\begin{eqnarray}
D_{in}^\pm(k)=\pm\frac{4\pi}{k\Omega}\Im\left[\Delta E_{i(n+1)}^\pm(k)+
r_i E_{in}^\pm(k)\right]
\label{noncoulD+-}
\end{eqnarray}
\end{center}
\vskip 3mm

Finally, using integration by part, one can define $E^\pm_{in}$ recursively
\begin{center}
\vskip 3mm
\begin{eqnarray}
E^\pm_{in}(k)=\frac{1}{ik}\left[(\pm)^ne^{ik(r_i\pm\Delta)}-\frac{n}{\Delta}
E^\pm_{i(n-1)}(k)\right]
\label{nthEpm}
\end{eqnarray}
\end{center}
\vskip 3mm
\noindent
starting from the $n=0$ term
\vskip 3mm
\begin{center}
\begin{eqnarray}
E^\pm_{i0}(k)=\frac{1}{ik}e^{ikr_i}\left(e^{\pm ik\Delta}-1\right)
\label{0thEpm}
\end{eqnarray}
\end{center}
\vskip 3mm
\subsubsection{$k\ne 0$, coulomb case.}
To efficiently treat the coulomb divergence at the origin it is convenient to use
a basis set $c_{i\alpha}^{\text{coul}}$ of the form 
\begin{equation}
c_{i\alpha}^{\text{coul}}=\frac{c_{i\alpha}}{r}
\end{equation}
An equation identical to Eq.\ref{D+-} holds but with the modified definition
\begin{equation}
D_{in}^\pm(k)=\pm\frac{4\pi}{k\Omega}\Im\left[\int_{r_i}^{r_i\pm\Delta}
dr\left(\frac{r-r_i}{\Delta}\right)^n e^{ikr}\right]
\end{equation}
which can be simply expressed using $E^\pm_{in}(k)$ as
\vskip 3mm
\begin{center}
\begin{eqnarray}
D_{in}^\pm(k)=\pm\frac{4\pi}{k\Omega}\Im\left[E_{in}^\pm(k)\right]
\label{coulD+-}
\end{eqnarray}
\end{center}
\vskip 3mm
\subsubsection{$k=0$ coulomb and non coulomb case.}
The definitions of $D_{in}(k)$ given so far are clearly incompatible 
with the choice $k=0$ (they involve division by $k$). For the non-coulomb
case the starting definition is
\begin{equation}
D^\pm_{in}(0)=\pm\frac{4\pi}{\Omega}\int_{r_i}^{r_i\pm\Delta}r^2
\left(\frac{r-r_i}{\Delta}\right)^ndr
\end{equation}
Using the definition $I_n^\pm=(\pm)^{n+1}\Delta/(n+1)$ we can express this
as
\begin{center}
\vskip 3mm
\begin{eqnarray}
D^\pm_{in}(0)=\pm\frac{4\pi}{\Omega}\left[\Delta^2 I_{n+2}^\pm
+2r_i\Delta I_{n+1}^\pm+2r_i^2I_n^\pm\right]
\label{noncoul_k=0D+-}
\end{eqnarray}
\end{center}
\vskip 3mm
For the coulomb case one get

\vskip 3mm
\begin{center}
\begin{eqnarray}
D^\pm_{in}(0)=\pm\frac{4\pi}{\Omega}\left(
\Delta I^\pm_{n+1} + r_i I^\pm_n\right)
\label{coul_k=0D+-}
\end{eqnarray}
\end{center}
\vskip 3mm
\subsection{Fourier components of the basis functions in 2D}
Eq.\ref{fourier_transform} still holds provided we define  
\begin{equation}
D^\pm_{in}(k)=\pm\frac{2\pi}{\Omega \Delta^n} \sum_{j=0}^n \binom{n}{j}
(-r_i)^{n-j}\int_{r_i}^{r_i\pm \Delta}\negthickspace \negthickspace 
\negthickspace \negthickspace \negthickspace \negthickspace \negthickspace 
dr r^{j+1-C} J_0(kr)
\label{2DD+-}
\end{equation}
where $C=1(=0)$ for the coulomb(non coulomb) case.
Eq.\ref{2DD+-} is obtained using the integral definition of the 
zero order Bessel function of the first kind 
\begin{equation}
J_0(z)=\frac{1}{\pi}\int_0^\pi e^{iz\cos\theta}d\theta
\end{equation}
and the binomial expansion for $(r-r_i)^n$.
The integrals can be computed recursively using the following identities
\begin{center}
\begin{minipage}{0.7\textwidth}
\begin{align}
&\int dz J_0(z)=\frac{z}{2}\left[\pi J_1(z)H_0(z)+J_0(z)(2-\pi H_1(z))\right]
\label{0thmoment}\\
&\int dz z J_0(z)= z J_1(z)
\label{1stmoment}\\
&\int dz z^n J_0(z)= z^nJ_1(z)+(n-1)x^{n-1}J_0(z)
-(n-1)^2\int dz z^{n-2} J_0(z)
\label{nthmoment}
\end{align}
\end{minipage}
\end{center}
Eq.\ref{nthmoment} is obtained using Eq.\ref{1stmoment}, integration by part and 
the identity $\int J_1(z) dz =-J_0(z)$. In Eq.\ref{0thmoment} $H_0$ and $H_1$ are Struve functions.

\subsection{Construction of the matrix elements}
Using the above equations one can construct the matrix elements in Eq.\ref{matrix_elements}
and proceed solving for the $t_i$. It is sometimes desirable to put some constraints
on the value of $t_i$. For example, when the coulomb potential is concerned one may 
want to set $t_{0}=1$. If the first $g$ variable are constrained by $t_{m}=\gamma_m$ 
with $m=[1,g]$ one can simply redefine Eq.\ref{matrix_elements} as
\begin{equation}
\begin{split}
A_{ij}=&\sum_{k>k_c} \tc_i(k)\tc_j(k)  \quad i,j\notin[1,g] \\
b_j=&\sum_{k>k_c} \left(\tV(k)-\sum_{m=1}^g \gamma_m \tc_m(k)\right)\tc_j(k)\quad j\notin[1,g]
\end{split}
\label{modified_matrix_elements}
\end{equation}



% discussion below of (fortran) routines kept for now (18 Oct 2017)
% possibly these map onto routines in qmcpack also

%\subsection*{The routines}
%\subsubsection*{fitpnnew}
%This routine constructs the $t_i$ and $\tY(k)$. Previously a routine, 
%let us call it {\em shells}, generating a grid of $\kv$ points has to
%be called. {\em shells} stores $\kv$ vectors
%in order of increasing magnitude and defines a shell as the 
%set of vectors having the same magnitude $k$ (in practice their difference 
%in magnitude must be below a given threshold). The total number of
%shells $N_\text{shell}$ has to be large enough to represents $V(\rv)$
%accurately using $\tV(k)$. The number of vector
%in a given shell is called $w(k)$. The following variables are passed as
%input: $\tV(k),k,w(k),N_\text{shell},m,r_c,N_\text{knot},\Omega$ and are called
%\verb!v(0:nk),rk(0:nk),wt(0:nk),nk,m,rad,nknots!. Note that the vectors all
%start from $0$ which corresponds to $k=0$. The number of shells such that
%$k\le k_c$ is also passed as input and called \verb!nf!. Additional input variables 
%are  \verb!coul! a logical variable specifying if the potential is coulombic; 
%\verb!vmad! the exact value of the Madelung constant;
%\verb!t0,t1! logical variables specifying if a constraint has to be put
%on element $t_0$ or $t_1$ and \verb!vt0,vt1! the value at which $t_0$ and $t_1$
%have to be set if corresponding constraints are active. 
%The routine works in this way:
%\begin{itemize}
%\item it calls {\em basis} and gets the coefficients $S_{\alpha n}$ (the $n$-th
%      coefficient of the $\alpha$-th polynomials) for the desired value of $m$.
%\item for every $k$ point, knot $i$ and polynomial $\alpha$ compute $\tc_{i\alpha}(k)$
%      using Eq.\ref{fourier_transform}. $D^\pm_{in}(k)$ is provided by {\em splint3D}
%      or {\em splint2D}. The routine uses \verb!ialpha!$=i(m+1)+\alpha$ (the range of 
%      variability of $\alpha$ and $i$ is specified above Eq.\ref{basisdef}).
%\item Matrix elements are constructed according to Eq.\ref{matrix_elements}
%\item Matrix elements are modified according to Eq.\ref{modified_matrix_elements} 
%      if constraints are active
%\item $t_i$ are computed solving the linear system. $\tY(k)$ are computed.
%\item A comparison with the exact Madelung constant is performed.
%\end{itemize}
%
%\subsubsection*{splint3D}
%Called by {\em fitpnnew}. This routine compute $D^\pm_{in}(k)$ for given 
%$k$ and $i$ and for all
%$n$ (going from $0$ to $\mathcal{N}=2m+1$). $D^\pm_{in}(k)$ are called 
%\verb!ddplus(0:maxn)! and \verb!ddminus(0:maxn)! and are given as output
%by the routine. In input one is required to specify $\mathcal{N},r_i,\Delta,k,\Omega$,
%respectively named \verb!maxn,r,delta,k,vol!. A logical input flag called 
%\verb!coul! specify if the potential is coulombic or not. The routine works
%in this way:
%\begin{itemize}
%\item it checks if $k$ is equal to 0
%\item if $k\ne 0$ then
%  \begin{itemize}
%  \item it computes $E^\pm_{in}(k)$ for the specified $i$ and $k$ using Eqs.\ref{nthEpm} and
%      \ref{0thEpm}. $E^\pm_{in}(k)$ are called \verb!ee(0:maxn,!$\pm$\verb!1)! 
%      (\verb!ee(:,0)! are never used).
%  \item Depending on the value of \verb!coul! either Eq.\ref{noncoulD+-} or Eq.\ref{coulD+-} 
%      is used to construct $D^\pm_{in}(k)$. The prefactor $\frac{4\pi}{k \Omega}$ 
%      is precomputed and called \verb!dnorm!.
%  \end{itemize}
%\item if $k=0$ the code uses either Eq.\ref{noncoul_k=0D+-} or Eq.\ref{coul_k=0D+-}.
%\end{itemize}
%
%\subsubsection*{splint2D}
%Called by {\em fitpnnew}. This routine compute $D^\pm_{in}(k)$ in the 2D case.
%The \verb!i\o! format is identical to {\em splint3D}. Equations from~\ref{0thmoment}
%to~\ref{nthmoment} are used to generate the required integrals.
%
%
%\subsubsection*{basis}
%Called by {\em fitpnnew}. It computes the coefficients $S_{\alpha n}$ (the $n$-th 
%coefficient of the $\alpha$-th polynomials) using Eqs.\ref{Smatrix1} and~\ref{Smatrix2}.
%These coefficients are stored in \verb!s(0:m,0:2m+1)!. $m$ (called \verb!m!) 
%is required in input.
%
%\subsubsection*{computespl}
%This compute $W(r)$ at any $r$. $r$ is named \verb!rpos! internally. It requires
%$m,2m+1,N_\text{knot},S_{\alpha n},r_i,t_i,\Delta$. These are internally called
%\verb!m,maxn,nknots,s(0:m,0:maxn),r(0:nknots),t(0:nknots(m+1)-1),delta!. 
%\verb!coul! is also needed: it is a logical variable 
%to specify if $c_{i\alpha}^\text{coul}(r)$ have to be used instead of $c_{i\alpha}(r)$.
%The value of $W(r)$ is stored in \verb!w!.




\newpage
\section{Feature: Cubic Spline Interpolation}
% Written by Kenneth P .Esler Jr.
% Originally titled ``Cubic Spline Interpolation in 1, 2 and 3 Dimensions''

\newenvironment{DMatrix}{\begin{array}|{*{20}{c}}|}{\end{array}}
\newenvironment{MyMatrix}{\begin{array}({*{20}{c}})}{\end{array}}
\newenvironment{LMatrix}{\begin{array}({*{20}{l}})}{\end{array}}

We present the basic equations and algorithms necessary to
construct and evaluate cubic interpolating splines in one, two, and
three dimensions.  Equations are provided for both natural and
periodic boundary conditions.

\subsection{One Dimension}
Let us consider the problem in which we have a function $y(x)$
specified at a discrete set of points $x_i$, such that $y(x_i) = y_i$.
We wish to construct a piece-wise cubic polynomial interpolating
function, $f(x)$, which satisfies the following conditions:
\begin{itemize}
\item $f(x_i) = y_i$
\item $f'(x_i^-) = f'(x_i^+)$
\item $f''(x_i^-) = f''(x_i+)$
\end{itemize}

\subsubsection{Hermite Interpolants}
In our piecewise representation, we wish to store only the values,
$y_i$, and first derivatives, $y'_i$, of our function at each point
$x_i$, which we call {\em knots}.  Given this data, we wish to
construct the piecewise cubic function to use between $x_i$ and
$x_{i+1}$ which satisfies the above conditions.  In particular, we
wish to find the unique cubic polynomial, $P(x)$ satisfying
\begin{eqnarray}
P(x_i)      & = & y_i      \label{eq:c1} \\
P(x_{i+1})  & = & y_{i+1}  \label{eq:c2} \\
P'(x_i)     & = & y'_i     \label{eq:c3} \\
P'(x_{i+1}) & = & y'_{i+1} \label{eq:c4}
\end{eqnarray}
\begin{eqnarray}
h_i & \equiv & x_{i+1} - {x_i} \\
t & \equiv & \frac{x-x_i}{h_i}.
\end{eqnarray}
We then define the basis functions,
\begin{eqnarray}
p_1(t) & = & (1+2t)(t-1)^2  \label{eq:p1}\\
q_1(t) & = & t (t-1)^2      \\
p_2(t) & = & t^2(3-2t)      \\
q_2(t) & = & t^2(t-1)      \label{eq:q2}
\end{eqnarray}
On the interval, $(x_i, x_{i+1}]$, we define the interpolating
function,
\begin{equation}
P(x) = y_i p_1(t) + y_{i+1}p_2(t) + h\left[y'_i q_1(t) + y'_{i+1} q_2(t)\right]
\end{equation}
It can be easily verified that $P(x)$ satisfies the conditions Eq.~\ref{eq:c1}
through Eq.~\ref{eq:c4}.  It is now left to
determine the proper values for the $y'_i\,$s such that the continuity
conditions given above are satisfied.

By construction, the value of the function and derivative will match
at the knots, i.e.
\begin{equation}
P(x_i^-) = P(x_i^+), \ \ \ \ P'(x_i^-) = P'(x_i^+).
\end{equation}
Then we must now enforce only the second derivative continuity:
\begin{eqnarray}
P''(x_i^-) & = & P''(x_i^+) \\
\frac{1}{h_{i-1}^2}\left[\rule{0pt}{0.3cm}6 y_{i-1} -6 y_i + h_{i-1}\left(2 y'_{i-1} +4 y'_i\right) \right]& = &
\frac{1}{h_i^2}\left[\rule{0pt}{0.3cm}-6 y_i + 6 y_{i+1} +h_i\left( -4 y'_i -2 y'_{i+1} \right)\right] \nonumber
\end{eqnarray}
Let us define
\begin{eqnarray}
\lambda_i & \equiv & \frac{h_i}{2(h_i+h_{i-1})} \\
\mu_i & \equiv & \frac{h_{i-1}}{2(h_i+h_{i-1})}  = \frac{1}{2} - \lambda_i.
\end{eqnarray}
Then we may rearrange,
\begin{equation}
\lambda_i y'_{i-1} + y'_i + \mu_i y'_{i+1} = \underbrace{3 \left[\lambda_i \frac{y_i - y_{i-1}}{h_{i-1}} + \mu_i \frac{y_{i+1}
    - y_i}{h_i} \right] }_{d_i}
\end{equation}
This equation holds for all $0<i<(N-1)$, so we have a tridiagonal set of
equations.  The equations for $i=0$ and $i=N-1$ depend on the boundary
conditions we are using.  
\subsubsection{Periodic boundary conditions}
For periodic boundary conditions, we have
\begin{equation}
\begin{matrix}
y'_0           & +  & \mu_0 y'_1     &   &                   &            & \dots                   & +  \lambda_0 y'_{N-1} & = & d_0 \\
\lambda_1 y'_0 & +  & y'_1           & + &  \mu_1 y'_2       &            & \dots                   &                       & = & d_1 \\
               &    & \lambda_2 y'_1 & + &  y'_2           + & \mu_2 y'_3 & \dots                   &                       & = & d_2 \\
               &    &                &   &  \vdots           &            &                         &                       &   &     \\
\mu_{N-1} y'_0 &    &                &   &                   &            & +\lambda_{N-1} y'_{N-1} & +  y'_{N-2}           & = & d_3 
\end{matrix}
\end{equation}
Or, in matrix form, we have,
\begin{equation}
\begin{MyMatrix}
1         & \mu_0     &    0   &   0           & \dots         &      0        & \lambda_0 \\
\lambda_1 &  1        & \mu_1  &   0           & \dots         &      0        &     0     \\
0         & \lambda_2 &   1    & \mu_2         & \dots         &      0        &     0     \\
\vdots    & \vdots    & \vdots & \vdots        & \ddots        &   \vdots      &  \vdots   \\
0         &   0       &   0    & \lambda_{N-3} &      1        & \mu_{N-3}     &    0      \\
0         &   0       &   0    &   0           & \lambda_{N-2} &      1        & \mu_{N-2} \\
\mu_{N-1} &   0       &   0    &   0           &   0           & \lambda_{N-1} &  1     
\end{MyMatrix}
\begin{MyMatrix} y'_0 \\ y'_1 \\ y'_2 \\ \vdots \\ y'_{N-3} \\ y'_{N-2} \\ y'_{N-1} \end{MyMatrix} =
\begin{MyMatrix} d_0  \\  d_1 \\  d_2 \\ \vdots \\  d_{N-3} \\  d_{N-2} \\  d_{N-1} \end{MyMatrix} .
\end{equation}
The system is tridiagonal except for the two elements in the upper
right and lower left corners.  These terms complicate the solution a
bit, although it can still be done in $\mathcal{O}(N)$ time.  We first
proceed down the rows, eliminating the the first non-zero term in each
row by subtracting the appropriate multiple of the previous row.  At
the same time, we also eliminate the first element in the last row,
shifting the position of the first non-zero element to the right with
each iteration.  When we get to the final row, we will have the value
for $y'_{N-1}$.  We can then proceed back upward, backsubstituting
values from the rows below to calculate all the derivatives.

\subsubsection{Complete boundary conditions}
If we specify the first derivatives of our function at the end points,
we have what is known as {\em complete} boundary conditions.  The
equations in that case are trivial to solve:
\begin{equation}
\begin{MyMatrix}
1         &  0        &    0   &   0           & \dots         &      0        &     0     \\
\lambda_1 &  1        & \mu_1  &   0           & \dots         &      0        &     0     \\
0         & \lambda_2 &   1    & \mu_2         & \dots         &      0        &     0     \\
\vdots    & \vdots    & \vdots & \vdots        & \ddots        &   \vdots      &  \vdots   \\
0         &   0       &   0    & \lambda_{N-3} &      1        & \mu_{N-3}     &    0      \\
0         &   0       &   0    &   0           & \lambda_{N-2} &      1        & \mu_{N-2} \\
0         &   0       &   0    &   0           &   0           &      0        &  1     
\end{MyMatrix}
\begin{MyMatrix} y'_0 \\ y'_1 \\ y'_2 \\ \vdots \\ y'_{N-3} \\ y'_{N-2} \\ y'_{N-1} \end{MyMatrix} =
\begin{MyMatrix} d_0  \\  d_1 \\  d_2 \\ \vdots \\  d_{N-3} \\  d_{N-2} \\  d_{N-1} \end{MyMatrix} .
\end{equation}
This system is completely tridiagonal and we may solve trivially by
performing row eliminations downward, then proceeding upward as
before.

\subsubsection{Natural boundary conditions}
If we do not have information about the derivatives at the boundary
conditions, we may construct a {\em natural spline}, which assumes the
the second derivatives are zero at the end points of our spline.  In
this case our system of equations is the following:
\begin{equation}
\begin{MyMatrix}
1         & \frac{1}{2} &    0   &   0           & \dots         &      0        &     0     \\
\lambda_1 &  1          & \mu_1  &   0           & \dots         &      0        &     0     \\
0         & \lambda_2   &   1    & \mu_2         & \dots         &      0        &     0     \\
\vdots    & \vdots      & \vdots & \vdots        & \ddots        &   \vdots      &  \vdots   \\
0         &   0         &   0    & \lambda_{N-3} &      1        & \mu_{N-3}     &    0      \\
0         &   0         &   0    &   0           & \lambda_{N-2} &      1        & \mu_{N-2} \\
0         &   0         &   0    &   0           &   0           &  \frac{1}{2}  &  1     
\end{MyMatrix}
\begin{MyMatrix} y'_0 \\ y'_1 \\ y'_2 \\ \vdots \\ y'_{N-3} \\ y'_{N-2} \\ y'_{N-1} \end{MyMatrix} =
\begin{MyMatrix} d_0  \\  d_1 \\  d_2 \\ \vdots \\  d_{N-3} \\  d_{N-2} \\  d_{N-1} \end{MyMatrix} ,
\end{equation}
with
\begin{equation}
d_0 = \frac{3}{2} \frac{y_1-y_1}{h_0}, \ \ \ \ \ d_{N-1} = \frac{3}{2} \frac{y_{N-1}-y_{N-2}}{h_{N-1}}.
\end{equation}

\subsection{Bicubic Splines}
It is possible to extend the cubic spline interpolation method to
functions of two variables, i.e. $F(x,y)$.  In this case, we have a
rectangular mesh of points given by $F_{ij} \equiv F(x_i,y_j)$.  In
the case of 1D splines, we needed to store the value of the first
derivative of the function at each point, in addition to the value.
In the case of {\em bicubic splines}, we need to store four
quantities for each mesh point:  
\begin{eqnarray}
F_{ij}    & \equiv & F(x_i, y_i)            \\
F^x_{ij}  & \equiv & \partial_x F(x_i, y_i) \\
F^y_{ij}  & \equiv & \partial_y F(x_i, y_i) \\
F^{xy}    & \equiv & \partial_x \partial_y F(x_i, y_i)
\end{eqnarray}

Consider the point $(x,y)$ at which we wish to interpolate $F$.  We
locate the rectangle which contains this point, such that $x_i <= x <
x_{i+1}$ and $y_i <= x < y_{i+1}$.  Let 
\begin{eqnarray}
h & \equiv & x_{i+1}-x_i \\
l & \equiv & y_{i+1}-y_i \\
u & \equiv & \frac{x-x_i}{h} \\
v & \equiv & \frac{y-y_i}{l}
\end{eqnarray}
Then, we calculate the interpolated value as
\begin{equation}
F(x,y) = 
\begin{MyMatrix}
p_1(u) \\ p_2(u) \\ h q_1(u) \\ h q_2(u) 
\end{MyMatrix}^T
\begin{MyMatrix}
F_{i,j}     & F_{i+1,j}     & F^y_{i,j}      & F^y_{i,j+1}     \\
F_{i+1,j}   & F_{i+1,j+1}   & F^y_{i+1,j}    & F^y_{i+1,j+1}   \\
F^x_{i,j}   & F^x_{i,j+1}   & F^{xy}_{i,j}   & F^{xy}_{i,j+1}  \\
F^x_{i+1,j} & F^x_{i+1,j+1} & F^{xy}_{i+1,j} & F^{xy}_{i+1,j+1} 
\end{MyMatrix}
\begin{MyMatrix}
p_1(v)\\ p_2(v)\\ k q_1(v) \\ k q_2(v)
\end{MyMatrix}
\end{equation}
\subsubsection{Construction bicubic splines}
We now address the issue of how to compute the derivatives that are
needed for the interpolation.  The algorithm is quite simple.  For
every $x_i$, we perform the tridiagonal solution as we did in the 1D
splines to compute $F^y_{ij}$.  Similarly, we perform a tridiagonal
solve for every value of $F^x_{ij}$.  Finally, in order to compute the
cross-derivative we may {\em either} to the tridiagonal solve in the $y$
direction of $F^x_{ij}$, {\em or} solve in the $x$ direction for
$F^y_{ij}$ to obtain the cross-derivatives, $F^{xy}_{ij}$.  Hence,
only minor modifications to the $1D$ interpolations are necessary.

\subsection{Tricubic Splines}
Bicubic interpolation required two four-component vectors and a 4x4
matrix.  By extension, tricubic interpolation requires three
4-component vectors and a 4x4x4 tensor.  We summarize the forms of
these vectors below.
\begin{eqnarray}
h & \equiv & x_{i+1}-x_i \\
l & \equiv & y_{i+1}-y_i \\
m & \equiv & z_{i+1}-z_i \\
u & \equiv & \frac{x-x_i}{h} \\
v & \equiv & \frac{y-y_i}{l} \\
w & \equiv & \frac{z-z_i}{m}
\end{eqnarray}
\begin{eqnarray}
\vec{a} & = & 
\begin{MyMatrix}
p_1(u) & p_2(u) & h q_1(u) & h q_2(u) 
\end{MyMatrix}^T \\
\vec{b} & = & 
\begin{MyMatrix}
p_1(v) & p_2(v) & k q_1(v) & k q_2(v) 
\end{MyMatrix}^T \\
\vec{c} & = & 
\begin{MyMatrix}
p_1(w) & p_2(w) & l q_1(w) & l q_2(w) 
\end{MyMatrix}^T 
\end{eqnarray}
\begin{equation}
\begin{LMatrix}
A_{000} = F_{i,j,k}     & A_{001}=F_{i,j,k+1}     & A_{002}=F^z_{i,j,k}      & A_{003}=F^z_{i,j,k+1}      \\
A_{010} = F_{i,j+1,k}   & A_{011}=F_{i,j+1,k+1}   & A_{012}=F^z_{i,j+1,k}    & A_{013}=F^z_{i,j+1,k+1}    \\
A_{020} = F^y_{i,j,k}   & A_{021}=F^y_{i,j,k+1}   & A_{022}=F^{yz}_{i,j,k}   & A_{023}=F^{yz}_{i,j,k+1}   \\
A_{030} = F^y_{i,j+1,k} & A_{031}=F^y_{i,j+1,k+1} & A_{032}=F^{yz}_{i,j+1,k} & A_{033}=F^{yz}_{i,j+1,k+1} \\
                        &                         &                          &                            \\
A_{100} = F_{i+1,j,k}     & A_{101}=F_{i+1,j,k+1}     & A_{102}=F^z_{i+1,j,k}      & A_{103}=F^z_{i+1,j,k+1}      \\
A_{110} = F_{i+1,j+1,k}   & A_{111}=F_{i+1,j+1,k+1}   & A_{112}=F^z_{i+1,j+1,k}    & A_{113}=F^z_{i+1,j+1,k+1}    \\
A_{120} = F^y_{i+1,j,k}   & A_{121}=F^y_{i+1,j,k+1}   & A_{122}=F^{yz}_{i+1,j,k}   & A_{123}=F^{yz}_{i+1,j,k+1}   \\
A_{130} = F^y_{i+1,j+1,k} & A_{131}=F^y_{i+1,j+1,k+1} & A_{132}=F^{yz}_{i+1,j+1,k} & A_{133}=F^{yz}_{i+1,j+1,k+1} \\
                        &                         &                          &                            \\
A_{200} = F^x_{i,j,k}      & A_{201}=F^x_{i,j,k+1}      & A_{202}=F^{xz}_{i,j,k}      & A_{203}=F^{xz}_{i,j,k+1}    \\
A_{210} = F^x_{i,j+1,k}    & A_{211}=F^x_{i,j+1,k+1}    & A_{212}=F^{xz}_{i,j+1,k}    & A_{213}=F^{xz}_{i,j+1,k+1}  \\
A_{220} = F^{xy}_{i,j,k}   & A_{221}=F^{xy}_{i,j,k+1}   & A_{222}=F^{xyz}_{i,j,k}     & A_{223}=F^{xyz}_{i,j,k+1}   \\
A_{230} = F^{xy}_{i,j+1,k} & A_{231}=F^{xy}_{i,j+1,k+1} & A_{232}=F^{xyz}_{i,j+1,k}   & A_{233}=F^{xyz}_{i,j+1,k+1} \\
                        &                         &                          &                                      \\
A_{300} = F^x_{i+1,j,k}      & A_{301}=F^x_{i+1,j,k+1}      & A_{302}=F^{xz}_{i+1,j,k}    & A_{303}=F^{xz}_{i+1,j,k+1}   \\
A_{310} = F^x_{i+1,j+1,k}    & A_{311}=F^x_{i+1,j+1,k+1}    & A_{312}=F^{xz}_{i+1,j+1,k}  & A_{313}=F^{xz}_{i+1,j+1,k+1} \\
A_{320} = F^{xy}_{i+1,j,k}   & A_{321}=F^{xy}_{i+1,j,k+1}   & A_{322}=F^{xyz}_{i+1,j,k}   & A_{323}=F^{xyz}_{i+1,j,k+1}  \\
A_{330} = F^{xy}_{i+1,j+1,k} & A_{331}=F^{xy}_{i+1,j+1,k+1} & A_{332}=F^{xyz}_{i+1,j+1,k} & A_{333}=F^{xyz}_{i+1,j+1,k+1} 
\end{LMatrix}
\end{equation}
Now, we can write
\begin{equation}
F(x,y,z) = \sum_{i=0}^3 a_i \sum_{j=0}^3 b_j \sum_{k=0}^3 c_k \ A_{i,j,k} 
\end{equation}
The appropriate derivatives of $F$ may be computed by a generalization
of the method used for bicubic splines above.




\newpage
\section{Feature: B-spline Orbital Tiling (Band Unfolding)}

% Written by Kenneth P .Esler Jr.
% Originally titled ``Generalized band unfolding for quantum Monte Carlo simulation of solids''

In continuum quantum Monte Carlo simulations, it is necessary to
evaluate the electronic orbitals of a system at real-space positions
hundreds of millions of times.  It has been found that if
these orbitals are represented in a localized, B-spline basis, each
evaluation takes a small, constant time which is independent of system
size.

Unfortunately, the memory required for storing the B-spline grows with
the second power of the system size.  If we are studying perfect
crystals, however, this can be reduced to linear scaling if we {\em
  tile} the primitive cell.  In this approach, 
%implemented in the CASINO QMC simulation suite, 
a supercell is constructed by tiling the
primitive cell $N_1 \times N_2 \times N_3$ in the three lattice
directions.  The orbitals are then represented in real space only in
the primitive cell, and an $N_1 \times N_2 \times N_3$ k-point mesh.
To evaluate an orbital at any point in the supercell, it is only
necessary to wrap that point back into the primitive cell, evaluate
the spline, and then multiply the phase factor,
$e^{-i\mathbf{k}\cdot\mathbf{r}}$.  

Here, we show that this approach can be generalized to a tiling
constructed with a $3\times 3$ nonsingular matrix of integers, of which
the above approach is a special case.  This generalization brings with
it a number of advantages.  The primary reason for performing
supercell calculations in QMC is to reduce finite-size errors.  These
errors result from three sources:  1) the quantization of the crystal
momentum;  2) the unphysical periodicity of the exchange-correlation
hole of the electron; and 3) the kinetic-energy contribution from the
periodicity of the long-range jastrow correlation functions.  The first
source of error can be largely eliminated by twist averaging.  If the
simulation cell is large enough that XC hole does not ``leak'' out of
the simulation cell, the second source can be eliminated either
through use of the MPC interaction or the {\em a postiori} correction
of Chiesa et. al.  

The satisfaction of the leakage requirement is controlled by whether
the minimum distance, $L_{\text{min}}$ from one supercell image to the
next is greater than the width of the XC hole.  Therefore, given a
choice, it is best to use a cell which is as nearly cubic as possible,
since this choice maximizes $L_{\text{min}}$ for a given number of
atoms.  Most often, however, the primitive cell is not cubic.  In
these cases, if we wish to choose the optimal supercell to reduce
finite size effects, we cannot utilize the simple primitive tiling
scheme.  In the generalized scheme we present, it is possible to
choose far better supercells (from the standpoint of finite-size
errors), while retaining the storage efficiency of the original tiling
scheme.

\subsection{The mathematics}
\renewcommand{\vp}{\mathbf{a}^{\text{p}}}
\renewcommand{\vs}{\mathbf{a}^{\text{s}}} 
\renewcommand{\Smat}{\mathbf{S}}
Consider the set of primitive lattice vectors, $\{\vp_1, \vp_2,
\vp_3\}$.  We may write these vectors in a matrix, $\mathbf{L}_p$, whose
rows are the primitive lattice vectors.  Consider a non-singular
matrix of integers, $\Smat$.  A corresponding set of supercell lattice
vectors, $\{\vs_1, \vs_2, \vs_3\}$, can be constructed by the matrix
product 
\begin{equation}
\vs_i = S_{ij} \vp_j
\end{equation}
If the primitive cell contains $N_p$ atoms, the supercell will then
contain $N_s = |\det(\Smat)| N_p$ atoms.

\subsection{Example: FeO}
As an example, consider the primitive cell for antiferromagnetic FeO
(wustite) in the rocksalt structure.  The primitive vectors, given in
units of the lattice constant, are given by
\newcommand{\xv}{\hat{\mathbf{x}}} 
\newcommand{\yv}{\hat{\mathbf{y}}}
\newcommand{\zv}{\hat{\mathbf{z}}}
\begin{eqnarray}
\vs_1 & = & \frac{1}{2}\xv + \frac{1}{2}\yv +      \ \   \zv \\
\vs_2 & = & \frac{1}{2}\xv +      \ \   \yv + \frac{1}{2}\zv \\
\vs_3 & = &   \ \      \xv + \frac{1}{2}\yv + \frac{1}{2}\zv 
\end{eqnarray}
This primitive cell contains two iron atoms and two oxygen atoms. It
is a very elongated cell with acute angles, and thus has a short
minimum distance between adjacent images.

The smallest cubic cell consistent with the AFM ordering can be
constructed with the matrix
\begin{equation}
\Smat = \left[\begin{array}{rrr}
  -1 & -1 &  3 \\
  -1 &  3 & -1 \\
   3 & -1 & -1 
  \end{array}\right]
\end{equation}
This cell has $2\det(\Smat) = 32$ iron atoms and 32 oxygen atoms.  In
this example, we may perform the simulation in the 32-iron supercell,
while storing the orbitals only in the 2-iron primitive cell, for a
savings of a factor of 16.  
%On current multicore supercomputers, with
%1-2GB RAM per core, this is literally the difference between be able
%to perform the simulation or not.

\subsubsection{The k-point mesh}
In order to be able to use the generalized tiling scheme, we need to
have the appropriate number of bands to occupy in the supercell.
This may be achieved by appropriately choosing the k-point mesh.  In
this section, we explain how these points are chosen.  

For simplicity, let us assume that the supercell calculation will be
performed at the $\Gamma$-point.  We may lift this restriction very
easily later.  The fact that supercell calculation is performed at
$\Gamma$ implies that the k-points used in the primitive-cell
calculation must be $\mathbf{G}$-vectors of the superlattice.  This
still leaves us with an infinite set of vectors.  We may reduce this
set to a finite number by considering that the orbitals must form an
linearly independent set.  Orbitals with k-vectors $\mathbf{k}^p_1$
and $\mathbf{k}^p_2$ will differ by at most a constant factor if
$\mathbf{k}^p_1 - \mathbf{k}^p_2 = \mathbf{G}^p$, where $\mathbf{G}^p$
is a reciprocal lattice vector of the primitive cell.  

Combining these two considerations gives us a prescription for
generating our k-point mesh.  The mesh may be taken to be the set of
k-point which are G-vectors of the superlattice, reside within the
first Brillouin zone (FBZ) of the primitive lattice, whose members do
not differ a G-vector of the primitive lattice.  Upon constructing
such a set, we find that the number of included k-points is equal to
$|\det(\Smat)|$, precisely the number we need.  This can by considering
the fact that the supercell has a volume $|\det(\Smat)|$ times that of
the primitive cell.  This implies that the volume of the supercell's
FBZ is $|\det(\Smat)|^{-1}$ times that of the primitive cell.  Hence,
$|\det(\Smat)|$ G-vectors of the supercell will fit in the FBZ of the
primitive cell.  Removing duplicate k-vectors, which differ from
another by a reciprocal lattice vector, avoids double-counting vectors
which lie on zone faces.

\subsubsection{Formulae}
\newcommand{\Amat}{\mathbf{A}} 
\newcommand{\Bmat}{\mathbf{B}} 
\renewcommand{\vk}{\mathbf{k}}
\newcommand{\vt}{\mathbf{t}}

Let $\Amat$ be the matrix whose rows are the direct lattice vectors,
$\{\mathbf{a}_i\}$.  The, let the matrix $\Bmat$ be defined as
$2\pi(\Amat^{-1})^\dagger$.  Its rows are the primitive reciprocal
lattice vectors.  Let $\Amat_p$ and $\Amat_s$ represent the primitive
and superlattice matrices, respectively, and similarly for their
reciprocals.  Then we have
\begin{eqnarray}
\Amat_s & = & \Smat \Amat_p \\
\Bmat_s & = & 2\pi\left[(\Smat \Amat_p)^{-1}\right]^\dagger \\
        & = & 2\pi\left[\Amat_p^{-1} \Smat{-1}\right]^\dagger \\
        & = & 2\pi(\Smat^{-1})^\dagger (\Amat_p^{-1})^\dagger \\
        & = & (\Smat^{-1})^\dagger \Bmat_p
\end{eqnarray}  
Consider a k-vector, $\vk$.  It may be alternatively be written in
basis of reciprocal lattice vectors as $\vt$.  
\begin{eqnarray}
\vk & = & (\vt^\dagger \Bmat)^\dagger \\
    & = & \Bmat^\dagger \vt           \\
\vt & = & (\Bmat^\dagger)^{-1} \vk    \\
    & = & (\Bmat^{-1})^\dagger \vk    \\
    & = & \frac{\Amat \vk}{2\pi}
\end{eqnarray}
We may then express a twist vector of the primitive lattice, $\vt_p$ in terms
of the superlattice.
\begin{eqnarray}
\vt_s & = & \frac{\Amat_s \vk}{2\pi}                           \\
      & = & \frac{\Amat_s \Bmat_p^\dagger \vt_p}{2\pi}         \\
      & = & \frac{\Smat \Amat_p \Bmat_p^\dagger \vt_p}{2\pi}   \\
      & = & \frac{2\pi \Smat \Amat_p \Amat_p^{-1} \vt_p}{2\pi} \\
      & = & \Smat \vt_p
\end{eqnarray}
This gives the simple result that twist-vectors transform in precisely
the same way as direct lattice vectors.




\newpage
\section{Feature: Hybrid orbital representation}

% Written by Kenneth P. Esler, Jr.
% Document originally included in QMCPACK at src/QMCWaveFunctions/AtomicOrbital.tex
% Originally titled ``Hybrid orbital representation''

\renewcommand{\vr}{\mathbf{r}}

\begin{equation}
\phi(\vr) = \sum_{\ell=0}^{\ell_\text{max}} \sum_{m=-\ell}^\ell Y_\ell^m (\hat{\Omega})
u_{\ell m}(r),
\end{equation}
where $u_{lm}(r)$ are complex radial functions represented in some
radial basis (e.g. splines).

\subsection{Real Spherical Harmonics}
\renewcommand{\Re}{\rm Re}
\renewcommand{\Im}{\rm Im}
If $\phi(\vr)$ can be written as purely real, we can change the
representation so that
\begin{equation}
\phi(\vr) = \sum_{l=0}^{l_\text{max}} \sum_{m=-\ell}^\ell Y_{\ell m}(\hat{\Omega})
\bar{u}_{lm}(r),
\end{equation}
where $\bar{Y}_\ell^m$ are the {\em real} spherical harmonics defined by
\begin{equation}
Y_{\ell m} = \begin{cases}
Y_\ell^0 & \mbox{if } m=0\\
{1\over 2}\left(Y_\ell^m+(-1)^m \, Y_\ell^{-m}\right) \ = \Re\left[Y_\ell^m\right]
%\sqrt{2} N_{(\ell,m)} P_\ell^m(\cos \theta) \cos m\varphi 
& \mbox{if } m>0 \\
{1\over i 2}\left(Y_\ell^{-m}-(-1)^{m}\, Y_\ell^{m}\right) = \Im\left[Y_\ell^{-m}\right]
%\sqrt{2} N_{(\ell,m)} P_\ell^{-m}(\cos \theta) \sin m\varphi 
&\mbox{if } m<0.
\end{cases}
\end{equation}
We need then to relate $\bar{u}_{\ell m}$ to $u_{\ell m}$.  We wish
to express,
\begin{equation}
\Re\left[\phi(\vr)\right] = \sum_{\ell=0}^{\ell_\text{max}} \sum_{m=-\ell}^\ell
\Re\left[Y_\ell^m (\hat{\Omega}) u_{\ell m}(r)\right]
\end{equation}
in terms of $\bar{u}_{\ell m}(r)$ and $Y_{\ell m}$.
\begin{eqnarray}
\Re\left[Y_\ell^m u_{\ell m}\right] & = & \Re\left[Y_\ell^m\right]
\Re\left[u_{\ell m}\right] - \Im\left[Y_\ell^m\right] \Im\left[u_{\ell m}\right]
\end{eqnarray}
For $m>0$,
\begin{equation}
\Re\left[Y_\ell^m\right] = Y_{\ell m} \qquad \text{and} \qquad \Im\left[Y_\ell^m\right] = Y_{\ell\,-m}.
\end{equation}
For $m<0$,
\begin{equation}
\Re\left[Y_\ell^m\right] = (-1)^m Y_{\ell\, -m} \qquad \text and \qquad \Im\left[Y_\ell^m\right] = -(-1)^m Y_{\ell m}.
\end{equation}
Then for $m > 0$,
\begin{eqnarray}
\bar{u}_{\ell m} & = & \Re\left[u_{\ell m}\right] + (-1)^m \Re\left[u_{\ell\,-m}\right] \\
\bar{u}_{\ell\, -m} & = & -\Im\left[u_{\ell m}\right] + (-1)^m \Im\left[u_{\ell\,-m}\right].
\end{eqnarray}


\subsection{Projecting to atomic orbitals}

% Written by Ken Esler as part of the Common codebase used in wfconvert
% Originally titled ``Notes on projecting to atomic orbitals''
% Dated July 19, 2009

\renewcommand{\vr}{\mathbf{r}}
\newcommand{\vI}{\mathbf{I}}
\renewcommand{\vk}{\mathbf{k}}
\newcommand{\vG}{\mathbf{G}}

%\subsubsection{Form for orbitals}
Inside a muffin tin, orbitals are represented as product of spherical
harmonics and 1D radial functions, primarily represented by splines.
For a muffin tin centered at $\vI$, 
\begin{equation}
\phi_n(\vr) = \sum_{\ell,m} Y_\ell^m(\hat{\vr -\vI})
u_{lm}\left(\left|\vr - \vI\right|\right) \label{eq:ulm}
\end{equation}
Let use consider the case that our original representation for
$\phi(\vr)$ is of the form
\begin{equation}
\phi_{n,\vk}(\vr) = \sum_\vG c_{\vG+\vk}^n e^{i(\vG + \vk)\cdot \vr}
\end{equation}
Recall that
\begin{equation}
e^{i\vk\cdot\vr} = 4\pi \sum_{\ell,m} i^\ell j_\ell(|\vr||\vk|)
Y_\ell^m(\hat{\vk}) \left[Y_\ell^m(\hat{\vr})\right]^*.
\end{equation}
Conjugating,
\begin{equation}
e^{-i\vk\cdot\vr} = 4\pi\sum_{\ell,m} (-i)^\ell j_\ell(|\vr||\vk|)
\left[Y_\ell^m(\hat{\vk})\right]^* Y_\ell^m(\hat{\vr}).
\end{equation}
Setting $\vk \rightarrow -k$,
\begin{equation}
e^{i\vk\cdot\vr} = 4\pi\sum_{\ell,m} i^\ell j_\ell(|\vr||\vk|)
\left[Y_\ell^m(\hat{\vk})\right]^* Y_\ell^m(\hat{\vr}).
\end{equation}

Then,
\begin{equation}
e^{i\vk\cdot(\vr-\vI)} = 4\pi\sum_{\ell,m} i^\ell j_\ell(|\vr-\vI||\vk|)
\left[Y_\ell^m(\hat{\vk})\right]^* Y_\ell^m(\hat{\vr-\vI}).
\end{equation}

\begin{equation}
e^{i\vk\cdot\vr} = 4\pi e^{i\vk\cdot\vI} \-\sum_{\ell,m} i^\ell j_\ell(|\vr-\vI||\vk|)
\left[Y_\ell^m(\hat{\vk})\right]^* Y_\ell^m(\hat{\vr-\vI}).
\end{equation}

Then
\begin{equation}
\phi_{n,\vk}(\vr) =  \sum_\vG 4\pi c_{\vG+\vk}^n
e^{i(\vG+\vk)\cdot\vI} \sum_{\ell,m}
  i^\ell j_\ell(|\vG +\vk||\vr-\vI|)
  \left[Y_\ell^m(\hat{\vG+\vk})\right]^*
Y_\ell^m(\hat{\vr - \vI})
\end{equation}
Comparing to Eq.~\ref{eq:ulm},
\begin{equation}
u_{\ell m}^n(r) = 4\pi i^\ell \sum_G c_{\vG+\vk}^n e^{i(\vG+\vk)\cdot\vI}  j_\ell\left(|\vG + \vk|r|\right)
\left[Y_\ell^m(\hat{\vG + \vk})\right]^*.
\end{equation}
If we had adopted the opposite sign convention for Fourier transforms
(as is unfortunately the case in wfconvert), we would have
\begin{equation}
u_{\ell m}^n(r) = 4\pi (-i)^\ell \sum_G c_{\vG+\vk}^n e^{-i(\vG+\vk)\cdot\vI}  j_\ell\left(|\vG + \vk|r|\right)
\left[Y_\ell^m(\hat{\vG + \vk})\right]^*.
\end{equation}




\newpage
\section{Feature: Electron-electron-ion Jastrow factor}

% Written by Kenneth P. Esler, Jr.
% Document originally included in QMCPACK at src/QMCWaveFunctions/Jastrow/eeI_Jastrow.tex
% Originally titled ``Electron-electron-ion Jastrow factor''

\newcommand{\riI}{r_{iI}}
\newcommand{\briI}{\mathbf{r}_{iI}}
\newcommand{\rjI}{r_{jI}}
\newcommand{\brjI}{\mathbf{r}_{jI}}
\newcommand{\rij}{r_{ij}}
\newcommand{\brij}{\mathbf{r}_{ij}}
%\section{Form of the Jastrow}
The general form of the 3-body Jastrow we describe here depends on the
three inter-particle distances, $(\rij, \riI, \rjI)$.
\begin{equation}
J_3 = \sum_{I\in\text{ions}} \sum_{i,j \in\text{elecs};i\neq j} U(\rij, \riI,
\rjI)
\end{equation}
Note that we constrain the form of $U$ such that
$U(\rij, \riI,\rjI) = U(\rij, \rjI,\riI)$, so as to preserve the
particle symmetry of the wave function.  We then compute the gradient as
\begin{equation}
\nabla_i J_3 =  \sum_{I\in\text{ions}} \sum_{j \neq i}
\left[\frac{\partial U(\rij, \riI,\rjI)}{\partial\rij}
  \frac{\mathbf{r}_i - \mathbf{r}_j}{|\mathbf{r}_i - \mathbf{r}_j|} 
+ \frac{\partial U(\rij, \riI,\rjI)}{\partial\riI}
  \frac{\mathbf{r}_i - \mathbf{I}}{|\mathbf{r}_i - \mathbf{I}|}  \right]
\end{equation}
To compute the laplacian, we take
\begin{eqnarray}
\nabla_i^2 J_3 & = & \nabla_i \cdot \left(\nabla_i J_3\right) \\
& = & \sum_{I\in\text{ions}} \sum_{j\neq i } \left[
\frac{\partial^2 U}{\partial \rij^2} + \frac{2}{\rij} \frac{\partial
  U}{\partial \rij} + 2 \frac{\partial^2 U}{\partial \rij \partial
  \riI}\frac{\brij\cdot\briI}{\rij\riI} +\frac{\partial^2 U}{\partial
  \riI^2}
+ \frac{2}{\riI}\frac{\partial U}{\partial \riI} \nonumber
\right]
\end{eqnarray}
We now wish to compute the gradient of these terms w.r.t. the ion position, $I$.
\begin{equation}
\nabla_I J_3 = -\sum_{j\neq i} \left[ \frac{\partial U(\rij, \riI,\rjI)}{\partial\riI}
  \frac{\mathbf{r}_i - \mathbf{I}}{|\mathbf{r}_i - \mathbf{I}|} 
+\frac{\partial U(\rij, \riI,\rjI)}{\partial\rjI}
  \frac{\mathbf{r}_j - \mathbf{I}}{|\mathbf{r}_j - \mathbf{I}|} \right]
\end{equation}
For the gradient w.r.t. $i$ of the gradient w.r.t. $I$, the result is a tensor,
\begin{eqnarray}
\nabla_I \nabla_i J_3 & = & \nabla_I \sum_{j \neq i}
\left[\frac{\partial U(\rij, \riI,\rjI)}{\partial\rij}
  \frac{\mathbf{r}_i - \mathbf{r}_j}{|\mathbf{r}_i - \mathbf{r}_j|} 
+ \frac{\partial U(\rij, \riI,\rjI)}{\partial\riI}
  \frac{\mathbf{r}_i - \mathbf{I}}{|\mathbf{r}_i - \mathbf{I}|}  \right] \\\nonumber \\\nonumber
& = & -\sum_{j\neq i} \left[ 
\frac{\partial^2 U}{\partial \rij \riI} \hat{\mathbf{r}}_{ij} \otimes
\hat{\mathbf{r}}_{iI} + \left(\frac{\partial^2 U}{\partial \riI^2} -
\frac{1}{\riI} \frac{\partial U}{\partial \riI}\right)
\hat{\mathbf{r}}_{iI} \otimes \hat{\mathbf{r}}_{iI} \right. + \\\nonumber
& & \left. \qquad \ \ \  \frac{\partial^U}{\partial \rij \rjI} \hat{\mathbf{r}}_{ij} \otimes \hat{\mathbf{r}}_{jI} + \frac{\partial^2 U}{\partial \riI \partial \rjI}
\hat{\mathbf{r}}_{iI}\otimes \hat{\mathbf{r}}_{jI}  +
\frac{1}{\riI} \frac{\partial U}{\partial \riI} \overleftrightarrow{\mathbf{1}}\right]
\end{eqnarray}

\begin{eqnarray}
\nabla_I \nabla_i J_3 & = & \nabla_I \sum_{j \neq i}
\left[\frac{\partial U(\rij, \riI,\rjI)}{\partial\rij}
  \frac{\mathbf{r}_i - \mathbf{r}_j}{|\mathbf{r}_i - \mathbf{r}_j|} 
+ \frac{\partial U(\rij, \riI,\rjI)}{\partial\riI}
  \frac{\mathbf{r}_i - \mathbf{I}}{|\mathbf{r}_i - \mathbf{I}|}  \right] \\\nonumber 
& = & \sum_{j\neq i} \left[ -\frac{\partial^2 U}{\partial \rij \partial \riI} \hat{\mathbf{r}}_{ij} \otimes \hat{\mathbf{r}}_{iI} +
\left(-\frac{\partial^2 U}{\partial \riI^2}  + \frac{1}{\riI}\frac{\partial U}{\partial \riI} \right) 
\hat{\mathbf{r}}_{iI} \otimes \hat{\mathbf{r}}_{iI} - \frac{1}{\riI}\frac{\partial U}{\partial \riI} \overleftrightarrow{\mathbf{1}}
\right]
\end{eqnarray}
For the laplacian,
\begin{eqnarray}
\nabla_I \nabla_i^2 J_3 & = & \nabla_I\left[\nabla_i \cdot \left(\nabla_i J_3\right)\right] \\
& = & \nabla_I \sum_{j\neq i } \left[
\frac{\partial^2 U}{\partial \rij^2} + \frac{2}{\rij} \frac{\partial
  U}{\partial \rij} + 2 \frac{\partial^2 U}{\partial \rij \partial
  \riI}\frac{\brij\cdot\briI}{\rij\riI} +\frac{\partial^2 U}{\partial
  \riI^2}
+ \frac{2}{\riI}\frac{\partial U}{\partial \riI} \nonumber
\right] \\
& = & \sum_{j\neq i } 
\left[ \frac{\partial^3 U}{\partial r_{iI} \partial^2 r_{ij}} +
\frac{2}{r_{ij}} \frac{\partial^2 U}{\partial r_{iI} \partial r_{ij}}
+ 2\left(\frac{\partial^3 U}{\partial \rij \partial^2 \riI} -\frac{1}{\riI} \frac{\partial^2 U}{\partial \rij \partial \riI}\right)\frac{\brij\cdot\briI}{\rij\riI} + \frac{\partial^3 U}{\partial^3 \riI} - \frac{2}{\riI^2} \frac{\partial U}{ \partial \riI} + \frac{2}{\riI} \frac{\partial^2 U}{\partial^2 \riI}
\right] \frac{\mathbf{I} - \mathbf{r}_i}{|\mathbf{I} - \mathbf{r}_i|} + \nonumber \\\nonumber 
 & & \sum_{j\neq i } \left[ \frac{\partial^3U}{\partial \rij^2 \partial \rjI} + \frac{2}{\rij}\frac{\partial^2 U}{\partial \rjI \partial \rij} 
+ 2\frac{\partial^3 U}{\partial \rij \partial \riI \partial \rjI}\frac{\brij\cdot\briI}{\rij\riI}
+\frac{\partial^3 U}{\partial \riI^2 \partial \rjI} + \frac{2}{\riI}\frac{\partial^2 U}{\partial \riI \partial \rjI} \right] 
\frac{\mathbf{I} - \mathbf{r}_j}{|\mathbf{r}_j - \mathbf{I}|} + \\\nonumber 
& & \sum_{j\neq i } \left[ -\frac{2}{\riI}\frac{\partial^2 U}{\partial \rij \partial \riI}\right] \frac{\mathbf{r}_{ij}}{r_{ij}}
\end{eqnarray}




\newpage
\section{Feature: Reciprocal-Space Jastrow Factors}
\label{sec:feature_kspace_jastrow}

% Written by Kenneth P. Esler, Jr.
% Document originally included in QMCPACK at src/QMCWaveFunctions/Jastrow/kSpaceJastrowNotes.tex
% Originally titled ``Notes on Reciprocal-Space Jastrow Factors''

\renewcommand{\vG}{\mathbf{G}}
\renewcommand{\vr}{\mathbf{r}}
\renewcommand{\vI}{\mathbf{I}}

\subsection{Two-body Jastrow}
\begin{equation}
J_2 = \sum_{\vG\neq \mathbf{0}}\sum_{i\neq j} a_\vG e^{i\vG\cdot(\vr_i-\vr_j)}
\end{equation}
This may be rewritten as
\begin{eqnarray}
J_2 & = & \sum_{\vG\neq \mathbf{0}}\sum_{i\neq j} a_\vG e^{i\vG\cdot\vr_i}e^{-i\vG\cdot\vr_j} \\
& = & \sum_{\vG\neq \mathbf{0}} a_\vG \left\{
\underbrace{\left[\sum_i e^{i\vG\cdot\vr_i} \right]}_{\rho_\vG}
\underbrace{\left[\sum_j e^{-i\vG\cdot\vr_j} \right]}_{\rho_{-\vG}}  -1 \right\}
\end{eqnarray}
The $-1$ is just a constant term and may be subsumed into the $a_\vG$
coefficient by a simple redefinition.  This leaves a simple, but
general, form:
\begin{equation}
J_2 = \sum_{\vG\neq\mathbf{0}} a_\vG \rho_\vG \rho_{-\vG}
\end{equation}
We may now further constrain this on physical grounds.  First, we
recognize that $J_2$ should be real.  Since $\rho_{-\vG} =
\rho_\vG^*$, it follows that $\rho_{\vG}\rho_{-\vG} = |\rho_\vG|^2$ is
real, so that $a_\vG$ must be real.  Furthermore, we group the $\vG$'s
into $(+\vG, -\vG)$ pairs, and sum over only the positive vectors to
save time.

\subsection{One-body Jastrow}
The one-body Jastrow has a similar form, but depends on the
displacement from the electrons to the ions in the system.
\begin{equation}
J_1 = \sum_{\vG\neq\mathbf{0}} \sum_{\alpha}
\sum_{i\in\vI^\alpha}\sum_{j\in\text{elec.}} b^{\alpha}_\vG
  e^{i\vG\cdot(\vI^{\alpha}_i - \vr_j)},
\end{equation}
where $\alpha$ denotes the different ionic species.
We may rewrite this in terms of $\rho^{\alpha}_\vG$, 
\begin{equation}
J_1 = \sum_{\vG\neq\mathbf{0}} \left[\sum_\alpha b^\alpha_\vG
  \rho_\vG^\alpha\right] \rho_{-\vG},
\end{equation}
where
\begin{equation}
\rho^\alpha_\vG = \sum_{i\in\vI^\alpha} e^{i\vG\cdot\vI^\alpha_i}.
\end{equation}
We note that in the above equation, for a single configuration of the
ions, the sum in brackets can be rewritten as a single constant.  This
implies that the per-species one-body coefficients, $b^\alpha_\vG$, are
underdetermined for single configuration of the ions.  In general, if
we have $N$ species, we need $N$ linearly independent ion
configurations to uniquely determine $b^{\alpha}_\vG$.  For this
reason, we will drop the $\alpha$ superscript of $b_\vG$ for now.  

If we do desire to find a reciprocal space one-body Jastrow that is
transferable to systems with different ion positions and $N$ 
ionic species, we must perform compute $b_\vG$ for $N$ different ion
configurations.  We may then construct $N$ equations at each value of
$\vG$ to solve for the $N$ unknown values, $b^\alpha_\vG$.

In the two-body case, $a_\vG$ was constrained to be real by the fact
that $\rho_\vG \rho_{-\vG}$ was real.  However, in the one-body case,
there is no such guarantee about $\rho^\alpha_\vG \rho_\vG$.
Therefore, in general, $b_\vG$ may be complex.

\subsection{Symmetry considerations}
For a crystal, many of the $\vG$-vectors will be equivalent by
symmetry.  It is useful then, to divide the $\vG$-vectors into
symmetry-related groups and then to require that they share a common
coefficient.  Two vectors, $\vG$ and $\vG'$, may be considered
symmetry related if, for all $\alpha$ and $\beta$,
\begin{equation}
\rho^\alpha_\vG \rho^\beta_{-\vG} = \rho^\alpha_{\vG'} \rho^\beta_{-\vG'}. 
\end{equation}
For the one-body term, we may also omit from our list of $\vG$-vectors
those for which all species structure factors are zero.  This is
equivalent to saying that, if we are tiling a primitive cell, we
should include only the $\vG$-vectors of the primitive cell, and not
the supercell.  Note that this is not the case for the two-body term,
since the exchange-correlation hole should not have the periodicity of
the primitive cell.

\subsection{Gradients and Laplacians}
\begin{eqnarray}
\nabla_{\vr_i} J_2 & = & \sum_{\vG \neq 0} a_\vG \left[\left(\nabla_{\vr_i}\rho_\vG\right) \rho_{-\vG} + \text{c.c.}\right] \\
& = & \sum_{\vG\neq \mathbf{0}} 2\vG a_\vG \mathbf{Re}\left(i e^{i\vG\cdot\vr_i} \rho_{-\vG} \right) \\
& = & \sum_{\vG\neq \mathbf{0}} -2\vG a_\vG\mathbf{Im}\left(e^{i\vG\cdot\vr_i} \rho_{-\vG} \right)
\end{eqnarray}
The Laplacian is then given by
\begin{eqnarray}
  \nabla^2 J_2 & = & \sum_{\vG\neq\mathbf{0}} a_\vG \left[\left(\nabla^2 \rho_\vG\right) \rho_{-\vG} + \text{c.c.} 
  + 2\left(\nabla \rho_\vG)\cdot(\nabla \rho_{-\vG}\right)\right] \\
& = & \sum_{\vG\neq\mathbf{0}} a_\vG \left[ -2G^2\mathbf{Re}(e^{i\vG\cdot\vr_i}\rho_{-\vG}) + 
    2\left(i\vG e^{i\vG\cdot\vr_i}\right) \cdot \left(-i\vG e^{-i\vG\cdot\vr_i}\right)
\right] \\
& = & 2 \sum_{\vG\neq\mathbf{0}} G^2 a_\vG  \left[-\mathbf{Re}\left(e^{i\vG\cdot\vr_i}\rho_{-\vG}\right) + 1\right] 
%  \nabla^2_{\vr_i} J_2 & = & \nabla_{\vr_i} \cdot \nabla_{\vr_i} J_2 \\
%  & = & -2\sum_{\vG \neq \mathbf{0}} a_\vG \vG \cdot \nabla_{\vr_i} \mathbf{Im}\left(e^{i\vG\cdot\vr_i} \rho_{-\vG}\right)
%  & = & -2\sum_{\vG \neq \mathbf{0}} a_\vG \vG \cdot \mathbf{Im}\left(i\vG e^{i\vG\cdot\vr_i}\rho_{-\vG} -i\vG \right)
\end{eqnarray}


\chapter{Development guide}
\label{chap:developguide}

The section gives guidance on how to extend the functionality of QMCPACK. Future examples will likely include topics such as the addition of a Jastrow function or a new QMC method.

\section{QMCPACK coding standards}

This chapter presents what we collectively have agreed are best practices for the code. This includes formatting style, naming conventions, documentation conventions, and certain prescriptions for C++ language use. At the moment only the formatting can be enforced in an objective fashion.

New development should follow these guidelines, and contributors are expected to adhere to them as they represent an integral part of our effort to continue \qmcpack as a world-class, sustainable QMC code. Although some of the source code has a ways to go to live up to these ideas, new code, even in old files, should follow the new conventions not the local conventions of the file whenever possible. Work on the code with continuous improvement in mind rather than a commitment to stasis.

The \href{https://github.com/QMCPACK/qmcpack/wiki/Development-workflow}{current workflow conventions} for the project are described in the wiki on the GitHub repository. It will save you and all the maintainers considerable time if you read these and ask questions up front.

A PR should follow these standards before inclusion in the mainline. You can be sure of properly following the formatting conventions if you use clang-format.  The mechanics of clang-format setup and use can be found at \url{https://github.com/QMCPACK/qmcpack/wiki/Source-formatting}.

The clang-format file found at \ishell{qmcpack/src/.clang-format} should be run over all code touched in a PR before a pull request is prepared. We also encourage developers to run clang-tidy with the \ishell{qmcpack/src/.clang-tidy} configuration over all new code.

As much as possible, try to break up refactoring, reformatting, feature, and bugs into separate, small PR                                                                              s. Aim for something that would take a reviewer no more than an hour. In this way we can maintain a good collective development velocity.

\section{Files}
Each file should start with the header.
\lstset{language=C++,style=C++}
\begin{lstlisting}
//////////////////////////////////////////////////////////////////////////////////////
// This file is distributed under the University of Illinois/NCSA Open Source License.
// See LICENSE file in top directory for details.
//
// Copyright (c) 2018 QMCPACK developers
//
// File developed by: Name, email, affiliation
//
// File created by: Name, email, affiliation
//////////////////////////////////////////////////////////////////////////////////////
\end{lstlisting}
If you make significant changes to an existing file, add yourself to the list of "developed by" authors.

\subsection{File organization}
Header files should be placed in the same directory as their implementations. 
Unit tests should be written for all new functionality. These tests should be placed in a \inlinecode{tests} subdirectory below the implementations.

\subsection{File names}
Each class should be defined in a separate file with the same name as the class name. Use separate \inlinecode{.cpp} implementation files whenever possible to aid in incremental compilation. 

The filenames of tests are composed by the filename of the object tested and the prefix \inlinecode{test_}.
The filenames of \emph{fake} and \emph{mock} objects used in tests are composed by the prefixes \inlinecode{fake_} and \inlinecode{mock_}, respectively, and the filename of the object that is imitated.

\subsection{Header files}
All header files should be self-contained (i.e., not dependent on following any other header when it is included). Nor should they include files that are not necessary for their use (i.e., headers needed only by the implementation). Implementation files should not include files only for the benefit of files they include.

There are many header files that currently violate this.
Each header must use \inlinecode{\#define} guards to prevent multiple inclusion.
The symbol name of the \inlinecode{\#define} guards should be \inlinecode{NAMESPACE(s)_CLASSNAME_H}.

\subsection{Includes}
Header files should be included with the full path based on the \verb|src| directory.
For example, the file \verb|qmcpack/src/QMCWaveFunctions/SPOSet.h| should be included as
\begin{lstlisting}
#include "QMCWaveFunctions/SPOSet.h"
\end{lstlisting}
Even if the included file is located in the same directory as the including file, this rule should be obeyed. Header files from external projects and standard libraries should be includes using the \inlinecode{<iostream>} convention, while headers that are part of the QMCPACK project should be included using the \verb|"our_header.h"| convention.

For readability, we suggest using the following standard order of includes:
\begin{enumerate}
	\item related header
	\item std C library headers
	\item std C++ library headers
	\item Other libraries' headers
	\item QMCPACK headers
\end{enumerate}

In each section the included files should be sorted in alphabetical order.

\section{Naming}
The balance between description and ease of implementation should be balanced such that the code remains self-documenting within a single terminal window.  If an extremely short variable name is used, its scope must be shorter than $\sim 40$ lines. An exception is made for template parameters, which must be in all CAPS.

\subsection{Namespace names}
Namespace names should be one word, lowercase.

\subsection{Type and class names}
Type and class names should start with a capital letter and have a capital letter for each new word.
Underscores (\inlinecode{_}) are not allowed. 

\subsection{Variable names}
Variable names should not begin with a capital letter, which is reserved for type and class names. Underscores (\inlinecode{_}) should be used to separate words.

\subsection{Class data members}
Class private/protected data members names should follow the convention of variable names with a trailing underscore (\inlinecode{_}).

\subsection{(Member) function names}
Function names should start with a lowercase character and have a capital letter for each new word.

\subsection{Lambda expressions}
Named lambda expressions follow the naming convention for functions:

\begin{lstlisting}[showspaces=false]
auto myWhatever = [](int i) { return i + 4; };
\end{lstlisting}

\subsection{Macro names}
Macro names should be all uppercase and can include underscores (\inlinecode{_}).
The underscore is not allowed as first or last character.

\subsection{Test case and test names}
Test code files should be named as follows:
\begin{lstlisting}[showspaces=false]
class DiracMatrix;
//leads to
test_dirac_matrix.cpp
//which contains test cases named
TEST_CASE("DiracMatrix_update_row","[wavefunction][fermion]")
\end{lstlisting}
where the test case covers the \inlinecode{updateRow} and  \inlinecode{[wavefunction][fermion]} indicates the test belongs to the fermion wavefunction functionality.

\section{Comments}
\subsection{Comment style}
Use the \inlinecode{// Comment} syntax for actual comments.
Use
\begin{lstlisting}
/** base class for Single-particle orbital sets
 *
 * SPOSet stands for S(ingle)P(article)O(rbital)Set which contains
 * a number of single-particle orbitals with capabilities of
 * evaluating \f$ \psi_j({\bf r}_i)\f$
 */
\end{lstlisting}
or
\begin{lstlisting}
///index in the builder list of sposets
int builder_index;
\end{lstlisting}

\subsection{Documentation}
Doxygen will be used for source documentation. Doxygen commands should be used when appropriate guidance on this has been decided.

\subsubsection{File docs}
Do not put the file name after the \verb|\file| Doxygen command. Doxygen will fill it in for the file the tag appears in.
\begin{lstlisting}
/** \file
 *  File level documentation 
 */
\end{lstlisting}

\subsubsection{Class docs}
Every class should have a short description (in the header of the file) of what it is and what is does.
Comments for public class member functions follow the same rules as general function comments.
Comments for private members are allowed but are not mandatory.

\subsubsection{Function docs}
For function parameters whose type is non-const reference or pointer to non-const memory,
it should be specified if they are input (In:), output (Out:) or input-output parameters (InOut:).

Example:
\begin{lstlisting}
/** Updates foo and computes bar using in_1 .. in_5.
 * \param[in] in_3
 * \param[in] in_5
 * \param[in,out] foo
 * \param[out] bar
 */

//This is probably not what our clang-format would do
void computeFooBar(Type in_1, const Type& in_2, Type& in_3,
                   const Type* in_4, Type* in_5, Type& foo,
                   Type& bar);
\end{lstlisting}

\subsubsection{Variable documentation}
Name should be self-descriptive.  If you need documentation consider renaming first.

\subsubsection{Golden rule of comments}
If you modify a piece of code, also adapt the comments that belong to it if necessary.

\section{Formatting and ``style''}
Use the provided clang-format style in \inlinecode{src/.clang-format} to format \verb|.h|, \verb|.hpp|, \verb|.cu|, and \verb|.cpp| files. Many of the following rules will be applied to the code by clang-format, which should allow you to ignore most of them if you always run it on your modified code.

You should use clang-format support and the \inlinecode{.clangformat} file with your editor, use a Git precommit hook to run clang-format or run clang-format manually on every file you modify.  However, if you see numerous formatting updates outside of the code you have modified, first commit the formatting changes in a separate PR.

\subsection{Indentation}
Indentation consists of two spaces.
Do not use tabs in the code.

\subsection{Line length}
The length of each line of your code should be at most \emph{120} characters.

\subsection{Horizontal spacing}
No trailing white spaces should be added to any line.
Use no space before a comma (\inlinecode{,}) and a semicolon (\inlinecode{;}), and add a space after them if they are not at the end of a line.

\subsection{Preprocessor directives}
The preprocessor directives are not indented.
The hash is the first character of the line.

\subsection{Binary operators}
The assignment operators should always have spaces around them.

\subsection{Unary operators}
Do not put any space between an unary operator and its argument.

\subsection{Types}
The \inlinecode{using} syntax is preferred to \inlinecode{typedef} for type aliases.
If the actual type is not excessively long or complex, simply use it; renaming simple types makes code less understandable.

\subsection{Pointers and references}
Pointer or reference operators should go with the type. But understand the compiler reads them from right to left.

\begin{lstlisting}
Type* var;
Type& var;

//Understand this is incompatible with multiple declarations
Type* var1, var2; // var1 is a pointer to Type but var2 is a Type.
\end{lstlisting}

\subsection{Templates}
The angle brackets of templates should not have any external or internal padding.
\begin{lstlisting}
template<class C>
class Class1;

Class1<Class2<type1>> object;
\end{lstlisting}

\subsection{Vertical spacing}
Use empty lines when it helps to improve the readability of the code, but do not use too many.
Do not use empty lines after a brace that opens a scope
or before a brace that closes a scope.
Each file should contain an empty line at the end of the file.
Some editors add an empty line automatically, some do not.

\subsection{Variable declarations and definitions}
\begin{itemize}
\item Avoid declaring multiple variables in the same declaration, especially if they are not fundamental types:

\begin{lstlisting}[showspaces=false]
int x, y;                        // Not recommended
Matrix a("my-matrix"), b(size);  // Not allowed

// Preferred
int x;
int y;
Matrix a("my-matrix");
Matrix b(10);
\end{lstlisting}

\item Use the following order for keywords and modifiers in  variable declarations:

\begin{lstlisting}[showspaces=false]
// General type
[static] [const/constexpr] Type variable_name;

// Pointer
[static] [const] Type* [const] variable_name;

// Integer
// the int is not optional not all platforms support long, etc.
[static] [const/constexpr] [signedness] [size] int variable_name;

// Examples:
static const Matrix a(10);
const double* const d(3.14);
constexpr unsigned long l(42);
\end{lstlisting}

\end{itemize}

\subsection{Function declarations and definitions}

The return type should be on the same line as the function name.
Parameters should also be on the same line unless they do not fit on it, in which case one parameter
per line aligned with the first parameter should be used.

Also include the parameter names in the declaration of a function, that is,
\begin{lstlisting}
// calculates a*b+c
double function(double a, double b, double c);

// avoid
double function(double, double, double);

//dont do this
double function(BigTemplatedSomething<double> a, BigTemplatedSomething<double> b,
                BigTemplatedSomething<double> c);

//do this
double function(BigTemplatedSomething<double> a,
                BigTemplatedSomething<double> b,
                BigTemplatedSomething<double> c);

\end{lstlisting}

\subsection{Conditionals}

Examples:
\begin{lstlisting}
if (condition)
  statement;
else
  statement;

if (condition)
{
  statement;
}
else if (condition2)
{
  statement;
}
else
{
  statement;
}
\end{lstlisting}

\subsection{Switch statement}

Switch statements should always have a default case.

Example:
\begin{lstlisting}
switch (var)
{
  case 0:
    statement1;
    statement2;
    break;

  case 1:
    statement1;
    statement2;
    break;

  default:
    statement1;
    statement2;
}
\end{lstlisting}

\subsection{Loops}

Examples:
\begin{lstlisting}
for (statement; condition; statement)
  statement;

for (statement; condition; statement)
{
  statement1;
  statement2;
}

while (condition)
  statement;

while (condition)
{
  statement1;
  statement2;
}

do
{
  statement;
}
while (condition);
\end{lstlisting}

\subsection{Class format}
\label{subsec:class_format}
\inlinecode{public}, \inlinecode{protected}, and \inlinecode{private} keywords are not indented.

Example:
\begin{lstlisting}
class Foo : public Bar 
{
public:
  Foo();
  explicit Foo(int var);

  void function();
  void emptyFunction() {}

  void setVar(const int var)
  {
    var_ = var;
  }
  int getVar() const
  {
    return var_;
  }

private:
  bool privateFunction();

  int var_;
  int var2_;
};
\end{lstlisting}

\subsubsection{Constructor initializer lists}

Examples:
\begin{lstlisting}
// When everything fits on one line:
Foo::Foo(int var) : var_(var) 
{
  statement;
}

// If the signature and the initializer list do not
// fit on one line, the colon is indented by 4 spaces:
Foo::Foo(int var)
    : var_(var), var2_(var + 1)
{
  statement;
}

// If the initializer list occupies more lines,
// they are aligned in the following way:
Foo::Foo(int var)
    : some_var_(var),
      some_other_var_(var + 1) 
{
  statement;
}

// No statements:
Foo::Foo(int var)
    : some_var_(var) {}
\end{lstlisting}

\subsection{Namespace formatting}
The content of namespaces is not indented.
A comment should indicate when a namespace is closed. (clang-format will add these if absent).
If nested namespaces are used, a comment with the full namespace is required after opening a set of namespaces or an inner namespace.

Examples:
\begin{lstlisting}
namespace ns
{
void foo();
}  // ns
\end{lstlisting}

\begin{lstlisting}
namespace ns1
{
namespace ns2
{
// ns1::ns2::
void foo();

namespace ns3
{
// ns1::ns2::ns3::
void bar();
}  // ns3
}  // ns2

namespace ns4
{
namespace ns5
{
// ns1::ns4::ns5::
void foo();
}  // ns5
}  // ns4
}  // ns1
\end{lstlisting}


\section{QMCPACK C++ guidance}
The guidance here, like any advice on how to program, should not be treated as a set of rules but rather the hard-won wisdom of many hours of suffering development. In the past, many rules were ignored, and the absolute worst results of that will affect whatever code you need to work with. Your PR should go much smoother if you do not ignore them.

\subsection{Encapsulation}
A class is not just a naming scheme for a set of variables and functions. It should provide a logical set of methods, could contain the state of a logical object, and might allow access to object data through a well-defined interface related variables, while preserving maximally ability to change internal implementation of the class.

Do not use \inlinecode{struct} as a way to avoid controlling access to the class. Only in rare cases where a class is a fully public data structure \inlinecode{struct} is this appropriate. Ignore (or fix one) the many examples of this in QMCPACK.

Do not use inheritance primarily as a means to break encapsulation. If your class could aggregate or compose another class, do that, and access it solely through its public interface. This will reduce dependencies.

\subsection{Casting}
In C++ source, avoid C style casts; they are difficult to search for and imprecise in function.
An exception is made for controlling implicit conversion of simple numerical types.

Explicit C++ style casts make it clear what the safety of the cast is and what sort of conversion is expected to be possible.

\begin{lstlisting}
int c = 2;
int d = 3;
double a;
a = (double)c / d;  // Ok

const class1 c1;
class2* c2;
c2 = (class2*)&c1; // NO
SPOSetAdvanced* spo_advanced = new SPOSetAdvanced();

SPOSet* spo = (SPOSet*)spo_advanced; // NO
SPOSet* spo = static_cast<SPOSet*>(spo_advanced); // OK if upcast, dangerous if downcast
\end{lstlisting}

\subsection{Pre-increment and pre-decrement}
Use the pre-increment (pre-decrement) operator when a variable is incremented (decremented) and the value of the expression is not used.
In particular, use the pre-increment (pre-decrement) operator for loop counters where i is not used:

\begin{lstlisting}[showspaces=false]
for (int i = 0; i < N; ++i)
{
  doSomething();
}

for (int i = 0; i < N; i++)
{
  doSomething(i);
}
\end{lstlisting}

The post-increment and post-decrement operators create an unnecessary copy that the compiler cannot optimize away in the case of iterators or other classes with overloaded increment and decrement operators.

\subsection{Alternative operator representations}
Alternative representations of operators and other tokens such as \inlinecode{and}, \inlinecode{or}, and \inlinecode{not} instead of \inlinecode{&&}, \inlinecode{||}, and \inlinecode{!} are not allowed.
For the reason of consistency, the far more common primary tokens should always be used.

\subsection{Use of const}
\begin{itemize}
\item Add the \inlinecode{const} qualifier to all function parameters that are not modified in the function body.
\item For parameters passed by value, add only the keyword in the function definition.
\item Member functions should be specified const whenever possible.

\begin{lstlisting}[showspaces=false]
// Declaration
int computeFoo(int bar, const Matrix& m)

// Definition
int computeFoo(const int bar, const Matrix& m)
{
  int foo = 42;

  // Compute foo without changing bar or m.
  // ...

  return foo;
}

class MyClass
{
  int count_
  ...
  int getCount() const { return count_;}
}
\end{lstlisting}

\end{itemize}



\section{Scalar estimator implementation}
\subsection{Introduction: Life of a specialized OperatorBase}

Almost all observables in QMCPACK are implemented as specialized derived classes of the OperatorBase base class. Each observable is instantiated in HamiltonianFactory and added to QMCHamiltonian for tracking. QMCHamiltonian tracks two types of observables: main and auxiliary. Main observables contribute to the local energy. These observables are elements of the simulated Hamiltonian such as kinetic or potential energy. Auxiliary observables are expectation values of matrix elements that do not contribute to the local energy. These Hamiltonians do not affect the dynamics of the simulation. In the code, the main observables are labeled by ``physical'' flag; the auxiliary observables have ``physical'' set to false.

\subsubsection{Initialization}
When an \verb|<estimator type="est_type" name="est_name" other_stuff="value"/>| tag is present in the \verb|<hamiltonian/>| section, it is first read by HamiltonianFactory. In general, the \verb|type| of the estimator will determine which specialization of OperatorBase should be instantiated, and a derived class with \verb|myName="est_name"| will be constructed. Then, the put() method of this specific class will be called to read any other parameters in the \verb|<estimator/>| XML node. Sometimes these parameters will instead be read by HamiltonianFactory because it can access more objects than OperatorBase.

\subsubsection{Cloning}
When \verb|OpenMP| threads are spawned, the estimator will be cloned by the \verb|CloneManager|, which is a parent class of many QMC drivers. 
\begin{lstlisting}[style=C++]
// In CloneManager.cpp
#pragma omp parallel for shared(w,psi,ham)
for(int ip=1; ip<NumThreads; ++ip)
{
  wClones[ip]=new MCWalkerConfiguration(w);
  psiClones[ip]=psi.makeClone(*wClones[ip]);
  hClones[ip]=ham.makeClone(*wClones[ip],*psiClones[ip]);
}
\end{lstlisting}
In the preceding snippet, \verb|ham| is the reference to the estimator on the master thread. If the implemented estimator does not allocate memory for any array, then the default constructor should suffice for the \verb|makeClone| method.
\begin{lstlisting}[style=C++]
// In SpeciesKineticEnergy.cpp
OperatorBase* SpeciesKineticEnergy::makeClone(ParticleSet& qp, TrialWaveFunction& psi)
{
  return new SpeciesKineticEnergy(*this);
}
\end{lstlisting}
If memory is allocated during estimator construction (usually when parsing the XML node in the \verb|put| method), then the \verb|makeClone| method should perform the same initialization on each thread.
\begin{lstlisting}[style=C++]
OperatorBase* LatticeDeviationEstimator::makeClone(ParticleSet& qp, TrialWaveFunction& psi)
{
  LatticeDeviationEstimator* myclone = new LatticeDeviationEstimator(qp,spset,tgroup,sgroup);
  myclone->put(input_xml);
  return myclone;
}
\end{lstlisting}

\subsubsection{Evaluate}
After the observable class (derived class of OperatorBase) is constructed and prepared (by the put() method), it is ready to be used in a QMCDriver. A QMCDriver will call \verb|H.auxHevaluate(W,thisWalker)| after every accepted move, where H is the QMCHamiltonian that holds all main and auxiliary Hamiltonian elements, W is a MCWalkerConfiguration, and thisWalker is a pointer to the current walker being worked on. As shown in the following, this function goes through each auxiliary Hamiltonian element and evaluates it using the current walker configuration. Under the hood, observables are calculated and dumped to the main particle set's property list for later collection.

\begin{lstlisting}[style=C++]
// In QMCHamiltonian.cpp
// This is more efficient. 
// Only calculate auxH elements if moves are accepted.
void QMCHamiltonian::auxHevaluate(ParticleSet& P, Walker_t& ThisWalker)
{
#if !defined(REMOVE_TRACEMANAGER)
  collect_walker_traces(ThisWalker,P.current_step);
#endif
  for(int i=0; i<auxH.size(); ++i)
  {
    auxH[i]->setHistories(ThisWalker);
    RealType sink = auxH[i]->evaluate(P);
    auxH[i]->setObservables(Observables);
#if !defined(REMOVE_TRACEMANAGER)
    auxH[i]->collect_scalar_traces();
#endif
    auxH[i]->setParticlePropertyList(P.PropertyList,myIndex);
  }
}
\end{lstlisting}

For estimators that contribute to the local energy (main observables), the return value of evaluate() is used in accumulating the local energy. For auxiliary estimators, the return value is not used (\verb|sink| local variable above); only the value of Value is recorded property lists by the setObservables() method as shown in the preceding code snippet. By default, the setObservables() method will transfer \verb|auxH[i]->Value| to \verb|P.PropertyList[auxH[i]->myIndex]|. The same property list is also kept by the particle set being moved by QMCDriver. This list is updated by \verb|auxH[i]->setParticlePropertyList(P.PropertyList,myIndex)|, where myIndex is the starting index of space allocated to this specific auxiliary Hamiltonian in the property list kept by the target particle set P.

\subsubsection{Collection}
The actual statistics are collected within the QMCDriver, which owns
an EstimatorManager object. This object (or a clone in the case of
multithreading) will be registered with each mover it owns. For each mover
(such as VMCUpdatePbyP derived from QMCUpdateBase), an accumulate() call
is made, which by default, makes an accumulate(walkerset) call to the
EstimatorManager it owns. Since each walker has a property set, EstimatorManager uses that local copy to calculate statistics. The EstimatorManager performs block averaging and file I/O.

\subsection{Single scalar estimator implementation guide}
Almost all of the defaults can be used for a single scalar observable. With any luck, only the put() and evaluate() methods need to be implemented. As an example, this section presents a step-by-step guide for implementing a \verb|SpeciesKineticEnergy| estimator that calculates the kinetic energy of a specific species instead of the entire particle set. For example, a possible input to this estimator can be:

\verb|<estimator type="specieskinetic" name="ukinetic" group="u"/>|

\verb|<estimator type="specieskinetic" name="dkinetic" group="d"/>|\:.

This should create two extra columns in the \inlinecode{scalar.dat} file that contains the kinetic energy of the up and down electrons in two separate columns. If the estimator is properly implemented, then the sum of these two columns should be equal to the default \verb|Kinetic| column.

\subsubsection{Barebone}

The first step is to create a barebone class structure for this simple scalar estimator. The goal is to be able to instantiate this scalar estimator with an XML node and have it print out a column of zeros in the \inlinecode{scalar.dat} file. 

To achieve this, first create a header file ``SpeciesKineticEnergy.h" in the QMCHamiltonians folder, with only the required functions declared as follows: 

\begin{lstlisting}[style=C++]
// In SpeciesKineticEnergy.h
#ifndef QMCPLUSPLUS_SPECIESKINETICENERGY_H
#define QMCPLUSPLUS_SPECIESKINETICENERGY_H

#include <Particle/WalkerSetRef.h>
#include <QMCHamiltonians/OperatorBase.h>

namespace qmcplusplus
{

class SpeciesKineticEnergy: public OperatorBase
{
public:
  
  SpeciesKineticEnergy(ParticleSet& P):tpset(P){ };
  
  bool put(xmlNodePtr cur);         // read input xml node, required
  bool get(std::ostream& os) const; // class description, required
  
  Return_t evaluate(ParticleSet& P);
  inline Return_t evaluate(ParticleSet& P, std::vector<NonLocalData>& Txy)
  { // delegate responsity inline for speed
    return evaluate(P);
  } 
  
  // pure virtual functions require overrider
  void resetTargetParticleSet(ParticleSet& P) { }                         // required
  OperatorBase* makeClone(ParticleSet& qp, TrialWaveFunction& psi); // required

private:
  ParticleSet& tpset;

}; // SpeciesKineticEnergy

} // namespace qmcplusplus
#endif
\end{lstlisting}

Notice that a local reference \verb|tpset| to the target particle set \verb|P| is saved in the constructor. The target particle set carries much information useful for calculating observables. Next, make ``SpeciesKineticEnergy.cpp," and make vacuous definitions.
\begin{lstlisting}[style=C++]
// In SpeciesKineticEnergy.cpp
#include <QMCHamiltonians/SpeciesKineticEnergy.h>
namespace qmcplusplus
{

bool SpeciesKineticEnergy::put(xmlNodePtr cur)
{
  return true;
} 

bool SpeciesKineticEnergy::get(std::ostream& os) const
{ 
  return true;
}

SpeciesKineticEnergy::Return_t SpeciesKineticEnergy::evaluate(ParticleSet& P)
{
  Value = 0.0;
  return Value;
}

OperatorBase* SpeciesKineticEnergy::makeClone(ParticleSet& qp, TrialWaveFunction& psi)
{
  // no local array allocated, default constructor should be enough
  return new SpeciesKineticEnergy(*this);
}

} // namespace qmcplusplus
\end{lstlisting}

Now, head over to HamiltonianFactory and instantiate this observable if an XML node is found requesting it. Look for ``gofr" in HamiltonianFactory.cpp, for example, and follow the if block.
\begin{lstlisting}[style=C++]
// In HamiltonianFactory.cpp
#include <QMCHamiltonians/SpeciesKineticEnergy.h>
else if(potType =="specieskinetic")
{        
  SpeciesKineticEnergy* apot = new SpeciesKineticEnergy(*target_particle_set);
  apot->put(cur);
  targetH->addOperator(apot,potName,false);
}
\end{lstlisting}
The last argument of addOperator() (i.e., the \verb|false| flag) is \textbf{crucial}. This tells QMCPACK that the observable we implemented is not a physical Hamiltonian; thus, it will not contribute to the local energy. Changes to the local energy will alter the dynamics of the simulation. Finally, add ``SpeciesKineticEnergy.cpp" to HAMSRCS in ``CMakeLists.txt" located in the QMCHamiltonians folder. Now, recompile QMCPACK and run it on an input that requests \verb|<estimator type="specieskinetic" name="ukinetic"/>| in the \verb|hamiltonian| block. A column of zeros should appear in the \inlinecode{scalar.dat} file under the name ``ukinetic."

\subsubsection{Evaluate}
The evaluate() method is where we perform the calculation of the desired observable. In a first iteration, we will simply hard-code the name and mass of the particles.
\begin{lstlisting}[style=C++]
// In SpeciesKineticEnergy.cpp
#include <QMCHamiltonians/BareKineticEnergy.h> // laplaician() defined here
SpeciesKineticEnergy::Return_t SpeciesKineticEnergy::evaluate(ParticleSet& P)
{
  std::string group="u";
  RealType minus_over_2m = -0.5;
  
  SpeciesSet& tspecies(P.getSpeciesSet());
  
  Value = 0.0;
  for (int iat=0; iat<P.getTotalNum(); iat++)
  {
    if (tspecies.speciesName[ P.GroupID(iat) ] == group)
    {
      Value += minus_over_2m*laplacian(P.G[iat],P.L[iat]);
    }
  }
  return Value;
  
  // Kinetic column has:
  // Value = -0.5*( Dot(P.G,P.G) + Sum(P.L) );
}
\end{lstlisting}
\textit{Voila}---you should now be able to compile QMCPACK, rerun, and see that the values in the ``ukinetic'' column are no longer zero. Now, the only task left to make this basic observable complete is to read in the extra parameters instead of hard-coding them.

\subsubsection{Parse extra input}
The preferred method to parse extra input parameters in the XML node is to implement the put() function of our specific observable. Suppose we wish to read in a single string that tells us whether to record the kinetic energy of the up electron (group=``u") or the down electron (group=``d"). This is easily achievable using the OhmmsAttributeSet class,
\begin{lstlisting}[style=C++]
// In SpeciesKineticEnergy.cpp
#include <OhmmsData/AttributeSet.h>
bool SpeciesKineticEnergy::put(xmlNodePtr cur)
{ 
  // read in extra parameter "group"
  OhmmsAttributeSet attrib;
  attrib.add(group,"group");
  attrib.put(cur);
  
  // save mass of specified group of particles
  SpeciesSet& tspecies(tpset.getSpeciesSet());
  int group_id  = tspecies.findSpecies(group);
  int massind   = tspecies.getAttribute("mass");
  minus_over_2m = -1./(2.*tspecies(massind,group_id));
  
  return true;
}
\end{lstlisting}
where we may keep ``group'' and ``minus\_over\_2m'' as local variables to our specific class.
\begin{lstlisting}[style=C++]
// In SpeciesKineticEnergy.h
private:
  ParticleSet& tpset;
  std::string  group;
  RealType minus_over_2m;
\end{lstlisting}
Notice that the previous operations are made possible by the saved reference to the target particle set. Last but not least, compile and run a full example (i.e., a short DMC calculation) with the following XML nodes in your input:

\verb|<estimator type="specieskinetic" name="ukinetic" group="u"/>|

\verb|<estimator type="specieskinetic" name="dkinetic" group="d"/>|\:.\\

Make sure the sum of the ``ukinetic" and ``dkinetic" columns is \textbf{exactly} the same as the Kinetic columns at \textbf{every block}.

For easy reference, a summary of the complete list of changes follows:
\begin{lstlisting}[style=C++]
// In HamiltonianFactory.cpp
#include "QMCHamiltonians/SpeciesKineticEnergy.h"
else if(potType =="specieskinetic")
{
	SpeciesKineticEnergy* apot = new SpeciesKineticEnergy(*targetPtcl);
	apot->put(cur);
	targetH->addOperator(apot,potName,false);
}
\end{lstlisting}
\begin{lstlisting}[style=C++]
// In SpeciesKineticEnergy.h
#include <Particle/WalkerSetRef.h>
#include <QMCHamiltonians/OperatorBase.h>

namespace qmcplusplus
{

class SpeciesKineticEnergy: public OperatorBase
{
public:

  SpeciesKineticEnergy(ParticleSet& P):tpset(P){ };

  // xml node is read by HamiltonianFactory, eg. the sum of following should be equivalent to Kinetic
  // <estimator name="ukinetic" type="specieskinetic" target="e" group="u"/>
  // <estimator name="dkinetic" type="specieskinetic" target="e" group="d"/>
  bool put(xmlNodePtr cur);         // read input xml node, required
  bool get(std::ostream& os) const; // class description, required
  
  Return_t evaluate(ParticleSet& P);
  inline Return_t evaluate(ParticleSet& P, std::vector<NonLocalData>& Txy)
  { // delegate responsity inline for speed
    return evaluate(P);
  } 
  
  // pure virtual functions require overrider
  void resetTargetParticleSet(ParticleSet& P) { }                         // required
  OperatorBase* makeClone(ParticleSet& qp, TrialWaveFunction& psi); // required
  
private:
  ParticleSet&       tpset; // reference to target particle set
  std::string        group; // name of species to track
  RealType   minus_over_2m; // mass of the species !! assume same mass
  // for multiple species, simply initialize multiple estimators
  
}; // SpeciesKineticEnergy

} // namespace qmcplusplus
#endif
\end{lstlisting}
\begin{lstlisting}[style=C++]
// In SpeciesKineticEnergy.cpp
#include <QMCHamiltonians/SpeciesKineticEnergy.h>
#include <QMCHamiltonians/BareKineticEnergy.h> // laplaician() defined here
#include <OhmmsData/AttributeSet.h>

namespace qmcplusplus
{

bool SpeciesKineticEnergy::put(xmlNodePtr cur)
{
  // read in extra parameter "group"
  OhmmsAttributeSet attrib;
  attrib.add(group,"group");
  attrib.put(cur);
  
  // save mass of specified group of particles
  int group_id  = tspecies.findSpecies(group);
  int massind   = tspecies.getAttribute("mass");
  minus_over_2m = -1./(2.*tspecies(massind,group_id)); 

  return true;
}

bool SpeciesKineticEnergy::get(std::ostream& os) const
{ // class description
  os << "SpeciesKineticEnergy: " << myName << " for species " << group;
  return true;
}

SpeciesKineticEnergy::Return_t SpeciesKineticEnergy::evaluate(ParticleSet& P)
{
  Value = 0.0;
  for (int iat=0; iat<P.getTotalNum(); iat++)
  {
    if (tspecies.speciesName[ P.GroupID(iat) ] == group)
    {
      Value += minus_over_2m*laplacian(P.G[iat],P.L[iat]);
    }
  }
  return Value;
}

OperatorBase* SpeciesKineticEnergy::makeClone(ParticleSet& qp, TrialWaveFunction& psi)
{ //default constructor
  return new SpeciesKineticEnergy(*this);
}

} // namespace qmcplusplus
\end{lstlisting}

\subsection{Multiple scalars}
It is fairly straightforward to create more than one column in the \inlinecode{scalar.dat} file with a single observable class. For example, if we want a single SpeciesKineticEnergy estimator to simultaneously record the kinetic energies of all species in the target particle set, we only have to write two new methods: addObservables() and setObservables(), then tweak the behavior of evaluate(). First, we will have to override the default behavior of addObservables() to make room for more than one column in the \inlinecode{scalar.dat} file as follows,
\begin{lstlisting}[style=C++]
// In SpeciesKineticEnergy.cpp
void SpeciesKineticEnergy::addObservables(PropertySetType& plist, BufferType& collectables)
{
  myIndex = plist.size();
  for (int ispec=0; ispec<num_species; ispec++)
  { // make columns named "$myName_u", "$myName_d" etc.
    plist.add(myName + "_" + species_names[ispec]);
  }
}
\end{lstlisting}
where ``num\_species'' and ``species\_name'' can be local variables initialized in the constructor. We should also initialize some local arrays to hold temporary data.
\begin{lstlisting}[style=C++]
// In SpeciesKineticEnergy.h
private:
  int num_species;
  std::vector<std::string> species_names;
  std::vector<RealType> species_kinetic,vec_minus_over_2m;
  
// In SpeciesKineticEnergy.cpp
SpeciesKineticEnergy::SpeciesKineticEnergy(ParticleSet& P):tpset(P)
{
  SpeciesSet& tspecies(P.getSpeciesSet());
  int massind = tspecies.getAttribute("mass");

  num_species = tspecies.size();
  species_kinetic.resize(num_species);
  vec_minus_over_2m.resize(num_species);
  species_names.resize(num_species);
  for (int ispec=0; ispec<num_species; ispec++)
  {
    species_names[ispec] = tspecies.speciesName[ispec];
    vec_minus_over_2m[ispec] = -1./(2.*tspecies(massind,ispec));   
  }
}
\end{lstlisting}
Next, we need to override the default behavior of \icode{setObservables()} to transfer multiple values to the property list kept by the main particle set, which eventually goes into the \inlinecode{scalar.dat} file.
\begin{lstlisting}[style=C++]
// In SpeciesKineticEnergy.cpp
void SpeciesKineticEnergy::setObservables(PropertySetType& plist)
{ // slots in plist must be allocated by addObservables() first
  copy(species_kinetic.begin(),species_kinetic.end(),plist.begin()+myIndex);
}
\end{lstlisting}
Finally, we need to change the behavior of evaluate() to fill the local vector ``species\_kinetic'' with appropriate observable values.
\begin{lstlisting}[style=C++]
SpeciesKineticEnergy::Return_t SpeciesKineticEnergy::evaluate(ParticleSet& P)
{
  std::fill(species_kinetic.begin(),species_kinetic.end(),0.0);

  for (int iat=0; iat<P.getTotalNum(); iat++)
  {
    int ispec = P.GroupID(iat);
    species_kinetic[ispec] += vec_minus_over_2m[ispec]*laplacian(P.G[iat],P.L[iat]);
  }
  
  Value = 0.0; // Value is no longer used
  return Value;
}
\end{lstlisting}
That's it! The SpeciesKineticEnergy estimator no longer needs the ``group'' input and will automatically output the kinetic energy of every species in the target particle set in multiple columns. You should now be able to run with 
\verb|<estimator type="specieskinetic" name="skinetic"/>| and check that the sum of all columns that start with ``skinetic'' is equal to the default ``Kinetic'' column.

\subsection{HDF5 output}
If we desire an observable that will output hundreds of scalars per simulation step (e.g., SkEstimator), then it is preferred to output to the \inlinecode{stat.h5} file instead of the \inlinecode{scalar.dat} file for better organization. A large chunk of data to be registered in the \inlinecode{stat.h5} file is called a ``Collectable'' in QMCPACK. In particular, if a OperatorBase object is initialized with \verb|UpdateMode.set(COLLECTABLE,1)|, then the ``Collectables'' object carried by the main particle set will be processed and written to the \inlinecode{stat.h5} file, where ``UpdateMode'' is a bit set (i.e., a collection of flags) with the following enumeration:
\begin{lstlisting}[style=C++]
// In OperatorBase.h
///enum for UpdateMode
enum {PRIMARY=0,
  OPTIMIZABLE=1,
  RATIOUPDATE=2,
  PHYSICAL=3,
  COLLECTABLE=4,
  NONLOCAL=5,
  VIRTUALMOVES=6
};
\end{lstlisting}

As a simple example, to put the two columns we produced in the previous section into the \inlinecode{stat.h5} file, we will first need to declare that our observable uses ``Collectables.''
\begin{lstlisting}[style=C++]
// In constructor add: 
hdf5_out = true;
UpdateMode.set(COLLECTABLE,1);
\end{lstlisting}
Then make some room in the ``Collectables'' object carried by the target particle set.
\begin{lstlisting}[style=C++]
// In addObservables(PropertySetType& plist, BufferType& collectables) add:
if (hdf5_out)
{
  h5_index = collectables.size();
  std::vector<RealType> tmp(num_species);
  collectables.add(tmp.begin(),tmp.end());
}
\end{lstlisting}
Next, make some room in the \inlinecode{stat.h5} file by overriding the registerCollectables() method.
\begin{lstlisting}[style=C++]
// In SpeciesKineticEnergy.cpp
void SpeciesKineticEnergy::registerCollectables(std::vector<observable_helper*>& h5desc, hid_t gid) const
{
  if (hdf5_out)
  {
    std::vector<int> ndim(1,num_species);
    observable_helper* h5o=new observable_helper(myName);
    h5o->set_dimensions(ndim,h5_index);
    h5o->open(gid);
    h5desc.push_back(h5o);
  }
}
\end{lstlisting}
Finally, edit evaluate() to use the space in the ``Collectables'' object.
\begin{lstlisting}[style=C++]
// In SpeciesKineticEnergy.cpp
SpeciesKineticEnergy::Return_t SpeciesKineticEnergy::evaluate(ParticleSet& P)
{
  RealType wgt = tWalker->Weight; // MUST explicitly use DMC weights in Collectables!
  std::fill(species_kinetic.begin(),species_kinetic.end(),0.0);

  for (int iat=0; iat<P.getTotalNum(); iat++)
  {
    int ispec = P.GroupID(iat);
    species_kinetic[ispec] += vec_minus_over_2m[ispec]*laplacian(P.G[iat],P.L[iat]);
    P.Collectables[h5_index + ispec] += vec_minus_over_2m[ispec]*laplacian(P.G[iat],P.L[iat])*wgt;
  }

  Value = 0.0; // Value is no longer used
  return Value;
}
\end{lstlisting}
There should now be a new entry in the \inlinecode{stat.h5} file containing the same columns of data as the \inlinecode{stat.h5} file. After this check, we should clean up the code by
\begin{enumerate}
\item making ``hdf5\_out'' and input flag by editing the put() method and
\item disabling output to \inlinecode{scalar.dat} when the ``hdf5\_out'' flag is on.
\end{enumerate}



\section{Estimator output}
\subsection{Estimator definition}
For simplicity, consider a local property $O(\bs{R})$, where $\bs{R}$ is the collection of all particle coordinates. An \textit{estimator} for $O(\bs{R}) $ is a weighted average over walkers:
\begin{align}
E[O] = \left(\sum\limits_{i=1}^{N^{tot}_{walker}} w_i O(\bs{R}_i) \right) / \left( \sum \limits_{i=1}^{N^{tot}_{walker}} w_i \right). \label{eq:estimator}
\end{align}
$N^{tot}_{walker}$ is the total number of walkers collected in the entire simulation. Notice that $N^{tot}_{walker}$ is typically far larger than the number of walkers held in memory at any given simulation step. $w_i$ is the weight of walker $i$.

In a VMC simulation, the weight of every walker is 1.0. Further, the number of walkers is constant at each step. Therefore, Equation~\ref{eq:estimator} simplifies to
\begin{align}
E_{VMC}[O] = \frac{1}{N_{step}N_{walker}^{ensemble}} \sum_{s,e} O(\bs{R}_{s,e})\:.
\end{align}
Each walker $\bs{R}_{s,e}$ is labeled by \textit{step index} s and \textit{ensemble index} e.

In a DMC simulation, the weight of each walker is different and may change from step to step. Further, the ensemble size varies from step to step. Therefore, Equation~\ref{eq:estimator} simplifies to
\begin{align}
E_{DMC}[O] = \frac{1}{N_{step}} \sum_{s} \left\{ \left(\sum_e w_{s,e} O(\bs{R}_{s,e})  \right) / \left( \sum \limits_{e} w_{s,e} \right)  \right\}\:.
\end{align}

We will refer to the average in the $\{\}$ as \textit{ensemble average} and to the remaining averages as \textit{block average}. The process of calculating $O(\bs{R})$ is \textit{evaluate}.

\subsection{Class relations}
A large number of classes are involved in the estimator collection process. They often have misleading class or method names. Check out the document gotchas in the following list:
\begin{enumerate}
\item \icode{EstimatorManager} is an unused copy of \icode{EstimatorManagerBase}. \icode{EstimatorManagerBase} is the class used in the QMC drivers. (PR \#371 explains this.)
\item \icode{EstimatorManagerBase::Estimators} is completely different from \icode{QMCDriver::Estimators}, which is subtly different from \icode{QMCHamiltonianBase::Estimators}. The first is a list of pointers to \icode{ScalarEstimatorBase}. The second is the master estimator (one per MPI group). The third is the slave estimator that exists one per OpenMP thread.
\item \icode{QMCHamiltonian} is NOT a parent class of \icode{QMCHamiltonianBase}. Instead, \icode{QMCHamiltonian} owns two lists of \icode{QMCHamiltonianBase} named \icode{H} and \icode{auxH}.
\item \icode{QMCDriver::H} is NOT the same as \icode{QMCHamiltonian::H}. The first is a pointer to a \icode{QMCHamiltonian}. \icode{QMCHamiltonian::H} is a list.
\item \icode{EstimatorManager::stopBlock(std::vector)} is completely different from \icode{EstimatorManager::}
\icode{stopBlock(RealType)}, which is the same as \icode{stopBlock(RealType, true)} but that is subtly different from \icode{stopBlock(RealType, false)}. The first three methods are intended to be called by the master estimator, which exists one per MPI group. The last method is intended to be called by the slave estimator, which exists one per OpenMP thread.
\end{enumerate}

\subsection{Estimator output stages}
%In QMCPACK, evaluation is done by \icode{QMCHamiltonianBase}; ensemble average is done either by a ``CloneDriver'' (e.g. \icode{VMCSingleOMP}, \icode{DMCOMP}) or \icode{ScalarEstimatorBase}; block average is done by \icode{ScalarEstimatorBase} or \icode{EstimatorManagerBase}. Walkers can be accessed by ``CloneDriver'' and \icode{QMCHamiltonianBase} but not by \icode{EstimatorManagerBase} or \icode{ScalarEstimatorBase}. Output files can be accessed by the latter two classes but not the former two. Therefore, in order to output estimators to file, data must be transferred from \textit{evaluate} classes to \textit{average} classes.

Estimators take four conceptual stages to propagate to the output files: evaluate, load ensemble, unload ensemble, and collect. They are easier to understand in reverse order.

\subsubsection{Collect stage}
File output is performed by the master \icode{EstimatorManager} owned by \icode{QMCDriver}. The first 8+ entries in \icode{EstimatorManagerBase::AverageCache} will be written to \icode{scalar.dat}. The remaining entries in \icode{AverageCache} will be written to \icode{stat.h5}. File writing is triggered by \icode{EstimatorManagerBase}\\ \icode{::collectBlockAverages} inside \icode{EstimatorManagerBase::stopBlock}.

\begin{lstlisting}
// In EstimatorManagerBase.cpp::collectBlockAverages
  if(Archive)
  {
    *Archive << std::setw(10) << RecordCount;
    int maxobjs=std::min(BlockAverages.size(),max4ascii);
    for(int j=0; j<maxobjs; j++)
      *Archive << std::setw(FieldWidth) << AverageCache[j];
    for(int j=0; j<PropertyCache.size(); j++)
      *Archive << std::setw(FieldWidth) << PropertyCache[j];
    *Archive << std::endl;
    for(int o=0; o<h5desc.size(); ++o)
      h5desc[o]->write(AverageCache.data(),SquaredAverageCache.data());
    H5Fflush(h_file,H5F_SCOPE_LOCAL);
  }
\end{lstlisting}

\icode{EstimatorManagerBase::collectBlockAverages} is triggered from the master-thread estimator via either \icode{stopBlock(std::vector)} or \icode{stopBlock(RealType, true)}. Notice that file writing is NOT triggered by the slave-thread estimator method \icode{stopBlock(RealType, false)}.

\begin{lstlisting}
// In EstimatorManagerBase.cpp
void EstimatorManagerBase::stopBlock(RealType accept, bool collectall)
{
  //take block averages and update properties per block
  PropertyCache[weightInd]=BlockWeight;
  PropertyCache[cpuInd] = MyTimer.elapsed();
  PropertyCache[acceptInd] = accept;
  for(int i=0; i<Estimators.size(); i++)
    Estimators[i]->takeBlockAverage(AverageCache.begin(),SquaredAverageCache.begin());
  if(Collectables)
  { 
    Collectables->takeBlockAverage(AverageCache.begin(),SquaredAverageCache.begin());
  }
  if(collectall)
    collectBlockAverages(1);
}
\end{lstlisting}

\begin{lstlisting}
// In ScalarEstimatorBase.h
template<typename IT>
inline void takeBlockAverage(IT first, IT first_sq)
{
  first += FirstIndex;
  first_sq += FirstIndex;
  for(int i=0; i<scalars.size(); i++)
  {
    *first++ = scalars[i].mean();
    *first_sq++ = scalars[i].mean2();
    scalars_saved[i]=scalars[i]; //save current block
    scalars[i].clear();
  }
}
\end{lstlisting}

At the collect stage, \icode{ScalarEstimatorBase::scalars} must be populated with ensemble-averaged data. Two derived classes of \icode{ScalarEstimatorBase} are crucial: \icode{LocalEnergyEstimator} will carry \icode{Properties}, where as \icode{CollectablesEstimator} will carry \icode{Collectables}.

\subsubsection{Unload ensemble stage}
\icode{LocalEnergyEstimator::scalars} are populated by
\icode{ScalarEstimatorBase::accumulate}, whereas
\icode{CollectablesEstimator::scalars} are populated by
\icode{CollectablesEstimator::} \icode{accumulate_all}. Both
accumulate methods are triggered by
\icode{EstimatorManagerBase::accumulate}. One confusing aspect about
the unload stage is that \icode{EstimatorManagerBase::accumulate} has
a master and a slave call signature. A slave estimator such as
\icode{QMCUpdateBase::Estimators} should unload a subset of
walkers. Thus, the slave estimator should call
\icode{accumulate(W,it,it_end)}. However, the master estimator, such
as \icode{SimpleFixedNodeBranch::myEstimator}, should unload data from
the entire walker ensemble. This is achieved by calling
\icode{accumulate(W)}.

\begin{lstlisting}
void EstimatorManagerBase::accumulate(MCWalkerConfiguration& W)
{ // intended to be called by master estimator only
  BlockWeight += W.getActiveWalkers();
  RealType norm=1.0/W.getGlobalNumWalkers();
  for(int i=0; i< Estimators.size(); i++)
    Estimators[i]->accumulate(W,W.begin(),W.end(),norm);
  if(Collectables)//collectables are normalized by QMC drivers
    Collectables->accumulate_all(W.Collectables,1.0);
}
\end{lstlisting}

\begin{lstlisting}
void EstimatorManagerBase::accumulate(MCWalkerConfiguration& W
 , MCWalkerConfiguration::iterator it
 , MCWalkerConfiguration::iterator it_end)
{ // intended to be called slaveEstimator only
  BlockWeight += it_end-it;
  RealType norm=1.0/W.getGlobalNumWalkers();
  for(int i=0; i< Estimators.size(); i++)
    Estimators[i]->accumulate(W,it,it_end,norm);
  if(Collectables)
    Collectables->accumulate_all(W.Collectables,1.0);
}
\end{lstlisting}

\begin{lstlisting}
// In LocalEnergyEstimator.h
inline void accumulate(const Walker_t& awalker, RealType wgt)
{ // ensemble average W.Properties
  // expect ePtr to be W.Properties; expect wgt = 1/GlobalNumberOfWalkers
  const RealType* restrict ePtr = awalker.getPropertyBase();
  RealType wwght= wgt* awalker.Weight;
  scalars[0](ePtr[LOCALENERGY],wwght);
  scalars[1](ePtr[LOCALENERGY]*ePtr[LOCALENERGY],wwght);
  scalars[2](ePtr[LOCALPOTENTIAL],wwght);
  for(int target=3, source=FirstHamiltonian; target<scalars.size(); ++target, ++source)
    scalars[target](ePtr[source],wwght);
}
\end{lstlisting}

\begin{lstlisting}
// In CollectablesEstimator.h
inline void accumulate_all(const MCWalkerConfiguration::Buffer_t& data, RealType wgt)
{ // ensemble average W.Collectables
  // expect data to be W.Collectables; expect wgt = 1.0
  for(int i=0; i<data.size(); ++i)
    scalars[i](data[i], wgt);
}
\end{lstlisting}

At the unload ensemble stage, the data structures \icode{Properties} and \icode{Collectables} must be populated by appropriately normalized values so that the ensemble average can be correctly taken. \icode{QMCDriver} is responsible for the correct loading of data onto the walker ensemble.

\subsubsection{Load ensemble stage}
\icode{Properties} in the MC ensemble of walkers \icode{QMCDriver::W} is populated by \icode{QMCHamiltonian}\\ \icode{::saveProperties}. The master \icode{QMCHamiltonian::LocalEnergy}, \icode{::KineticEnergy}, and \icode{::Observables} must be properly populated at the end of the evaluate stage.
\begin{lstlisting}
// In QMCHamiltonian.h
  template<class IT>
  inline
  void saveProperty(IT first)
  { // expect first to be W.Properties
    first[LOCALPOTENTIAL]= LocalEnergy-KineticEnergy;
    copy(Observables.begin(),Observables.end(),first+myIndex);
  }
\end{lstlisting}

\icode{Collectables}'s load stage is combined with its evaluate stage.

\subsubsection{Evaluate stage}

The master \icode{QMCHamiltonian::Observables} is populated by slave \icode{QMCHamiltonianBase}
\icode{::setObservables}. However, the call signature must be \icode{QMCHamiltonianBase::setObservables}
\icode{(QMCHamiltonian::} \\\icode{Observables)}. This call signature is enforced by \icode{QMCHamiltonian::evaluate} and \icode{QMCHamiltonian::} \\\icode{auxHevaluate}.

\begin{lstlisting}
// In QMCHamiltonian.cpp
QMCHamiltonian::Return_t
QMCHamiltonian::evaluate(ParticleSet& P)
{
  LocalEnergy = 0.0;
  for(int i=0; i<H.size(); ++i)
  {
    myTimers[i]->start();
    LocalEnergy += H[i]->evaluate(P);
    H[i]->setObservables(Observables);
#if !defined(REMOVE_TRACEMANAGER)
    H[i]->collect_scalar_traces();
#endif
    myTimers[i]->stop();
    H[i]->setParticlePropertyList(P.PropertyList,myIndex);
  }
  KineticEnergy=H[0]->Value;
  P.PropertyList[LOCALENERGY]=LocalEnergy;
  P.PropertyList[LOCALPOTENTIAL]=LocalEnergy-KineticEnergy;
  // auxHevaluate(P);
  return LocalEnergy;
}
\end{lstlisting}

\begin{lstlisting}
// In QMCHamiltonian.cpp
void QMCHamiltonian::auxHevaluate(ParticleSet& P, Walker_t& ThisWalker)
{
#if !defined(REMOVE_TRACEMANAGER)
  collect_walker_traces(ThisWalker,P.current_step);
#endif
  for(int i=0; i<auxH.size(); ++i)
  {
    auxH[i]->setHistories(ThisWalker);
    RealType sink = auxH[i]->evaluate(P);
    auxH[i]->setObservables(Observables);
#if !defined(REMOVE_TRACEMANAGER)
    auxH[i]->collect_scalar_traces();
#endif
    auxH[i]->setParticlePropertyList(P.PropertyList,myIndex);
  }
}
\end{lstlisting}

\subsection{Estimator use cases}

\subsubsection{VMCSingleOMP pseudo code}
\begin{lstlisting}
bool VMCSingleOMP::run()
{
  masterEstimator->start(nBlocks);
  for (int ip=0; ip<NumThreads; ++ip)
    Movers[ip]->startRun(nBlocks,false);  // slaveEstimator->start(blocks, record)
  
  do // block
  {
    #pragma omp parallel
    {
      Movers[ip]->startBlock(nSteps);  // slaveEstimator->startBlock(steps)
      RealType cnorm = 1.0/static_cast<RealType>(wPerNode[ip+1]-wPerNode[ip]);
      do // step
      {
        wClones[ip]->resetCollectables();
        Movers[ip]->advanceWalkers(wit, wit_end, recompute);
        wClones[ip]->Collectables *= cnorm;
        Movers[ip]->accumulate(wit, wit_end);
      } // end step
      Movers[ip]->stopBlock(false);  // slaveEstimator->stopBlock(acc, false)
    } // end omp
    masterEstimator->stopBlock(estimatorClones);  // write files
  } // end block
  masterEstimator->stop(estimatorClones);
}
\end{lstlisting}

\subsubsection{DMCOMP  pseudo code}
\begin{lstlisting}
bool DMCOMP::run()
{
  masterEstimator->setCollectionMode(true);
  
  masterEstimator->start(nBlocks);
  for(int ip=0; ip<NumThreads; ip++)
    Movers[ip]->startRun(nBlocks,false);  // slaveEstimator->start(blocks, record)
  
  do // block
  {
    masterEstimator->startBlock(nSteps);
    for(int ip=0; ip<NumThreads; ip++)
      Movers[ip]->startBlock(nSteps);  // slaveEstimator->startBlock(steps)
    
    do // step
    {
      #pragma omp parallel
      {
      wClones[ip]->resetCollectables();
      // advanceWalkers
      } // end omp
      
      //branchEngine->branch
      { // In WalkerControlMPI.cpp::branch
      wgt_inv=WalkerController->NumContexts/WalkerController->EnsembleProperty.Weight;
      walkers.Collectables *= wgt_inv;
      slaveEstimator->accumulate(walkers);
      }
      masterEstimator->stopBlock(acc)  // write files
    }  // end for step
  }  // end for block
  
  masterEstimator->stop();
}
\end{lstlisting}

\subsection{Summary}

Two ensemble-level data structures, \icode{ParticleSet::Properties} and \icode{::Collectables}, serve as intermediaries between evaluate classes and output classes to \icode{scalar.dat} and \icode{stat.h5}. \icode{Properties} appears in both \icode{scalar.dat} and \icode{stat.h5}, whereas \icode{Collectables} appears only in \icode{stat.h5}. \icode{Properties} is overwritten by \icode{QMCHamiltonian::Observables} at the end of each step. \icode{QMCHamiltonian::Observables} is filled upon call to \icode{QMCHamiltonian::evaluate} and \icode{::auxHevaluate}. \icode{Collectables} is zeroed at the beginning of each step and accumulated upon call to \icode{::auxHevaluate}.

Data are output to \icode{scalar.dat} in four stages: evaluate, load, unload, and collect. In the evaluate stage, \icode{QMCHamiltonian::Observables} is populated by a list of \icode{QMCHamiltonianBase}. In the load stage, \icode{QMCHamiltonian::Observables} is transfered to \icode{Properties} by \icode{QMCDriver}. In the unload stage, \icode{Properties} is copied to \icode{LocalEnergyEstimator::scalars}. In the collect stage, \icode{LocalEnergyEstimator::scalars} is block-averaged to \icode{EstimatorManagerBase}\\ \icode{::AverageCache} and dumped to file. For \icode{Collectables}, the evaluate and load stages are combined in a call to \icode{QMCHamiltonian::auxHevaluate}. In the unload stage, \icode{Collectables} is copied to \icode{CollectablesEstimator::scalars}. In the collect stage, \icode{CollectablesEstimator}\\ \icode{::scalars} is block-averaged to \icode{EstimatorManagerBase::AverageCache} and dumped to file.

\subsection{Appendix: dmc.dat}

\begin{sloppypar}
There is an additional data structure, \icode{ParticleSet::EnsembleProperty}, that is managed by \icode{WalkerControlBase::EnsembleProperty} and directly dumped to \icode{dmc.dat} via its own averaging procedure. \icode{dmc.dat} is written by \icode{WalkerControlBase::measureProperties}, which is called by \icode{WalkerControlBase::branch}, which is called by \icode{SimpleFixedNodeBranch}\\ \icode{::branch}, for example.
\end{sloppypar}
\section{Slater-Backflow Wavefunction Implementation Details}

For simplicity, consider $N$ identical fermions of the same spin (e.g. up electrons) at spatial locations $\{\bs{r}_1,\bs{r}_2,\dots,\bs{r}_{N}\}$. Then the Slater determinant can be written as
\begin{align}
S=\det M,
\end{align}
where each entry in the determinant is a single-particle orbital evaluated at a particle position
\begin{align}
M_{ij} = \phi_i(\bs{r}_j).
\end{align}

When backflow transformation is applied to the Determinant, the particle coordinates $\bs{r}_i$ that go into the single-particle orbitals are replaced by quasi-particle coordinates $\bs{x}_i$
\begin{align}
M_{ij} = \phi_i(\bs{x}_j), \label{eq:psiM}
\end{align}
where
\begin{align}
\bs{x}_i=\bs{r}_i+\sum\limits_{j=1,j\neq i}^N\eta(r_{ij})(\bs{r}_i-\bs{r}_j). \label{eq:quasi}
\end{align}
$r_{ij}=\vert\bs{r}_i-\bs{r}_j\vert$. The integers i,j label the particle/quasi-particle. There is a one-to-one correspondence between the particles and the quasi-particles, which is simplest when $\eta=0$.

\subsection{Value}
The evaluation of the Slater-Backflow wavefunction is almost identical to that of a Slater wavefunction. The only difference is that the quasi-particles coordinates are used to evaluate the single-particle orbitals. The actual value of the determinant is stored during the inversion of the matrix $M$ (\verb|cgetrf|$\rightarrow$\verb|cgetri|). Suppose $M=LU$, then $S=\prod\limits_{i=1}^N L_{ii} U_{ii}$. \\

\begin{lstlisting}
// In DiracDeterminantWithBackflow::evaluateLog(P,G,L)
Phi->evaluate(BFTrans->QP, FirstIndex, LastIndex, psiM,dpsiM,grad_grad_psiM);
psiMinv = psiM;
LogValue=InvertWithLog(psiMinv.data(),NumPtcls,NumOrbitals
  ,WorkSpace.data(),Pivot.data(),PhaseValue);
\end{lstlisting}

QMCPACK represents the complex value of the wavefunction in polar coordinates $S=e^Ue^{i\theta}$. Specifically, \verb|LogValue| $U$ and \verb|PhaseValue| $\theta$ are handled separately. In the following, I will consider derivatives of the log value only.

\subsection{Gradient}
To evaluate particle gradient of the log value of the Slater-Backflow wavefunction, we can use the $\log\det$ identity in eq.~\ref{eq:logdet}. This identity maps the derivative of $\log\det M$ with respect to a real variable $p$ to a trace over $M^{-1}dM$
\begin{align}
\frac{\partial}{\partial p}\log\det M = \tr\left( M^{-1} \frac{\partial M}{\partial p} \right) \label{eq:logdet}.
\end{align}

Following Kwon, Ceperley, and Martin~\cite{Kwon1993backflow}, the particle gradient
\begin{align}
G_i^\alpha \equiv \frac{\partial}{\partial r_i^\alpha} \log\det M = \sum\limits_{j=1}^N \sum\limits_{\beta=1}^3 F_{jj}^\beta A_{jj}^{\alpha\beta}, \label{eq:grad}
\end{align}
where the quasi-particle gradient matrix
\begin{align}
A_{ij}^{\alpha\beta} \equiv \frac{\partial x_j^\beta}{\partial r_i^\alpha},
\end{align}
and the intermediate matrix
\begin{align}
F_{ij}^\alpha\equiv\sum\limits_k M^{-1}_{ik} dM_{kj}^\alpha,
\end{align}
with the single-particle orbital derivatives (w.r. to quasi-particle coordinates)
\begin{align}
dM_{ij}^\alpha \equiv \frac{\partial M_{ij}}{\partial x_j^\alpha}.
\end{align}
Notice, I have made the name change of $\phi\rightarrow M$ from the notations of ref.~\cite{Kwon1993backflow}. This name change is intended to help the reader associate M with the QMCPACK variable \verb|psiM|.
\begin{lstlisting}
// In DiracDeterminantWithBackflow::evaluateLog(P,G,L)
for(int i=0; i<num; i++) // k in above formula
{
  for(int j=0; j<NumPtcls; j++)
  {
    for(int k=0; k<OHMMS_DIM; k++) // alpha in above formula
    {
      myG(i) += dot(BFTrans->Amat(i,FirstIndex+j),Fmat(j,j));
    }
  }
}
\end{lstlisting}

Eq.~\ref{eq:grad} is still relatively simple to understand. The $A$ matrix maps changes in particle coordinates $d\bs{r}$ to changes in quasi-particle coordinates $d\bs{x}$. Dotting A into F propagates $d\bs{x}$ to $dM$. Thus $F\cdot A$ is the term inside the trace operator of eq.~\ref{eq:logdet}. Finally, performing the trace completes the evaluation of the derivative.

\subsection{Laplacian}
The particle laplacian is given in ref.~\cite{Kwon1993backflow} as
\begin{align}
L_i \equiv \sum\limits_{\beta} \frac{\partial^2}{\partial (r_i^\beta)^2} \log\det M = \sum\limits_{j\alpha} B_{ij}^\alpha F_{jj}^\alpha - \sum\limits_{jk}\sum\limits_{\alpha\beta\gamma} A_{ij}^{\alpha\beta}A_{ik}^{\alpha\gamma}\times\left(F_{kj}^\alpha F_{jk}^\gamma -\delta_{jk}\sum\limits_m M^{-1}_{jm} d2M_{mj}^{\beta\gamma}\right), \label{eq:lap}
\end{align}
where the quasi-particle laplacian matrix
\begin{align}
B_{ij}^{\alpha} \equiv \sum\limits_\beta \frac{\partial^2 x_j^\alpha}{\partial (r_i^\beta)^2},
\end{align}
with the second derivatives of the single-particles orbitals being
\begin{align}
d2M_{ij}^{\alpha\beta} \equiv \frac{\partial^2 M_{ij}}{\partial x_j^\alpha\partial x_j^\beta}.
\end{align}

Schematically, $L_i$ has contributions from 3 terms of the form $BF, AAFF, \tr(AA,Md2M)$, respectively. $A, B, M ,d2M,$ and $F$ can be calculated and stored before the calculations of $L_i$. The first $BF$ term can be directly calculated in a loop over quasi-particle coordinates $j\alpha$.
\begin{lstlisting}
// In DiracDeterminantWithBackflow::evaluateLog(P,G,L)
for(int j=0; j<NumPtcls; j++)
  for(int a=0; a<OHMMS_DIM; k++)
    myL(i) += BFTrans->Bmat_full(i,FirstIndex+j)[a]*Fmat(j,j)[a];
\end{lstlisting}
Notice $B_{ij}^\alpha$ is stored in \verb|Bmat_full|, NOT \verb|Bmat|. 

The remaining two terms both involve $AA$. Thus, it is best to define a temporary tensor $AA$
\begin{align}
{}_iAA_{jk}^{\beta\gamma} \equiv \sum\limits_\alpha A_{ij}^{\alpha\beta} A_{ij}^{\alpha\gamma},
\end{align}
which we will overwrite for each particle $i$. Similarly, define $FF$
\begin{align}
FF_{jk}^{\alpha\gamma} \equiv F_{kj}^\alpha F_{jk}^\gamma,
\end{align}
which is simply the outer product of $F\otimes F$. Then the $AAFF$ term can be calculated by fully contracting $AA$ with $FF$.
\begin{lstlisting}
// In DiracDeterminantWithBackflow::evaluateLog(P,G,L)
for(int j=0; j<NumPtcls; j++)
  for(int k=0; k<NumPtcls; k++)
    for(int i=0; i<num; i++)
    {
      Tensor<RealType,OHMMS_DIM> AA = dot(transpose(BFTrans->Amat(i,FirstIndex+j)),BFTrans->Amat(i,FirstIndex+k));
      HessType FF = outerProduct(Fmat(k,j),Fmat(j,k));
      myL(i) -= traceAtB(AA,FF);
    }
\end{lstlisting}
Finally, define the single-particle orbital derivative term
\begin{align}
Md2M_j^{\beta\gamma} \equiv \sum\limits_m M^{-1}_{jm} d2M_{mj}^\beta,
\end{align}
then the last term is given by the contraction of $Md2M$ (\verb|q_j|) with the diagonal of $AA$.
\begin{lstlisting}
for(int j=0; j<NumPtcls; j++)
{
  HessType q_j;
  q_j=0.0;
  for(int k=0; k<NumPtcls; k++)
    q_j += psiMinv(j,k)*grad_grad_psiM(j,k);
  for(int i=0; i<num; i++)
  {
    Tensor<RealType,OHMMS_DIM> AA = dot(
      transpose(BFTrans->Amat(i,FirstIndex+j)),
      BFTrans->Amat(i,FirstIndex+j)
    );
    myL(i) += traceAtB(AA,q_j);
  }
}
\end{lstlisting}

\subsection{Wavefunction Parameter Derivative}
In order to use the robust linear optimization method of ref.~\cite{Toulouse2007linear}, the trial wavefunction needs to know its contributions to the overlap and hamiltonian matrices. In particular, we need derivatives of these matrices with respect to wavefunction parameters. As a consequence, the wavefunction $\psi$ needs to be able to evaluate $\frac{\partial}{\partial p} \ln \psi$ and $\frac{\partial}{\partial p} \frac{\mathcal{H}\psi}{\psi}$, where $p$ is a parameter.

When two-body backflow is considered, a wavefunction parameter $p$ enters the $\eta$ function only eq.~\ref{eq:quasi}. $\bs{r}$, $\phi$, and $M$ are do not explicitly dependent on $p$. Derivative of the log value is almost identical to particle gradient. Namely, eq.~\ref{eq:grad} applies upon the substitution $r_i^\alpha\rightarrow p$
\begin{align}
\frac{\partial}{\partial p} \ln\det M = \sum\limits_{j=1}^N \sum\limits_{\beta=1}^3 F_{jj}^\beta \left({}_pC_{j}^{\beta}\right),
\end{align}
where the quasi-particle derivatives are stored in \verb|Cmat|
\begin{align}
{}_pC_{i}^{\alpha} \equiv \frac{\partial}{\partial p} x_{i}^{\alpha}.
\end{align}

The change in local kinetic energy is a lot more difficult to calculate
\begin{align}
\frac{\partial T_{\text{local}}}{\partial p} = \frac{\partial}{\partial p} \left\{ \left( \sum\limits_{i=1}^N \frac{1}{2m_i} \nabla^2_i \right) \ln \det M \right\} = \sum\limits_{i=1}^N \frac{1}{2m_i} \frac{\partial}{\partial p} L_i, \label{eq:dK}
\end{align}
where $L_i$ is the particle laplacian defined in eq.~\ref{eq:lap}. In order to evaluate eq.~\ref{eq:dK}, we need to calculate parameter derivatives of all three terms defined in the laplacian evalaution. Namely $(B)(F)$, $(AA)(FF)$, and $\tr(AA,Md2M)$, where I have put parentheses around previously identified data structures. After $\frac{\partial}{\partial p}$ hits, each of the 3 terms will split into two terms by the product rule. Each smaller term will contain a contraction of two data structures. Therefore, we will need to calculate the parameter derivatives of each data structure defined in the laplacian evaluation
\begin{align}
{}_pX_{ij}^{\alpha\beta} \equiv \frac{\partial}{\partial p} A_{ij}^{\alpha\beta} \\
{}_pY_{ij}^{\alpha} \equiv \frac{\partial}{\partial p} B_{ij}^{\alpha} \\
{}_pdF_{ij}^{\alpha} \equiv \frac{\partial}{\partial p} F_{ij}^{\alpha} \\
{}_{pi}{AA'}_{jk}^{\beta\gamma} \equiv \frac{\partial}{\partial p}  {}_iAA_{jk}^{\beta\gamma} \\
{}_p {FF'}_{jk}^{\alpha\gamma} \equiv \frac{\partial}{\partial p} FF_{jk}^{\alpha\gamma} \\
{}_p {Md2M'}_{j}^{\beta\gamma} \equiv \frac{\partial}{\partial p} Md2M_j^{\beta\gamma}
\end{align}
X and Y are stored as \verb|Xmat|, and \verb|Ymat_full| (NOT \verb|Ymat|) in the code. dF is \verb|dFa|. $AA'$ is not fully store, intermediate values are stored in \verb|Aij_sum| and \verb|a_j_sum|. $FF'$ is calculated on-the-fly as $dF\otimes F+F\otimes dF$. $Md2M'$ is not stored, intermediate values are stored in \verb|q_j_prime|.

\section{Particles and distance tables}
\label{sec:distance_tables}

\subsection{ParticleSets}
The \texttt{ParticleSet} class stores particle positions and attributes (charge, mass, etc).

The \texttt{R} member stores positions.
For calculations, the \texttt{R} variable needs to be transferred to the structure-of-arrays (SoA) storage in \texttt{RSoA}.   This is done by the \texttt{update} method.
In the future the interface may change to use functions to set and retrieve positions
so the SoA transformation of the particle data can happen automatically.

A particular distance table is retrieved with \texttt{getDistTable}.
Use \texttt{addTable} to add a \texttt{ParticleSet} and return the index of the distance table.
If the table already exists the index of the existing table will be returned.

The mass and charge of each particle is stored in \texttt{Mass} and \texttt{Z}.
The flag, \texttt{SameMass}, indicates if all the particles have the same mass (true for electrons).

\subsubsection{Groups}

Particles can belong to different groups.
For electrons, the groups are up and down spins.
For ions, the groups are the atomic element.
The group type for each particle can be accessed through the \texttt{GroupID} member.
The number of groups is returned from \texttt{groups()}.
The total number particles is accessed with \texttt{getTotalNum()}.
The number of particles in a group is \texttt{groupsize(int igroup)}.

The particle indices for each group are found with \texttt{first(int igroup)} and \texttt{last(int igroup)}.
These functions only work correctly if the particles are packed according to group.
The flag, \texttt{IsGrouped}, indicates if the particles are grouped or not.
The particles will not be grouped if the elements are not grouped together in the input file.
This ordering is usually the responsibility of the converters.

Code can be written to only handle the grouped case, but put an assert or failure check if the particles are not grouped.  Otherwise the code will give wrong answers and it can be time-consuming to debug.

\subsection{Distance tables}

Distance tables store distances between particles.
There are symmetric (AA) tables for distance between like particles (electron-electron or ion-ion) and asymmetric (BA) tables for distance between unlike particles (electron-ion)

The \texttt{Distances} and \texttt{Displacements} members contain the data.
The indexing order is target index first, then source.
For electron-ion tables, the sources are the ions and the targets are the electrons.

\subsection{Looping over particles}

Some sample code on how to loop over all the particles in an electron-ion distance table:
\begin{verbatim}

// d_table is an electron-ion distance table

for (int jat = 0; j < d_table.targets(); jat++) { // Loop over electrons
  for (int iat = 0; i < d_table.sources(); iat++) { // Loop over ions
     d_table.Distances[jat][iat];
  }
}
\end{verbatim}

Interactions sometimes depend on the type of group of the particles.
The code can loop over all particles and use \texttt{GroupID[idx]} to choose the interaction.
Alternately, the code can loop over the number of groups and then loop from the first to last index for those groups.  This method can attain higher performance by effectively hoisting tests for group ID out of the loop.

An example of the first approach is
\begin{verbatim}

// P is a ParticleSet

for (int iat = 0; iat < P.getTotalNum(); iat++) {
  int group_idx = P.GroupID[iat];
  // Code that depends on the group index
}
\end{verbatim}


An example of the second approach is
\begin{verbatim}
// P is a ParticleSet
assert(P.IsGrouped == true); // ensure particles are grouped

for (int ig = 0; ig < P.groups(); ig++) { // loop over groups
  for (int iat = P.first(ig); iat < P.last(ig); iat++) { // loop over elements in each group
     // Code that depends on group
  }
}

\end{verbatim}



\section{Adding a wavefunction}
\label{sec:adding_wavefunction}

The total wavefunction is stored in \texttt{TrialWaveFunction} as a product of all
the components.  Each component derives from \texttt{WaveFunctionComponent}.
The code contains an example of a wavefunction component for a Helium atom using a simple form and
is described in Section \ref{sec:helium_wavefunction_example}.


\subsection{Mathematical preliminaries}

The wavefunction evaluation functions compute the log of the wavefunction,
the gradient and the Laplacian of the log of the wavefunction.
Expanded, the gradient and Laplacian are
\begin{eqnarray}
G &=& \grad \log(\psi) = \frac{\grad \psi}{\psi} \\
L &=& \lap \log(\psi) = \frac{\lap \psi}{\psi} - \frac{\grad \psi}{\psi} \cdot \frac{\grad \psi}{\psi} \\
                &=& \frac{\lap \psi}{\psi} - G \cdot G
\end{eqnarray}

However, the local energy formula needs $\frac{\lap \psi}{\psi}$.  The conversion from the Laplacian of the log of the wavefunction to the local energy value is performed
in \texttt{QMCHamiltonians/BareKineticEnergy.h} (i.e. $L + G \cdot G$.)


\subsection{Wavefunction evaluation}

The process for creating a new wavefunction component class is to derive
from WaveFunctionComponent and implement a number pure virtual functions.
To start most of them can be empty.

The following four functions evaluate the wavefunction values and spatial derivatives:

\begin{description}
\item{\texttt{evaluateLog}} Computes the log of the wavefunction and the gradient
and Laplacian (of the log of the wavefunction) for all particles.
The input is the\texttt{ParticleSet}(\texttt{P}) (of the electrons).
The return value is the log of wavefunction, and the gradient is in \texttt{G} and Laplacian in \texttt{L}.

\item{\texttt{ratio}} Computes the wavefunction ratio (not the log) for a single particle move ($\psi_{new}/\psi_{old}$).
The inputs are the \texttt{ParticleSet}(\texttt{P}) and the particle index (\texttt{iat}).

\item{\texttt{evalGrad}} Computes the gradient for a given particle.
The inputs are the \texttt{ParticleSet}(\texttt{P}) and the particle index (\texttt{iat}).

\item{\texttt{ratioGrad}} Computes the wavefunction ratio and the gradient at the new position for a single particle move.
The inputs are the \texttt{ParticleSet}(\texttt{P}) and the particle index (\texttt{iat}).
The output gradient is in \texttt{grad\_iat};
\end{description}

The \texttt{updateBuffer} function needs to be implemented, but to start it can simply
call \texttt{evaluateLog}.

The \texttt{put} function should be implemented to read parameter specifics from the input XML file.

\subsection{Function use}

For debugging it can be helpful to know the under what conditions the various
routines are called.

The VMC and DMC loops initialize the walkers by calling \texttt{evaluateLog}.
For all-electron moves, each timestep advance calls \texttt{evaluateLog}.
If the \texttt{use\_drift} parameter is no, then only the wavefunction value is used for sampling.
The gradient and Laplacian are used for computing the local energy.

For particle-by-particle moves, each timestep advance
\begin{enumerate}
\item calls \texttt{evalGrad}
\item computes a trial move
\item calls \texttt{ratioGrad} for the wavefunction ratio and the gradient at the trial position.
(If the \texttt{use\_drift} parameter is no, the \texttt{ratio} function is called instead.)
\end{enumerate}


The following example shows part of an input block for VMC with all-electron moves and drift.

\begin{verbatim}
<qmc method="vmc" target="e" move="alle">
  <parameter name="use_drift">yes</parameter>
</qmc>
\end{verbatim}



\subsection{Particle distances}

The \texttt{ParticleSet} parameter in these functions refers to the electrons.
The distance tables that store the inter-particle distances are stored as an array.

To get the electron-ion distances, add the ion \texttt{ParticleSet} using \texttt{addTable}
and save the returned index. Use that index to get the ion-electron distance table.
\begin{verbatim}
const int ei_id = elecs.addTable(ions, DT_SOA); // in the constructor only
const auto& ei_table = elecs.getDistTable(ei_id); // when consuming a distance table
\end{verbatim}
Getting the electron-electron distances is very similar, just add the electron \texttt{ParticleSet} using \texttt{addTable}.

Only the lower triangle for the electron-electron table should be used.
It is the only part of the distance table valid throughout the run.
During particle-by-particle move, there are extra restrictions.
When a move of electron iel is proposed, only the lower triangle parts [0,iel)[0,iel) [iel, Nelec)[iel, Nelec) and the row [iel][0:Nelec) are valid.
In fact, the current implementation of distance based two and three body Jastrow factors in QMCPACK only needs the row [iel][0:Nelec).

In \texttt{ratioGrad}, the new distances are stored in the \texttt{Temp\_r} and \texttt{Temp\_dr}
members of the distance tables.

\subsection{Setup}

A builder processes XML input, creates the wavefunction, and adds it to \texttt{targetPsi}.
Builders derive from \texttt{WaveFunctionComponentBuilder}.

The new builder hooks into the XML processing in \texttt{WaveFunctionFactory.cpp} in the \texttt{build} function.


\subsection{Caching values}
The \texttt{acceptMove} and \texttt{restore} methods are called on accepted and rejected moves for
the component to update cached values.


\subsection{Threading}
The \texttt{makeClone} function needs to be implemented to work correctly with OpenMP threading.
There will be one copy of the component created for each thread.
If there is no extra storage, calling the copy constructor will be sufficient.
If there are cached values, the clone call may need to create space.


\subsection{Parameter optimization}

The \texttt{checkInVariables}, \texttt{checkOutVariables}, and \texttt{resetParameters} functions manage the variational parameters.
Optimizable variables also need to be registered when the XML is processed.

Variational parameter derivatives are computed in the \texttt{evaluateDerivatives} function.
The first output value is an array with parameter derivatives of log of the wavefunction.
The second output values is an array with parameter derivatives of
the Laplacian divided by the wavefunction (and not the Laplacian of the log of the wavefunction)
The kinetic energy term contains a $-1/2m$ factor.
The $1/m$ factor is applied in \texttt{TrialWaveFunction.cpp}, but the $-1/2$ is not and must be accounted for in this function.


%See \texttt{QMCWaveFunctions/Jastrow/DiffTwoBodyJastrowOrbital.h} for an
%application.
%Also note that it calls into the radial functions for derivatives, but
%it expects the derivative of the logs.

%The conversion is
%\begin{equation}
%\frac{\partial }{\partial B} = \frac{\partial}{\partial B} \lap \log(\psi) +
%2 \frac{\grad \psi}{\psi} \frac{\partial}{\partial B} \frac{\grad \psi}{\psi}
%\end{equation}

\subsection{Helium Wavefunction Example}
\label{sec:helium_wavefunction_example}
The code contains an example of a wavefunction component for a Helium atom using STO orbitals and a Pade Jastrow.

The wavefunction is
\begin{equation}
\psi = \frac{1}{\sqrt{\pi}} \exp(-Z r_1) \exp(-Z r_2) \exp(A / (1 + B r_{12}))
\end{equation}
where $Z = 2$ is the nuclear charge, $A=1/2$ is the electron-electron cusp, and $B$ is a variational parameter.
The electron-ion distances are $r_1$ and $r_2$, and $r_{12}$ is the electron-electron distance.
The wavefunction is the same as the one expressed with built-in components in \texttt{examples/molecules/He/he\_simple\_opt.xml}.

The code is in \texttt{src/QMCWaveFunctions/ExampleHeComponent.cpp}.
The builder is in \texttt{src/QMCWaveFunctions/ExampleHeBuilder.cpp}.
The input file is in \texttt{examples/molecules/He/he\_example\_wf.xml}.
A unit test compares results from the wavefunction evaluation functions for
consistency in \texttt{src/QMCWaveFunctions/tests/test\_example\_he.cpp}.

The recommended approach for creating a new wavefunction component is to copy the example and the unit test.
Implement the evaluation functions and ensure the unit test passes.



%% \renewcommand{\chaptername}{}
%% \renewcommand{\thechapter}{}
\chapter*{References}
\addcontentsline{toc}{chapter}{References}
\begin{btSect}{bibliography}
\btPrintCited
\end{btSect}
\end{btUnit}

\appendix
\chapter{Derivation of twist averaging efficiency}
\label{sec:app_ta_efficiency}
In this appendix we derive the relative statistical efficiency of 
twist averaging with an irreducible (weighted) set of k-points 
versus using uniform weights over an unreduced set of k-points 
(e.g., a full Monkhorst-Pack mesh).

Consider the weighted average of a set of statistical variables 
$\{x_m\}$ with weights $\{w_m\}$:
\begin{align}
  x_{TA} = \frac{\sum_mw_mx_m}{\sum_mw_m}\:.
\end{align} 
If produced by a finite QMC run at a set of 
twist angles/k-points $\{k_m\}$, each variable mean $\mean{x_m}$ 
has a statistical error bar $\sigma_m$, and we can also obtain 
the statistical error bar of the mean of the twist-averaged 
quantity $\mean{x_{TA}}$:
\begin{align}
  \sigma_{TA} = \frac{\left(\sum_mw_m^2\sigma_m^2\right)^{1/2}}{\sum_mw_m}\:.
\end{align}
The error bar of each individual twist $\sigma_m$ is related to the 
autocorrelation time $\kappa_m$,  intrinsic variance $v_m$, and the number 
of postequilibration MC steps $N_{step}$ in the following way:
\begin{align}
  \sigma_m^2=\frac{\kappa_mv_m}{N_{step}}\:.
\end{align}
In the setting of twist averaging, the autocorrelation time and 
variance for different twist angles are often very similar across 
twists, and we have
\begin{align}
  \sigma_m^2=\sigma^2=\frac{\kappa v}{N_{step}}\:.
\end{align} 
If we define the total weight as $W$, that is, $W\equiv\sum_{m=1}^Mw_m$, 
for the weighted case with $M$ irreducible twists, the error bar is
\begin{align}
  \sigma_{TA}^{weighted}=\frac{\left(\sum_{m=1}^Mw_m^2\right)^{1/2}}{W}\sigma\:.
\end{align}
For uniform weighting with $w_m=1$, the number of twists is $W$ and 
we have
\begin{align}
  \sigma_{TA}^{uniform}=\frac{1}{\sqrt{W}}\sigma\:.
\end{align}
We are interested in comparing the efficiency of choosing weights 
uniformly or based on the irreducible multiplicity of each twist angle 
for a given target error bar $\sigma_{target}$.  The number of MC  
steps required to reach this target for uniform weighting is
\begin{align}
  N_{step}^{uniform} = \frac{1}{W}\frac{\kappa v}{\sigma_{target}^2}\:,
\end{align}
while for nonuniform weighting we have
\begin{align}\label{eq:weighted_step}
  N_{step}^{weighted} &= \frac{\sum_{m=1}^Mw_m^2}{W^2}\frac{\kappa v}{\sigma_{target}^2} \nonumber\:,\\
                  &=\frac{\sum_{m=1}^Mw_m^2}{W}N_{step}^{uniform}\:.
\end{align}
The MC efficiency is defined as 
\begin{align}
  \xi = \frac{1}{\sigma^2t}\:,
\end{align}
where $\sigma$ is the error bar and $t$ is the total CPU time required 
for the MC run.  

The main advantage made possible by irreducible twist weighting is to 
reduce the equilibration time overhead by having fewer twists and, 
hence, fewer MC runs to equilibrate.  In the context of twist 
averaging, the total CPU time for a run can be considered to be
\begin{align}
  t=N_{twist}(N_{eq}+N_{step})t_{step}\:,
\end{align}
where $N_{twist}$ is the number of twists, $N_{eq}$ is the number of MC steps required to reach equilibrium, $N_{step}$ is the number 
of MC steps included in the statistical averaging as before, 
and $t_{step}$ is the wall clock time required to complete a single 
MC step. For uniform weighting $N_{twist}=W$; while for irreducible 
weighting $N_{twist}=M$.

We can now calculate the relative efficiency ($\eta$) of irreducible vs. 
uniform twist weighting with the aim of obtaining a target error bar 
$\sigma_{target}$:
\begin{align}
  \eta &= \frac{\xi_{TA}^{weighted}}{\xi_{TA}^{uniform}} \nonumber\:, \\
       &= \frac{\sigma_{target}^2t_{TA}^{uniform}}{\sigma_{target}^2t_{TA}^{weighted}} \nonumber\:, \\
       &= \frac{W(N_{eq}+N_{step}^{uniform})}{M(N_{eq}+N_{step}^{weighted})} \nonumber\:, \\
       &= \frac{W(N_{eq}+N_{step}^{uniform})}{M(N_{eq}+\frac{\sum_{m=1}^Mw_m^2}{W}N_{step}^{uniform})} \nonumber\:, \\
       &= \frac{W}{M}\frac{1+f}{1+\frac{\sum_{m=1}^Mw_m^2}{W}f}\:.
\end{align}
In this last expression, $f$ is the ratio of the number of usable 
MC steps to the number that must be discarded during equilibration 
($f=N_{step}^{uniform}/N_{eq}$); and as before, $W=\sum_mw_m$, which is the number of 
twist angles in the uniform weighting case.  It is important to recall 
that $N_{step}^{uniform}$ in $f$ is defined relative to uniform weighting and is 
the number of MC steps required to reach a target accuracy in the 
case of uniform twist weights.

The formula for $\eta$ in the preceding can be easily changed with the help of 
Equation~\ref{eq:weighted_step} to reflect the 
number of MC steps obtained in an irreducibly weighted run 
instead.  A good exercise is to consider runs that have already completed 
with either uniform or irreducible weighting and calculate the 
expected efficiency change had the opposite type of weighting been used.

The break even point $(\eta=1)$ can be found at a usable step fraction of 
\begin{align}
  f=\frac{W-M}{M\frac{\sum_{m=1}^Mw_m^2}{W}-W}\:.
\end{align}

The relative efficiency $(\eta)$ is useful to consider in view of certain 
scenarios.  An important case is where the number of required sampling 
steps is no larger than the number of equilibration steps (i.e., 
$f\approx 1$).  For a very simple case with eight uniform twists with 
irreducible multiplicities of $w_m\in\{1,3,3,1\}$ ($W=8$, $M=4$), the 
relative efficiency of irreducible vs. uniform weighting is 
$\eta=\frac{8}{4}\frac{2}{1+20/8}\approx 1.14$.  In this case, 
irreducible weighting is about $14$\% more efficient than uniform weighting.

Another interesting case is one in which the number of sampling steps you can 
reach with uniform twists before wall clock time runs out is small 
relative to the number of equilibration steps ($f\rightarrow 0$). 
In this limit, $\eta\approx W/M$.  For our eight-uniform-twist example, this would 
result in a relative efficiency of $\eta=8/4=2$, making irreducible 
weighting twice as efficient.

A final case of interest is one in which the equilibration time is short 
relative to the available sampling time $(f\rightarrow\infty)$, 
giving $\eta\approx W^2/(M\sum_{m=1}^Mw_m^2)$.  Again, for our simple example 
we find $\eta=8^2/(4\times 20)\approx 0.8$, with uniform weighting being 
$25$\% more efficient than irreducible weighting. For this example, the crossover point for irreducible weighting being 
more efficient than uniform weighting is $f<2$, that is, when the 
available sampling period is less than twice the length of the equilibration 
period.  The expected efficiency ratio and crossover point should be checked 
for the particular case under consideration to inform the choice between   
twist averaging methods.




\begin{btUnit}
\chapter{QMCPACK papers}
% Uset the bibtopic package to generate a bibliography local to this btsection below.
% btPrintAll creates the bibliography citing all entries in the files qmcpack_papers.bib
\begin{btSect}{qmcpack_papers}

  The following is a list of all papers, theses, and book chapters
  known to use QMCPACK. Please let the developers know if your paper
  is missing, if you know of other works, or an entry is incorrect. We
  list papers whether they cite QMCPACK directly or not. This list
  will be placed on the \url{http://www.qmcpack.org} website.

\btPrintAll

\end{btSect}
 

\end{btUnit}
\end{document}
